% !TeX root = ../introduccion.tex


\begin{enunciado}
    Clasifica los siguientes procesos en un nivel de madurez dentro del estándar ISO/IEC 5504.
    Justifica la respuesta.
    \begin{enumerate}
        \item Proceso consistente en cuatro fases: definición, planificación, desarrollo y entrega.
        La definición genera un documento de funcionalidades del software.
        Esta definición se usa para crear un esquema temporal del desarrollo hasta la entrega.
        Durante la fase de desarrollo se completan las tareas descritas en el esquema temporal.
        El producto del desarrollo se entrega al cliente.

        \item Proceso consistente en siete fases: análisis, proposición, decisión, prototipado, validación, desarrollo y entrega.
        Las fases de desarrollo se ejecutan de manera consecutiva consiguiendo entregas de valor cada 6 meses.

        \item Proceso iterativo consistente en cinco fases: análisis, diseño, planificación, desarrollo y entrega.
        Cada fase del proceso genera un artefacto de trabajo que alimenta la siguiente fase.
        En cada iteración se establecen unos objetivos de entrega de valor.
        Se miden los tiempos de ejecución de cada una de las fases.
    \end{enumerate}
\end{enunciado}

\begin{solucion}
    \begin{description}
        \item[1.] \textbf{Nivel 2: Gestionado.}
        El proceso sigue una estructura clara con fases bien definidas.
        Se generan artefactos como el documento de funcionalidades y un esquema temporal, que son controlados y utilizados para guiar el desarrollo.
        No se menciona recogida de datos cuantitativos ni mecanismos de mejora continua, por lo que no alcanza un nivel superior.

        \item[2.] \textbf{Nivel 1: Realizado.}
        Las fases están claramente delimitadas y ejecutadas de forma secuencial, con entregas regulares de valor cada 6 meses.
        No hay ningún mecanismo de control, por lo que no se puede considerar un proceso gestionado.


        \item[3.] \textbf{Nivel 3: Establecido.}
        El proceso iterativo ha sido \textbf{definido} y \textbf{estabilizado}.
        Cada fase del proceso genera un artefacto de trabajo que alimenta la siguiente fase.
        Aunque no oficial, hay una estandarización del flujo de trabajo.
        Se miden los tiempos de ejecución de cada una de las fases, lo que permite una mejora continua del proceso.
        La recogida de tiempos indica un inicio de control, pero no se relaciona con su integración en los objetivos del proyecto.
    \end{description}
\end{solucion}
