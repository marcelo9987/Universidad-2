\subsection{Metodología y procesos}\label{subsec:metodologia-y-procesos}
Para el desarrollo exitoso de un software se requiere de ciertos \textquote{componentes} que permiten que la \textquote{máquina} funciones sin \textquote{atascarse}.
Estos \textquote{engranajes} son:
\begin{enumerate}
    \item Herramientas: Tanto control de versiones (git, subversion,\dots), como IDEs y otros \textbf{programas o dispositivos} que ayudan al desarrollo.
    \item Métodos: \textbf{Técnicas} usadas para facilitar el desarrollo.
    \item Proceso: \textbf{Actividades, acciones y tareas} realizadas con el fin de crear productos.
\end{enumerate}

\subsection{Proceso esencial}\label{subsec:proceso-esencial}
En un desarrollo habitual, el proceso suele ser el siguiente:

\begin{enumerate}
    \item \textbf{Entender el problema:} Comunicación y análisis (participantes, incógnitas).
    \item \textbf{Planear la solución:} Modelado y diseño (descomposición del problema).
    \item \textbf{Ejecutar el plan:} Creación de código (seguimiento y revisión).
    \item \textbf{Examinar el resultado:} Pruebas y aseguramiento de la calidad.
\end{enumerate}

\subsection{Definiciones}\label{subsec:definiciones-metodologia-y-procesos}
\begin{definicion}
    Un \textbf{proceso} es aquel \textbf{conjunto de actividades, acciones y tareas} que se ejecutan cuando va a crearse algún producto del trabajo.
\end{definicion}
\begin{definicion}
    Una \textbf{actividad} busca lograr un \textbf{objetivo amplio} sin importar el dominio de la aplicación, tamaño del proyecto, complejidad del esfuerzo.
\end{definicion}

\begin{definicion}
    Una \textbf{acción} es un \textbf{conjunto de tareas} que producen un producto importante del trabajo.
\end{definicion}
\begin{definicion}
    Una \textbf{tarea} se centra en un objetivo \textbf{pequeño pero bien definido} que produce un \textbf{resultado tangible}.
\end{definicion}
\begin{definicion}
    Un \textbf{proceso} no es una prescripción rígida.
    Es un enfoque\textbf{adaptable} que permite entregar el software de \textbf{forma oportuna} y con \textbf{calidad suficiente}.
\end{definicion}

\subsection{Fases del proceso}\label{subsec:fases-del-proceso}
El proceso de ingeniería del software se divide en 5 sencillas fases: \textbf{Comunicación}, \textbf{planificación}, \textbf{diseño}, \textbf{desarrollo} y, \textbf{despliegue}.
Se puede observar que es y que objetivo tiene cada fase en la \autoref{tab:fases-desarrollo}

\begin{table}[hbtp]
    \centering

    \begin{tabularx}{\textwidth}{|l l X|}\toprule
    Fase & Objetivo& Detalles \\\midrule

    \textbf{Comunicación} & Entender requisitos.& Colaboración con participantes, definición de características. \\
    \textbf{Planificación} & Definir el traballo& Tareas técnicas, riesgos, recursos, cronograma.\\
    \textbf{Diseño}& Plantear solución & Diagramas de alto nivel, refinamiento iterativo.\\
    \textbf{Desarrollo}& Construír y verificar& Porgramación y pruebas.\\
    \textbf{Despliegue}& Entregar produto & Evaluación y \textit{feedback}.\\ \bottomrule

    \end{tabularx}
    \caption{Comparación de las distintas fases del proceso de desarrollo de software}
    \label{tab:fases-desarrollo}

\end{table}


\subsection{Tipos de flujo de fases}\label{subsec:tipos-de-flujo-de-fases}
Existen varios tipos de flujo de fases.
A continuación, varios diagramas con los distintos flujos:

\subsubsection{Lineal}

De una en una, sencillo.

\deactivatequoting
\tikz
{
    \node [rectangle, draw] (A) {Comunicación};
    \node [rectangle, draw] (B) [right= of A] {Planeación};
    \node [rectangle, draw] (C) [right=of B] {Modelado};
    \node [rectangle, draw] (D) [right=of C] {Construcción};
    \node [rectangle, draw] (E) [right=of D] {Despliegue};

    \draw[
        -{Latex}
    ,draw=black
    , thick
    ]
    % Básicos
    (A) edge (B)
    (B) edge (C)
    (C) edge (D)
    (D) edge (E)
}
\activatequoting

\subsubsection{Iterativo}

\deactivatequoting
\tikz
{
    \node [rectangle, draw] (A) {Comunicación};
    \node [rectangle, draw] (B) [right= of A] {Planeación};
    \node [rectangle, draw] (C) [right=of B] {Modelado};
    \node [rectangle, draw] (D) [right=of C] {Construcción};
    \node [rectangle, draw] (E) [right=of D] {Despliegue};

    \draw[
        -{Latex}
    ,draw=black
    , thick
    ]
    % Básicos
    (A) edge (B)
    (B) edge (C)
    (C) edge (D)
    (D) edge (E)

%Iterativo
    (B) edge[bend left = 45] (A)
    (C) edge[in=-170, out=-10,looseness=5] (C)
    (D) edge[bend left = 45] (A)
}
\activatequoting

\subsubsection{Evolutivo}
Permite un prototipado rápido.
Resiliente y con una planificación adaptativa.

\deactivatequoting
\tikz
{
    \node [rectangle, draw] (A) {Comunicación};
    \node [rectangle, draw] (B) [above right    = of A] {Planeación};
    \node [rectangle, draw] (C) [below right    = of B] {Modelado};
    \node [rectangle, draw] (D) [below          = of C] {Construcción};
    \node [rectangle, draw] (E) [below left     = of D] {Despliegue};
    \node [rectangle]       (F) [left           = of E] {Incremento obtenido};

    \draw[
        -{Latex}
    ,draw=black
    , thick
    ]
    % Básicos
    (A) edge (B)
    (B) edge (C)
    (C) edge (D)
    (D) edge (E)
    (E) edge (A)
    (E) edge (F)

}
\activatequoting

\subsubsection{Paralelo}
Fases en paralelo para optimizar la eficiencia

\deactivatequoting
\tikz
{
    \node [rectangle, draw] (A) {Comunicación};
    \node [rectangle, draw] (B) [right          = of A] {Planeación};
    \node [rectangle, draw] (C) [below right    = of A] {Modelado};
    \node [rectangle, draw] (D) [below right    = of C] {Construcción};
    \node [rectangle, draw] (E) [right          = of D] {Despliegue};
    \node [rectangle]       (F) [right          = of C] {Tiempo};

    \draw[
        -{Latex}
    ,draw=black
    , thick
    ]
    % Básicos
    (A) edge (B)
    (A) edge (C)
    (B) edge (C)
    (C) edge (D)
    (D) edge (E)

}
\activatequoting

\subsection{Actividades transversales}\label{subsec:actividades-transversales}
Las siguientes tareas se llevan a cabo de forma mas o menos contínua durante la duración total del proyecto:
\begin{itemize}
    \item \textbf{Seguimiento y control:} Evaluación del estado del proyecto con respecto al plan original.

    \item \textbf{Administración de riesgos:} Identificación y mitigación.

    \item \textbf{Seguimiento de calidad del software:} Revisiones y estándares que \textbf{garantizan} que el software tiene la calidad prometida.


    \item \textbf{Medición:} Métricas de proceso, proyecto y produto.

    \item \textbf{Gestión de la configuración:} Control de cambios.

    \item \textbf{Reutilización:} Creación e uso de componentes reutilizables.
    \item \textbf{Preparación y producción del producto del trabajo:} Actividades  para crear productos del trabajo, tales como modelos y documentación.

\end{itemize}