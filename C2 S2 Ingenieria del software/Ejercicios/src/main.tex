%!root = ./main.tex
\documentclass[a4paper,11pt]{article}

% Codificación e idioma
%\usepackage[T1]{fontenc}
\usepackage[spanish]{babel}

% Estilo y herramientas
\usepackage{geometry}
\geometry{margin=2.5cm}
\usepackage{amsmath, amssymb}
\usepackage{enumitem}
\usepackage{tcolorbox}
\usepackage{fancyhdr}
\usepackage{tikz}
\usepackage{hyperref}
\usepackage{graphicx}
\usepackage{booktabs}
\usepackage{csquotes}
\usepackage{xcolor}
\usepackage{pgf-umlsd}

\usetikzlibrary{graphs,shapes,arrows,positioning,shadows,calc}

% Definir estilos
\tikzset{
    service/.style={
        rectangle,
        draw=black,
        fill=white,
        minimum width=2.5cm,
        minimum height=1cm,
        align=center,
        font=\small,
        drop shadow
    },
    database/.style={
        cylinder,
        shape border rotate=90,
        aspect=0.25,
        draw=black,
        fill=gray!20,
        minimum width=0.8cm,
        minimum height=1cm
    },
    interface/.style={
        rectangle,
        draw=black,
        fill=blue!20,
        minimum width=4cm,
        minimum height=0.8cm,
        font=\small\bfseries
    },
    arrow/.style={
        ->,
        >=stealth,
        thick
    },
    bidirectional/.style={
        <->,
        >=stealth,
        thick
    }
}

% Configuración de colores personalizados
\definecolor{primaryblue}{RGB}{25, 118, 210}
\definecolor{lightblue}{RGB}{227, 242, 253}
\definecolor{darkgray}{RGB}{66, 66, 66}
\definecolor{lightgray}{RGB}{245, 245, 245}
\definecolor{successgreen}{RGB}{76, 175, 80}
\definecolor{warningorange}{RGB}{255, 152, 0}

% Configuración mejorada de tcolorbox
\tcbuselibrary{skins,breakable}

% Estilos personalizados para cajas
\newtcolorbox{enunciado}{
    enhanced,
    breakable,
    colback=lightblue,
    colframe=primaryblue,
    arc=4mm,
    boxrule=1.5pt,
    title={\textbf{Enunciado}},
    fonttitle=\Large\bfseries,
    attach boxed title to top left={yshift=-3mm, xshift=4mm},
    boxed title style={
        enhanced,
        colback=primaryblue,
        colframe=primaryblue,
        arc=2mm
    },
    drop shadow,
    left=8pt,
    right=8pt,
    top=12pt,
    bottom=8pt
}

\newtcolorbox{solucion}[1][Solución]{
    enhanced,
    breakable,
    colback=white,
    colframe=successgreen,
    arc=3mm,
    boxrule=1pt,
    title={\textbf{Solución}},
    fonttitle=\large\bfseries,
    attach boxed title to top left={yshift=-2mm, xshift=4mm},
    boxed title style={
        enhanced,
        colback=successgreen,
        colframe=successgreen,
        arc=2mm
    },
    left=6pt,
    right=6pt,
    top=8pt,
    bottom=6pt,
    title=#1
}

\newtcolorbox{importante}{
    enhanced,
    breakable,
    colback=orange!10,
    colframe=warningorange,
    arc=3mm,
    boxrule=1pt,
    title={\textbf{¡Importante!}},
    fonttitle=\large\bfseries,
    attach boxed title to top left={yshift=-2mm, xshift=4mm},
    boxed title style={
        enhanced,
        colback=warningorange,
        colframe=warningorange,
        arc=2mm
    },
    left=6pt,
    right=6pt,
    top=8pt,
    bottom=6pt
}

%-- --

% Configuración de siunitx
\usepackage[per-mode=symbol, group-minimum-digits=4]{siunitx}
\sisetup{group-separator = {\:\!}}
\DeclareSIUnit\puntosFuncion{PF}
\DeclareSIUnit\porMes{pm}
\DeclareSIUnit\linesofcode{LOC}
\DeclareSIUnit\personameses{persona\text{\textbf{-}}meses}
\DeclareSIUnit\meses{meses}

% Configuración de pgfgantt
\usepackage{pgfgantt}

% Encabezado y pie de página mejorado
\pagestyle{fancy}
\fancyhf{}
\fancyhead[L]{\textcolor{primaryblue}{\textbf{Ingeniería del Software}}}
\fancyhead[R]{\textcolor{darkgray}{Ejercicios para el examen}}
\fancyfoot[C]{\textcolor{darkgray}{\thepage}}
\renewcommand{\headrulewidth}{0.8pt}
\renewcommand{\footrulewidth}{0.4pt}
\renewcommand{\headrule}{\hbox to\headwidth{\color{primaryblue}\leaders\hrule height \headrulewidth\hfill}}

% Configuración de secciones
\setcounter{secnumdepth}{0}

%configuracion de indice
\setcounter{tocdepth}{1}

% para que se calle el pesado de fancyhdr
\setlength{\headheight}{24pt}

% Configuración de hyperref
\hypersetup{
    colorlinks=true,
    linkcolor=primaryblue,
    filecolor=primaryblue,
    urlcolor=primaryblue,
    citecolor=primaryblue,
    pdfpagemode=FullScreen,
}

% Metadatos
\title{\Huge\textbf{Ejercicios} \\ \Large\textcolor{primaryblue}{Ingeniería del Software}}
\author{\textbf{Marcelo Fort Muñoz}}
\date{\today}

\begin{document}

% Página de título personalizada
    \begin{titlepage}
        \centering
        \vspace*{2cm}

        {\Huge\textbf{Ejercicios}}\\[0.5cm]
        {\Large\textcolor{primaryblue}{\textbf{Ingeniería del Software}}}\\[2cm]

        \begin{tikzpicture}
            \draw[primaryblue, line width=2pt] (0,0) -- (8,0);
        \end{tikzpicture}\\[2cm]

        {\Large\textbf{Marcelo Fort Muñoz}}\\[1cm]
        {\large\textcolor{darkgray}{\today}}

        \vfill

        \begin{tikzpicture}
            \draw[lightgray, line width=1pt] (0,0) -- (8,0);
        \end{tikzpicture}
    \end{titlepage}

    \tableofcontents
    \newpage


    \part{\textcolor{primaryblue}{Introducción}}\label{part:introduccion}
    % !TeX root = ./main.tex


\section{Ejercicio 1}\label{sec:intro-ej1}
% !TeX root = ../examen-parcial-2023.tex


\begin{itemize}
    \item \textbf{Puntos:} 2
\end{itemize}
\begin{enunciado}
    El actual equipo de desarrollo está aplicando las siguientes prácticas:
    \begin{enumerate}
        \item El equipo planifica entregas trimestrales que incluyen un conjunto de funcionalidades acordadas entre el director de ingeniería y el director de producción dentro un plan anual.
        \item El equipo se asegura que el conjunto de funcionalidades de las entregas trimestrales se comporta correctamente y es usado por los usuarios finales sin dificultades.
        \item El equipo se comunica directamente con el director de producción cuando tiene dudas acerca de cómo debe comportarse una funcionalidad concreta.
        \item Dentro del equipo cada miembro tiene su función: una persona diseña la solución, otro la construye, otro la prueba y otro la despliega y mantiene en producción.
    \end{enumerate}
    Lee detenidamente las prácticas e:
    \begin{enumerate}
        \item Identifica, para cada una de ellas, si se corresponden con prácticas ágiles.
        \item Justifica las respuestas en base al manifiesto ágil.
        \item \textbf{0,3 cada respuesta correcta con justificación.}
        \item Indica qué cambios aplicarías en las que no son ágiles (si hay alguna) para que sí lo sean.
        \item \textbf{0,4 por cada práctica convertida en ágil.}
    \end{enumerate}
\end{enunciado}

\begin{solucion}
    \begin{enumerate}
        \item \textbf{Práctica 1:} NO ágil.
        Se incumple el valor Respuesta ante el cambio sobre seguir un plan.
        \begin{itemize}
            \item \textbf{Cambios para que fuera ágil:}
            \begin{itemize}
                \item Ciclos de desarrollo más cortos (2 a 4 semanas).
                \item Identificación y priorización de funcionalidades en cada ciclo.
            \end{itemize}
        \end{itemize}

        \item \textbf{Práctica 2:} Ágil.
        Se cumple el valor Software funcionando sobre documentación extensiva.

        \item \textbf{Práctica 3:} Ágil.
        Se cumple el valor Colaboración con el cliente sobre negociación contractual.

        \item \textbf{Práctica 4:} NO ágil.
        Se incumple el valor Individuos e interacciones sobre procesos y herramientas.
        \begin{itemize}
            \item \textbf{Cambios para que fuera ágil:}
            \begin{itemize}
                \item Individuos multidisciplinares.
                \item Colaboración entre los miembros para realizar las diferentes funciones.
            \end{itemize}
        \end{itemize}
    \end{enumerate}
\end{solucion}



\section{Ejercicio 2: ISO/IEC 15504}\label{sec:intro-ej2}
% !TeX root = ../examen-parcial-2023.tex

\begin{itemize}
    \item \textbf{Puntos:} 3
\end{itemize}

\begin{enunciado}
    Se han identificado los siguientes requisitos como parte de la mejora de la gestión de la
    producción:
    \begin{enumerate}
        \item Los gestores deben poder acceder al sistema a través de una interfaz web mientras que
        los agricultores deben poder hacerlo mediante una aplicación móvil disponible para iOS\@.
        \item Los gestores de producción deben poder añadir y eliminar campos de cultivo al sistema.
        \item Los agricultores deben poder registrar las labores realizadas en los campos de cultivo
        (arado, siembra, riego, abonado, recolección, etc.) mediante geolocalización.
        \item Los agricultores deben poder notificar incidencias que afecten a la producción (plagas,
        eventos climatológicos, etc.).
        \item El equipo de desarrollo debe poder saber si el sistema está funcionando correctamente.
        \item Los gestores deben poder anotar la producción recolectada en cada campo.
        \item La aplicación debe tener un porcentaje de disponibilidad anual del 99.99\%.
        \item Los gestores deben poder marcar el estado de un campo (barbecho, activo, etc.).
    \end{enumerate}
    Lee detenidamente los requisitos y:
    \begin{enumerate}
        \item Clasifica los requisitos en funcionales, no funcionales u otros.
        \item $0.2$ puntos por cada respuesta correcta.
        \item Desarrolla la especificación del caso de uso de uno de los requisitos que hayas
        clasificado como funcional.
        \item $0.2$ puntos por cada campo simple; $0.3$ puntos por cada campo
    \end{enumerate}
\end{enunciado}
\begin{solucion}
    \begin{enumerate}
        \item Clasificación de los requisitos:
        \begin{itemize}
            \item Requisito 1: No funcional.
            \item Requisito 2: Funcional.
            \item Requisito 3: Funcional.
            \item Requisito 4: Funcional.
            \item Requisito 5: Otros.
            \item Requisito 6: Funcional.
            \item Requisito 7: No funcional.
            \item Requisito 8: Funcional.
        \end{itemize}

        \item Especificación del caso de uso (ejemplo para el requisito 2):
        \begin{itemize}
            \item Nombre: Añadir campo de cultivo.
            \item Actor: Gestor de producción.
            \item Descripción: Los gestores de producción deben poder añadir y eliminar campos de cultivo al sistema.
            \item Precondiciones: El gestor de producción debe haber iniciado sesión en el sistema.
            \item Dependencias: No especificado.
            \item Escenario:
            \begin{enumerate}
                \item El gestor de campo comienza el proceso de añadir un campo.
                \item El gestor de campo rellena los detalles del campo (nombre, localización,\ldots ).
                \item El gestor graba el campo en el sistema.
            \end{enumerate}
            \item Excepciones:
            \begin{enumerate}
                \item El gestor de campo comienza el proceso de añadir un campo.
                \item El gestor de campo no rellena todos los detalles del campo.
                \item El gestor de campo intenta grabar el campo en el sistema.
                \item El sistema indica que faltan detalles del campo.
            \end{enumerate}
        \end{itemize}
    \end{enumerate}
    \begin{itemize}
        \item Prioridad: No especificado.
    \end{itemize}
\end{solucion}


    \clearpage


    \part{\textcolor{primaryblue}{El proceso del software}}\label{part:el-proceso-del-software}
    % !TeX root = ../main.tex


\section{Ejercicio 1: Scrum a Kanban}\label{sec:el-proceso-del-software-ej1}
% !TeX root = ../examen-parcial-2023.tex


\begin{itemize}
    \item \textbf{Puntos:} 2
\end{itemize}
\begin{enunciado}
    El actual equipo de desarrollo está aplicando las siguientes prácticas:
    \begin{enumerate}
        \item El equipo planifica entregas trimestrales que incluyen un conjunto de funcionalidades acordadas entre el director de ingeniería y el director de producción dentro un plan anual.
        \item El equipo se asegura que el conjunto de funcionalidades de las entregas trimestrales se comporta correctamente y es usado por los usuarios finales sin dificultades.
        \item El equipo se comunica directamente con el director de producción cuando tiene dudas acerca de cómo debe comportarse una funcionalidad concreta.
        \item Dentro del equipo cada miembro tiene su función: una persona diseña la solución, otro la construye, otro la prueba y otro la despliega y mantiene en producción.
    \end{enumerate}
    Lee detenidamente las prácticas e:
    \begin{enumerate}
        \item Identifica, para cada una de ellas, si se corresponden con prácticas ágiles.
        \item Justifica las respuestas en base al manifiesto ágil.
        \item \textbf{0,3 cada respuesta correcta con justificación.}
        \item Indica qué cambios aplicarías en las que no son ágiles (si hay alguna) para que sí lo sean.
        \item \textbf{0,4 por cada práctica convertida en ágil.}
    \end{enumerate}
\end{enunciado}

\begin{solucion}
    \begin{enumerate}
        \item \textbf{Práctica 1:} NO ágil.
        Se incumple el valor Respuesta ante el cambio sobre seguir un plan.
        \begin{itemize}
            \item \textbf{Cambios para que fuera ágil:}
            \begin{itemize}
                \item Ciclos de desarrollo más cortos (2 a 4 semanas).
                \item Identificación y priorización de funcionalidades en cada ciclo.
            \end{itemize}
        \end{itemize}

        \item \textbf{Práctica 2:} Ágil.
        Se cumple el valor Software funcionando sobre documentación extensiva.

        \item \textbf{Práctica 3:} Ágil.
        Se cumple el valor Colaboración con el cliente sobre negociación contractual.

        \item \textbf{Práctica 4:} NO ágil.
        Se incumple el valor Individuos e interacciones sobre procesos y herramientas.
        \begin{itemize}
            \item \textbf{Cambios para que fuera ágil:}
            \begin{itemize}
                \item Individuos multidisciplinares.
                \item Colaboración entre los miembros para realizar las diferentes funciones.
            \end{itemize}
        \end{itemize}
    \end{enumerate}
\end{solucion}



%\section{Ejercicio 2: Flujos}\label{sec:el-proceso-del-software-ej2}
%% !TeX root = ../examen-parcial-2023.tex

\begin{itemize}
    \item \textbf{Puntos:} 3
\end{itemize}

\begin{enunciado}
    Se han identificado los siguientes requisitos como parte de la mejora de la gestión de la
    producción:
    \begin{enumerate}
        \item Los gestores deben poder acceder al sistema a través de una interfaz web mientras que
        los agricultores deben poder hacerlo mediante una aplicación móvil disponible para iOS\@.
        \item Los gestores de producción deben poder añadir y eliminar campos de cultivo al sistema.
        \item Los agricultores deben poder registrar las labores realizadas en los campos de cultivo
        (arado, siembra, riego, abonado, recolección, etc.) mediante geolocalización.
        \item Los agricultores deben poder notificar incidencias que afecten a la producción (plagas,
        eventos climatológicos, etc.).
        \item El equipo de desarrollo debe poder saber si el sistema está funcionando correctamente.
        \item Los gestores deben poder anotar la producción recolectada en cada campo.
        \item La aplicación debe tener un porcentaje de disponibilidad anual del 99.99\%.
        \item Los gestores deben poder marcar el estado de un campo (barbecho, activo, etc.).
    \end{enumerate}
    Lee detenidamente los requisitos y:
    \begin{enumerate}
        \item Clasifica los requisitos en funcionales, no funcionales u otros.
        \item $0.2$ puntos por cada respuesta correcta.
        \item Desarrolla la especificación del caso de uso de uno de los requisitos que hayas
        clasificado como funcional.
        \item $0.2$ puntos por cada campo simple; $0.3$ puntos por cada campo
    \end{enumerate}
\end{enunciado}
\begin{solucion}
    \begin{enumerate}
        \item Clasificación de los requisitos:
        \begin{itemize}
            \item Requisito 1: No funcional.
            \item Requisito 2: Funcional.
            \item Requisito 3: Funcional.
            \item Requisito 4: Funcional.
            \item Requisito 5: Otros.
            \item Requisito 6: Funcional.
            \item Requisito 7: No funcional.
            \item Requisito 8: Funcional.
        \end{itemize}

        \item Especificación del caso de uso (ejemplo para el requisito 2):
        \begin{itemize}
            \item Nombre: Añadir campo de cultivo.
            \item Actor: Gestor de producción.
            \item Descripción: Los gestores de producción deben poder añadir y eliminar campos de cultivo al sistema.
            \item Precondiciones: El gestor de producción debe haber iniciado sesión en el sistema.
            \item Dependencias: No especificado.
            \item Escenario:
            \begin{enumerate}
                \item El gestor de campo comienza el proceso de añadir un campo.
                \item El gestor de campo rellena los detalles del campo (nombre, localización,\ldots ).
                \item El gestor graba el campo en el sistema.
            \end{enumerate}
            \item Excepciones:
            \begin{enumerate}
                \item El gestor de campo comienza el proceso de añadir un campo.
                \item El gestor de campo no rellena todos los detalles del campo.
                \item El gestor de campo intenta grabar el campo en el sistema.
                \item El sistema indica que faltan detalles del campo.
            \end{enumerate}
        \end{itemize}
    \end{enumerate}
    \begin{itemize}
        \item Prioridad: No especificado.
    \end{itemize}
\end{solucion}
%todo: complétame


\section{Ejercicio 3: Identificar fases del proceso}\label{sec:el-proceso-del-software-ej3}
% !TeX root = ../examen-parcial-2023.tex
%
%Analiza detenidamente el diagrama y:
%A) Asocia los componentes con los requisitos que cubren. - 0,4 por cada requisito
%Interfaz web e interfaz móvil. Requisito 1.
%Gestor de campos. Requisitos 2 y 8.
%Registrador de labores. Requisito 3.
%Administrador de incidencias. Requisito 4.
%B) Completa el diagrama con los componentes necesarios para cubrir todos los requisitos
%funcionales del Ejercicio 2. - 1 punto
%Gestor de producción. Conectado al canal de comunicación. Requisito 6.

\begin{itemize}
    \item \textbf{Puntos:} 3
\end{itemize}

\begin{enunciado}
    El equipo de desarrollo ha realizado esta primera versión del diagrama de arquitectura para
    cubrir los requisitos funcionales del Ejercicio 2:


    \deactivatequoting

    \begin{tikzpicture}[
    % Estilos máis simples pero modernos
        interface/.style={
            rectangle,
            rounded corners=5pt,
            fill=blue!10,
            draw=blue!50,
            line width=1pt,
            font=\sffamily,
            minimum width=3cm,
            minimum height=1cm,
            align=center
        },
        communication/.style={
            rectangle,
            rounded corners=8pt,
            fill=blue!80,
            text=white,
            font=\sffamily\bfseries,
            minimum width=7cm,
            minimum height=1.2cm,
            align=center
        },
        component/.style={
            rectangle,
            rounded corners=3pt,
            fill=gray!10,
            draw=gray!50,
            font=\sffamily\small,
            minimum width=2.5cm,
            minimum height=0.8cm,
            align=center
        }
    ]

        % Interfaces superiores
        \node[interface] (interfaz_movil) at (0, 3) {Interfaz Móvil};
        \node[interface] (interfaz_web) at (5, 3) {Interfaz Web};

        % Canal de comunicación
        \node[communication] (canal) at (2.5, 1) {Canal de comunicación};

        % Compoñentes inferiores
        \node[component] (registrador) at (0, -1) {Registrador de\\labores};
        \node[component] (gestor) at (2.5, -1) {Gestor de\\Campos};
        \node[component] (administrador) at (5, -1) {Administrador de\\incidencias};

        % Conexiones
        \draw[->, thick, blue] (interfaz_movil) -- (canal);
        \draw[->, thick, blue] (interfaz_web) -- (canal);
        \draw[->, thick, blue] (canal) -- (registrador);
        \draw[->, thick, blue] (canal) -- (gestor);
        \draw[->, thick, blue] (canal) -- (administrador);

    \end{tikzpicture}

    \begin{enumerate}
        \item Analiza detenidamente el diagrama y:
        \begin{enumerate}
            \item Asocia los componentes con los requisitos que cubren.
            \item Completa el diagrama con los componentes necesarios para cubrir todos los requisitos
            funcionales del Ejercicio 2.
        \end{enumerate}
        \item \textbf{0,4 puntos por cada requisito asociado.}
        \item \textbf{1 punto por completar el diagrama.}
    \end{enumerate}

\end{enunciado}

\begin{solucion}
    \begin{enumerate}
        \item Análisis del diagrama:
        \begin{enumerate}
            \item Asociaciones de componentes con requisitos:
            \begin{itemize}
                \item Interfaz web e interfaz móvil: Requisito 1.
                \item Gestor de campos: Requisitos 2 y 8.
                \item Registrador de labores: Requisito 3.
                \item Administrador de incidencias: Requisito 4.
            \end{itemize}

            \item Componentes necesarios para cubrir todos los requisitos funcionales:
            \begin{itemize}
                \item Gestor de producción: Conectado al canal de comunicación: Requisito 6.
            \end{itemize}
        \end{enumerate}
    \end{enumerate}
\end{solucion}


\section{Ejercicio 4: Escoger proceso de sofware}\label{sec:el-proceso-del-software-ej4}
% !TeX root = ../mantenimiento.tex


\begin{enunciado}
    ¿Qué tipo de mantenimiento representan cada una de las siguientes acciones?
    \begin{enumerate}
        \item Adaptar la aplicación para que no incumpla la nueva normativa de protección de datos que entrará en vigor el próximo mes.
        \item Expandir la aplicación a un país nuevo, donde el idioma usado para comunicarse con el usuario debe ser diferente a los soportados.
        \item Añadir test unitarios, que no existían en la versión inicial.
        \item Mejorar la accesibilidad de nuestra web para usuarios con ceguera o deficiencia visual.
        \item Solucionar un problema que causa que uno de cada cien registros falle
    \end{enumerate}
\end{enunciado}
\begin{solucion}
    \begin{description}
        \item[1] Adaptativo
        \item[2] Adaptativo
        \item[3] Preventivo
        \item[4] Perfectivo
        \item[5] Correctivo
    \end{description}
\end{solucion}


\section{Ejercicio 5: Scrum vérsus Kanban}\label{sec:el-proceso-del-software-ej5}
% !TeX root = ../planificacion.tex

\begin{enunciado}
    Desarrolla el diagrama de PERT del conjunto de tareas del ejercicio 3.
    ¿Qué tareas están en el camino crítico?
\end{enunciado}

\begin{solucion}

    La duración total del proyecto es de \textbf{33 días}, determinada por el camino crítico:

    A → C → E → F → G → H → I → J\@.

    Diagrama PERT de las \textbf{tareas del ejercicio 3}:

    \deactivatequoting
    \tikz[>={To[sep]}, rotate=90, xscale=-1]
    \graph [nodes={circle,draw},
        edges={nodes={inner sep=1pt, anchor=mid}}]
    {
        A ->
            {
                {
                B ->
                    {
                    D
                }
            }
            ,
                {
                C ->
                    {
                    E ->
                        {
                        F
                    }
                }
            }
        } ->
            {
            G ->
                {
                H ->
                    {
                    I ->
                        {
                        J
                    }
                }
            }
        }
    };
    \activatequoting
\end{solucion}



    \clearpage


    \part{\textcolor{primaryblue}{Modelado}}\label{part:modelado}
    % !TeX root = ../main.tex


\section{Ejercicio 1: Requisitos}\label{sec:ejercicio-1:-requisitos}
% !TeX root = ../examen-parcial-2023.tex


\begin{itemize}
    \item \textbf{Puntos:} 2
\end{itemize}
\begin{enunciado}
    El actual equipo de desarrollo está aplicando las siguientes prácticas:
    \begin{enumerate}
        \item El equipo planifica entregas trimestrales que incluyen un conjunto de funcionalidades acordadas entre el director de ingeniería y el director de producción dentro un plan anual.
        \item El equipo se asegura que el conjunto de funcionalidades de las entregas trimestrales se comporta correctamente y es usado por los usuarios finales sin dificultades.
        \item El equipo se comunica directamente con el director de producción cuando tiene dudas acerca de cómo debe comportarse una funcionalidad concreta.
        \item Dentro del equipo cada miembro tiene su función: una persona diseña la solución, otro la construye, otro la prueba y otro la despliega y mantiene en producción.
    \end{enumerate}
    Lee detenidamente las prácticas e:
    \begin{enumerate}
        \item Identifica, para cada una de ellas, si se corresponden con prácticas ágiles.
        \item Justifica las respuestas en base al manifiesto ágil.
        \item \textbf{0,3 cada respuesta correcta con justificación.}
        \item Indica qué cambios aplicarías en las que no son ágiles (si hay alguna) para que sí lo sean.
        \item \textbf{0,4 por cada práctica convertida en ágil.}
    \end{enumerate}
\end{enunciado}

\begin{solucion}
    \begin{enumerate}
        \item \textbf{Práctica 1:} NO ágil.
        Se incumple el valor Respuesta ante el cambio sobre seguir un plan.
        \begin{itemize}
            \item \textbf{Cambios para que fuera ágil:}
            \begin{itemize}
                \item Ciclos de desarrollo más cortos (2 a 4 semanas).
                \item Identificación y priorización de funcionalidades en cada ciclo.
            \end{itemize}
        \end{itemize}

        \item \textbf{Práctica 2:} Ágil.
        Se cumple el valor Software funcionando sobre documentación extensiva.

        \item \textbf{Práctica 3:} Ágil.
        Se cumple el valor Colaboración con el cliente sobre negociación contractual.

        \item \textbf{Práctica 4:} NO ágil.
        Se incumple el valor Individuos e interacciones sobre procesos y herramientas.
        \begin{itemize}
            \item \textbf{Cambios para que fuera ágil:}
            \begin{itemize}
                \item Individuos multidisciplinares.
                \item Colaboración entre los miembros para realizar las diferentes funciones.
            \end{itemize}
        \end{itemize}
    \end{enumerate}
\end{solucion}



\section{Ejercicio 2: Casos de uso e historias de usuario}\label{sec:ejercicio-2:-casos-de-uso-e-historias-de-usuario}
% !TeX root = ../examen-parcial-2023.tex

\begin{itemize}
    \item \textbf{Puntos:} 3
\end{itemize}

\begin{enunciado}
    Se han identificado los siguientes requisitos como parte de la mejora de la gestión de la
    producción:
    \begin{enumerate}
        \item Los gestores deben poder acceder al sistema a través de una interfaz web mientras que
        los agricultores deben poder hacerlo mediante una aplicación móvil disponible para iOS\@.
        \item Los gestores de producción deben poder añadir y eliminar campos de cultivo al sistema.
        \item Los agricultores deben poder registrar las labores realizadas en los campos de cultivo
        (arado, siembra, riego, abonado, recolección, etc.) mediante geolocalización.
        \item Los agricultores deben poder notificar incidencias que afecten a la producción (plagas,
        eventos climatológicos, etc.).
        \item El equipo de desarrollo debe poder saber si el sistema está funcionando correctamente.
        \item Los gestores deben poder anotar la producción recolectada en cada campo.
        \item La aplicación debe tener un porcentaje de disponibilidad anual del 99.99\%.
        \item Los gestores deben poder marcar el estado de un campo (barbecho, activo, etc.).
    \end{enumerate}
    Lee detenidamente los requisitos y:
    \begin{enumerate}
        \item Clasifica los requisitos en funcionales, no funcionales u otros.
        \item $0.2$ puntos por cada respuesta correcta.
        \item Desarrolla la especificación del caso de uso de uno de los requisitos que hayas
        clasificado como funcional.
        \item $0.2$ puntos por cada campo simple; $0.3$ puntos por cada campo
    \end{enumerate}
\end{enunciado}
\begin{solucion}
    \begin{enumerate}
        \item Clasificación de los requisitos:
        \begin{itemize}
            \item Requisito 1: No funcional.
            \item Requisito 2: Funcional.
            \item Requisito 3: Funcional.
            \item Requisito 4: Funcional.
            \item Requisito 5: Otros.
            \item Requisito 6: Funcional.
            \item Requisito 7: No funcional.
            \item Requisito 8: Funcional.
        \end{itemize}

        \item Especificación del caso de uso (ejemplo para el requisito 2):
        \begin{itemize}
            \item Nombre: Añadir campo de cultivo.
            \item Actor: Gestor de producción.
            \item Descripción: Los gestores de producción deben poder añadir y eliminar campos de cultivo al sistema.
            \item Precondiciones: El gestor de producción debe haber iniciado sesión en el sistema.
            \item Dependencias: No especificado.
            \item Escenario:
            \begin{enumerate}
                \item El gestor de campo comienza el proceso de añadir un campo.
                \item El gestor de campo rellena los detalles del campo (nombre, localización,\ldots ).
                \item El gestor graba el campo en el sistema.
            \end{enumerate}
            \item Excepciones:
            \begin{enumerate}
                \item El gestor de campo comienza el proceso de añadir un campo.
                \item El gestor de campo no rellena todos los detalles del campo.
                \item El gestor de campo intenta grabar el campo en el sistema.
                \item El sistema indica que faltan detalles del campo.
            \end{enumerate}
        \end{itemize}
    \end{enumerate}
    \begin{itemize}
        \item Prioridad: No especificado.
    \end{itemize}
\end{solucion}


\section{Ejercicio 3: Diagrama de secuencia}\label{sec:ejercicio-3:-diagrama-de-secuencia}
% !TeX root = ../examen-parcial-2023.tex
%
%Analiza detenidamente el diagrama y:
%A) Asocia los componentes con los requisitos que cubren. - 0,4 por cada requisito
%Interfaz web e interfaz móvil. Requisito 1.
%Gestor de campos. Requisitos 2 y 8.
%Registrador de labores. Requisito 3.
%Administrador de incidencias. Requisito 4.
%B) Completa el diagrama con los componentes necesarios para cubrir todos los requisitos
%funcionales del Ejercicio 2. - 1 punto
%Gestor de producción. Conectado al canal de comunicación. Requisito 6.

\begin{itemize}
    \item \textbf{Puntos:} 3
\end{itemize}

\begin{enunciado}
    El equipo de desarrollo ha realizado esta primera versión del diagrama de arquitectura para
    cubrir los requisitos funcionales del Ejercicio 2:


    \deactivatequoting

    \begin{tikzpicture}[
    % Estilos máis simples pero modernos
        interface/.style={
            rectangle,
            rounded corners=5pt,
            fill=blue!10,
            draw=blue!50,
            line width=1pt,
            font=\sffamily,
            minimum width=3cm,
            minimum height=1cm,
            align=center
        },
        communication/.style={
            rectangle,
            rounded corners=8pt,
            fill=blue!80,
            text=white,
            font=\sffamily\bfseries,
            minimum width=7cm,
            minimum height=1.2cm,
            align=center
        },
        component/.style={
            rectangle,
            rounded corners=3pt,
            fill=gray!10,
            draw=gray!50,
            font=\sffamily\small,
            minimum width=2.5cm,
            minimum height=0.8cm,
            align=center
        }
    ]

        % Interfaces superiores
        \node[interface] (interfaz_movil) at (0, 3) {Interfaz Móvil};
        \node[interface] (interfaz_web) at (5, 3) {Interfaz Web};

        % Canal de comunicación
        \node[communication] (canal) at (2.5, 1) {Canal de comunicación};

        % Compoñentes inferiores
        \node[component] (registrador) at (0, -1) {Registrador de\\labores};
        \node[component] (gestor) at (2.5, -1) {Gestor de\\Campos};
        \node[component] (administrador) at (5, -1) {Administrador de\\incidencias};

        % Conexiones
        \draw[->, thick, blue] (interfaz_movil) -- (canal);
        \draw[->, thick, blue] (interfaz_web) -- (canal);
        \draw[->, thick, blue] (canal) -- (registrador);
        \draw[->, thick, blue] (canal) -- (gestor);
        \draw[->, thick, blue] (canal) -- (administrador);

    \end{tikzpicture}

    \begin{enumerate}
        \item Analiza detenidamente el diagrama y:
        \begin{enumerate}
            \item Asocia los componentes con los requisitos que cubren.
            \item Completa el diagrama con los componentes necesarios para cubrir todos los requisitos
            funcionales del Ejercicio 2.
        \end{enumerate}
        \item \textbf{0,4 puntos por cada requisito asociado.}
        \item \textbf{1 punto por completar el diagrama.}
    \end{enumerate}

\end{enunciado}

\begin{solucion}
    \begin{enumerate}
        \item Análisis del diagrama:
        \begin{enumerate}
            \item Asociaciones de componentes con requisitos:
            \begin{itemize}
                \item Interfaz web e interfaz móvil: Requisito 1.
                \item Gestor de campos: Requisitos 2 y 8.
                \item Registrador de labores: Requisito 3.
                \item Administrador de incidencias: Requisito 4.
            \end{itemize}

            \item Componentes necesarios para cubrir todos los requisitos funcionales:
            \begin{itemize}
                \item Gestor de producción: Conectado al canal de comunicación: Requisito 6.
            \end{itemize}
        \end{enumerate}
    \end{enumerate}
\end{solucion}


\section{Ejercicio 5: Tiempo medio entre fallos y tiempo de recuperación}\label{sec:ejercicio-5:-tiempo-medio-entre-fallos-y-tiempo-de-recuperacion}
% !TeX root = ../planificacion.tex

\begin{enunciado}
    Desarrolla el diagrama de PERT del conjunto de tareas del ejercicio 3.
    ¿Qué tareas están en el camino crítico?
\end{enunciado}

\begin{solucion}

    La duración total del proyecto es de \textbf{33 días}, determinada por el camino crítico:

    A → C → E → F → G → H → I → J\@.

    Diagrama PERT de las \textbf{tareas del ejercicio 3}:

    \deactivatequoting
    \tikz[>={To[sep]}, rotate=90, xscale=-1]
    \graph [nodes={circle,draw},
        edges={nodes={inner sep=1pt, anchor=mid}}]
    {
        A ->
            {
                {
                B ->
                    {
                    D
                }
            }
            ,
                {
                C ->
                    {
                    E ->
                        {
                        F
                    }
                }
            }
        } ->
            {
            G ->
                {
                H ->
                    {
                    I ->
                        {
                        J
                    }
                }
            }
        }
    };
    \activatequoting
\end{solucion}


\clearpage


\section{Ejercicio 6: Requisitos}\label{sec:ejercicio-6-:-requisitos}
% !TeX root = ../modelado.tex


\begin{enunciado}
    Clasifica los siguientes requisitos no funcionales en su correspondiente subcategoría:
    \begin{enumerate}
        \item La aplicación debe funcionar en Windows, Unix, MacOs, Android e iOS
        \item La aplicación debe usar lenguaje inclusivo
        \item La aplicación debe estar disponible de lunes a viernes de 8h a 18h
        \item La operación de registro debe realizarse en menos de 1 segundo
        \item La versión móvil de la aplicación debe ocupar menos de 100MB
        \item La aplicación debe ser compatible con el sistema de videoconferencia Zoom
        \item Se deben entregar tanto los ejecutables de las diferentes versiones como el código fuente
        \item La aplicación debe ser accesible para personas con discapacidades visuales o motoras.
    \end{enumerate}
\end{enunciado}

\begin{solucion}
    \begin{enumerate}
        \item De proceso, implementación, usabilidad
        \item Externos, legislativo, seguridad
        \item Producto, usabilidad
        \item Producto, eficiencia, espacio
        \item Producto, usabilidad, implementación
        \item Proceso, delivery
        \item Ética?
    \end{enumerate}
\end{solucion}


\clearpage


\section{Ejercicio 7: Diagrama de arquitectura}\label{sec:ejercicio-7:-diagrama-de-arquitectura}
% !TeX root = ../modelado.tex

\begin{enunciado}
    Completa el diagrama de arquitectura para añadir la siguiente funcionalidad:

    Aplicación de ofertas en función de la cantidad de productos comprados y el país desde el
    que se realiza la compra.
\end{enunciado}


\begin{solucion}

    \begin{tikzpicture}[node distance=1.25cm]
% Interfaz Gráfica
        \hspace{1em}
        \node[interface] (gui) {Interfaz Gráfica};

% Primera fila de servicios
        \node[service, below left=2cm and 1.5cm of gui] (catalog) {CatalogService};
        \node[service, below=2cm of gui] (cart) {CartService};
        \node[service, below right=2cm and 1.5cm of gui] (order) {OrderService};

% Segunda fila de servicios
        \node[service, below =2.0cm of catalog] (pricing) {PricingService};
        \node[service, below =2cm of order] (payment) {PaymentService};

% Nuevo servicio de ofertas (añadido)
        \node[service, below=2.0cm of cart, fill=green!20] (offers) {OffersService};

% Servicio de geolocalización (añadido)
        \node[service, below left=1.5cm and -0.5cm of offers, fill=orange!20] (geo) {GeoService};

% Bases de datos
        \node[database, left=0.5cm of catalog] (db1) {};
        \node[database, above= 0.25cm of cart,xshift=0.5cm] (db2) {};
        \node[database, right=0.5cm of order] (db3) {};
        \node[database, left=0.5cm of pricing] (db4) {};
        \node[database, right=0.5cm of offers] (db6) {};
        \node[database, left=0.5cm of geo] (db7) {};

% Conexiones desde la interfaz
        \draw[arrow] (gui) -- (catalog);
        \draw[arrow] (gui) -- (cart);
        \draw[arrow] (gui) -- (order);

% Conexiones entre servicios
        \draw[bidirectional] (catalog) -- (cart);
        \draw[bidirectional] (cart) -- (order);
        \draw[bidirectional] (catalog) -- (pricing);
        \draw[bidirectional] (order) -- (payment);

% Nuevas conexiones para ofertas
        \draw[bidirectional] (cart) -- (offers);
        \draw[bidirectional] (order) -- (offers);
        \draw[bidirectional] (offers) -- (geo);
        \draw[bidirectional] (offers) -- (pricing);

% Conexiones a bases de datos
        \draw[arrow] (catalog) -- (db1);
        \draw[arrow] (cart) -- (db2);
        \draw[arrow] (order) -- (db3);
        \draw[arrow] (pricing) -- (db4);
        \draw[arrow] (offers) -- (db6);
        \draw[arrow] (geo) -- (db7);

% Etiquetas para las nuevas funcionalidades
        \node[above=0.2cm of offers, font=\tiny, text=green!60!black] {Gestión de ofertas};
        \node[above=0.2cm of geo, font=\tiny, text=orange!60!black] {Geolocalización};

    \end{tikzpicture}

\end{solucion}


\section{Ejercicio 8: Interfaz textual}\label{sec:ejercicio-8:-interfaz-textual}
% !TeX root = ../modelado.tex


\begin{enunciado}
    Diseña una interfaz textual para las siguientes funcionalidades:

    \begin{itemize}
        \item Registrar películas vistas con su título y fecha de visualización
        \item Listar las películas vistas permitiendo ordenarlas por título o por fecha de visualización
    \end{itemize}
\end{enunciado}

\begin{solucion}
    Componente visual, imagen de la portada de la película


    Componente caja de texto para introducir el texto


    Componente campo de texto para introducir la fecha


    Búsqueda de datos para almacenar películas


    Componente visual para elegir si mostrar por fecha o por titulo
\end{solucion}


    \clearpage


    \part{\textcolor{primaryblue}{Planificación}}\label{part:planificacion}
    % !TeX root = ../main.tex


\section{Ejercicio 1: Estimación basada en problema}\label{sec:ejercicio-1:-estimacion-basada-en-problema}
% !TeX root = ../examen-parcial-2023.tex


\begin{itemize}
    \item \textbf{Puntos:} 2
\end{itemize}
\begin{enunciado}
    El actual equipo de desarrollo está aplicando las siguientes prácticas:
    \begin{enumerate}
        \item El equipo planifica entregas trimestrales que incluyen un conjunto de funcionalidades acordadas entre el director de ingeniería y el director de producción dentro un plan anual.
        \item El equipo se asegura que el conjunto de funcionalidades de las entregas trimestrales se comporta correctamente y es usado por los usuarios finales sin dificultades.
        \item El equipo se comunica directamente con el director de producción cuando tiene dudas acerca de cómo debe comportarse una funcionalidad concreta.
        \item Dentro del equipo cada miembro tiene su función: una persona diseña la solución, otro la construye, otro la prueba y otro la despliega y mantiene en producción.
    \end{enumerate}
    Lee detenidamente las prácticas e:
    \begin{enumerate}
        \item Identifica, para cada una de ellas, si se corresponden con prácticas ágiles.
        \item Justifica las respuestas en base al manifiesto ágil.
        \item \textbf{0,3 cada respuesta correcta con justificación.}
        \item Indica qué cambios aplicarías en las que no son ágiles (si hay alguna) para que sí lo sean.
        \item \textbf{0,4 por cada práctica convertida en ágil.}
    \end{enumerate}
\end{enunciado}

\begin{solucion}
    \begin{enumerate}
        \item \textbf{Práctica 1:} NO ágil.
        Se incumple el valor Respuesta ante el cambio sobre seguir un plan.
        \begin{itemize}
            \item \textbf{Cambios para que fuera ágil:}
            \begin{itemize}
                \item Ciclos de desarrollo más cortos (2 a 4 semanas).
                \item Identificación y priorización de funcionalidades en cada ciclo.
            \end{itemize}
        \end{itemize}

        \item \textbf{Práctica 2:} Ágil.
        Se cumple el valor Software funcionando sobre documentación extensiva.

        \item \textbf{Práctica 3:} Ágil.
        Se cumple el valor Colaboración con el cliente sobre negociación contractual.

        \item \textbf{Práctica 4:} NO ágil.
        Se incumple el valor Individuos e interacciones sobre procesos y herramientas.
        \begin{itemize}
            \item \textbf{Cambios para que fuera ágil:}
            \begin{itemize}
                \item Individuos multidisciplinares.
                \item Colaboración entre los miembros para realizar las diferentes funciones.
            \end{itemize}
        \end{itemize}
    \end{enumerate}
\end{solucion}



\section{Ejercicio 2: Tiempo de desarrollo usando COCOMO II}\label{sec:ejercicio-2:-tiempo-de-desarrollo-usando-cocomo-ii}
% !TeX root = ../examen-parcial-2023.tex

\begin{itemize}
    \item \textbf{Puntos:} 3
\end{itemize}

\begin{enunciado}
    Se han identificado los siguientes requisitos como parte de la mejora de la gestión de la
    producción:
    \begin{enumerate}
        \item Los gestores deben poder acceder al sistema a través de una interfaz web mientras que
        los agricultores deben poder hacerlo mediante una aplicación móvil disponible para iOS\@.
        \item Los gestores de producción deben poder añadir y eliminar campos de cultivo al sistema.
        \item Los agricultores deben poder registrar las labores realizadas en los campos de cultivo
        (arado, siembra, riego, abonado, recolección, etc.) mediante geolocalización.
        \item Los agricultores deben poder notificar incidencias que afecten a la producción (plagas,
        eventos climatológicos, etc.).
        \item El equipo de desarrollo debe poder saber si el sistema está funcionando correctamente.
        \item Los gestores deben poder anotar la producción recolectada en cada campo.
        \item La aplicación debe tener un porcentaje de disponibilidad anual del 99.99\%.
        \item Los gestores deben poder marcar el estado de un campo (barbecho, activo, etc.).
    \end{enumerate}
    Lee detenidamente los requisitos y:
    \begin{enumerate}
        \item Clasifica los requisitos en funcionales, no funcionales u otros.
        \item $0.2$ puntos por cada respuesta correcta.
        \item Desarrolla la especificación del caso de uso de uno de los requisitos que hayas
        clasificado como funcional.
        \item $0.2$ puntos por cada campo simple; $0.3$ puntos por cada campo
    \end{enumerate}
\end{enunciado}
\begin{solucion}
    \begin{enumerate}
        \item Clasificación de los requisitos:
        \begin{itemize}
            \item Requisito 1: No funcional.
            \item Requisito 2: Funcional.
            \item Requisito 3: Funcional.
            \item Requisito 4: Funcional.
            \item Requisito 5: Otros.
            \item Requisito 6: Funcional.
            \item Requisito 7: No funcional.
            \item Requisito 8: Funcional.
        \end{itemize}

        \item Especificación del caso de uso (ejemplo para el requisito 2):
        \begin{itemize}
            \item Nombre: Añadir campo de cultivo.
            \item Actor: Gestor de producción.
            \item Descripción: Los gestores de producción deben poder añadir y eliminar campos de cultivo al sistema.
            \item Precondiciones: El gestor de producción debe haber iniciado sesión en el sistema.
            \item Dependencias: No especificado.
            \item Escenario:
            \begin{enumerate}
                \item El gestor de campo comienza el proceso de añadir un campo.
                \item El gestor de campo rellena los detalles del campo (nombre, localización,\ldots ).
                \item El gestor graba el campo en el sistema.
            \end{enumerate}
            \item Excepciones:
            \begin{enumerate}
                \item El gestor de campo comienza el proceso de añadir un campo.
                \item El gestor de campo no rellena todos los detalles del campo.
                \item El gestor de campo intenta grabar el campo en el sistema.
                \item El sistema indica que faltan detalles del campo.
            \end{enumerate}
        \end{itemize}
    \end{enumerate}
    \begin{itemize}
        \item Prioridad: No especificado.
    \end{itemize}
\end{solucion}


\section{Ejercicio 3: Diagrama de Gantt}\label{sec:ejercicio-3:-diagrama-de-gantt}
% !TeX root = ../examen-parcial-2023.tex
%
%Analiza detenidamente el diagrama y:
%A) Asocia los componentes con los requisitos que cubren. - 0,4 por cada requisito
%Interfaz web e interfaz móvil. Requisito 1.
%Gestor de campos. Requisitos 2 y 8.
%Registrador de labores. Requisito 3.
%Administrador de incidencias. Requisito 4.
%B) Completa el diagrama con los componentes necesarios para cubrir todos los requisitos
%funcionales del Ejercicio 2. - 1 punto
%Gestor de producción. Conectado al canal de comunicación. Requisito 6.

\begin{itemize}
    \item \textbf{Puntos:} 3
\end{itemize}

\begin{enunciado}
    El equipo de desarrollo ha realizado esta primera versión del diagrama de arquitectura para
    cubrir los requisitos funcionales del Ejercicio 2:


    \deactivatequoting

    \begin{tikzpicture}[
    % Estilos máis simples pero modernos
        interface/.style={
            rectangle,
            rounded corners=5pt,
            fill=blue!10,
            draw=blue!50,
            line width=1pt,
            font=\sffamily,
            minimum width=3cm,
            minimum height=1cm,
            align=center
        },
        communication/.style={
            rectangle,
            rounded corners=8pt,
            fill=blue!80,
            text=white,
            font=\sffamily\bfseries,
            minimum width=7cm,
            minimum height=1.2cm,
            align=center
        },
        component/.style={
            rectangle,
            rounded corners=3pt,
            fill=gray!10,
            draw=gray!50,
            font=\sffamily\small,
            minimum width=2.5cm,
            minimum height=0.8cm,
            align=center
        }
    ]

        % Interfaces superiores
        \node[interface] (interfaz_movil) at (0, 3) {Interfaz Móvil};
        \node[interface] (interfaz_web) at (5, 3) {Interfaz Web};

        % Canal de comunicación
        \node[communication] (canal) at (2.5, 1) {Canal de comunicación};

        % Compoñentes inferiores
        \node[component] (registrador) at (0, -1) {Registrador de\\labores};
        \node[component] (gestor) at (2.5, -1) {Gestor de\\Campos};
        \node[component] (administrador) at (5, -1) {Administrador de\\incidencias};

        % Conexiones
        \draw[->, thick, blue] (interfaz_movil) -- (canal);
        \draw[->, thick, blue] (interfaz_web) -- (canal);
        \draw[->, thick, blue] (canal) -- (registrador);
        \draw[->, thick, blue] (canal) -- (gestor);
        \draw[->, thick, blue] (canal) -- (administrador);

    \end{tikzpicture}

    \begin{enumerate}
        \item Analiza detenidamente el diagrama y:
        \begin{enumerate}
            \item Asocia los componentes con los requisitos que cubren.
            \item Completa el diagrama con los componentes necesarios para cubrir todos los requisitos
            funcionales del Ejercicio 2.
        \end{enumerate}
        \item \textbf{0,4 puntos por cada requisito asociado.}
        \item \textbf{1 punto por completar el diagrama.}
    \end{enumerate}

\end{enunciado}

\begin{solucion}
    \begin{enumerate}
        \item Análisis del diagrama:
        \begin{enumerate}
            \item Asociaciones de componentes con requisitos:
            \begin{itemize}
                \item Interfaz web e interfaz móvil: Requisito 1.
                \item Gestor de campos: Requisitos 2 y 8.
                \item Registrador de labores: Requisito 3.
                \item Administrador de incidencias: Requisito 4.
            \end{itemize}

            \item Componentes necesarios para cubrir todos los requisitos funcionales:
            \begin{itemize}
                \item Gestor de producción: Conectado al canal de comunicación: Requisito 6.
            \end{itemize}
        \end{enumerate}
    \end{enumerate}
\end{solucion}


\section{Ejercicio 4: Clasificación de riesgos}\label{sec:ejercicio-4:-clasificacion-de-riesgos}
% !TeX root = ../mantenimiento.tex


\begin{enunciado}
    ¿Qué tipo de mantenimiento representan cada una de las siguientes acciones?
    \begin{enumerate}
        \item Adaptar la aplicación para que no incumpla la nueva normativa de protección de datos que entrará en vigor el próximo mes.
        \item Expandir la aplicación a un país nuevo, donde el idioma usado para comunicarse con el usuario debe ser diferente a los soportados.
        \item Añadir test unitarios, que no existían en la versión inicial.
        \item Mejorar la accesibilidad de nuestra web para usuarios con ceguera o deficiencia visual.
        \item Solucionar un problema que causa que uno de cada cien registros falle
    \end{enumerate}
\end{enunciado}
\begin{solucion}
    \begin{description}
        \item[1] Adaptativo
        \item[2] Adaptativo
        \item[3] Preventivo
        \item[4] Perfectivo
        \item[5] Correctivo
    \end{description}
\end{solucion}


\section{Ejercicio 5: Camino crítico PERT}\label{sec:ejercicio-5:-camino-critico-pert}
% !TeX root = ../planificacion.tex

\begin{enunciado}
    Desarrolla el diagrama de PERT del conjunto de tareas del ejercicio 3.
    ¿Qué tareas están en el camino crítico?
\end{enunciado}

\begin{solucion}

    La duración total del proyecto es de \textbf{33 días}, determinada por el camino crítico:

    A → C → E → F → G → H → I → J\@.

    Diagrama PERT de las \textbf{tareas del ejercicio 3}:

    \deactivatequoting
    \tikz[>={To[sep]}, rotate=90, xscale=-1]
    \graph [nodes={circle,draw},
        edges={nodes={inner sep=1pt, anchor=mid}}]
    {
        A ->
            {
                {
                B ->
                    {
                    D
                }
            }
            ,
                {
                C ->
                    {
                    E ->
                        {
                        F
                    }
                }
            }
        } ->
            {
            G ->
                {
                H ->
                    {
                    I ->
                        {
                        J
                    }
                }
            }
        }
    };
    \activatequoting
\end{solucion}



    \clearpage


    \part{\textcolor{primaryblue}{Calidad}}\label{part:calidad}
    \localtableofcontents
\begin{definicion}
    La calidad en software se define como un proceso eficaz que, al ser bien aplicado, crea un producto útil con valor mesurable tanto para los productores como para los usuarios.
\end{definicion}

La ingeniería de software tiene como objetivo garantizar esa calidad a lo largo de todo el ciclo de vida:

\begin{itemize}
    \item Planificación
    \item Diseño
    \item Desarrollo
    \item Pruebas
    \item Despliegue
    \item Mantenimiento
\end{itemize}


\section{Aseguramiento de la Calidad}\label{sec:aseguramiento-de-la-calidad}
\begin{definicion}
    Según la norma ISO/IEC 9126, la calidad de un producto software se puede evaluar mediante seis atributos principales:
\end{definicion}

\begin{enumerate}
    \item \textbf{Funcionalidad:} Grado en que el software cumple los requisitos funcionales esperados.
    \item \textbf{Confiabilidad:} Estabilidad del software ante fallos, por ejemplo, el tiempo medio de funcionamiento antes de un fallo.
    \item \textbf{Usabilidad:} Facilidad de uso para los usuarios.
    \item \textbf{Eficiencia:} Uso óptimo de los recursos disponibles (CPU, memoria, etc.).
    \item \textbf{Mantenibilidad:} Facilidad para modificar, corregir o mejorar el software.
    \item \textbf{Portabilidad:} Facilidad para trasladar el software entre diferentes entornos.
\end{enumerate}

\begin{nota}
    Existe un dilema clásico entre producir software rápido y barato o producir software de alta calidad, ya que la calidad requiere tiempo y recursos.
\end{nota}

\begin{nota}
    Los errores pequeños no detectados a tiempo pueden amplificarse y causar problemas mayores en fases posteriores, por lo que la detección temprana es fundamental.
\end{nota}

\subsection{Principios rectores de la calidad}\label{subsec:principios-rectores-de-la-calidad}
\begin{itemize}
    \item \textbf{Formulación.} La derivación de medidas y métricas de software apropiadas
    \item para la representación del software que se está construyendo.
    \item \textbf{Recolección. }Mecanismo que se usa para acumular datos requeridos para derivar las métricas formuladas.
    \item \textbf{Análisis.} El cálculo de métricas y la aplicación de herramientas
    \item matemáticas.
    \item \textbf{Interpretación.} Evaluación de las métricas resultantes para comprender la calidad de la representación.
    \item \textbf{Retroalimentación.} Recomendaciones derivadas de la interpretación de las métricas del producto, transmitidas al equipo de software.
\end{itemize}

\subsection{Atributos de las métricas}\label{subsec:atributos-de-las-metricas}
\begin{enumerate}
    \item \textbf{Medible. }Debe ser simple poder recolectar los datos que componen la métrica y realizar su cálculo.
    \item \textbf{Intuitiva.} Los usuarios de la métrica deben poder identificar su significado y su valor.
    \item \textbf{Objetiva.} Siempre debe producir resultados que no tengan ambigüedades.
    \item \textbf{Coherente.} El cálculo matemático de la métrica debe usar medidas que no conduzcan a combinaciones extrañas de unidades.
    \item \textbf{Tecnológicamente agnóstica.} Debe basarse en el modelo de requerimientos, el modelo de diseño o la estructura del programa en sí.
    \item \textbf{Accionable.} Debe proporcionar información que pueda conducir a un producto final de mayor calidad.
\end{enumerate}

\subsection{Control vs. Aseguramiento de la Calidad}\label{subsec:control-vs.-aseguramiento-de-la-calidad}

\begin{center}
    \begin{tabularx}{\textwidth}{|X|X|}
        \hline
        \textbf{Control de Calidad (QC)}                         & \textbf{Aseguramiento de Calidad (QA)}               \\
        \hline
        Reactivo                                                 & Proactivo                                            \\
        Detección y corrección de errores después de que ocurren & Prevención de errores mediante estándares y procesos \\
        Inspección de productos                                  & Mejora continua de procesos                          \\
        \hline
    \end{tabularx}
\end{center}

\subsection{Revisiones técnicas}\label{subsec:revisiones-tecnicas}

\begin{itemize}
    \item \textbf{Informales:} Conversaciones espontáneas, revisiones en escritorio; baja eficacia que mejora con listas de verificación.
    \item \textbf{Formales:} Reuniones estructuradas y preparadas; alta eficacia, más tiene un alto coste en tiempo y esfuerzo; se utiliza normalmente una muestra representativa.
\end{itemize}

\subsection{Revisiones durante el desarrollo}\label{subsec:revisiones-durante-el-desarrollo}

\begin{itemize}
    \item \textbf{Revisión de código:} Entre compañeros (pair review) para detectar errores y mejorar el aprendizaje del equipo.
    \item \textbf{Análisis estático de código:} Herramientas automáticas que detectan errores potenciales, complejidad y duplicaciones.
\end{itemize}

\subsection{Métricas de calidad (DORA Metrics)}\label{subsec:metricas-de-calidad-(dora-metrics)}

\begin{center}
    \begin{tabular}{|l|l|}
        \hline
        \textbf{Métrica}                  & \textbf{Significado}                                 \\
        \hline
        MTTR (Mean Time to Recover)       & Tiempo medio para recuperar un sistema tras un fallo \\
        MTBF (Mean Time Between Failures) & Tiempo medio entre fallos                            \\
        Disponibilidad                    & Proporción de tiempo que el sistema está disponible  \\
        \hline
    \end{tabular}
\end{center}

\begin{definicion}
    La disponibilidad se calcula con la fórmula:
    \[
        \text{Disponibilidad} = \frac{\text{MTBF}}{\text{MTBF} + \text{MTTR}} \cdot 100\%
    \]
\end{definicion}

\subsection{Buenas prácticas de desarrollo}\label{subsec:buenas-practicas-de-desarrollo}

\begin{itemize}
    \item \textbf{Clean Code} (Robert C. Martin - Uncle Bob):
    \begin{itemize}
        \item KISS: “Keep It Simple, Stupid”
        \item DRY: “Don’t Repeat Yourself”
        \item YAGNI: “You Aren’t Gonna Need It”
        \item SoC: “Separation of Concerns”
    \end{itemize}
    \item \textbf{Documentación:} Comentarios explicativos (no descriptivos), explicaciones en los commits y ejemplos de uso en tests.
    \item \textbf{Control de versiones:} Uso de herramientas como Git para seguir cambios y facilitar la colaboración.
\end{itemize}

\subsection{Métricas de desarrollo}\label{subsec:metricas-de-desarrollo}

\begin{center}
    \begin{tabular}{|l|l|}
        \hline
        \textbf{Métrica}                             & \textbf{Descripción}                       \\
        \hline
        Densidad de comentarios                      & \% de comentarios respecto al código total \\
        Duplicidad de código                         & Código repetido                            \\
        Cobertura de pruebas                         & \% de código ejecutado durante pruebas     \\
        Complejidad ciclomática                      & Mide rutas lógicas (condiciones y bucles)  \\
        IMS (Índice de Madurez del código(Software)) & Evalúa la estabilidad de una release       \\
        \hline
    \end{tabular}
\end{center}

\begin{definicion}
    La fórmula para calcular el IMS es:
    \[
        IMS = \frac{M_T - (F_a + F_c + F_d)}{M_T}
    \]
    Donde:
    \begin{itemize}
        \item $M_T$: Número total de pruebas planificadas.
        \item $F_a$: Fallos críticos encontrados.
        \item $F_c$: Fallos menores encontrados.
        \item $F_d$: Fallos detectados en desarrollo.
    \end{itemize}
\end{definicion}

\subsection{Deuda técnica}\label{subsec:deuda-tecnica}

\begin{definicion}
    Según Martin Fowler, la deuda técnica es una metáfora financiera: tomar atajos en el diseño genera un \textquote{interés} que se paga con mayor esfuerzo futuro.
    Se puede:
    \begin{itemize}
        \item Seguir pagando intereses (mantener mal diseño).
        \item Pagar el principal (refactorizar y mejorar).
    \end{itemize}
\end{definicion}

Se clasifica según dos ejes:

\begin{center}
    \begin{tabular}{|c|c|c|}
        \hline
        & \textbf{Temeraria}                              & \textbf{Prudente}                                    \\
        \hline
        \textbf{Deliberada}  & \textquote{No tenemos tiempo, entregamos ahora} & \textquote{Lo haremos rápido y mejoraremos después}  \\
        \hline
        \textbf{Inadvertida} & \textquote{¿Qué componentes tiene esto?}        & \textquote{Ahora sabemos cómo debería haberse hecho} \\
        \hline
    \end{tabular}
\end{center}


\section{Estrategias de Prueba}\label{sec:estrategias-de-prueba}

\subsection{Verificación vs. Validación}\label{subsec:verificacion-vs.-validacion}

\begin{center}
    %! suppress = LineBreak
    \begin{tabular}{|l|l|}
        \hline
        \textbf{Verificación}                     & \textbf{Validación}                         \\
        \hline
        ¿Construimos \textbf{bien} el producto?   & ¿Construimos el \textbf{producto correcto}? \\
        Garantía de implementación correcta       & Cumplimiento de requisitos del cliente      \\
        Incluye pruebas, revisiones, simulaciones & Incluye pruebas de aceptación y prototipos  \\
        \hline
    \end{tabular}
\end{center}

\subsection{Malas prácticas}\label{subsec:malas-practicas}

\begin{itemize}
    \item Suponer que hay partes que no se pueden probar.
    \item Que el desarrollador no haga pruebas.
    \item Aislar al equipo de pruebas.
    \item Involucrar a los testers solo al final.
\end{itemize}

\subsection{Estrategias de prueba (pirámide de pruebas)}\label{subsec:estrategias-de-prueba-(piramide-de-pruebas)}

Martin Fowler propone priorizar así:

\begin{enumerate}
    \item Pruebas unitarias (muchas, automáticas).
    \item Pruebas de integración.
    \item Pruebas de interfaz/aceptación (pocas, más caras de automatizar).
\end{enumerate}

\subsection{Dimensiones de la prueba}\label{subsec:dimensiones-de-la-prueba}

\begin{center}
    \begin{tabularx}{\textwidth}{|l|l|X|}
        \hline
        \textbf{Dimensión} & \textbf{Subdimensión}                   & \textbf{Descripción}                                    \\
        \hline
        Tipo               & Funcional / No funcional                & Comprobación de funciones / parámetros como rendimiento \\
        Granularidad       & Unitaria / Integración / Validación     & Nivel del sistema probado                               \\
        Alcance            & Progresión / Regresión / Smoke / Sanity & Cobertura funcional                                     \\
        Ejecución          & Manual / Asistida / Automática          & Nivel de automatización                                 \\
        Metodología        & Guiada / Exploratoria                   & Nivel de formalización                                  \\
        \hline
    \end{tabularx}
\end{center}

\subsection{Pruebas específicas}\label{subsec:pruebas-especificas}

\begin{itemize}
    \item \textbf{Unitarias:} \textbf{Caja blanca}.
    Son las pruebas más simples y frecuentes (sobre código individual).
    Uso de \emph{stubs} para módulos dependientes.
    Ejemplos: métodos, condiciones de frontera, errores.
    \item \textbf{Integración:} \textbf{Caja gris}.
    Pruebas de cómo interactúan los componentes.
    Importante la integración incremental (mayor control de errores).
    Puede ser ascendente o descendente.
    \item \textbf{Validación (aceptación):} \textbf{Caja negra}.
    Realistas y orientadas al usuario final.
    Difíciles de automatizar.
    Usadas en fases alfa y beta.
    \item \textbf{No funcionales:} Rendimiento, seguridad, recuperación, esfuerzo, despliegue, etc.
\end{itemize}


\section{Administración de la Configuración}\label{sec:administracion-de-la-configuracion}
\begin{definicion}
    Conjunto de actividades para gestionar los cambios durante el ciclo de vida del software, garantizando:
    \begin{itemize}
        \item Trazabilidad
        \item Control de versiones
        \item Información actualizada
    \end{itemize}

    Se diferencia del mantenimiento en que este aplica cambios, y la administración de configuración controla y registra dichos cambios.
\end{definicion}

\subsection*{Conceptos clave}
\begin{itemize}
    \item \textbf{Elemento de configuración (EC / ICS):} Unidad identificable que debe ser controlada (código, documentación, etc.).
    \item \textbf{Base de datos de configuración (BCD):} Lugar donde se almacenan los EC y sus versiones.
    \item \textbf{Línea base:} Versión estable y acordada de un conjunto de EC\@.
    \item \textbf{Cambio:} Modificación propuesta a un EC\@.
    \item \textbf{Control de cambios:} Proceso de evaluar y aprobar cambios.
\end{itemize}

\subsection{Ítems de Configuración (IC)}\label{subsec:items-de-configuracion-(ic)}
\begin{definicion}
    Cada uno de los elementos que comprenden toda la información
    producida como parte del proceso de software.
\end{definicion}
\begin{itemize}
    \item Programas de cómputo
    \item Productos de trabajo
    \item Contenido
\end{itemize}

\subsection{Sistema de Administración}\label{subsec:sistema-de-administracion}
\begin{itemize}
    \item \textbf{Elementos componentes:} Herramientas que permiten el acceso y gestión de cada ítem de configuración del software.
    \item \textbf{Elementos de proceso:} Acciones y tareas necesarias para realizar una gestión efectiva del cambio.
    \item \textbf{Elementos de construcción:} Herramientas que automatizan la compilación, empaquetado y generación de versiones correctas del software.
    \item \textbf{Elementos humanos:} Procesos y herramientas que utiliza el equipo para implementar de manera efectiva la administración de configuración.
\end{itemize}

\subsection{Repositorio ACS}\label{subsec:repositorio-acs}
\begin{definicion}
    Almacén centralizado o distribuido donde se gestionan los ítems de configuración.
\end{definicion}

\paragraph{Características clave:}
\begin{itemize}
    \item Control de versiones.
    \item Rastreo de dependencias entre componentes.
    \item Trazabilidad de requisitos y cambios.
    \item Apoyo a auditorías e inspecciones de calidad.
\end{itemize}

\subsection{Proceso de Administración de la Configuración}\label{subsec:proceso-de-administracion-de-la-configuracion}
\begin{enumerate}
    \item \textbf{Identificación:} Asignar nombres unívocos a los ítems de configuración.
    \item \textbf{Control de cambios:} Registrar, aprobar e implementar modificaciones.
    \item \textbf{Control de versiones:} Gestionar diferentes versiones de todos los objetos.
    \item \textbf{Auditoría:} Verificar que los cambios cumplen con los estándares establecidos.
    \item \textbf{Reporte de estado:} Informar del estado actual y del histórico de cambios.
\end{enumerate}

\subsection{Administración del contenido}\label{subsec:administracion-del-contenido}
\begin{itemize}
    \item \textbf{Subsistema de recopilación:} Permite almacenar y organizar el contenido asociado a cada Ítem de Configuración del Software (ICS).
    \item \textbf{Subsistema de administración:} Gestiona los cambios realizados sobre el contenido de cada ICS, manteniendo un histórico completo y trazable.
    \item \textbf{Subsistema de publicación:} Hace accesible el contenido relevante a los distintos interesados del proyecto, permitiendo su consulta o reutilización.
\end{itemize}

\subsection{Conflictos dominantes}\label{subsec:conflictos-dominantes}
\begin{itemize}
    \item \textbf{Contenido:} No está claro qué información debe formar parte del sistema de administración de configuración.
    \item \textbf{Personas:} El equipo desconoce las herramientas o procesos de ACS, o no sabe cuándo aplicarlos.
    \item \textbf{Escalabilidad:} A medida que crece el proyecto, se complica la gestión del gran volumen de cambios.
    \item \textbf{Políticas:} Faltan reglas claras sobre quién es responsable de llevar a cabo la administración de configuración y cómo aplicarla de manera coherente.
\end{itemize}





    \clearpage


    \part{\textcolor{primaryblue}{Mantenimiento}}\label{part:mantenimiento}
    \localtableofcontents

\section{Métricas de producto}\label{sec:tema-6.1---metricas-de-producto}
    \begin{definicion}
        \textbf{Métrica:} Medida cuantitativa del grado en que un sistema, componente o proceso posee un atributo determinado (IEEE Standard).
    \end{definicion}

    \subsection{Conceptos clave}\label{subsec:conceptos-clave}
    \begin{itemize}
        \item \textbf{Medida:} Un único dato.
        \item \textbf{Medición:} Recolección de varias medidas.
        \item \textbf{Indicador:} Combinación de métricas que proporciona comprensión sobre procesos, proyectos o productos.
    \end{itemize}

    \subsection{Principios de medición}\label{subsec:principios-de-medicion}
    \begin{enumerate}
        \item \textbf{Formulación:} Derivación de medidas apropiadas
        \item \textbf{Recolección:} Mecanismo para acumular datos
        \item \textbf{Análisis:} Cálculo de métricas con herramientas matemáticas
        \item \textbf{Interpretación:} Evaluación de resultados
        \item \textbf{Retroalimentación:} Recomendaciones al equipo
    \end{enumerate}

    \subsection{Atributos de métricas}\label{subsec:atributos-de-metricas}
    \begin{itemize}
        \item Medible
        \item Intuitiva
        \item Objetiva
        \item Coherente \item
        Tecnológicamente agnóstica
        \item Accionable
    \end{itemize}

    \subsection{Métricas específicas}\label{subsec:metricas-especificas}

    \begin{definicion}

        \textit{Los puntos de definición son una forma de medir el tamaño, la complejidad y la calidad del software}.

    \end{definicion}
    \begin{itemize}
        \item \textbf{Modelo de requisitos:}
        \[PF = conteo \cdot (0.65 + (0.01 \cdot \text{FAV}))\]
        \begin{itemize}
            \item Componentes:
            \begin{description}
                \item[Entradas externas (EE)]: Información que se origina de un usuario o se transmite desde otra aplicación.
                Se usan para actualizar archivos lógicos internos (ALI). Las entradas deben distinguirse de las consultas.
                \item[Salidas externas (SE)]: Datos derivados dentro de la aplicación que ofrecen información al usuario.
                \item [Consultas externas (CE)]: Entrada que da como resultado la generación de alguna respuesta (con frecuencia recuperada de un ALI).
                \item[Número de archivos lógicos internos (ALI)]:Agrupamiento lógico de datos que reside dentro de la frontera de la aplicación y se mantiene mediante entradas externas
                \item[Número de archivos de interfaz externos (AIE)]:
                Agrupamiento lógico de datos que reside fuera de la aplicación, pero que proporciona información que puede usar la aplicación.
                \item [Factor de ajuste de valor (FAV)]: Indica la complejidad del sistema en su conjunto en base a unas preguntas a las que se asigna un valor entre 0 (irrelevante) y 5 (esencial).
            \end{description}
        \end{itemize}

        \item \textbf{Diseño arquitectónico:}
        \begin{itemize}
            \item Complejidad de módulo: $S(i) = f^2_{out}(i)$, $D(i) = \frac{v(i)}{f_{out}(i) + 1}$ (Estructural (S) y Datos (D))
            \item Complejidad de sistema: $C(i) = S(i) + D(i)$
        \end{itemize}
        Donde:
        \item $f_{out}(i)$ es el número de módulos que dependen del módulo $i$.
        \item $v(i)$ es el número de módulos de los que depende el módulo $i$.
        \item $\text{Tamaño} = n + a$; $n$ es el número de nodos y $a$ es el número de arcos.
        \item Profundidad: trayectoria más larga desde el nodo raíz hasta un nodo hoja
        \item Ancho: número máximo de nodos en cualquier nivel de la arquitectura

        \item \textbf{Orientadas a clase (CK):}
        \begin{table}[!ht]
            \centering
            \caption{Métricas de la Suite CK}
            \label{tab:ck_metrics}
            \begin{tabularx}{\linewidth}{lX}
                \toprule
                \textbf{Acrónimo} & \textbf{Descripción}                                                                                                                                                          \\
                \midrule
                MPC               & Métodos Ponderados por Clase: Suma de las complejidades nominales de todos los métodos de una clase ($MPC = \sum c_i$), donde $c_i$ es la complejidad nominal de cada método. \\
                \addlinespace[0.3cm]
                PAH               & Profundidad del Árbol de Herencia: Longitud máxima desde el nodo de la clase actual hasta la raíz del árbol de herencia.                                                      \\
                \addlinespace[0.3cm]
                NDH               & Número de Hijos Directos: Cantidad de subclases que heredan directamente de la clase actual (subclases inmediatas).                                                           \\
                \addlinespace[0.3cm]
                FCOM              & Falta de Cohesión en Métodos: Número de pares de métodos que acceden a uno o más atributos de clase en común.                                                                 \\
                \bottomrule
            \end{tabularx}
        \end{table}

        \item \textbf{Código fuente, métricas de Halstead:}
        \begin{itemize}
            \begin{table}[!ht]
                \centering
                \caption{Métricas de Halstead}
                \label{tab:halstead_metrics}
                \begin{tabular}{>{\bfseries}l l l}
                    \toprule
                    \multicolumn{1}{c}{\textbf{Símbolo}} &
                    \multicolumn{1}{c}{\textbf{Descripción}} &
                    \multicolumn{1}{c}{\textbf{Fórmula}} \\
                    \midrule
                    n1 & Número de operadores únicos    & Cantidad de tipos diferentes de operadores        \\
                    & en un programa                 &                                                   \\
                    \addlinespace

                    n2 & Número de operandos únicos     & Cantidad de tipos diferentes de operandos         \\
                    & en un programa                 &                                                   \\
                    \addlinespace

                    N1 & Frecuencia total de operadores & $N1 = \sum (\text{ocurrencias de cada operador})$ \\
                    &                                &                                                   \\
                    \addlinespace

                    N2 & Frecuencia total de operandos  & $N2 = \sum (\text{ocurrencias de cada operando})$ \\
                    &                                &                                                   \\
                    \bottomrule
                \end{tabular}
            \end{table}

            \item Longitud: $N = n_1 \log_2 n_1 + n_2 \log_2 n_2$
            \begin{nota}
                \textit{(Coa estimación de $N$ anterior. Se se usa $N = N_1 + N_2$, o volume sería real.)}
            \end{nota}
            \item Volumen: $V = N \log_2 (n_1 + n_2)$
        \end{itemize}

        \item \textbf{Pruebas:}
        \begin{itemize}
            \item Nivel de programa: $PL = 1 / ((n_1/2) \cdot (N_2/n_2))$
            \item Esfuerzo de prueba: $e = V / PL$
        \end{itemize}

        \item \textbf{Mantenimiento:}
        \begin{itemize}
            \item Índice de madurez: $IMS = (M_1 - (F_a + F_c + F_d)) / M_1$
        \end{itemize}

        \item \textbf{SLA/SLO/SLI:}
        \begin{itemize}
            \item SLA: Acuerdo con cliente (Ejemplo: disponibilidad >99.9\%)
            \item SLO: Objetivo específico dentro de SLA
            \item SLI: Medida real de cumplimiento
        \end{itemize}
    \end{itemize}

% todo: completar


\section{Tipos de mantenimiento}\label{sec:tipos-de-mantenimiento}
\subsection{Leyes de Lehman}\label{subsec:leyes-de-lehman}
\begin{itemize}
    \item \textbf{Ley del cambio continuo.} En un entorno real, un sistema debe necesariamente cambiar para mantener su utilidad.
    \item \textbf{Ley de complejidad creciente.} Cuando el sistema evoluciona se hace más complejo.
    Hay que tomar medidas para evitarlo.
    \item \textbf{Ley de evolución. }La evolución es un proceso autorregulado.
    El tamaño, tiempo entre versiones, errores detectados, etc., se mantienen en el tiempo.
    \item \textbf{Ley de estabilidad organizacional.} Durante el tiempo de vida del sistema su velocidad de desarrollo es constante e independiente de los recursos dedicados su desarrollo.
    \item \textbf{Ley de conservación de la familiaridad.} A medida que un sistema evoluciona todo lo que está asociado con ello debe mantener un conocimiento total de su contenido y su comportamiento.
    \item \textbf{Ley de crecimiento continuado. }La funcionalidad ofrecida por los sistemas tiene que crecer continuamente para
    mantener la satisfacción de los usuarios.
    \item \textbf{Ley de decremento de la calidad.} La calidad de los sistemas software comenzará a disminuir a menos que dichos
    sistemas se adapten a los cambios de su entorno de funcionamiento.
    \item \textbf{Ley de retroalimentación.} Los procesos de evolución incorporan sistemas de retroalimentación.
\end{itemize}

\subsection{Proceso de mantenimiento}\label{subsec:proceso-de-mantenimiento}
\begin{itemize}
    \item \textbf{Causas de modificación:}
    \begin{itemize}
        \item Nuevos requisitos/cambios solicitados
        \item Corrección de errores
    \end{itemize}

    \item \textbf{Incorporación al desarrollo:}
    \begin{itemize}
        \item Implementación formal u hotfixes
    \end{itemize}

    \item \textbf{Factores de esfuerzo:}
    \begin{itemize}
        \item Diseño del sistema
        \item  Mecanismos de prueba
        \item Documentación
        \item  Estabilidad del personal
    \end{itemize}
\end{itemize}

\subsection{Clasificación de mantenimiento}\label{subsec:clasificacion-de-mantenimiento}
\begin{tabular}{|l|l|l|p{6cm}|c|}
    \hline
    \textbf{Categoría} & \textbf{Tipo} & \textbf{Descripción}          & \textbf{Esfuerzo} \\
    \hline
    Evolutivo          & Perfectivo    & Añadir nuevas funcionalidades & 50\%              \\
    \hline
    & Adaptativo    & Adaptar a nuevos entornos     & 25\%              \\
    \hline
    & Preventivo    & Mejorar mantenibilidad futura & 5\%               \\
    \hline
    Tradicional        & Correctivo    & Corregir errores              & 20\%              \\
    \hline
\end{tabular}

\subsection{Release Notes}\label{subsec:release-notes}
\begin{itemize}
    \item \textbf{Contenido esencial:}
    \begin{itemize}
        \item Novedades
        \item Mejoras
        \item Correcciones de errores
    \end{itemize}
\end{itemize}

\subsection{Sistemas heredados}\label{subsec:sistemas-heredados}
\begin{itemize}
    \item \textbf{Problemas comunes:}
    \begin{itemize}
        \item Código espagueti
        \item  Falta de documentación
        \item Estructura deficiente
        \item Especificaciones ausentes
    \end{itemize}

    \item \textbf{Solución:} Ingeniería inversa
\end{itemize}

\subsection{Reingeniería de sistemas}\label{subsec:reingenieria-de-sistemas}
\begin{definicion}
    Reestructuración, reescritura o re-documentación sin cambiar funcionalidad.
\end{definicion}

\begin{itemize}
    \item \textbf{Ventajas:} Más económico que desarrollo nuevo • Reemplazo gradual
    \item \textbf{Factores coste:} Personal experto, herramientas disponibles
    \item \textbf{¿Por qué?:} Más barato que el desarrollo.
\end{itemize}

\subsection{Tipos de reingeniería}\label{subsec:tipos-de-reingenieria}
\begin{itemize}
    \item Traducción de código
    \item Ingeniería inversa
    \item Reestructuración
    \item Ingeniería hacia adelante
    \item Migración de datos
\end{itemize}

\subsection{Flujo de reingeniería}\label{subsec:flujo-de-reingenieria}
\begin{enumerate}
    \item Código fuente sucio $\rightarrow$ Reestructuración $\rightarrow$ Código limpio
    \item Extracción de abstracciones $\rightarrow$ Especificación inicial
    \item Refinamiento $\rightarrow$ Especificación final
\end{enumerate}



    \clearpage

    \part{\textcolor{orange}{Examen parcial 2023}}\label{part:examen-parcial-2023}
    \begin{itemize}
    \item \textbf{Asignatura:} Ingeniería del Software (G0460021)
    \item \textbf{Curso:} 2022/2023
    \item \textbf{Examen:} Parcial
    \item \textbf{Fecha:} 15 de marzo de 2023
    \item \textbf{Semestre:} Segundo
    \item \textbf{Convocatoria:} Ordinaria
\end{itemize}


\section{Ejercicio 1: Desarrollo ágil}\label{sec:ejercicio-1}
% !TeX root = ../examen-parcial-2023.tex


\begin{itemize}
    \item \textbf{Puntos:} 2
\end{itemize}
\begin{enunciado}
    El actual equipo de desarrollo está aplicando las siguientes prácticas:
    \begin{enumerate}
        \item El equipo planifica entregas trimestrales que incluyen un conjunto de funcionalidades acordadas entre el director de ingeniería y el director de producción dentro un plan anual.
        \item El equipo se asegura que el conjunto de funcionalidades de las entregas trimestrales se comporta correctamente y es usado por los usuarios finales sin dificultades.
        \item El equipo se comunica directamente con el director de producción cuando tiene dudas acerca de cómo debe comportarse una funcionalidad concreta.
        \item Dentro del equipo cada miembro tiene su función: una persona diseña la solución, otro la construye, otro la prueba y otro la despliega y mantiene en producción.
    \end{enumerate}
    Lee detenidamente las prácticas e:
    \begin{enumerate}
        \item Identifica, para cada una de ellas, si se corresponden con prácticas ágiles.
        \item Justifica las respuestas en base al manifiesto ágil.
        \item \textbf{0,3 cada respuesta correcta con justificación.}
        \item Indica qué cambios aplicarías en las que no son ágiles (si hay alguna) para que sí lo sean.
        \item \textbf{0,4 por cada práctica convertida en ágil.}
    \end{enumerate}
\end{enunciado}

\begin{solucion}
    \begin{enumerate}
        \item \textbf{Práctica 1:} NO ágil.
        Se incumple el valor Respuesta ante el cambio sobre seguir un plan.
        \begin{itemize}
            \item \textbf{Cambios para que fuera ágil:}
            \begin{itemize}
                \item Ciclos de desarrollo más cortos (2 a 4 semanas).
                \item Identificación y priorización de funcionalidades en cada ciclo.
            \end{itemize}
        \end{itemize}

        \item \textbf{Práctica 2:} Ágil.
        Se cumple el valor Software funcionando sobre documentación extensiva.

        \item \textbf{Práctica 3:} Ágil.
        Se cumple el valor Colaboración con el cliente sobre negociación contractual.

        \item \textbf{Práctica 4:} NO ágil.
        Se incumple el valor Individuos e interacciones sobre procesos y herramientas.
        \begin{itemize}
            \item \textbf{Cambios para que fuera ágil:}
            \begin{itemize}
                \item Individuos multidisciplinares.
                \item Colaboración entre los miembros para realizar las diferentes funciones.
            \end{itemize}
        \end{itemize}
    \end{enumerate}
\end{solucion}


\section{Ejercicio 2: Requisitos}\label{sec:ejercicio-2-ex2023}
% !TeX root = ../examen-parcial-2023.tex

\begin{itemize}
    \item \textbf{Puntos:} 3
\end{itemize}

\begin{enunciado}
    Se han identificado los siguientes requisitos como parte de la mejora de la gestión de la
    producción:
    \begin{enumerate}
        \item Los gestores deben poder acceder al sistema a través de una interfaz web mientras que
        los agricultores deben poder hacerlo mediante una aplicación móvil disponible para iOS\@.
        \item Los gestores de producción deben poder añadir y eliminar campos de cultivo al sistema.
        \item Los agricultores deben poder registrar las labores realizadas en los campos de cultivo
        (arado, siembra, riego, abonado, recolección, etc.) mediante geolocalización.
        \item Los agricultores deben poder notificar incidencias que afecten a la producción (plagas,
        eventos climatológicos, etc.).
        \item El equipo de desarrollo debe poder saber si el sistema está funcionando correctamente.
        \item Los gestores deben poder anotar la producción recolectada en cada campo.
        \item La aplicación debe tener un porcentaje de disponibilidad anual del 99.99\%.
        \item Los gestores deben poder marcar el estado de un campo (barbecho, activo, etc.).
    \end{enumerate}
    Lee detenidamente los requisitos y:
    \begin{enumerate}
        \item Clasifica los requisitos en funcionales, no funcionales u otros.
        \item $0.2$ puntos por cada respuesta correcta.
        \item Desarrolla la especificación del caso de uso de uno de los requisitos que hayas
        clasificado como funcional.
        \item $0.2$ puntos por cada campo simple; $0.3$ puntos por cada campo
    \end{enumerate}
\end{enunciado}
\begin{solucion}
    \begin{enumerate}
        \item Clasificación de los requisitos:
        \begin{itemize}
            \item Requisito 1: No funcional.
            \item Requisito 2: Funcional.
            \item Requisito 3: Funcional.
            \item Requisito 4: Funcional.
            \item Requisito 5: Otros.
            \item Requisito 6: Funcional.
            \item Requisito 7: No funcional.
            \item Requisito 8: Funcional.
        \end{itemize}

        \item Especificación del caso de uso (ejemplo para el requisito 2):
        \begin{itemize}
            \item Nombre: Añadir campo de cultivo.
            \item Actor: Gestor de producción.
            \item Descripción: Los gestores de producción deben poder añadir y eliminar campos de cultivo al sistema.
            \item Precondiciones: El gestor de producción debe haber iniciado sesión en el sistema.
            \item Dependencias: No especificado.
            \item Escenario:
            \begin{enumerate}
                \item El gestor de campo comienza el proceso de añadir un campo.
                \item El gestor de campo rellena los detalles del campo (nombre, localización,\ldots ).
                \item El gestor graba el campo en el sistema.
            \end{enumerate}
            \item Excepciones:
            \begin{enumerate}
                \item El gestor de campo comienza el proceso de añadir un campo.
                \item El gestor de campo no rellena todos los detalles del campo.
                \item El gestor de campo intenta grabar el campo en el sistema.
                \item El sistema indica que faltan detalles del campo.
            \end{enumerate}
        \end{itemize}
    \end{enumerate}
    \begin{itemize}
        \item Prioridad: No especificado.
    \end{itemize}
\end{solucion}

\section{Ejercicio 3: Diagrama de arquitectura}\label{sec:ejercicio-3-ex2023}
% !TeX root = ../examen-parcial-2023.tex
%
%Analiza detenidamente el diagrama y:
%A) Asocia los componentes con los requisitos que cubren. - 0,4 por cada requisito
%Interfaz web e interfaz móvil. Requisito 1.
%Gestor de campos. Requisitos 2 y 8.
%Registrador de labores. Requisito 3.
%Administrador de incidencias. Requisito 4.
%B) Completa el diagrama con los componentes necesarios para cubrir todos los requisitos
%funcionales del Ejercicio 2. - 1 punto
%Gestor de producción. Conectado al canal de comunicación. Requisito 6.

\begin{itemize}
    \item \textbf{Puntos:} 3
\end{itemize}

\begin{enunciado}
    El equipo de desarrollo ha realizado esta primera versión del diagrama de arquitectura para
    cubrir los requisitos funcionales del Ejercicio 2:


    \deactivatequoting

    \begin{tikzpicture}[
    % Estilos máis simples pero modernos
        interface/.style={
            rectangle,
            rounded corners=5pt,
            fill=blue!10,
            draw=blue!50,
            line width=1pt,
            font=\sffamily,
            minimum width=3cm,
            minimum height=1cm,
            align=center
        },
        communication/.style={
            rectangle,
            rounded corners=8pt,
            fill=blue!80,
            text=white,
            font=\sffamily\bfseries,
            minimum width=7cm,
            minimum height=1.2cm,
            align=center
        },
        component/.style={
            rectangle,
            rounded corners=3pt,
            fill=gray!10,
            draw=gray!50,
            font=\sffamily\small,
            minimum width=2.5cm,
            minimum height=0.8cm,
            align=center
        }
    ]

        % Interfaces superiores
        \node[interface] (interfaz_movil) at (0, 3) {Interfaz Móvil};
        \node[interface] (interfaz_web) at (5, 3) {Interfaz Web};

        % Canal de comunicación
        \node[communication] (canal) at (2.5, 1) {Canal de comunicación};

        % Compoñentes inferiores
        \node[component] (registrador) at (0, -1) {Registrador de\\labores};
        \node[component] (gestor) at (2.5, -1) {Gestor de\\Campos};
        \node[component] (administrador) at (5, -1) {Administrador de\\incidencias};

        % Conexiones
        \draw[->, thick, blue] (interfaz_movil) -- (canal);
        \draw[->, thick, blue] (interfaz_web) -- (canal);
        \draw[->, thick, blue] (canal) -- (registrador);
        \draw[->, thick, blue] (canal) -- (gestor);
        \draw[->, thick, blue] (canal) -- (administrador);

    \end{tikzpicture}

    \begin{enumerate}
        \item Analiza detenidamente el diagrama y:
        \begin{enumerate}
            \item Asocia los componentes con los requisitos que cubren.
            \item Completa el diagrama con los componentes necesarios para cubrir todos los requisitos
            funcionales del Ejercicio 2.
        \end{enumerate}
        \item \textbf{0,4 puntos por cada requisito asociado.}
        \item \textbf{1 punto por completar el diagrama.}
    \end{enumerate}

\end{enunciado}

\begin{solucion}
    \begin{enumerate}
        \item Análisis del diagrama:
        \begin{enumerate}
            \item Asociaciones de componentes con requisitos:
            \begin{itemize}
                \item Interfaz web e interfaz móvil: Requisito 1.
                \item Gestor de campos: Requisitos 2 y 8.
                \item Registrador de labores: Requisito 3.
                \item Administrador de incidencias: Requisito 4.
            \end{itemize}

            \item Componentes necesarios para cubrir todos los requisitos funcionales:
            \begin{itemize}
                \item Gestor de producción: Conectado al canal de comunicación: Requisito 6.
            \end{itemize}
        \end{enumerate}
    \end{enumerate}
\end{solucion}

\section{Ejercicio 4: Interfaz textual}\label{sec:ejercicio-4-ex2023}
% !TeX root = ../mantenimiento.tex


\begin{enunciado}
    ¿Qué tipo de mantenimiento representan cada una de las siguientes acciones?
    \begin{enumerate}
        \item Adaptar la aplicación para que no incumpla la nueva normativa de protección de datos que entrará en vigor el próximo mes.
        \item Expandir la aplicación a un país nuevo, donde el idioma usado para comunicarse con el usuario debe ser diferente a los soportados.
        \item Añadir test unitarios, que no existían en la versión inicial.
        \item Mejorar la accesibilidad de nuestra web para usuarios con ceguera o deficiencia visual.
        \item Solucionar un problema que causa que uno de cada cien registros falle
    \end{enumerate}
\end{enunciado}
\begin{solucion}
    \begin{description}
        \item[1] Adaptativo
        \item[2] Adaptativo
        \item[3] Preventivo
        \item[4] Perfectivo
        \item[5] Correctivo
    \end{description}
\end{solucion}


    \clearpage

    \part{\textcolor{orange}{Examen parcial 2024}}\label{part:examen-parcial-2024}
    \begin{itemize}
    \item \textbf{Asignatura:} Ingeniería del Software (G0460021)
    \item \textbf{Curso:} 2024/2025
    \item \textbf{Examen:} Parcial
    \item \textbf{Fecha:} 20 de marzo de 2024
    \item \textbf{Semestre:} Segundo
    \item \textbf{Convocatoria:} Ordinaria
\end{itemize}


\section{Ejercicio 1: Historias de usuario y actores}\label{sec:ejercicio-1-ex2024}
% !TeX root = ../examen-parcial-2023.tex


\begin{itemize}
    \item \textbf{Puntos:} 2
\end{itemize}
\begin{enunciado}
    El actual equipo de desarrollo está aplicando las siguientes prácticas:
    \begin{enumerate}
        \item El equipo planifica entregas trimestrales que incluyen un conjunto de funcionalidades acordadas entre el director de ingeniería y el director de producción dentro un plan anual.
        \item El equipo se asegura que el conjunto de funcionalidades de las entregas trimestrales se comporta correctamente y es usado por los usuarios finales sin dificultades.
        \item El equipo se comunica directamente con el director de producción cuando tiene dudas acerca de cómo debe comportarse una funcionalidad concreta.
        \item Dentro del equipo cada miembro tiene su función: una persona diseña la solución, otro la construye, otro la prueba y otro la despliega y mantiene en producción.
    \end{enumerate}
    Lee detenidamente las prácticas e:
    \begin{enumerate}
        \item Identifica, para cada una de ellas, si se corresponden con prácticas ágiles.
        \item Justifica las respuestas en base al manifiesto ágil.
        \item \textbf{0,3 cada respuesta correcta con justificación.}
        \item Indica qué cambios aplicarías en las que no son ágiles (si hay alguna) para que sí lo sean.
        \item \textbf{0,4 por cada práctica convertida en ágil.}
    \end{enumerate}
\end{enunciado}

\begin{solucion}
    \begin{enumerate}
        \item \textbf{Práctica 1:} NO ágil.
        Se incumple el valor Respuesta ante el cambio sobre seguir un plan.
        \begin{itemize}
            \item \textbf{Cambios para que fuera ágil:}
            \begin{itemize}
                \item Ciclos de desarrollo más cortos (2 a 4 semanas).
                \item Identificación y priorización de funcionalidades en cada ciclo.
            \end{itemize}
        \end{itemize}

        \item \textbf{Práctica 2:} Ágil.
        Se cumple el valor Software funcionando sobre documentación extensiva.

        \item \textbf{Práctica 3:} Ágil.
        Se cumple el valor Colaboración con el cliente sobre negociación contractual.

        \item \textbf{Práctica 4:} NO ágil.
        Se incumple el valor Individuos e interacciones sobre procesos y herramientas.
        \begin{itemize}
            \item \textbf{Cambios para que fuera ágil:}
            \begin{itemize}
                \item Individuos multidisciplinares.
                \item Colaboración entre los miembros para realizar las diferentes funciones.
            \end{itemize}
        \end{itemize}
    \end{enumerate}
\end{solucion}


\section{Ejercicio 2: Desarrollo ágil y Scrum}\label{sec:ejercicio-2-ex2024}
% !TeX root = ../examen-parcial-2023.tex

\begin{itemize}
    \item \textbf{Puntos:} 3
\end{itemize}

\begin{enunciado}
    Se han identificado los siguientes requisitos como parte de la mejora de la gestión de la
    producción:
    \begin{enumerate}
        \item Los gestores deben poder acceder al sistema a través de una interfaz web mientras que
        los agricultores deben poder hacerlo mediante una aplicación móvil disponible para iOS\@.
        \item Los gestores de producción deben poder añadir y eliminar campos de cultivo al sistema.
        \item Los agricultores deben poder registrar las labores realizadas en los campos de cultivo
        (arado, siembra, riego, abonado, recolección, etc.) mediante geolocalización.
        \item Los agricultores deben poder notificar incidencias que afecten a la producción (plagas,
        eventos climatológicos, etc.).
        \item El equipo de desarrollo debe poder saber si el sistema está funcionando correctamente.
        \item Los gestores deben poder anotar la producción recolectada en cada campo.
        \item La aplicación debe tener un porcentaje de disponibilidad anual del 99.99\%.
        \item Los gestores deben poder marcar el estado de un campo (barbecho, activo, etc.).
    \end{enumerate}
    Lee detenidamente los requisitos y:
    \begin{enumerate}
        \item Clasifica los requisitos en funcionales, no funcionales u otros.
        \item $0.2$ puntos por cada respuesta correcta.
        \item Desarrolla la especificación del caso de uso de uno de los requisitos que hayas
        clasificado como funcional.
        \item $0.2$ puntos por cada campo simple; $0.3$ puntos por cada campo
    \end{enumerate}
\end{enunciado}
\begin{solucion}
    \begin{enumerate}
        \item Clasificación de los requisitos:
        \begin{itemize}
            \item Requisito 1: No funcional.
            \item Requisito 2: Funcional.
            \item Requisito 3: Funcional.
            \item Requisito 4: Funcional.
            \item Requisito 5: Otros.
            \item Requisito 6: Funcional.
            \item Requisito 7: No funcional.
            \item Requisito 8: Funcional.
        \end{itemize}

        \item Especificación del caso de uso (ejemplo para el requisito 2):
        \begin{itemize}
            \item Nombre: Añadir campo de cultivo.
            \item Actor: Gestor de producción.
            \item Descripción: Los gestores de producción deben poder añadir y eliminar campos de cultivo al sistema.
            \item Precondiciones: El gestor de producción debe haber iniciado sesión en el sistema.
            \item Dependencias: No especificado.
            \item Escenario:
            \begin{enumerate}
                \item El gestor de campo comienza el proceso de añadir un campo.
                \item El gestor de campo rellena los detalles del campo (nombre, localización,\ldots ).
                \item El gestor graba el campo en el sistema.
            \end{enumerate}
            \item Excepciones:
            \begin{enumerate}
                \item El gestor de campo comienza el proceso de añadir un campo.
                \item El gestor de campo no rellena todos los detalles del campo.
                \item El gestor de campo intenta grabar el campo en el sistema.
                \item El sistema indica que faltan detalles del campo.
            \end{enumerate}
        \end{itemize}
    \end{enumerate}
    \begin{itemize}
        \item Prioridad: No especificado.
    \end{itemize}
\end{solucion}

\end{document}
