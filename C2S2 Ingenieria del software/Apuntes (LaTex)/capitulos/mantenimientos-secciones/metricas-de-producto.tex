    \begin{definicion}
        \textbf{Métrica:} Medida cuantitativa del grado en que un sistema, componente o proceso posee un atributo determinado (IEEE Standard).
    \end{definicion}

    \subsection{Conceptos clave}\label{subsec:conceptos-clave}
    \begin{itemize}
        \item \textbf{Medida:} Un único dato.
        \item \textbf{Medición:} Recolección de varias medidas.
        \item \textbf{Indicador:} Combinación de métricas que proporciona comprensión sobre procesos, proyectos o productos.
    \end{itemize}

    \subsection{Principios de medición}\label{subsec:principios-de-medicion}
    \begin{enumerate}
        \item \textbf{Formulación:} Derivación de medidas apropiadas
        \item \textbf{Recolección:} Mecanismo para acumular datos
        \item \textbf{Análisis:} Cálculo de métricas con herramientas matemáticas
        \item \textbf{Interpretación:} Evaluación de resultados
        \item \textbf{Retroalimentación:} Recomendaciones al equipo
    \end{enumerate}

    \subsection{Atributos de métricas}\label{subsec:atributos-de-metricas}
    \begin{itemize}
        \item Medible
        \item Intuitiva
        \item Objetiva
        \item Coherente \item
        Tecnológicamente agnóstica
        \item Accionable
    \end{itemize}

    \subsection{Métricas específicas}\label{subsec:metricas-especificas}

    \begin{definicion}

        \textit{Los puntos de definición son una forma de medir el tamaño, la complejidad y la calidad del software}.

    \end{definicion}
    \begin{itemize}
        \item \textbf{Modelo de requisitos:}
        \[PF = conteo \cdot (0.65 + (0.01 \cdot \text{FAV}))\]
        \begin{itemize}
            \item Componentes:
            \begin{description}
                \item[Entradas externas (EE)]: Información que se origina de un usuario o se transmite desde otra aplicación.
                Se usan para actualizar archivos lógicos internos (ALI). Las entradas deben distinguirse de las consultas.
                \item[Salidas externas (SE)]: Datos derivados dentro de la aplicación que ofrecen información al usuario.
                \item [Consultas externas (CE)]: Entrada que da como resultado la generación de alguna respuesta (con frecuencia recuperada de un ALI).
                \item[Número de archivos lógicos internos (ALI)]:Agrupamiento lógico de datos que reside dentro de la frontera de la aplicación y se mantiene mediante entradas externas
                \item[Número de archivos de interfaz externos (AIE)]:
                Agrupamiento lógico de datos que reside fuera de la aplicación, pero que proporciona información que puede usar la aplicación.
                \item [Factor de ajuste de valor (FAV)]: Indica la complejidad del sistema en su conjunto en base a unas preguntas a las que se asigna un valor entre 0 (irrelevante) y 5 (esencial).
            \end{description}
        \end{itemize}

        \item \textbf{Diseño arquitectónico:}
        \begin{itemize}
            \item Complejidad de módulo: $S(i) = f^2_{out}(i)$, $D(i) = \frac{v(i)}{f_{out}(i) + 1}$ (Estructural (S) y Datos (D))
            \item Complejidad de sistema: $C(i) = S(i) + D(i)$
        \end{itemize}
        Donde:
        \item $f_{out}(i)$ es el número de módulos que dependen del módulo $i$.
        \item $v(i)$ es el número de módulos de los que depende el módulo $i$.
        \item $\text{Tamaño} = n + a$; $n$ es el número de nodos y $a$ es el número de arcos.
        \item Profundidad: trayectoria más larga desde el nodo raíz hasta un nodo hoja
        \item Ancho: número máximo de nodos en cualquier nivel de la arquitectura

        \item \textbf{Orientadas a clase (CK):}
        \begin{table}[!ht]
            \centering
            \caption{Métricas de la Suite CK}
            \label{tab:ck_metrics}
            \begin{tabularx}{\linewidth}{lX}
                \toprule
                \textbf{Acrónimo} & \textbf{Descripción}                                                                                                                                                          \\
                \midrule
                MPC               & Métodos Ponderados por Clase: Suma de las complejidades nominales de todos los métodos de una clase ($MPC = \sum c_i$), donde $c_i$ es la complejidad nominal de cada método. \\
                \addlinespace[0.3cm]
                PAH               & Profundidad del Árbol de Herencia: Longitud máxima desde el nodo de la clase actual hasta la raíz del árbol de herencia.                                                      \\
                \addlinespace[0.3cm]
                NDH               & Número de Hijos Directos: Cantidad de subclases que heredan directamente de la clase actual (subclases inmediatas).                                                           \\
                \addlinespace[0.3cm]
                FCOM              & Falta de Cohesión en Métodos: Número de pares de métodos que acceden a uno o más atributos de clase en común.                                                                 \\
                \bottomrule
            \end{tabularx}
        \end{table}

        \item \textbf{Código fuente, métricas de Halstead:}
        \begin{itemize}
            \begin{table}[!ht]
                \centering
                \caption{Métricas de Halstead}
                \label{tab:halstead_metrics}
                \begin{tabular}{>{\bfseries}l l l}
                    \toprule
                    \multicolumn{1}{c}{\textbf{Símbolo}} &
                    \multicolumn{1}{c}{\textbf{Descripción}} &
                    \multicolumn{1}{c}{\textbf{Fórmula}} \\
                    \midrule
                    n1 & Número de operadores únicos    & Cantidad de tipos diferentes de operadores        \\
                    & en un programa                 &                                                   \\
                    \addlinespace

                    n2 & Número de operandos únicos     & Cantidad de tipos diferentes de operandos         \\
                    & en un programa                 &                                                   \\
                    \addlinespace

                    N1 & Frecuencia total de operadores & $N1 = \sum (\text{ocurrencias de cada operador})$ \\
                    &                                &                                                   \\
                    \addlinespace

                    N2 & Frecuencia total de operandos  & $N2 = \sum (\text{ocurrencias de cada operando})$ \\
                    &                                &                                                   \\
                    \bottomrule
                \end{tabular}
            \end{table}

            \item Longitud: $N = n_1 \log_2 n_1 + n_2 \log_2 n_2$
            \begin{nota}
                \textit{(Coa estimación de $N$ anterior. Se se usa $N = N_1 + N_2$, o volume sería real.)}
            \end{nota}
            \item Volumen: $V = N \log_2 (n_1 + n_2)$
        \end{itemize}

        \item \textbf{Pruebas:}
        \begin{itemize}
            \item Nivel de programa: $PL = 1 / ((n_1/2) \cdot (N_2/n_2))$
            \item Esfuerzo de prueba: $e = V / PL$
        \end{itemize}

        \item \textbf{Mantenimiento:}
        \begin{itemize}
            \item Índice de madurez: $IMS = (M_1 - (F_a + F_c + F_d)) / M_1$
        \end{itemize}

        \item \textbf{SLA/SLO/SLI:}
        \begin{itemize}
            \item SLA: Acuerdo con cliente (Ejemplo: disponibilidad >99.9\%)
            \item SLO: Objetivo específico dentro de SLA
            \item SLI: Medida real de cumplimiento
        \end{itemize}
    \end{itemize}

% todo: completar