%Ejercicio 4 (Puntos: 2)
%El director de producción quiere poder consultar, a través de un bot de Telegram, la última labor
%realizada en un campo según la localización en que se encuentre en ese momento.
%Desarrolla la interfaz textual del bot teniendo en cuenta:
%•
%Los requisitos funcionales del Ejercicio 2.
%•
%Las condiciones de error que pueden darse.
%•
%La información mostrada debe incluir: nombre del campo, labor realizada, fecha en que
%se realizó y nombre del agricultor que hizo la labor.
%Bot (B): Bienvenido al Labrija. Selección opción.
%Usuario (U): Ver la última labor del campo en el que estoy.
%B: Esta operación requiere acceso a la localización del dispositivo. ¿Permitir?
%U: Sí.
%B: Se ha detectado que el campo en el que se encuentra es Campo 1. ¿Es correcto?
%U: Sí.
%B: Aquí está la última labor del campo:
%-
%Campo: Campo 1
%-
%Labor: Siembra
%-
%Fecha: 2 de febrero de 2023
%-
%Agricultor: Felipe Gómez Pérez

\begin{itemize}
    \item \textbf{Puntos:} 2
\end{itemize}

\begin{enunciado}
    El director de producción quiere poder consultar, a través de un bot de Telegram, la última labor
    realizada en un campo según la localización en que se encuentre en ese momento.
    Desarrolla la interfaz textual del bot teniendo en cuenta:
    \begin{itemize}
        \item Los requisitos funcionales del Ejercicio 2.
        \item Las condiciones de error que pueden darse.
        \item La información mostrada debe incluir: nombre del campo, labor realizada, fecha en que
        se realizó y nombre del agricultor que hizo la labor.
    \end{itemize}
\end{enunciado}

\begin{solucion}
    \begin{itemize}
        \item \textbf{Bot (B):} Bienvenido al Labrija.
        Selección opción.
        \item \textbf{Usuario (U):} Ver la última labor del campo en el que estoy.
        \item \textbf{B:} Esta operación requiere acceso a la localización del dispositivo.
        ¿Permitir?
        \item \textbf{U:} Sí.
        \item \textbf{B:} Se ha detectado que el campo en el que se encuentra es Campo 1.
        ¿Es correcto?
        \item \textbf{U:} Sí.
        \item \textbf{B:} Aquí está la última labor del campo:
        \begin{itemize}
            \item Campo: Campo 1
            \item Labor: Siembra
            \item Fecha: 2 de febrero de 2023
            \item Agricultor: Felipe Gómez Pérez
        \end{itemize}
    \end{itemize}
\end{solucion}
