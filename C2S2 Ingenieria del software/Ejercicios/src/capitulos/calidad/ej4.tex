% !TeX root = ../calidad.tex
\begin{enunciado}
    Indica si las siguientes pruebas funcionales son de caja blanca, caja gris o caja negra:
    \begin{enumerate}
        \item Comprobar que, al introducir un título erróneo, no se encuentra ninguna película.
        \item Comprobar el almacenamiento de los datos de cliente en base de datos.
        \item Comprobar el algoritmo de ordenación de registros.
        \item Comprobar la funcionalidad de login.
        \item Comprobar que el servicio de traducción de títulos de película funciona correctamente.
        \item Comprobar que es posible evaluar una película.
        \item Comprobar que el servicio de evaluación puede recuperar los atributos de una película.
        \item Comprobar que el tiempo de respuesta del servicio de búsqueda es inferior a 1 segundo
    \end{enumerate}
\end{enunciado}

\begin{solucion}
    \begin{enumerate}
        \item Comprobar que, al introducir un título erróneo, no se encuentra ninguna película.

        \textbf{Tipo:} Caja negra

        \textbf{Justificación:} Se prueba la funcionalidad externa (entrada/salida) sin analizar el código interno.
        Es un caso de validación desde la perspectiva del usuario.

        \item Comprobar el almacenamiento de los datos de cliente en base de datos.

        \textbf{Tipo:} Caja gris

        \textbf{Justificación:} Implica verificar la interacción entre componentes (aplicación y base de datos), típico de pruebas de integración donde se combina conocimiento externo e interno.

        \item Comprobar el algoritmo de ordenación de registros.

        \textbf{Tipo:} Caja blanca

        \textbf{Justificación:} Requiere analizar la lógica interna del algoritmo (estructuras, bucles, condiciones).
        Es una prueba unitaria de bajo nivel.

        \item Comprobar la funcionalidad de login.

        \textbf{Tipo:} Caja negra

        \textbf{Justificación:} Se valida el comportamiento externo del sistema (usuario/contraseña → acceso), sin revisar la implementación.
        Es una prueba de aceptación.

        \item Comprobar que el servicio de traducción de títulos de película funciona correctamente.

        \textbf{Tipo:} Caja negra

        \textbf{Justificación:} Se evalúa el resultado (traducción correcta/incorrecta) desde la interfaz, sin inspeccionar el código del servicio.

        \item Comprobar que es posible evaluar una película.

        \textbf{Tipo:} Caja negra

        \textbf{Justificación:} Prueba funcional de extremo a extremo (flujo de usuario), centrada en requisitos externos, no en implementación.

        \item Comprobar que el servicio de evaluación puede recuperar los atributos de una película.

        \textbf{Tipo:} Caja gris

        \textbf{Justificación:} Involucra la integración entre servicios (evaluación y recuperación de datos), combinando aspectos de caja negra y blanca.
        Si se realiza desde la interfaz sin conocimiento interno, también podría considerarse caja negra.

        \item Revisar que el tiempo de respuesta del servicio de búsqueda es inferior a 1 segundo.

        \textbf{Tipo:} Caja negra

        \textbf{Justificación:} Es una prueba no funcional (rendimiento), pero al medirse desde la interfaz externa sin analizar código interno, se clasifica como caja negra.
    \end{enumerate}

\end{solucion}