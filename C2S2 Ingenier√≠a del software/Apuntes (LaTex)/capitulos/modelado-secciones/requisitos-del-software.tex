    \subsection{Principios de comunicación}\label{subsec:principios-de-comunicacion}

    \begin{enumerate}

        \item Escuchar activamente.

        \item Prepararse previamente.

        \item Facilitar la comunicación (agenda, mediador).

        \item Comunicación cara a cara.

        \item Documentar decisiones clave.

        \item Fomentar colaboración.

        \item Seguir agenda.

        \item Apoyar con representación gráfica.

        \item Avanzar ante bloqueos.

        \item Negociar de forma equilibrada.

    \end{enumerate}

    \subsection{Definición de requisito}\label{subsec:definicion-de-requisito}

    \begin{definicion}

        \textbf{Requisito:} Descripción detallada de una funcionalidad o característica deseada.
        Guía el diseño, implementación y prueba.

    \end{definicion}

    \subsection{Tipos de Requisitos}\label{subsec:tipos-de-requisitos}

    \subsubsection{Funcionales}\label{subsubsec:requisitos-funcionales}

    \paragraph{Qué:} Describe qué debe hacer el software

    \begin{exemplo}
        \begin{itemize}
            \item Permitir ingresar información de contacto

            \item Calcular impuesto sobre renta

            \item Generar informe contable
        \end{itemize}
    \end{exemplo}

    \subsubsection{No Funcionales}

    \paragraph{Cómo:} Describe cómo debe comportarse el software

    \begin{exemplo}
        \begin{itemize}
            \item Disponibilidad 24/7

            \item Compatibilidad multi-SO/navegador

            \item Tiempo de respuesta < 1 minuto
        \end{itemize}
    \end{exemplo}

    \subsection{Clasificación de requisitos no funcionales}\label{subsec:clasificacion-de-requisitos-no-funcionales}

    Los Requisitos No Funcionales se dividen en tres categorías principales:

    \subsubsection{1. Product Requirements (Requisitos del Producto)}

    Definen las características intrínsecas del software:


    \begin{itemize}

        \item Usability Requirements (Requisitos de Usabilidad)
        \begin{itemize}

            \item Facilidad de uso e interfaz intuitiva

            \item Experiencia del usuario (UX)

            \item Accesibilidad

        \end{itemize}



        \item Portability Requirements (Requisitos de Portabilidad)
        \begin{itemize}

            \item Capacidad de ejecutarse en diferentes plataformas

            \item Adaptabilidad a distintos sistemas operativos

            \item Compatibilidad multiplataforma

        \end{itemize}



        \item Efficiency Requirements (Requisitos de Eficiencia)
        \begin{itemize}

            \item Uso optimizado de recursos del sistema

            \item Gestión eficiente de memoria y CPU

            \item Tiempos de respuesta

        \end{itemize}



        \item Reliability Requirements (Requisitos de Confiabilidad)
        \begin{itemize}

            \item Disponibilidad del sistema

            \item Tolerancia a fallos

            \item Recuperación ante errores

        \end{itemize}



        \item Performance Requirements (Requisitos de Rendimiento) (subtipo de Eficiencia)
        \begin{itemize}

            \item Velocidad de procesamiento

            \item Throughput (rendimiento)

            \item Tiempos de respuesta específicos

        \end{itemize}



        \item Space Requirements (Requisitos de Espacio) (subtipo de Eficiencia)
        \begin{itemize}

            \item Limitaciones de almacenamiento

            \item Uso de memoria

            \item Capacidad de datos

        \end{itemize}


    \end{itemize}

    \subsubsection{2. Process Requirements (Requisitos del Proceso)}

    Relacionados con el desarrollo y despliegue:


    \begin{itemize}

        \item Delivery Requirements (Requisitos de Entrega)
        \begin{itemize}

            \item Plazos de desarrollo

            \item Metodología de entrega

            \item Hitos del proyecto

        \end{itemize}



        \item Implementation Requirements (Requisitos de Implementación)
        \begin{itemize}

            \item Lenguajes de programación específicos

            \item Herramientas de desarrollo

            \item Estándares de codificación

        \end{itemize}



        \item Standards Requirements (Requisitos de Estándares)
        \begin{itemize}

            \item Cumplimiento de normativas

            \item Estándares de la industria

            \item Protocolos específicos

        \end{itemize}


    \end{itemize}

    \subsubsection{3. External Requirements (Requisitos Externos)}

    Impuestos por factores externos al sistema:


    \begin{itemize}

        \item Interoperability Requirements (Requisitos de Interoperabilidad)
        \begin{itemize}

            \item Integración con otros sistemas

            \item Intercambio de datos

            \item APIs y protocolos de comunicación

        \end{itemize}



        \item Ethical Requirements (Requisitos Éticos)
        \begin{itemize}

            \item Consideraciones morales

            \item Impacto social

            \item Responsabilidad corporativa

        \end{itemize}



        \item Legislative Requirements (Requisitos Legislativos)
        \begin{itemize}

            \item Cumplimiento legal

            \item Regulaciones gubernamentales

            \item Normativas específicas del sector

        \end{itemize}



        \item Privacy Requirements (Requisitos de Privacidad) (subtipo de Legislativos)
        \begin{itemize}

            \item Protección de datos personales

            \item GDPR/LOPD

            \item Gestión de información sensible

        \end{itemize}



        \item Safety Requirements (Requisitos de Seguridad) (subtipo de Legislativos)
        \begin{itemize}

            \item Protección contra amenazas

            \item Seguridad de datos

            \item Control de acceso
        \end{itemize}

    \end{itemize}

    \subsection{Dificultades Comunes}\label{subsec:dificultades-comunes}

    \begin{itemize}

        \item Alcance: Fronteras mal definidas

        \item Entendimiento: Usuarios inseguros, información \textquote{obvia} omitida, requisitos ambiguos o en conflicto, \ldots

        \item Volatilidad: Requisitos cambian con el tiempo
    \end{itemize}

    \subsubsection{Impacto}
    Estas dificultades pueden llevar a:

    \begin{itemize}

        \item 15\% del tiempo en definir requisitos.

        \item Cambiar requisitos tras entrega cuesta de 60 a 100 veces más.

        \item 56\% de errores por especificación incorrecta.

        \item Solo 2\% del software entregado cumple completamente.

    \end{itemize}

    \subsection{Ingeniería de requisitos}\label{subsec:ingenieria-de-requisitos}
    Para desarrollar un software de calidad, es fundamental una buena ingeniería de requisitos.
    Para ello, se sigue un proceso iterativo que incluye:

    \begin{itemize}
        \item Concepción Necesidad inicial.
        Identificar la necesidad del software y su contexto.

        \item \textbf{Indagación} Preguntas organizadas sobre objetivos.
        ¿Qué, por qué y cómo?

        \item \textbf{Elaboración:} Modelo refinado, escenarios usuario, diagramas

        \item \textbf{Negociación:} Priorizar, resolver conflictos

        \item \textbf{Especificación:} Documentación formal

        \item \textbf{Validación:} Detectar ambigüedades, inconsistencias y verificar con usuarios

        \item \textbf{Administración:} Controlar cambios, versiones y trazabilidad
    \end{itemize}

    \subsection*{1. Indagación de requisitos}\label{subsec:indagacion-de-requisitos}
%    \etoctoccontentsline{subsection}{Indagación de requisitos}{2}
%    \etocimmediatetoccontentsline{subsection}{Indagación de requisitos}
    \addcontentsline{toc}{subsubsection}{1. Indagación de requisitos}

    La indagación de requisitos es una fase crítica donde se obtienen y refinan los requisitos del software.
    En esta fase distinguen dos actividades principales:

    \subsubsection*{1. Preparación}


    Antes de capturar los requisitos, es esencial una fase de preparación que garantice el éxito de la indagación.
    Esta comienza con la \textbf{identificación de los participantes}, es decir, todas las personas relevantes para el sistema: solicitantes, usuarios finales y otros perfiles implicados.


    A continuación, se realiza la \textbf{recepción y análisis de la concepción}, un documento preliminar que describe de forma general la necesidad o el problema.
    Esta revisión ayuda a entender el contexto, anticipar posibles temas clave y preparar la discusión.


    Una vez analizada la concepción, se elabora una \textbf{agenda de indagación} que guíe las reuniones y asegure que se aborden todos los puntos críticos:

    \begin{itemize}

        \item Quién es el solicitante.

        \item Qué perfil tienen los usuarios.

        \item Cuál es el problema a resolver.

        \item Cómo se definiría una solución exitosa.

        \item Qué otros participantes deben considerarse en el proceso.

    \end{itemize}

    \subsubsection*{2. Captura de requisitos}


    Durante la indagación, los requisitos pueden obtenerse mediante:

    \begin{itemize}

        \item Entrevistas estructuradas.

        \item Lluvia de ideas (brainstorming).

        \item Observación directa del entorno.

        \item Análisis de sistemas similares.

        \item Prototipos de baja fidelidad.

    \end{itemize}

    \subsubsection*{3. Tipos de requisitos según el cliente}


    \begin{itemize}

        \item \textbf{Normales:} Son los que el cliente solicita explícitamente.

        \item \textbf{Esperados:} No se mencionan, pero se asumen necesarios.

        \item \textbf{Emocionantes:} Superan las expectativas del cliente y generan satisfacción.

    \end{itemize}

    \subsection*{2. Elaboración}\label{subsec:elaboracion}
    \addcontentsline{toc}{subsubsection}{2. Elaboración}

    La elaboración es una fase donde se refinan los requisitos obtenidos en la indagación.

    \subsubsection*{Casos de uso}

    \begin{definicion}
        Un \textbf{caso de uso} describe una secuencia de acciones entre un actor y el sistema.
        Representa una unidad de funcionalidad y permite estimar, verificar y priorizar el trabajo.
        Cada caso puede dividirse en varios escenarios (principales o alternativos), y cada actor representa un conjunto de roles del usuario.
    \end{definicion}

    \paragraph{Características}

    \begin{itemize}

        \item Valorados por clientes

        \item Estimables (implementación independiente)

        \item Pequeños (fácil estimación)

        \item Verificables
    \end{itemize}



    \begin{exemplo}
        \textbf{Especificación de un caso de uso:}

        \begin{itemize}

            \item Nombre: Nombre del caso de uso

            \item Actor/es: Rol del usuario (o entidad(es)) que interactúa con el sistema

            \item Descripción: Breve resumen de la funcionalidad

            \item Precondiciones: Condiciones que deben cumplirse antes de iniciar el caso de uso, se asumen ciertas.

            \item Dependencias: Otros casos de uso o requisitos que deben cumplirse

            \item Escenario principal: Secuencia de pasos que describe la interacción normal entre el actor y el sistema

            \item Excepciones: Condiciones especiales o errores que pueden ocurrir durante la ejecución del caso de uso

            \item Prioridad (P0, P1, \ldots): Importancia del caso de uso

        \end{itemize}
    \end{exemplo}

    \subsubsection*{Historias de usuario}

    \begin{definicion}
        Descripción breve y sencilla, escrita en lenguaje natural y enfocada en
        las necesidades del usuario, que sirve como base para el desarrollo de
        una característica del software.
    \end{definicion}

    \begin{exemplo}

        \textit{\textquote{[rol], quiero [función] para [beneficio]}}
    \end{exemplo}

    Una historia específica podría ser:

    \begin{exemplo}

        \begin{quote}

            Como usuario de la aplicación de compras quiero ver el estado de mis pedidos para saber cuáles están en proceso de envío.

        \end{quote}

    \end{exemplo}

    No son requisitos ni sustituyen a los casos de uso, pero ayudan a centrar el desarrollo en el valor para el usuario y permiten definir criterios de aceptación.
    En la \autoref{tab:casos-de-uso-vs-historias-de-usuario} se muestra una comparación entre casos de uso y historias de usuario.

    \subsubsection*{Casos de uso vs Historias de usuario}


    \begin{itemize}

        \item \textbf{Casos de uso:} Más detallados, útiles para stakeholders.

        \item \textbf{Historias de usuario:} Más ágiles, enfocadas al equipo de desarrollo.

    \end{itemize}


    \begin{table}
        \centering
        \begin{tabular}{lll}
            \toprule
            \textbf{Aspecto}       & \textbf{Casos Uso}          & \textbf{Historias Usuario} \\
            \midrule
            Agilismo               & Indiferente                 & Sí                         \\
            Representan requisitos & Sí                          & No                         \\
            Describen              & Interacciones actor-sistema & Capacidades usuario        \\
            Útil para              & Stakeholders/Clientes       & Equipo desarrollo          \\
            \bottomrule
        \end{tabular}
        \caption{Comparación entre Casos de Uso y Historias de Usuario}
        \label{tab:casos-de-uso-vs-historias-de-usuario}
    \end{table}




    Ambos permiten priorización y ponen al usuario en el centro.

    \subsection*{3.Negociación y especificación}\label{subsec:3.negociacion-y-especificacion}
    \addcontentsline{toc}{subsubsection}{3. Negociación y especificación}


    Durante la negociación, se ajustan expectativas, prioridades y limitaciones.
    Se analiza el coste, el tiempo y el alcance del producto.
    De este proceso resultan dos documentos clave:


    \begin{itemize}

        \item \textbf{DRU (Documento de Requisitos del Usuario):} Qué problema resolver.

        \item \textbf{ERS (Especificación de Requisitos del Software):} Qué debe hacer el sistema para alcanzar los objetivos del DRU\@.

    \end{itemize}


    \subsection*{4. Validación}\label{subsec:4.-validacion}
    \addcontentsline{toc}{subsubsection}{4. Validación}


    Se realiza para asegurar:

    \begin{itemize}

        \item Coherencia entre requisitos y resultado esperado.

        \item Priorización adecuada.

        \item Ausencia de ambigüedades.

        \item Viabilidad y resolución de conflictos.

    \end{itemize}


    \textbf{Técnicas:} revisión por expertos, prototipado, validación de modelos, verificación de testabilidad.


    \subsection*{5. Gestión}
    \addcontentsline{toc}{subsubsection}{5. Gestión}
    \phantomsection


    La gestión consiste en supervisar los requisitos a lo largo de todo el ciclo de vida del producto.
    Incluye:


    \begin{itemize}

        \item Registro de cambios y asegurar la consistencia entre los requisitos y el sistema.

        \item Repriorización.

        \item Adición o eliminación de requisitos.

        \item Seguimiento del estado de cada uno.

    \end{itemize}

    \subsection{Modelado de requisitos}\label{subsec:modelado-de-requisitos}


    El modelado de requisitos tiene como objetivo representar, de forma gráfica o estructurada, los distintos aspectos del sistema desde el punto de vista de los usuarios, del dominio de información y del comportamiento del software.
    Permite comunicar mejor las necesidades del sistema, detectar inconsistencias y preparar el diseño.


    \textbf{Tipos de diagramas:}

    \begin{itemize}

        \item \textbf{Basados en escenarios:} describen interacciones usuario-sistema.
        Incluyen casos de uso y sus escenarios.

        \item \textbf{De datos:} muestran entidades, atributos y relaciones (diagramas E–R).

        \item \textbf{Orientados a clases:} presentan clases, atributos, métodos y relaciones (herencia, asociación\ldots).

        \item \textbf{De flujo:} representan transformación y circulación de datos.

        \item \textbf{De comportamiento:} representan estados del sistema y transiciones ante eventos.

    \end{itemize}


    \textbf{Lenguaje de modelado UML:}

    \begin{itemize}

        \item \textbf{Diagramas estructurales:} clases, objetos, componentes, despliegue.

        \item \textbf{Diagramas de comportamiento:} casos de uso, secuencia, estados, colaboración.

    \end{itemize}


    \textbf{Ventajas del modelado:}

    \begin{itemize}

        \item Mejora la comprensión y comunicación.

        \item Detecta errores e inconsistencias tempranas.

        \item Sirve como base para diseño y validación.

    \end{itemize}


    % --- **** ---- **** ---- **** ---- **** ---- **** ---- **** ---- **** ---- **** ---- **** ---- **** ---- **** ----