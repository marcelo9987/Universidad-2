% !TEX root = ./apuntesIngenieriaDelSoftware_20242025.tex
\documentclass[a4paper,11pt]{report}


% Codificación e idioma
% \usepackage[utf8]{inputenc}
% \usepackage[T1]{fontenc}
\usepackage[spanish]{babel}

% Paquetes para estilo e utilidades
\usepackage{geometry}
\geometry{margin=2.5cm}
\usepackage{amsmath, amssymb}
\usepackage{graphicx}
\usepackage{float}
\usepackage{tcolorbox}
\usepackage{fancyhdr}
\usepackage{hyperref}
\usepackage{tikz}
\usepackage{caption}
\usepackage{subcaption}
\usepackage{booktabs}
\usepackage{tabularx}
\usepackage{csquotes}
\usepackage{multicol}

\usepackage{microtype}
\usepackage{parskip}
\usepackage{enumitem}

\usepackage{etoc}
\usepackage{appendix}

\usetikzlibrary
{
    shapes
    , shapes.arrows
    , arrows
    , arrows.meta
    , positioning
    , decorations.pathmorphing
    , shadows
}

\usetikzlibrary{babel}
\usetikzlibrary{graphs}
\usetikzlibrary {bending}

\pagestyle{fancy}
\fancyhead[L]{Apuntes de Ingeniería del Software}
\fancyhead[R]{Marcelo Fort Muñoz}
\fancyfoot[C]{\thepage}

\definecolor{exemploColor}{RGB}{230, 245, 255}
\definecolor{notaColor}{RGB}{255, 255, 230}
\definecolor{definicionColor}{RGB}{0, 255, 230}
\definecolor{azul}{RGB}{0, 230, 255}
\definecolor{rosa}{RGB}{170,80,120}
\definecolor{rojo}{RGB}{255,80,0}

% % Caixas personalizadas -- Código histórico
% \tcbset{
%   exemplo/.style={colback=exemploColor, colframe=blue!50!black, title=Ejemplo},
%   nota/.style={colback=notaColor, colframe=orange!60!black, title=Nota personal},
%   definicion/.style={colback=definicion,colframe=green!70!black,title=Definición}
% }

% Cajas personalizadas como entornos newtcolorbox
\newtcolorbox{exemplo}{
    colback=exemploColor,
    colframe=blue!50!black,
    title=Ejemplo
}

\newtcolorbox{nota}{
    colback=notaColor,
    colframe=orange!60!black,
    title=Nota personal
}

\newtcolorbox{cajaverde}[1][]{
    colback=exemploColor,
    colframe=blue!50!black,
    title=#1
}

\newtcolorbox{cajanaranja}[1][]{
    colback=notaColor,
    colframe=orange!60!black,
    title=#1
}

\newtcolorbox{cajaroja}[1][]{
    colback=rojo,
    colframe=red!60!black,
    title=#1
}

\newtcolorbox{cajarosa}[1][]{
    colback=rosa,
    colframe=pink!70!black,
    title=#1
}

\newtcolorbox{definicion}{
    colback=definicionColor,
    colframe=green!50!black,
    title=Definición
}

\newtcolorbox{cajaazul}[1][]{
    colback=azul,
    colframe=blue!60!black,
    title=#1
}

\setlength{\headheight}{13.7pt}


\setcounter{secnumdepth}{2} % Profundidad de numeración de secciones

\setcounter{tocdepth}{2} % TOC global ata section

% Metadatos
\title{Apuntes Completos \\ \large Ingeniería del Software \\ Universidad de Antonio de Nebrija}
\author{Marcelo Fort Muñoz}
\date{\today}

\begin{document}

    % las primeras páginas no tienen numeración
    \pagenumbering{gobble}

    \maketitle
    \tableofcontents
    \newpage

    \etocsettocdepth{3} % Profundidad de la tabla de contenidos local

    \pagenumbering{arabic} % Numeración de páginas a partir de aquí


    \chapter{Introducción}\label{ch:introduccion}
    % !TeX root = ./main.tex


\section{Ejercicio 1}\label{sec:intro-ej1}
% !TeX root = ../examen-parcial-2023.tex


\begin{itemize}
    \item \textbf{Puntos:} 2
\end{itemize}
\begin{enunciado}
    El actual equipo de desarrollo está aplicando las siguientes prácticas:
    \begin{enumerate}
        \item El equipo planifica entregas trimestrales que incluyen un conjunto de funcionalidades acordadas entre el director de ingeniería y el director de producción dentro un plan anual.
        \item El equipo se asegura que el conjunto de funcionalidades de las entregas trimestrales se comporta correctamente y es usado por los usuarios finales sin dificultades.
        \item El equipo se comunica directamente con el director de producción cuando tiene dudas acerca de cómo debe comportarse una funcionalidad concreta.
        \item Dentro del equipo cada miembro tiene su función: una persona diseña la solución, otro la construye, otro la prueba y otro la despliega y mantiene en producción.
    \end{enumerate}
    Lee detenidamente las prácticas e:
    \begin{enumerate}
        \item Identifica, para cada una de ellas, si se corresponden con prácticas ágiles.
        \item Justifica las respuestas en base al manifiesto ágil.
        \item \textbf{0,3 cada respuesta correcta con justificación.}
        \item Indica qué cambios aplicarías en las que no son ágiles (si hay alguna) para que sí lo sean.
        \item \textbf{0,4 por cada práctica convertida en ágil.}
    \end{enumerate}
\end{enunciado}

\begin{solucion}
    \begin{enumerate}
        \item \textbf{Práctica 1:} NO ágil.
        Se incumple el valor Respuesta ante el cambio sobre seguir un plan.
        \begin{itemize}
            \item \textbf{Cambios para que fuera ágil:}
            \begin{itemize}
                \item Ciclos de desarrollo más cortos (2 a 4 semanas).
                \item Identificación y priorización de funcionalidades en cada ciclo.
            \end{itemize}
        \end{itemize}

        \item \textbf{Práctica 2:} Ágil.
        Se cumple el valor Software funcionando sobre documentación extensiva.

        \item \textbf{Práctica 3:} Ágil.
        Se cumple el valor Colaboración con el cliente sobre negociación contractual.

        \item \textbf{Práctica 4:} NO ágil.
        Se incumple el valor Individuos e interacciones sobre procesos y herramientas.
        \begin{itemize}
            \item \textbf{Cambios para que fuera ágil:}
            \begin{itemize}
                \item Individuos multidisciplinares.
                \item Colaboración entre los miembros para realizar las diferentes funciones.
            \end{itemize}
        \end{itemize}
    \end{enumerate}
\end{solucion}



\section{Ejercicio 2: ISO/IEC 15504}\label{sec:intro-ej2}
% !TeX root = ../examen-parcial-2023.tex

\begin{itemize}
    \item \textbf{Puntos:} 3
\end{itemize}

\begin{enunciado}
    Se han identificado los siguientes requisitos como parte de la mejora de la gestión de la
    producción:
    \begin{enumerate}
        \item Los gestores deben poder acceder al sistema a través de una interfaz web mientras que
        los agricultores deben poder hacerlo mediante una aplicación móvil disponible para iOS\@.
        \item Los gestores de producción deben poder añadir y eliminar campos de cultivo al sistema.
        \item Los agricultores deben poder registrar las labores realizadas en los campos de cultivo
        (arado, siembra, riego, abonado, recolección, etc.) mediante geolocalización.
        \item Los agricultores deben poder notificar incidencias que afecten a la producción (plagas,
        eventos climatológicos, etc.).
        \item El equipo de desarrollo debe poder saber si el sistema está funcionando correctamente.
        \item Los gestores deben poder anotar la producción recolectada en cada campo.
        \item La aplicación debe tener un porcentaje de disponibilidad anual del 99.99\%.
        \item Los gestores deben poder marcar el estado de un campo (barbecho, activo, etc.).
    \end{enumerate}
    Lee detenidamente los requisitos y:
    \begin{enumerate}
        \item Clasifica los requisitos en funcionales, no funcionales u otros.
        \item $0.2$ puntos por cada respuesta correcta.
        \item Desarrolla la especificación del caso de uso de uno de los requisitos que hayas
        clasificado como funcional.
        \item $0.2$ puntos por cada campo simple; $0.3$ puntos por cada campo
    \end{enumerate}
\end{enunciado}
\begin{solucion}
    \begin{enumerate}
        \item Clasificación de los requisitos:
        \begin{itemize}
            \item Requisito 1: No funcional.
            \item Requisito 2: Funcional.
            \item Requisito 3: Funcional.
            \item Requisito 4: Funcional.
            \item Requisito 5: Otros.
            \item Requisito 6: Funcional.
            \item Requisito 7: No funcional.
            \item Requisito 8: Funcional.
        \end{itemize}

        \item Especificación del caso de uso (ejemplo para el requisito 2):
        \begin{itemize}
            \item Nombre: Añadir campo de cultivo.
            \item Actor: Gestor de producción.
            \item Descripción: Los gestores de producción deben poder añadir y eliminar campos de cultivo al sistema.
            \item Precondiciones: El gestor de producción debe haber iniciado sesión en el sistema.
            \item Dependencias: No especificado.
            \item Escenario:
            \begin{enumerate}
                \item El gestor de campo comienza el proceso de añadir un campo.
                \item El gestor de campo rellena los detalles del campo (nombre, localización,\ldots ).
                \item El gestor graba el campo en el sistema.
            \end{enumerate}
            \item Excepciones:
            \begin{enumerate}
                \item El gestor de campo comienza el proceso de añadir un campo.
                \item El gestor de campo no rellena todos los detalles del campo.
                \item El gestor de campo intenta grabar el campo en el sistema.
                \item El sistema indica que faltan detalles del campo.
            \end{enumerate}
        \end{itemize}
    \end{enumerate}
    \begin{itemize}
        \item Prioridad: No especificado.
    \end{itemize}
\end{solucion}


    \chapter{El proceso del software}\label{ch:el-proceso-del-software}
    % !TeX root = ../main.tex


\section{Ejercicio 1: Scrum a Kanban}\label{sec:el-proceso-del-software-ej1}
% !TeX root = ../examen-parcial-2023.tex


\begin{itemize}
    \item \textbf{Puntos:} 2
\end{itemize}
\begin{enunciado}
    El actual equipo de desarrollo está aplicando las siguientes prácticas:
    \begin{enumerate}
        \item El equipo planifica entregas trimestrales que incluyen un conjunto de funcionalidades acordadas entre el director de ingeniería y el director de producción dentro un plan anual.
        \item El equipo se asegura que el conjunto de funcionalidades de las entregas trimestrales se comporta correctamente y es usado por los usuarios finales sin dificultades.
        \item El equipo se comunica directamente con el director de producción cuando tiene dudas acerca de cómo debe comportarse una funcionalidad concreta.
        \item Dentro del equipo cada miembro tiene su función: una persona diseña la solución, otro la construye, otro la prueba y otro la despliega y mantiene en producción.
    \end{enumerate}
    Lee detenidamente las prácticas e:
    \begin{enumerate}
        \item Identifica, para cada una de ellas, si se corresponden con prácticas ágiles.
        \item Justifica las respuestas en base al manifiesto ágil.
        \item \textbf{0,3 cada respuesta correcta con justificación.}
        \item Indica qué cambios aplicarías en las que no son ágiles (si hay alguna) para que sí lo sean.
        \item \textbf{0,4 por cada práctica convertida en ágil.}
    \end{enumerate}
\end{enunciado}

\begin{solucion}
    \begin{enumerate}
        \item \textbf{Práctica 1:} NO ágil.
        Se incumple el valor Respuesta ante el cambio sobre seguir un plan.
        \begin{itemize}
            \item \textbf{Cambios para que fuera ágil:}
            \begin{itemize}
                \item Ciclos de desarrollo más cortos (2 a 4 semanas).
                \item Identificación y priorización de funcionalidades en cada ciclo.
            \end{itemize}
        \end{itemize}

        \item \textbf{Práctica 2:} Ágil.
        Se cumple el valor Software funcionando sobre documentación extensiva.

        \item \textbf{Práctica 3:} Ágil.
        Se cumple el valor Colaboración con el cliente sobre negociación contractual.

        \item \textbf{Práctica 4:} NO ágil.
        Se incumple el valor Individuos e interacciones sobre procesos y herramientas.
        \begin{itemize}
            \item \textbf{Cambios para que fuera ágil:}
            \begin{itemize}
                \item Individuos multidisciplinares.
                \item Colaboración entre los miembros para realizar las diferentes funciones.
            \end{itemize}
        \end{itemize}
    \end{enumerate}
\end{solucion}



%\section{Ejercicio 2: Flujos}\label{sec:el-proceso-del-software-ej2}
%% !TeX root = ../examen-parcial-2023.tex

\begin{itemize}
    \item \textbf{Puntos:} 3
\end{itemize}

\begin{enunciado}
    Se han identificado los siguientes requisitos como parte de la mejora de la gestión de la
    producción:
    \begin{enumerate}
        \item Los gestores deben poder acceder al sistema a través de una interfaz web mientras que
        los agricultores deben poder hacerlo mediante una aplicación móvil disponible para iOS\@.
        \item Los gestores de producción deben poder añadir y eliminar campos de cultivo al sistema.
        \item Los agricultores deben poder registrar las labores realizadas en los campos de cultivo
        (arado, siembra, riego, abonado, recolección, etc.) mediante geolocalización.
        \item Los agricultores deben poder notificar incidencias que afecten a la producción (plagas,
        eventos climatológicos, etc.).
        \item El equipo de desarrollo debe poder saber si el sistema está funcionando correctamente.
        \item Los gestores deben poder anotar la producción recolectada en cada campo.
        \item La aplicación debe tener un porcentaje de disponibilidad anual del 99.99\%.
        \item Los gestores deben poder marcar el estado de un campo (barbecho, activo, etc.).
    \end{enumerate}
    Lee detenidamente los requisitos y:
    \begin{enumerate}
        \item Clasifica los requisitos en funcionales, no funcionales u otros.
        \item $0.2$ puntos por cada respuesta correcta.
        \item Desarrolla la especificación del caso de uso de uno de los requisitos que hayas
        clasificado como funcional.
        \item $0.2$ puntos por cada campo simple; $0.3$ puntos por cada campo
    \end{enumerate}
\end{enunciado}
\begin{solucion}
    \begin{enumerate}
        \item Clasificación de los requisitos:
        \begin{itemize}
            \item Requisito 1: No funcional.
            \item Requisito 2: Funcional.
            \item Requisito 3: Funcional.
            \item Requisito 4: Funcional.
            \item Requisito 5: Otros.
            \item Requisito 6: Funcional.
            \item Requisito 7: No funcional.
            \item Requisito 8: Funcional.
        \end{itemize}

        \item Especificación del caso de uso (ejemplo para el requisito 2):
        \begin{itemize}
            \item Nombre: Añadir campo de cultivo.
            \item Actor: Gestor de producción.
            \item Descripción: Los gestores de producción deben poder añadir y eliminar campos de cultivo al sistema.
            \item Precondiciones: El gestor de producción debe haber iniciado sesión en el sistema.
            \item Dependencias: No especificado.
            \item Escenario:
            \begin{enumerate}
                \item El gestor de campo comienza el proceso de añadir un campo.
                \item El gestor de campo rellena los detalles del campo (nombre, localización,\ldots ).
                \item El gestor graba el campo en el sistema.
            \end{enumerate}
            \item Excepciones:
            \begin{enumerate}
                \item El gestor de campo comienza el proceso de añadir un campo.
                \item El gestor de campo no rellena todos los detalles del campo.
                \item El gestor de campo intenta grabar el campo en el sistema.
                \item El sistema indica que faltan detalles del campo.
            \end{enumerate}
        \end{itemize}
    \end{enumerate}
    \begin{itemize}
        \item Prioridad: No especificado.
    \end{itemize}
\end{solucion}
%todo: complétame


\section{Ejercicio 3: Identificar fases del proceso}\label{sec:el-proceso-del-software-ej3}
% !TeX root = ../examen-parcial-2023.tex
%
%Analiza detenidamente el diagrama y:
%A) Asocia los componentes con los requisitos que cubren. - 0,4 por cada requisito
%Interfaz web e interfaz móvil. Requisito 1.
%Gestor de campos. Requisitos 2 y 8.
%Registrador de labores. Requisito 3.
%Administrador de incidencias. Requisito 4.
%B) Completa el diagrama con los componentes necesarios para cubrir todos los requisitos
%funcionales del Ejercicio 2. - 1 punto
%Gestor de producción. Conectado al canal de comunicación. Requisito 6.

\begin{itemize}
    \item \textbf{Puntos:} 3
\end{itemize}

\begin{enunciado}
    El equipo de desarrollo ha realizado esta primera versión del diagrama de arquitectura para
    cubrir los requisitos funcionales del Ejercicio 2:


    \deactivatequoting

    \begin{tikzpicture}[
    % Estilos máis simples pero modernos
        interface/.style={
            rectangle,
            rounded corners=5pt,
            fill=blue!10,
            draw=blue!50,
            line width=1pt,
            font=\sffamily,
            minimum width=3cm,
            minimum height=1cm,
            align=center
        },
        communication/.style={
            rectangle,
            rounded corners=8pt,
            fill=blue!80,
            text=white,
            font=\sffamily\bfseries,
            minimum width=7cm,
            minimum height=1.2cm,
            align=center
        },
        component/.style={
            rectangle,
            rounded corners=3pt,
            fill=gray!10,
            draw=gray!50,
            font=\sffamily\small,
            minimum width=2.5cm,
            minimum height=0.8cm,
            align=center
        }
    ]

        % Interfaces superiores
        \node[interface] (interfaz_movil) at (0, 3) {Interfaz Móvil};
        \node[interface] (interfaz_web) at (5, 3) {Interfaz Web};

        % Canal de comunicación
        \node[communication] (canal) at (2.5, 1) {Canal de comunicación};

        % Compoñentes inferiores
        \node[component] (registrador) at (0, -1) {Registrador de\\labores};
        \node[component] (gestor) at (2.5, -1) {Gestor de\\Campos};
        \node[component] (administrador) at (5, -1) {Administrador de\\incidencias};

        % Conexiones
        \draw[->, thick, blue] (interfaz_movil) -- (canal);
        \draw[->, thick, blue] (interfaz_web) -- (canal);
        \draw[->, thick, blue] (canal) -- (registrador);
        \draw[->, thick, blue] (canal) -- (gestor);
        \draw[->, thick, blue] (canal) -- (administrador);

    \end{tikzpicture}

    \begin{enumerate}
        \item Analiza detenidamente el diagrama y:
        \begin{enumerate}
            \item Asocia los componentes con los requisitos que cubren.
            \item Completa el diagrama con los componentes necesarios para cubrir todos los requisitos
            funcionales del Ejercicio 2.
        \end{enumerate}
        \item \textbf{0,4 puntos por cada requisito asociado.}
        \item \textbf{1 punto por completar el diagrama.}
    \end{enumerate}

\end{enunciado}

\begin{solucion}
    \begin{enumerate}
        \item Análisis del diagrama:
        \begin{enumerate}
            \item Asociaciones de componentes con requisitos:
            \begin{itemize}
                \item Interfaz web e interfaz móvil: Requisito 1.
                \item Gestor de campos: Requisitos 2 y 8.
                \item Registrador de labores: Requisito 3.
                \item Administrador de incidencias: Requisito 4.
            \end{itemize}

            \item Componentes necesarios para cubrir todos los requisitos funcionales:
            \begin{itemize}
                \item Gestor de producción: Conectado al canal de comunicación: Requisito 6.
            \end{itemize}
        \end{enumerate}
    \end{enumerate}
\end{solucion}


\section{Ejercicio 4: Escoger proceso de sofware}\label{sec:el-proceso-del-software-ej4}
% !TeX root = ../mantenimiento.tex


\begin{enunciado}
    ¿Qué tipo de mantenimiento representan cada una de las siguientes acciones?
    \begin{enumerate}
        \item Adaptar la aplicación para que no incumpla la nueva normativa de protección de datos que entrará en vigor el próximo mes.
        \item Expandir la aplicación a un país nuevo, donde el idioma usado para comunicarse con el usuario debe ser diferente a los soportados.
        \item Añadir test unitarios, que no existían en la versión inicial.
        \item Mejorar la accesibilidad de nuestra web para usuarios con ceguera o deficiencia visual.
        \item Solucionar un problema que causa que uno de cada cien registros falle
    \end{enumerate}
\end{enunciado}
\begin{solucion}
    \begin{description}
        \item[1] Adaptativo
        \item[2] Adaptativo
        \item[3] Preventivo
        \item[4] Perfectivo
        \item[5] Correctivo
    \end{description}
\end{solucion}


\section{Ejercicio 5: Scrum vérsus Kanban}\label{sec:el-proceso-del-software-ej5}
% !TeX root = ../planificacion.tex

\begin{enunciado}
    Desarrolla el diagrama de PERT del conjunto de tareas del ejercicio 3.
    ¿Qué tareas están en el camino crítico?
\end{enunciado}

\begin{solucion}

    La duración total del proyecto es de \textbf{33 días}, determinada por el camino crítico:

    A → C → E → F → G → H → I → J\@.

    Diagrama PERT de las \textbf{tareas del ejercicio 3}:

    \deactivatequoting
    \tikz[>={To[sep]}, rotate=90, xscale=-1]
    \graph [nodes={circle,draw},
        edges={nodes={inner sep=1pt, anchor=mid}}]
    {
        A ->
            {
                {
                B ->
                    {
                    D
                }
            }
            ,
                {
                C ->
                    {
                    E ->
                        {
                        F
                    }
                }
            }
        } ->
            {
            G ->
                {
                H ->
                    {
                    I ->
                        {
                        J
                    }
                }
            }
        }
    };
    \activatequoting
\end{solucion}



    \chapter{Modelado del software}\label{ch:modelado-del-software}
        \localtableofcontents


    \section{Requisitos del software}\label{sec:requisitos-del-software}
        \subsection{Principios de comunicación}\label{subsec:principios-de-comunicacion}

    \begin{enumerate}

        \item Escuchar activamente.

        \item Prepararse previamente.

        \item Facilitar la comunicación (agenda, mediador).

        \item Comunicación cara a cara.

        \item Documentar decisiones clave.

        \item Fomentar colaboración.

        \item Seguir agenda.

        \item Apoyar con representación gráfica.

        \item Avanzar ante bloqueos.

        \item Negociar de forma equilibrada.

    \end{enumerate}

    \subsection{Definición de requisito}\label{subsec:definicion-de-requisito}

    \begin{definicion}

        \textbf{Requisito:} Descripción detallada de una funcionalidad o característica deseada.
        Guía el diseño, implementación y prueba.

    \end{definicion}

    \subsection{Tipos de Requisitos}\label{subsec:tipos-de-requisitos}

    \subsubsection{Funcionales}\label{subsubsec:requisitos-funcionales}

    \paragraph{Qué:} Describe qué debe hacer el software

    \begin{exemplo}
        \begin{itemize}
            \item Permitir ingresar información de contacto

            \item Calcular impuesto sobre renta

            \item Generar informe contable
        \end{itemize}
    \end{exemplo}

    \subsubsection{No Funcionales}

    \paragraph{Cómo:} Describe cómo debe comportarse el software

    \begin{exemplo}
        \begin{itemize}
            \item Disponibilidad 24/7

            \item Compatibilidad multi-SO/navegador

            \item Tiempo de respuesta < 1 minuto
        \end{itemize}
    \end{exemplo}

    \subsection{Clasificación de requisitos no funcionales}\label{subsec:clasificacion-de-requisitos-no-funcionales}

    Los Requisitos No Funcionales se dividen en tres categorías principales:

    \subsubsection{1. Product Requirements (Requisitos del Producto)}

    Definen las características intrínsecas del software:


    \begin{itemize}

        \item Usability Requirements (Requisitos de Usabilidad)
        \begin{itemize}

            \item Facilidad de uso e interfaz intuitiva

            \item Experiencia del usuario (UX)

            \item Accesibilidad

        \end{itemize}



        \item Portability Requirements (Requisitos de Portabilidad)
        \begin{itemize}

            \item Capacidad de ejecutarse en diferentes plataformas

            \item Adaptabilidad a distintos sistemas operativos

            \item Compatibilidad multiplataforma

        \end{itemize}



        \item Efficiency Requirements (Requisitos de Eficiencia)
        \begin{itemize}

            \item Uso optimizado de recursos del sistema

            \item Gestión eficiente de memoria y CPU

            \item Tiempos de respuesta

        \end{itemize}



        \item Reliability Requirements (Requisitos de Confiabilidad)
        \begin{itemize}

            \item Disponibilidad del sistema

            \item Tolerancia a fallos

            \item Recuperación ante errores

        \end{itemize}



        \item Performance Requirements (Requisitos de Rendimiento) (subtipo de Eficiencia)
        \begin{itemize}

            \item Velocidad de procesamiento

            \item Throughput (rendimiento)

            \item Tiempos de respuesta específicos

        \end{itemize}



        \item Space Requirements (Requisitos de Espacio) (subtipo de Eficiencia)
        \begin{itemize}

            \item Limitaciones de almacenamiento

            \item Uso de memoria

            \item Capacidad de datos

        \end{itemize}


    \end{itemize}

    \subsubsection{2. Process Requirements (Requisitos del Proceso)}

    Relacionados con el desarrollo y despliegue:


    \begin{itemize}

        \item Delivery Requirements (Requisitos de Entrega)
        \begin{itemize}

            \item Plazos de desarrollo

            \item Metodología de entrega

            \item Hitos del proyecto

        \end{itemize}



        \item Implementation Requirements (Requisitos de Implementación)
        \begin{itemize}

            \item Lenguajes de programación específicos

            \item Herramientas de desarrollo

            \item Estándares de codificación

        \end{itemize}



        \item Standards Requirements (Requisitos de Estándares)
        \begin{itemize}

            \item Cumplimiento de normativas

            \item Estándares de la industria

            \item Protocolos específicos

        \end{itemize}


    \end{itemize}

    \subsubsection{3. External Requirements (Requisitos Externos)}

    Impuestos por factores externos al sistema:


    \begin{itemize}

        \item Interoperability Requirements (Requisitos de Interoperabilidad)
        \begin{itemize}

            \item Integración con otros sistemas

            \item Intercambio de datos

            \item APIs y protocolos de comunicación

        \end{itemize}



        \item Ethical Requirements (Requisitos Éticos)
        \begin{itemize}

            \item Consideraciones morales

            \item Impacto social

            \item Responsabilidad corporativa

        \end{itemize}



        \item Legislative Requirements (Requisitos Legislativos)
        \begin{itemize}

            \item Cumplimiento legal

            \item Regulaciones gubernamentales

            \item Normativas específicas del sector

        \end{itemize}



        \item Privacy Requirements (Requisitos de Privacidad) (subtipo de Legislativos)
        \begin{itemize}

            \item Protección de datos personales

            \item GDPR/LOPD

            \item Gestión de información sensible

        \end{itemize}



        \item Safety Requirements (Requisitos de Seguridad) (subtipo de Legislativos)
        \begin{itemize}

            \item Protección contra amenazas

            \item Seguridad de datos

            \item Control de acceso
        \end{itemize}

    \end{itemize}

    \subsection{Dificultades Comunes}\label{subsec:dificultades-comunes}

    \begin{itemize}

        \item Alcance: Fronteras mal definidas

        \item Entendimiento: Usuarios inseguros, información \textquote{obvia} omitida, requisitos ambiguos o en conflicto, \ldots

        \item Volatilidad: Requisitos cambian con el tiempo
    \end{itemize}

    \subsubsection{Impacto}
    Estas dificultades pueden llevar a:

    \begin{itemize}

        \item 15\% del tiempo en definir requisitos.

        \item Cambiar requisitos tras entrega cuesta de 60 a 100 veces más.

        \item 56\% de errores por especificación incorrecta.

        \item Solo 2\% del software entregado cumple completamente.

    \end{itemize}

    \subsection{Ingeniería de requisitos}\label{subsec:ingenieria-de-requisitos}
    Para desarrollar un software de calidad, es fundamental una buena ingeniería de requisitos.
    Para ello, se sigue un proceso iterativo que incluye:

    \begin{itemize}
        \item Concepción Necesidad inicial.
        Identificar la necesidad del software y su contexto.

        \item \textbf{Indagación} Preguntas organizadas sobre objetivos.
        ¿Qué, por qué y cómo?

        \item \textbf{Elaboración:} Modelo refinado, escenarios usuario, diagramas

        \item \textbf{Negociación:} Priorizar, resolver conflictos

        \item \textbf{Especificación:} Documentación formal

        \item \textbf{Validación:} Detectar ambigüedades, inconsistencias y verificar con usuarios

        \item \textbf{Administración:} Controlar cambios, versiones y trazabilidad
    \end{itemize}

    \subsection*{1. Indagación de requisitos}\label{subsec:indagacion-de-requisitos}
%    \etoctoccontentsline{subsection}{Indagación de requisitos}{2}
%    \etocimmediatetoccontentsline{subsection}{Indagación de requisitos}
    \addcontentsline{toc}{subsubsection}{1. Indagación de requisitos}

    La indagación de requisitos es una fase crítica donde se obtienen y refinan los requisitos del software.
    En esta fase distinguen dos actividades principales:

    \subsubsection*{1. Preparación}


    Antes de capturar los requisitos, es esencial una fase de preparación que garantice el éxito de la indagación.
    Esta comienza con la \textbf{identificación de los participantes}, es decir, todas las personas relevantes para el sistema: solicitantes, usuarios finales y otros perfiles implicados.


    A continuación, se realiza la \textbf{recepción y análisis de la concepción}, un documento preliminar que describe de forma general la necesidad o el problema.
    Esta revisión ayuda a entender el contexto, anticipar posibles temas clave y preparar la discusión.


    Una vez analizada la concepción, se elabora una \textbf{agenda de indagación} que guíe las reuniones y asegure que se aborden todos los puntos críticos:

    \begin{itemize}

        \item Quién es el solicitante.

        \item Qué perfil tienen los usuarios.

        \item Cuál es el problema a resolver.

        \item Cómo se definiría una solución exitosa.

        \item Qué otros participantes deben considerarse en el proceso.

    \end{itemize}

    \subsubsection*{2. Captura de requisitos}


    Durante la indagación, los requisitos pueden obtenerse mediante:

    \begin{itemize}

        \item Entrevistas estructuradas.

        \item Lluvia de ideas (brainstorming).

        \item Observación directa del entorno.

        \item Análisis de sistemas similares.

        \item Prototipos de baja fidelidad.

    \end{itemize}

    \subsubsection*{3. Tipos de requisitos según el cliente}


    \begin{itemize}

        \item \textbf{Normales:} Son los que el cliente solicita explícitamente.

        \item \textbf{Esperados:} No se mencionan, pero se asumen necesarios.

        \item \textbf{Emocionantes:} Superan las expectativas del cliente y generan satisfacción.

    \end{itemize}

    \subsection*{2. Elaboración}\label{subsec:elaboracion}
    \addcontentsline{toc}{subsubsection}{2. Elaboración}

    La elaboración es una fase donde se refinan los requisitos obtenidos en la indagación.

    \subsubsection*{Casos de uso}

    \begin{definicion}
        Un \textbf{caso de uso} describe una secuencia de acciones entre un actor y el sistema.
        Representa una unidad de funcionalidad y permite estimar, verificar y priorizar el trabajo.
        Cada caso puede dividirse en varios escenarios (principales o alternativos), y cada actor representa un conjunto de roles del usuario.
    \end{definicion}

    \paragraph{Características}

    \begin{itemize}

        \item Valorados por clientes

        \item Estimables (implementación independiente)

        \item Pequeños (fácil estimación)

        \item Verificables
    \end{itemize}



    \begin{exemplo}
        \textbf{Especificación de un caso de uso:}

        \begin{itemize}

            \item Nombre: Nombre del caso de uso

            \item Actor/es: Rol del usuario (o entidad(es)) que interactúa con el sistema

            \item Descripción: Breve resumen de la funcionalidad

            \item Precondiciones: Condiciones que deben cumplirse antes de iniciar el caso de uso, se asumen ciertas.

            \item Dependencias: Otros casos de uso o requisitos que deben cumplirse

            \item Escenario principal: Secuencia de pasos que describe la interacción normal entre el actor y el sistema

            \item Excepciones: Condiciones especiales o errores que pueden ocurrir durante la ejecución del caso de uso

            \item Prioridad (P0, P1, \ldots): Importancia del caso de uso

        \end{itemize}
    \end{exemplo}

    \subsubsection*{Historias de usuario}

    \begin{definicion}
        Descripción breve y sencilla, escrita en lenguaje natural y enfocada en
        las necesidades del usuario, que sirve como base para el desarrollo de
        una característica del software.
    \end{definicion}

    \begin{exemplo}

        \textit{\textquote{[rol], quiero [función] para [beneficio]}}
    \end{exemplo}

    Una historia específica podría ser:

    \begin{exemplo}

        \begin{quote}

            Como usuario de la aplicación de compras quiero ver el estado de mis pedidos para saber cuáles están en proceso de envío.

        \end{quote}

    \end{exemplo}

    No son requisitos ni sustituyen a los casos de uso, pero ayudan a centrar el desarrollo en el valor para el usuario y permiten definir criterios de aceptación.
    En la \autoref{tab:casos-de-uso-vs-historias-de-usuario} se muestra una comparación entre casos de uso y historias de usuario.

    \subsubsection*{Casos de uso vs Historias de usuario}


    \begin{itemize}

        \item \textbf{Casos de uso:} Más detallados, útiles para stakeholders.

        \item \textbf{Historias de usuario:} Más ágiles, enfocadas al equipo de desarrollo.

    \end{itemize}


    \begin{table}
        \centering
        \begin{tabular}{lll}
            \toprule
            \textbf{Aspecto}       & \textbf{Casos Uso}          & \textbf{Historias Usuario} \\
            \midrule
            Agilismo               & Indiferente                 & Sí                         \\
            Representan requisitos & Sí                          & No                         \\
            Describen              & Interacciones actor-sistema & Capacidades usuario        \\
            Útil para              & Stakeholders/Clientes       & Equipo desarrollo          \\
            \bottomrule
        \end{tabular}
        \caption{Comparación entre Casos de Uso y Historias de Usuario}
        \label{tab:casos-de-uso-vs-historias-de-usuario}
    \end{table}




    Ambos permiten priorización y ponen al usuario en el centro.

    \subsection*{3.Negociación y especificación}\label{subsec:3.negociacion-y-especificacion}
    \addcontentsline{toc}{subsubsection}{3. Negociación y especificación}


    Durante la negociación, se ajustan expectativas, prioridades y limitaciones.
    Se analiza el coste, el tiempo y el alcance del producto.
    De este proceso resultan dos documentos clave:


    \begin{itemize}

        \item \textbf{DRU (Documento de Requisitos del Usuario):} Qué problema resolver.

        \item \textbf{ERS (Especificación de Requisitos del Software):} Qué debe hacer el sistema para alcanzar los objetivos del DRU\@.

    \end{itemize}


    \subsection*{4. Validación}\label{subsec:4.-validacion}
    \addcontentsline{toc}{subsubsection}{4. Validación}


    Se realiza para asegurar:

    \begin{itemize}

        \item Coherencia entre requisitos y resultado esperado.

        \item Priorización adecuada.

        \item Ausencia de ambigüedades.

        \item Viabilidad y resolución de conflictos.

    \end{itemize}


    \textbf{Técnicas:} revisión por expertos, prototipado, validación de modelos, verificación de testabilidad.


    \subsection*{5. Gestión}
    \addcontentsline{toc}{subsubsection}{5. Gestión}
    \phantomsection


    La gestión consiste en supervisar los requisitos a lo largo de todo el ciclo de vida del producto.
    Incluye:


    \begin{itemize}

        \item Registro de cambios y asegurar la consistencia entre los requisitos y el sistema.

        \item Repriorización.

        \item Adición o eliminación de requisitos.

        \item Seguimiento del estado de cada uno.

    \end{itemize}

    \subsection{Modelado de requisitos}\label{subsec:modelado-de-requisitos}


    El modelado de requisitos tiene como objetivo representar, de forma gráfica o estructurada, los distintos aspectos del sistema desde el punto de vista de los usuarios, del dominio de información y del comportamiento del software.
    Permite comunicar mejor las necesidades del sistema, detectar inconsistencias y preparar el diseño.


    \textbf{Tipos de diagramas:}

    \begin{itemize}

        \item \textbf{Basados en escenarios:} describen interacciones usuario-sistema.
        Incluyen casos de uso y sus escenarios.

        \item \textbf{De datos:} muestran entidades, atributos y relaciones (diagramas E–R).

        \item \textbf{Orientados a clases:} presentan clases, atributos, métodos y relaciones (herencia, asociación\ldots).

        \item \textbf{De flujo:} representan transformación y circulación de datos.

        \item \textbf{De comportamiento:} representan estados del sistema y transiciones ante eventos.

    \end{itemize}


    \textbf{Lenguaje de modelado UML:}

    \begin{itemize}

        \item \textbf{Diagramas estructurales:} clases, objetos, componentes, despliegue.

        \item \textbf{Diagramas de comportamiento:} casos de uso, secuencia, estados, colaboración.

    \end{itemize}


    \textbf{Ventajas del modelado:}

    \begin{itemize}

        \item Mejora la comprensión y comunicación.

        \item Detecta errores e inconsistencias tempranas.

        \item Sirve como base para diseño y validación.

    \end{itemize}


    % --- **** ---- **** ---- **** ---- **** ---- **** ---- **** ---- **** ---- **** ---- **** ---- **** ---- **** ----

    \section{Diseño del software}\label{sec:diseno-del-software}
    % !TeX root = ../../main.tex
\subsection{Definición y propósito}\label{subsec:definicion-y-proposito}

\begin{definicion}
    El diseño de software consiste en crear un modelo que describa la arquitectura del sistema, los componentes que lo conforman, sus interfaces y las relaciones entre ellos.
    Se busca traducir los requisitos en una estructura organizada que guíe a la implementación.
\end{definicion}

\textbf{Propósitos principales:}
\begin{itemize}
    \item Facilitar la comunicación entre los participantes del proyecto.
    \item Servir de base para el análisis y la toma de decisiones.
    \item Identificar y gestionar las abstracciones clave del sistema.
    \item Promover la reutilización a gran escala (familias de productos).
\end{itemize}

\subsection{Importancia del diseño}\label{subsec:importancia-del-diseno}

Un buen diseño:
\begin{itemize}
    \item Permite traducir correctamente los requisitos en soluciones funcionales.
    \item Reduce el riesgo de errores, inestabilidad y dificultad de pruebas.
    \item Sienta las bases de la calidad del software.
\end{itemize}

\subsection{Atributos de calidad del software}\label{subsec:atributos-de-calidad-del-software}
\begin{itemize}
    \item \textbf{Funcionalidad:} capacidades del sistema, cobertura y seguridad.
    \item \textbf{Usabilidad:} facilidad de uso, estética, documentación.
    \item \textbf{Confiabilidad:} frecuencia y gravedad de fallos (MTBF, MTTR).
    \item \textbf{Rendimiento:} velocidad, tiempo de respuesta, uso de recursos.
    \item \textbf{Mantenibilidad:} facilidad para modificar, adaptar y probar.
\end{itemize}

\subsection{Conceptos clave del diseño}\label{subsec:conceptos-clave-del-diseno}

\begin{itemize}
    \item \textbf{Abstracción:} ocultar detalles para centrarse en lo esencial.
    \item \textbf{Arquitectura:} organización general del sistema y sus relaciones.
    \item \textbf{Patrones de diseño:} soluciones reutilizables a problemas comunes.
    \item \textbf{División de problemas:} estrategia de \textquote{divide y vencerás}.
    \item \textbf{Modularidad:} división del software en componentes independientes.
    \item \textbf{Ocultamiento de información:} cada módulo expone solo lo necesario.
    \item \textbf{Independencia funcional:} cada módulo resuelve tareas bien definidas.
    \item \textbf{Cohesión y acoplamiento:} coherencia interna y dependencia entre módulos.
    \item \textbf{Rediseño:} mejora estructural sin alterar funcionalidad.
\end{itemize}

\subsection{Clases de diseño}\label{subsec:clases-de-diseno}

\begin{itemize}
    \item \textbf{Interfaz de usuario:} elementos visibles e interactivos.
    \item \textbf{Dominio de negocio:} lógica y entidades principales.
    \item \textbf{Procesos:} gestión de tareas o flujos.
    Abstracciones de alto nivel que definen cómo se ejecutan las tareas.
    \item \textbf{Persistencia:} almacenamiento y recuperación de datos.
    \item \textbf{Sistema:} gestión de la infraestructura software.
\end{itemize}

\subsection{Elementos del modelo de diseño}\label{subsec:elementos-del-modelo-de-diseno}

\begin{itemize}
    \item \textbf{Diseño de datos:} organización de estructuras y bases de datos.
    \item \textbf{Diseño arquitectónico:} definición de subsistemas.
    \item \textbf{Diseño de interfaces:} entre componentes, usuarios y otros sistemas.
    \item \textbf{Diseño de componentes:} estructuras internas, algoritmos, interfaces.
    \item \textbf{Diseño de despliegue:} mapeo lógico-físico (servidores, contenedores).
\end{itemize}

\subsection{Diseño de arquitectura y patrones}\label{subsec:diseno-de-arquitectura-y-patrones}

\begin{definicion}
    \textbf{Arquitectura:}
    Estructura o estructuras del sistema, lo que comprende a los
    componentes del software, sus propiedades externas visibles y las
    relaciones entre ellos.
\end{definicion}

\subsubsection{MVC (Modelo-Vista-Controlador)}
\begin{minipage}{0.65\textwidth}
    \begin{itemize}
        \item \textbf{Modelo}: Datos y lógica negocio
        \item \textbf{Vista}: Interfaz usuario (presentación)
        \item \textbf{Controlador}: Mediador (recibe inputs, actualiza modelo)
        \item \textbf{Flujo}:
        \begin{enumerate}
            \item Usuario → Acción → Controlador
            \item Controlador → Actualiza Modelo
            \item Modelo → Notifica → Vista
            \item Vista → Renderiza → Usuario
        \end{enumerate}
        \item \textbf{Ventajas}: Separación preocupaciones, escalable
        \item \textbf{Uso}: Aplicaciones con UI compleja
    \end{itemize}
\end{minipage}
\hfill
\begin{minipage}{0.5\textwidth}
    \includegraphics[width=\linewidth]{imagenes/model-view-controller-light-blue}
\end{minipage}

\subsubsection{SOA (Arquitectura Orientada a Servicios)}
\begin{itemize}
    \item Componentes autónomos (\textbf{Servicios})
    \item Comunicación mediante APIs
    \item \textbf{Niveles}:
    \begin{itemize}
        \item Integración
        \item Lógica negocio
        \item Acceso datos
    \end{itemize}
    \item \textbf{Ventajas}: Bajo acoplamiento, escalabilidad horizontal
    \item \textbf{Desafío}: Complejidad integración
\end{itemize}
\hfill
\begin{minipage}{0.5\textwidth}
    \includegraphics[width=\linewidth]{imagenes/soa}
\end{minipage}

\subsubsection{EDA (Arquitectura Dirigida por Eventos)}
\begin{itemize}
    \item Componentes: \textbf{Productores} (generan eventos) y \textbf{Consumidores} (reaccionan)
    \item Comunicación asíncrona mediante eventos
    \item \textbf{Ventajas}: Escalabilidad, tiempo real
    \item \textbf{Uso}: Sistemas IoT, alta concurrencia
\end{itemize}
\begin{minipage}{0.5\textwidth}
    \includegraphics[width=\linewidth]{imagenes/eda}
\end{minipage}

\subsubsection{Otros Patrones}
\begin{itemize}
    \item \textbf{Layered}: Organización en capas (presentación, negocio, datos)
    \item \textbf{Microservicios}: SOA con servicios más pequeños y especializados
    \item \textbf{Workflow}: Orquestación de procesos secuenciales
\end{itemize}

\subsection{Diseño de interfaz de usuario}\label{subsec:diseno-de-interfaz-de-usuario}

\begin{definicion}
    El diseño de interfaz de usuario (UI) es el proceso de crear componentes visuales y funcionales que permiten a los usuarios interactuar con el software.
    Busca optimizar la experiencia del usuario (UX) y facilitar la usabilidad.
    Recoge los datos del subsistema de lógica de negocio
    para mostrarla.
    Reenvía las interacciones del usuario al subsistema de lógica de negocio para su procesado.
\end{definicion}

\textbf{Tipos:} GUI (gráfica), TUI (texto), VUI (voz).

\textbf{Aspectos clave:}
\begin{itemize}
    \item Usabilidad (eficiencia, satisfacción, accesibilidad).
    \item Personas: perfiles de usuario y escenarios de uso.
    \item Aspectos: claridad, tiempo de respuesta, ayuda, errores, internacionalización.
    \item Principios: familiaridad, uniformidad, recuperabilidad, diversidad.
\end{itemize}

\subsubsection{Principio de familiaridad en el diseño de interfaces}


El principio de familiaridad busca que el usuario se sienta cómodo desde el primer uso de la interfaz, aprovechando convenciones conocidas o comportamientos esperados.
Esto facilita la adopción del sistema, mejora la usabilidad y reduce la curva de aprendizaje.


Algunas técnicas comunes asociadas a este principio son:


\begin{itemize}

    \item \textbf{Lazy registration:} permite que los usuarios accedan parcialmente a la funcionalidad sin necesidad de autenticarse inicialmente.
    Solo cuando la acción lo requiere, se solicita el registro o inicio de sesión.

    \begin{itemize}

        \item \textbf{Propósito:} evitar fricción en el primer uso y fomentar la adopción.

    \end{itemize}
    \item \textbf{Breadcrumbs (migas de pan):} muestran la ruta de navegación desde la página principal hasta la ubicación actual.
    \begin{itemize}
        \item \textbf{Propósito:} permitir al usuario orientarse y volver atrás con facilidad.
    \end{itemize}

    \item \textbf{Hover control:} oculta información secundaria y solo la muestra cuando el usuario sitúa el cursor encima del elemento.
    \begin{itemize}
        \item \textbf{Propósito:} reducir la carga cognitiva y evitar distracciones.
    \end{itemize}

    \item \textbf{Search filters:} ofrece comandos o filtros de búsqueda predeterminados o sugerencias.
    \begin{itemize}
        \item \textbf{Propósito:} facilitar la exploración y ayudar al usuario a reconocer lo relevante.
    \end{itemize}

    \item \textbf{Carrito de la compra (e-commerce):} el carrito es accesible desde cualquier sección del sitio.
    \begin{itemize}
        \item \textbf{Propósito:} mejorar la experiencia de compra y aumentar la conversión.
    \end{itemize}
\end{itemize}

\subsubsection{Proceso de diseño de interfaz de usuario}

\begin{enumerate}
    \item Comprensión de actividades del usuario.
    \item Arquitectura de información.
    \item Prototipado de baja fidelidad.
    \item Pruebas de usabilidad.
    \item Diseño final (alta fidelidad, posible código).
\end{enumerate}

\subsection{Componentes}\label{subsec:componentes}

\begin{definicion}
    \textbf{Componentes:} unidades reutilizables del sistema que encapsulan funcionalidad.
    Cada componente tiene una interfaz bien definida y puede interactuar con otros componentes.
    Los componentes pueden ser independientes o formar parte de un subsistema mayor.
\end{definicion}

\subsubsection{Identificación de componentes:}


La correcta identificación de componentes es fundamental para lograr un diseño robusto, modular y mantenible.
Se deben tener en cuenta los siguientes criterios:


\begin{itemize}

    \item \textbf{Relación directa con los requisitos:} cada componente debe corresponderse con una funcionalidad significativa descrita en los requisitos del sistema.

    \item \textbf{Agrupación lógica:} las responsabilidades del componente deben estar relacionadas funcional o temáticamente.

    \item \textbf{Fronteras bien definidas:} cada componente debe tener una interfaz clara y estable que delimite lo que ofrece al resto del sistema.

    \item \textbf{Bajo acoplamiento:} los componentes deben minimizar sus dependencias mutuas, de modo que un cambio en uno no afecte a los demás.

    \item \textbf{Alta cohesión:} todas las funciones internas del componente deben estar fuertemente relacionadas y contribuir a un único propósito.

    \item \textbf{Reutilización potencial:} debe evaluarse si el componente puede ser útil en otros proyectos o contextos.

    \item \textbf{Autonomía en el despliegue:} en arquitecturas modernas (como microservicios), es deseable que cada componente pueda desplegarse y escalarse de forma independiente.

\end{itemize}


\textbf{Ejemplos de componentes:}

\begin{itemize}

    \item Módulo de autenticación y gestión de usuarios.

    \item Servicio de generación de informes PDF\@.

    \item API de gestión de pedidos.

    \item Componente de acceso a base de datos (DAO).

    \item Microservicio de pagos.

\end{itemize}




\subsection{Revisión técnica formal}\label{subsec:revision-tecnica-formal}

\textbf{Objetivo:} detectar problemas antes de la implementación.

\textbf{Participantes:} equipo de desarrollo.

\textbf{Evaluación de:} mantenibilidad, seguridad, disponibilidad, rendimiento, funcionalidad, usabilidad, coste.


    \chapter{Planificación de proyectos}\label{ch:planificacion-de-proyectos}
    \localtableofcontents
\section{Estimación}\label{sec:estimacion}
\subsection{Objetivo y proceso de la planificación}\label{subsec:objetivo-y-proceso-de-la-planificacion}

La planificación busca determinar los recursos necesarios para completar el proyecto en plazo y con calidad aceptable.
Los elementos principales son:

\begin{itemize}
    \item Estimación del esfuerzo y tiempo
    \item Asignación de tareas
    \item Identificar las dependencias
    \item Gestión de riesgos
\end{itemize}

Sus fases son:

\begin{enumerate}
    \item \textbf{Establecer ámbito:} ¿Qué se va a hacer?
    (casos de uso + requisitos no funcionales)
    \item \textbf{Determinar viabilidad:} ¿Es posible?
    (tecnología, finanzas, tiempo)
    \item \textbf{Analizar riesgos:} ¿Qué puede salir mal?
    \item \textbf{Definir recursos:} Personal, hardware, herramientas y componentes reutilizables
    \item \textbf{Estimar coste/esfuerzo:} Técnicas como COCOMO o puntos de historia
    \item \textbf{Desarrollar calendario:} Con tareas, hitos y dependencias
\end{enumerate}

\subsection{Ámbito vs Factibilidad}\label{subsec:ambito-vs-factibilidad}

El ámbito puede ser tanto una descripción escrita como un conjunto de casos de uso.
Se deben tener en cuenta los detalles de los casos de uso y considerar las restricciones.

Por otro lado, la factibilidad es hasta qué punto es realizable el proyecto.
Esta puede ser:

\begin{itemize}
    \item Tecnológica
    \item Financiera
    \item Temporal
    \item Material (recursos)
\end{itemize}

Los recursos se pueden dividir de la siguiente forma dentro del proyecto:

\begin{itemize}
    \item \textbf{Personal:}
    \begin{itemize}
        \item Número
        \item Habilidades
        \item Ubicación
    \end{itemize}
    \item \textbf{Entorno:}
    \begin{itemize}
        \item Herramientas software
        \item Hardware
        \item Recursos de red
    \end{itemize}
    \item \textbf{Software reutilizable:}
    \begin{itemize}
        \item \textbf{Componentes nuevos}
        \item \textbf{Componentes de experiencia parcial:} Las especificaciones, los diseños, código o pruebas existentes de proyectos anteriores podrán ser usados para el proyecto actual, pero requerirán una modificación sustancial y los miembros del equipo han limitado su experiencia sólo al área de aplicación representada por los componentes.
        Por eso, las modificaciones tendrán un mayor riesgo
        \item \textbf{Componentes con experiencia completa:} Ya existentes y desarrollados para proyectos anteriores similares al software que se va a construir para el proyecto actual.
        Los miembros del equipo ya tienen experiencia en el área.
        Bajo riesgo
        \item \textbf{Componentes COTS:} Diseñados para un uso inmediato, no requiere modificaciones.
        Sin riesgo
    \end{itemize}
\end{itemize}

\subsection{Técnicas de Estimación}\label{subsec:tecnicas-de-estimacion}

\subsubsection{Ley de Parkinson}

El mismo trabajo requiere más tiempo cuando hay plazos más largos.

\subsubsection{Precio Oportunista}

\begin{itemize}
    \item Basado en lo que el cliente está dispuesto a pagar
    \item Común en concursos públicos y licitaciones
    \item Riesgo: Puede no reflejar el esfuerzo real requerido
\end{itemize}

\subsection{Técnicas de Descomposición}\label{subsec:tecnicas-de-descomposicion}

\subsubsection{Tipos de Descomposición}

\begin{itemize}
    \item \textbf{Lógica difusa:}
    \begin{itemize}
        \item Identificar tipo de aplicación
        \item Establecer magnitud en escala cualitativa
        \item Refinar dentro del rango original
    \end{itemize}

    \item \textbf{Puntos de Función (PF):}
    \begin{itemize}
        \item Medir características del dominio de información
        \item Componentes: Entradas, Salidas, Consultas, Archivos, Interfaces
    \end{itemize}

    \item \textbf{Componente Estándar:}
    \begin{itemize}
        \item Contar ocurrencias de componentes estándar
        \item Usar datos históricos para estimar tamaño por componente
    \end{itemize}

    \item \textbf{Dimensionamiento del Cambio:}
    \begin{itemize}
        \item Para proyectos que modifican software existente
        \item Estimar número y tipo de modificaciones requeridas
    \end{itemize}
\end{itemize}

\subsubsection{Estimación Basada en Problema}

\begin{itemize}
    \item \textbf{Métricas clave:}
    \begin{itemize}
        \item Líneas de Código (LOC)
        \item Puntos de Función (PF)
    \end{itemize}

    \item \textbf{Fórmula de estimación ponderada:}
    \[
        S = \frac{S_{\text{opt}} + 4S_m + S_{\text{pes}}}{6}
    \]
    donde:
    \begin{itemize}
        \item $S_{\text{opt}}$ = Estimación optimista
        \item $S_m$ = Estimación más probable
        \item $S_{\text{pes}}$ = Estimación pesimista
    \end{itemize}

    \item \textbf{Ejemplo práctico:}
    \begin{center}
        \begin{tabular}{lcccc}
            \toprule
            Componente     & Optimista & Probable & Pesimista & Estimación         \\
            \midrule
            GUI            & 4600      & 6900     & 8600      & 6800               \\
            Servicio 1     & 2200      & 2750     & 3300      & 2750               \\
            Servicio 2     & 2500      & 3300     & 4400      & 3350               \\
            Servicio 3     & 1800      & 2250     & 2700      & 2250               \\
            \midrule
            \textbf{Total} &           &          &           & \textbf{15150 LOC} \\
            \bottomrule
        \end{tabular}
    \end{center}

    \item \textbf{Cálculo de coste:}
    \[
        \text{Esfuerzo} = \frac{\text{Total LOC}}{\text{Productividad}} = \frac{15150}{750} = 20.2\ \text{persona-meses}
    \]
\end{itemize}

\subsubsection{Estimación Basada en Proceso}

\begin{itemize}
    \item Descomposición en actividades del proceso de software
    \item Asignación de esfuerzo a cada actividad
\end{itemize}

\subsubsection{Modelos Empíricos}
\label{subsubsec:empiricos}

\begin{itemize}
    \item \textbf{Fórmula general:}
    \[
        E = A + B \cdot (e_v)^C
    \]
    \begin{itemize}
        \item $E$: Esfuerzo en personas-mes.
        \item $e_v$: Variable de estimación (LOC o PF).
        \item $A, B, C$: Constantes derivadas de datos históricos.
    \end{itemize}
    \item \textbf{Ventaja:} Predictibilidad mediante regresión sobre proyectos pasados.
\end{itemize}

\subsection{COCOMO II}
\label{subsec:cocomo}

\subsubsection{Conceptos Básicos}
\label{subsubsec:cocomo-basico}

\begin{itemize}
    \item \textbf{Objetivo:} Estimar esfuerzo ($E$) y tiempo ($D$) en función del tamaño (KLOC) para ($N$) personas
    \item \textbf{Ecuaciones:}
    \begin{align*}
        E &= a \cdot (\text{KLOC})^b \\
        D &= c \cdot (E)^d
    \end{align*}
\end{itemize}

\subsubsection{Tipos de Proyectos}
\label{subsubsec:tipos-proyectos}

\begin{center}
    \begin{tabular}{lcccc}
        \toprule
        \textbf{Tipo de Proyecto}                         & \textbf{a} & \textbf{b} & \textbf{c} & \textbf{d} \\
        \midrule
        \textbf{Orgánico} (pequeño, requisitos flexibles) & 2,4        & 1,05       & 2,5        & 0,38       \\
        \textbf{Semi-acoplado} (complejidad media)         & 3,0        & 1,12       & 2.5        & 0,35       \\
        \textbf{Empotrado} (requisitos rígidos)           & 3,6        & 1,20       & 2,5        & 0,32       \\
        \bottomrule
    \end{tabular}
\end{center}

\subsubsection{Ejemplo Práctico}
\label{subsubsec:ejemplo-cocomo}

\begin{itemize}
    \item \textbf{Tamaño total:} 15.150 LOC = 15,15 KLOC
    \item \textbf{Proyecto orgánico:}
    \begin{align*}
        E &= 2.4 \cdot (15.15)^{1.05} = 42\ \text{personas-mes} \\
        D &= 2.5 \cdot (42)^{0.38} = 11\ \text{meses} \\
        N &= \frac{E}{D} = \frac{42}{11} = 4\ \text{personas}
    \end{align*}
\end{itemize}

\subsubsection{Limitaciones}
\label{subsubsec:limitaciones-cocomo}

\begin{itemize}
    \item No considera reusabilidad en programación orientada a objetos
    \item Basado en muestras limitadas (no aplicable a todos los entornos)
    \item Ignora paralelización de tareas y factores de productividad
\end{itemize}

\subsection{Estimación Ágil}
\label{subsec:agil}

\subsubsection{Poker Planning}
\label{subsubsec:poker}

\begin{description}
    \item[\textbf{Paso 1:}] Seleccionar una historia de usuario
    \item[\textbf{Paso 2:}] Discusión breve del equipo
    \item[\textbf{Paso 3:}] Estimación individual con tarjetas (ejemplo: Fibonacci: 1, 2, 3, 5, 8)
    \item[\textbf{Paso 4:}] Revelar estimaciones simultáneamente
    \item[\textbf{Paso 5:}] Si hay discrepancia, debatir y repetir
\end{description}

\paragraph{Puntos vs. Horas}
\label{par:puntos-vs-horas}

\begin{center}
    \begin{tabularx}{\textwidth}{lXlX}
        \toprule
        & \textbf{Puntos} & & \textbf{Horas} \\
        \midrule
        \textbf{Ventajas} &
        \begin{itemize}
            \item Capturan complejidad, riesgo y esfuerzo
            \item Enfocados en valor (no tiempo)
        \end{itemize} &
        \textbf{Ventajas} &
        \begin{itemize}[leftmargin=*]
            \item Fácil medición del trabajo
            \item Cálculo directo de productividad
        \end{itemize} \\
        \midrule
        \textbf{Desventajas} &
        \begin{itemize}[leftmargin=*]
            \item Abstractos (requieren equipo consolidado)
        \end{itemize} &
        \textbf{Desventajas} &
        \begin{itemize}[leftmargin=*]
            \item Ignoran trabajo no relacionado (reuniones, etc.)
        \end{itemize} \\
        \bottomrule
    \end{tabularx}
\end{center}


\section{Calendarización}
\label{sec:calendarizacion}
\subsection{Relación con la Estimación}
\label{subsec:relacion}

\textbf{Estimación vs Calendarización:}

\begin{table}[h]
    \centering
    \begin{tabularx}{\textwidth}{|l|X|X|}
        \hline
        \textbf{Aspecto} & \textbf{Estimación}             & \textbf{Calendarización}      \\
        \hline
        Propósito        & Calcular esfuerzo/tiempo        & Fijar fechas y paralelización \\
        \hline
        Paralelismo      & No considera tareas simultáneas & Gestiona dependencias         \\
        \hline
        Entregas         & Sin fechas específicas          & Establece hitos y entregas    \\
        \hline
        Ejemplo:          & ``Requiere 40 personas-mes''    & ``Versión 1.0 para 15/Oct''   \\
        \hline
    \end{tabularx}
    \label{tab:comparacion-estimacion-calendarizacion}
\end{table}

\subsection{Dificultades Comunes}
\label{subsec:dificultades}

\begin{enumerate}
    \item Fechas límite \textbf{irreales} impuestas externamente
    \item \textbf{Cambios constantes} en requisitos del cliente
    \item \textbf{Subestimación} de esfuerzo/recursos
    \item \textbf{Riesgos no considerados} inicialmente
    \item Problemas técnicos \textbf{imprevistos}
    \item \textbf{Falta de comunicación} en el equipo
    \item Negación de \textbf{retrasos} y correcciones
\end{enumerate}

\subsection{Manejo de Fechas Límite}
\label{subsec:fechas}

\textbf{Estrategias efectivas:}

\begin{itemize}
    \item \textbf{Estimación detallada:} Usar datos históricos de proyectos similares
    \item \textbf{Desarrollo incremental:}
    \begin{itemize}
        \item Establecer qué funcionalidad corresponde a cada entrega
        \item Asegurar las funcionalidades mínimas básicas antes de cada entrega
    \end{itemize}
    \item \textbf{Comunicación proactiva:}
    \begin{enumerate}
        \item Demostrar con datos por qué la fecha es irreal
        \item Proponer alternativas concretas
        \item Negociar alcance vs.\ tiempo
    \end{enumerate}
    \item \textbf{Enfoque ágil:}
    \begin{itemize}
        \item \textbf{Sprints} cortos con entregables concretos
        \item \textbf{Priorización flexible} con el cliente
    \end{itemize}
\end{itemize}

\subsection{Hitos y Entregas}
\label{subsec:hitos}

\begin{itemize}
    \item \textbf{Hito válido:}
    \begin{itemize}
        \item Final de etapa lógica (ejemplo: \textquote{Prototipo móvil terminado})
        \item Medible y verificable
        \item Responsable asignado
    \end{itemize}
    \item \textbf{Anti-ejemplo:}
    \begin{itemize}
        \item \textquote{50\% del código funciona} (no verificable)
    \end{itemize}
    \item \textbf{Entrega:}
    \begin{itemize}
        \item Hito entregado al cliente
        \item Ejemplo: \textquote{Versión con registro de usuarios}
        \item Debe incluir criterios de aceptación claros
    \end{itemize}
\end{itemize}

\subsection{Dependencias entre Tareas}
\label{subsec:dependencias}
\deactivatequoting
\begin{center}
    \begin{tabularx}{\textwidth}{|l|l|X|c|}
        \hline
        \textbf{Tipo} & \textbf{Notación} & \textbf{Descripción}                                                             \\
        \hline
        Fin-Inicio    & FS                & B no empieza hasta que A termina (\textit{ejemplo: Pruebas después de desarrollo})    \\
        \hline
        Inicio-Inicio & SS                & B no empieza hasta que A empieza (\textit{ejemplo: Diseño UI y \textquote{back-end}}) \\
        \hline
        Fin-Fin       & FF                & B no termina hasta que A termina (\textit{ejemplo: Documentación y codificación})     \\
        \hline
        Inicio-Fin & SF & B no termina hasta que A empieza (\textit{rara en software})
        \\
        \hline
    \end{tabularx}
\end{center}
\activatequoting

\textbf{Ejemplos prácticos:}

\begin{itemize}
    \item \textbf{FS:} Pruebas no pueden comenzar hasta que desarrollo termine
    \item \textbf{SS:} Diseño UI y diseño DB pueden comenzar simultáneamente
    \item \textbf{FF:} Documentación no puede completarse hasta que desarrollo finalice
    \item \textbf{SF:} Mantenimiento no puede terminar hasta que despliegue comience
\end{itemize}

\subsection{Diagrama de Gantt}
\label{subsec:gantt}
Se muestra el diagrama de Gantt en la \autoref{fig:diagrama-gantt}.
\begin{itemize}
    \item \textbf{Elementos clave:}
    \begin{itemize}
        \item Tareas en eje vertical
        \item Tiempo en eje horizontal
        \item Barras horizontales = duración
    \end{itemize}
    \item \textbf{Ventajas:}
    \begin{itemize}
        \item Visualización clara del cronograma
        \item Identificación de solapamientos
        \item Seguimiento de progreso
    \end{itemize}
    \item \textbf{Ejemplo:}
    \begin{figure}[h]
        \includegraphics[width=0.75\textwidth]{imagenes/gantt_ejemplo}
        \caption{Diagrama de Gantt}
        \label{fig:diagrama-gantt}
    \end{figure}
\end{itemize}

\subsection{Diagrama de PERT}
\label{subsec:pert}

Proporciona una representación visual del cronograma de un proyecto y desglosa las tareas individuales.

\subsubsection{Conceptos fundamentales}

\begin{itemize}
    \item \textbf{Nodos:} Representan eventos (no actividades)
    \item \textbf{Flechas:} Actividades con duración
    \item \textbf{Camino crítico:} Ruta más larga (determina duración total)
\end{itemize}

\subsubsection{Elementos}

Este diagrama tiene los elementos de la \autoref{tab:tabladiapert}.

\begin{table}[htbp]
    \centering
    \begin{tabularx}{\textwidth}{|l|X|}
        \hline
        \textbf{Término}   & \textbf{Significado}                                                                          \\
        \hline
        D (Duración)       & Tiempo requerido para realizar la actividad                                                   \\
        \hline
        ES (Early Start)   & Primer momento posible de inicio                                                              \\
        \hline
        EF (Early Finish)  & Primer momento posible de fin                                                                 \\
        \hline
        LS (Late Start)    & Último momento posible de inicio                                                              \\
        \hline
        LF (Late Finish)   & Último momento posible de fin                                                                 \\
        \hline
        TF (Holgura Total) & Máximo retraso sin afectar proyecto                                                           \\
        \hline
        FF (Holgura Libre) & Máximo retraso sin afectar al \textquote{inicio temprano} de \textquote{sucesoras} inmediatas \\
        \hline
        Camino Crítico     & Secuencia de tareas más larga.
        Determinará la duración mínima del proyecto                    \\
        \hline
    \end{tabularx}
    \caption{Elementos del diagrama PERT}
    \label{tab:tabladiapert}
\end{table}

\subsubsection{Beneficios}

\begin{itemize}
    \item Identifica cuellos de botella
    \item Calcula probabilidades de cumplimiento
    \item Gestiona riesgos de tiempo
\end{itemize}


\section{Administración del Riesgo}
\label{sec:riesgos}
\begin{itemize}
    \item \textbf{RAE:} \textquote{Contingencia o proximidad de un daño}
    \item \textbf{Ingeniería del software:}
    \begin{itemize}
        \item Posibilidad de eventos imprevistos con impacto negativo en el proyecto
        \item Implica previsión del futuro, cambio y toma de decisiones
    \end{itemize}

    \item \textbf{Elementos clave:}
    \begin{itemize}
        \item \textbf{Fuente:} Origen del riesgo (ejemplo: nueva tecnología)
        \item \textbf{Evento:} Qué podría ocurrir (ejemplo: incompatibilidad)
        \item \textbf{Impacto:} Consecuencias (ejemplo: retraso 2 semanas)
    \end{itemize}
\end{itemize}

\subsection{Clasificación de Riesgos}
\label{subsec:clasificacion}

\begin{center}
    \begin{tabular}{p{3.5cm}p{9cm}}
        \toprule
        \textbf{Tipo de Riesgo} & \textbf{Descripción y Ejemplos} \\
        \midrule
        \textbf{Proyecto} &
        \begin{itemize}[leftmargin=*]
            \item Calendario o recursos insuficientes
            \item Ejemplo: Desarrollo por detrás de las metas acordadas con el cliente
        \end{itemize} \\
        \midrule
        \textbf{Técnico} &
        \begin{itemize}[leftmargin=*]
            \item Amenaza calidad o planificación
            \item ejemplo: Rendimiento inadecuado bajo carga máxima
        \end{itemize} \\
        \midrule
        \textbf{Producto} &
        \begin{itemize}[leftmargin=*]
            \item Incumplimiento expectativas usuario
            \item ejemplo: Interfaz poco intuitiva para usuarios finales
        \end{itemize} \\
        \midrule
        \textbf{Negocio} &
        \begin{itemize}[leftmargin=*]
            \item Problemas organizativos o de mercado
            \item ejemplo: Cambio en estrategia corporativa que reduce la prioridad del proyecto
        \end{itemize} \\
        \bottomrule
    \end{tabular}
\end{center}

\begin{center}
    \begin{tabular}{p{3.5cm}p{9cm}}
        \toprule
        \textbf{Tipo de Riesgo} & \textbf{Descripción y Ejemplos} \\
        \midrule
        \textbf{Tecnologías} &
        \begin{itemize}[leftmargin=*]
            \item Librerías obsoletas sin soporte
            \item Dependencias con licencias restrictivas
        \end{itemize} \\
        \midrule
        \textbf{Personal} &
        \begin{itemize}[leftmargin=*]
            \item Baja capacitación en tecnología clave
            \item Conflictos interpersonales en equipo
        \end{itemize} \\
        \midrule
        \textbf{Organizativos} &
        \begin{itemize}[leftmargin=*]
            \item Cambios frecuentes en liderazgo
            \item Presupuesto inestable
        \end{itemize} \\
        \midrule
        \textbf{Herramientas} &
        \begin{itemize}[leftmargin=*]
            \item Incompatibilidad entre entornos
            \item Bugs en software de testing
        \end{itemize} \\
        \midrule
        \textbf{Requisitos} &
        \begin{itemize}[leftmargin=*]
            \item Alcance no definido (scope creep)
            \item Requerimientos contradictorios
        \end{itemize} \\
        \midrule
        \textbf{Estimación} &
        \begin{itemize}[leftmargin=*]
            \item Subestimación de complejidad
            \item Falta de datos históricos
        \end{itemize} \\
        \bottomrule
    \end{tabular}
\end{center}

\subsection{Descripción de Riesgos}\label{subsec:descripcion-de-riesgos}

Para cada riesgo identificado, se recomienda recopilar la siguiente información:

\begin{itemize}
    \item \textbf{Descripción:} Cuál es el riesgo y cuál es su origen.

    \item \textbf{Prioridad (impacto):} \textit{alta} | \textit{media} | \textit{baja}

    \item \textbf{Probabilidad:} \textit{alta} | \textit{media} | \textit{baja}

    \item \textbf{Plan de acción:}
    \begin{itemize}
        \item Tipo: \textit{contingencia}, \textit{minimización} o \textit{prevención}
        \item Estrategia: \textit{soportar}, \textit{reducir el impacto}, \textit{evitar}
        \item Acciones planificadas: descripción específica de los pasos a seguir.
    \end{itemize}
    \item \textbf{Responsable:} Persona a cargo de la implementación del plan
    \item \textbf{Estado:} \textit{abierto} | \textit{cerrado}
\end{itemize}

\begin{exemplo}
    \textbf{Descripción:} Posible ausencia prolongada de una persona clave del equipo.

    \textbf{Impacto:} Alto

    \textbf{Probabilidad:} Promedio

    \textbf{Plan de contingencia:}
    \begin{itemize}
        \item Tipo: Contingencia
        \item Estrategia: Reducir el impacto
        \item Acciones: Documentar el trabajo clave y la capacitación cruzada dentro del equipo
    \end{itemize}

    \textbf{Responsable:} Coordinador del proyecto

    \textbf{Estado:} Abierto
\end{exemplo}

\subsection{Estrategias de Gestión de Riesgos}\label{subsec:estrategias-de-gestion-de-riesgos}

La gestión de riesgos puede adoptar diferentes enfoques según el momento oportuno para actuar sobre el riesgo:

\begin{itemize}
    \item \textbf{Estrategias reactivas:} Se actúa tras la materialización del riesgo, evaluando sus consecuencias y tomando medidas correctivas

    \item \textbf{Estrategias proactivas:} El riesgo se considera un posible evento futuro.
    Se establece un plan de contingencia para evitarlo o, en caso de ocurrir, minimizar su impacto
\end{itemize}

Un análisis temprano, sistemático y profundo de los riesgos favorece una estrategia proactiva global y reduce la ocurrencia de problemas imprevistos graves.

\medskip

\textbf{La gestión de riesgos} es el proceso continuo que permite:

\begin{itemize}
    \item Identificar riesgos potenciales
    \item Analizar su naturaleza e impacto
    \item Proponer soluciones antes de que se conviertan en problemas reales
\end{itemize}

Este proceso debe aplicarse continuamente durante todo el ciclo de vida del proyecto.

\begin{nota}
    La integración temprana de la gestión de riesgos en la planificación permite anticipar fallos, tomar decisiones más informadas y reducir costos inesperados.
\end{nota}

\subsection{Proceso de Gestión de Riesgos}
\label{subsec:proceso}
\deactivatequoting
\begin{center}
    \begin{tikzpicture}[node distance=1.5cm, auto]
        \tikzstyle{block} = [rectangle, draw, text width=6em, text centered, rounded corners, minimum height=4em]
        \tikzstyle{line} = [draw, -latex']

        \node [block] (identificar) {Identificación};
        \node [block, right of=identificar, node distance=4cm] (analizar) {Análisis};
        \node [block, right of=analizar, node distance=4cm] (planificar) {Planificación};
        \node [block, below of=analizar, node distance=3cm] (monitorear) {Seguimiento};

        \path [line] (identificar) -- (analizar);
        \path [line] (analizar) -- (planificar);
        \path [line] (planificar) |- (monitorear);
        \path [line] (monitorear) -| (identificar);
        \path [line] (monitorear) -- (analizar);
    \end{tikzpicture}
\end{center}
\activatequoting

\subsection{Actividades por fase}\label{subsec:actividades-por-fase}
\begin{itemize}
    \item \textbf{Identificación}: Listar riesgos potenciales
    \item \textbf{Análisis}: Calcular probabilidad/impacto, priorizar
    \item \textbf{Planificación}: Definir estrategias y planes de acción
    \item \textbf{Seguimiento}:
    \begin{itemize}
        \item Revisión mensual de riesgos activos
        \item Actualizar matriz de riesgos
        \item Activar planes de contingencia si es necesario
    \end{itemize}
\end{itemize}

\clearpage




    \chapter{Calidad}\label{ch:calidad}
    \localtableofcontents
\begin{definicion}
    La calidad en software se define como un proceso eficaz que, al ser bien aplicado, crea un producto útil con valor mesurable tanto para los productores como para los usuarios.
\end{definicion}

La ingeniería de software tiene como objetivo garantizar esa calidad a lo largo de todo el ciclo de vida:

\begin{itemize}
    \item Planificación
    \item Diseño
    \item Desarrollo
    \item Pruebas
    \item Despliegue
    \item Mantenimiento
\end{itemize}


\section{Aseguramiento de la Calidad}\label{sec:aseguramiento-de-la-calidad}
\begin{definicion}
    Según la norma ISO/IEC 9126, la calidad de un producto software se puede evaluar mediante seis atributos principales:
\end{definicion}

\begin{enumerate}
    \item \textbf{Funcionalidad:} Grado en que el software cumple los requisitos funcionales esperados.
    \item \textbf{Confiabilidad:} Estabilidad del software ante fallos, por ejemplo, el tiempo medio de funcionamiento antes de un fallo.
    \item \textbf{Usabilidad:} Facilidad de uso para los usuarios.
    \item \textbf{Eficiencia:} Uso óptimo de los recursos disponibles (CPU, memoria, etc.).
    \item \textbf{Mantenibilidad:} Facilidad para modificar, corregir o mejorar el software.
    \item \textbf{Portabilidad:} Facilidad para trasladar el software entre diferentes entornos.
\end{enumerate}

\begin{nota}
    Existe un dilema clásico entre producir software rápido y barato o producir software de alta calidad, ya que la calidad requiere tiempo y recursos.
\end{nota}

\begin{nota}
    Los errores pequeños no detectados a tiempo pueden amplificarse y causar problemas mayores en fases posteriores, por lo que la detección temprana es fundamental.
\end{nota}

\subsection{Principios rectores de la calidad}\label{subsec:principios-rectores-de-la-calidad}
\begin{itemize}
    \item \textbf{Formulación.} La derivación de medidas y métricas de software apropiadas
    \item para la representación del software que se está construyendo.
    \item \textbf{Recolección. }Mecanismo que se usa para acumular datos requeridos para derivar las métricas formuladas.
    \item \textbf{Análisis.} El cálculo de métricas y la aplicación de herramientas
    \item matemáticas.
    \item \textbf{Interpretación.} Evaluación de las métricas resultantes para comprender la calidad de la representación.
    \item \textbf{Retroalimentación.} Recomendaciones derivadas de la interpretación de las métricas del producto, transmitidas al equipo de software.
\end{itemize}

\subsection{Atributos de las métricas}\label{subsec:atributos-de-las-metricas}
\begin{enumerate}
    \item \textbf{Medible. }Debe ser simple poder recolectar los datos que componen la métrica y realizar su cálculo.
    \item \textbf{Intuitiva.} Los usuarios de la métrica deben poder identificar su significado y su valor.
    \item \textbf{Objetiva.} Siempre debe producir resultados que no tengan ambigüedades.
    \item \textbf{Coherente.} El cálculo matemático de la métrica debe usar medidas que no conduzcan a combinaciones extrañas de unidades.
    \item \textbf{Tecnológicamente agnóstica.} Debe basarse en el modelo de requerimientos, el modelo de diseño o la estructura del programa en sí.
    \item \textbf{Accionable.} Debe proporcionar información que pueda conducir a un producto final de mayor calidad.
\end{enumerate}

\subsection{Control vs. Aseguramiento de la Calidad}\label{subsec:control-vs.-aseguramiento-de-la-calidad}

\begin{center}
    \begin{tabularx}{\textwidth}{|X|X|}
        \hline
        \textbf{Control de Calidad (QC)}                         & \textbf{Aseguramiento de Calidad (QA)}               \\
        \hline
        Reactivo                                                 & Proactivo                                            \\
        Detección y corrección de errores después de que ocurren & Prevención de errores mediante estándares y procesos \\
        Inspección de productos                                  & Mejora continua de procesos                          \\
        \hline
    \end{tabularx}
\end{center}

\subsection{Revisiones técnicas}\label{subsec:revisiones-tecnicas}

\begin{itemize}
    \item \textbf{Informales:} Conversaciones espontáneas, revisiones en escritorio; baja eficacia que mejora con listas de verificación.
    \item \textbf{Formales:} Reuniones estructuradas y preparadas; alta eficacia, más tiene un alto coste en tiempo y esfuerzo; se utiliza normalmente una muestra representativa.
\end{itemize}

\subsection{Revisiones durante el desarrollo}\label{subsec:revisiones-durante-el-desarrollo}

\begin{itemize}
    \item \textbf{Revisión de código:} Entre compañeros (pair review) para detectar errores y mejorar el aprendizaje del equipo.
    \item \textbf{Análisis estático de código:} Herramientas automáticas que detectan errores potenciales, complejidad y duplicaciones.
\end{itemize}

\subsection{Métricas de calidad (DORA Metrics)}\label{subsec:metricas-de-calidad-(dora-metrics)}

\begin{center}
    \begin{tabular}{|l|l|}
        \hline
        \textbf{Métrica}                  & \textbf{Significado}                                 \\
        \hline
        MTTR (Mean Time to Recover)       & Tiempo medio para recuperar un sistema tras un fallo \\
        MTBF (Mean Time Between Failures) & Tiempo medio entre fallos                            \\
        Disponibilidad                    & Proporción de tiempo que el sistema está disponible  \\
        \hline
    \end{tabular}
\end{center}

\begin{definicion}
    La disponibilidad se calcula con la fórmula:
    \[
        \text{Disponibilidad} = \frac{\text{MTBF}}{\text{MTBF} + \text{MTTR}} \cdot 100\%
    \]
\end{definicion}

\subsection{Buenas prácticas de desarrollo}\label{subsec:buenas-practicas-de-desarrollo}

\begin{itemize}
    \item \textbf{Clean Code} (Robert C. Martin - Uncle Bob):
    \begin{itemize}
        \item KISS: “Keep It Simple, Stupid”
        \item DRY: “Don’t Repeat Yourself”
        \item YAGNI: “You Aren’t Gonna Need It”
        \item SoC: “Separation of Concerns”
    \end{itemize}
    \item \textbf{Documentación:} Comentarios explicativos (no descriptivos), explicaciones en los commits y ejemplos de uso en tests.
    \item \textbf{Control de versiones:} Uso de herramientas como Git para seguir cambios y facilitar la colaboración.
\end{itemize}

\subsection{Métricas de desarrollo}\label{subsec:metricas-de-desarrollo}

\begin{center}
    \begin{tabular}{|l|l|}
        \hline
        \textbf{Métrica}                             & \textbf{Descripción}                       \\
        \hline
        Densidad de comentarios                      & \% de comentarios respecto al código total \\
        Duplicidad de código                         & Código repetido                            \\
        Cobertura de pruebas                         & \% de código ejecutado durante pruebas     \\
        Complejidad ciclomática                      & Mide rutas lógicas (condiciones y bucles)  \\
        IMS (Índice de Madurez del código(Software)) & Evalúa la estabilidad de una release       \\
        \hline
    \end{tabular}
\end{center}

\begin{definicion}
    La fórmula para calcular el IMS es:
    \[
        IMS = \frac{M_T - (F_a + F_c + F_d)}{M_T}
    \]
    Donde:
    \begin{itemize}
        \item $M_T$: Número total de pruebas planificadas.
        \item $F_a$: Fallos críticos encontrados.
        \item $F_c$: Fallos menores encontrados.
        \item $F_d$: Fallos detectados en desarrollo.
    \end{itemize}
\end{definicion}

\subsection{Deuda técnica}\label{subsec:deuda-tecnica}

\begin{definicion}
    Según Martin Fowler, la deuda técnica es una metáfora financiera: tomar atajos en el diseño genera un \textquote{interés} que se paga con mayor esfuerzo futuro.
    Se puede:
    \begin{itemize}
        \item Seguir pagando intereses (mantener mal diseño).
        \item Pagar el principal (refactorizar y mejorar).
    \end{itemize}
\end{definicion}

Se clasifica según dos ejes:

\begin{center}
    \begin{tabular}{|c|c|c|}
        \hline
        & \textbf{Temeraria}                              & \textbf{Prudente}                                    \\
        \hline
        \textbf{Deliberada}  & \textquote{No tenemos tiempo, entregamos ahora} & \textquote{Lo haremos rápido y mejoraremos después}  \\
        \hline
        \textbf{Inadvertida} & \textquote{¿Qué componentes tiene esto?}        & \textquote{Ahora sabemos cómo debería haberse hecho} \\
        \hline
    \end{tabular}
\end{center}


\section{Estrategias de Prueba}\label{sec:estrategias-de-prueba}

\subsection{Verificación vs. Validación}\label{subsec:verificacion-vs.-validacion}

\begin{center}
    %! suppress = LineBreak
    \begin{tabular}{|l|l|}
        \hline
        \textbf{Verificación}                     & \textbf{Validación}                         \\
        \hline
        ¿Construimos \textbf{bien} el producto?   & ¿Construimos el \textbf{producto correcto}? \\
        Garantía de implementación correcta       & Cumplimiento de requisitos del cliente      \\
        Incluye pruebas, revisiones, simulaciones & Incluye pruebas de aceptación y prototipos  \\
        \hline
    \end{tabular}
\end{center}

\subsection{Malas prácticas}\label{subsec:malas-practicas}

\begin{itemize}
    \item Suponer que hay partes que no se pueden probar.
    \item Que el desarrollador no haga pruebas.
    \item Aislar al equipo de pruebas.
    \item Involucrar a los testers solo al final.
\end{itemize}

\subsection{Estrategias de prueba (pirámide de pruebas)}\label{subsec:estrategias-de-prueba-(piramide-de-pruebas)}

Martin Fowler propone priorizar así:

\begin{enumerate}
    \item Pruebas unitarias (muchas, automáticas).
    \item Pruebas de integración.
    \item Pruebas de interfaz/aceptación (pocas, más caras de automatizar).
\end{enumerate}

\subsection{Dimensiones de la prueba}\label{subsec:dimensiones-de-la-prueba}

\begin{center}
    \begin{tabularx}{\textwidth}{|l|l|X|}
        \hline
        \textbf{Dimensión} & \textbf{Subdimensión}                   & \textbf{Descripción}                                    \\
        \hline
        Tipo               & Funcional / No funcional                & Comprobación de funciones / parámetros como rendimiento \\
        Granularidad       & Unitaria / Integración / Validación     & Nivel del sistema probado                               \\
        Alcance            & Progresión / Regresión / Smoke / Sanity & Cobertura funcional                                     \\
        Ejecución          & Manual / Asistida / Automática          & Nivel de automatización                                 \\
        Metodología        & Guiada / Exploratoria                   & Nivel de formalización                                  \\
        \hline
    \end{tabularx}
\end{center}

\subsection{Pruebas específicas}\label{subsec:pruebas-especificas}

\begin{itemize}
    \item \textbf{Unitarias:} \textbf{Caja blanca}.
    Son las pruebas más simples y frecuentes (sobre código individual).
    Uso de \emph{stubs} para módulos dependientes.
    Ejemplos: métodos, condiciones de frontera, errores.
    \item \textbf{Integración:} \textbf{Caja gris}.
    Pruebas de cómo interactúan los componentes.
    Importante la integración incremental (mayor control de errores).
    Puede ser ascendente o descendente.
    \item \textbf{Validación (aceptación):} \textbf{Caja negra}.
    Realistas y orientadas al usuario final.
    Difíciles de automatizar.
    Usadas en fases alfa y beta.
    \item \textbf{No funcionales:} Rendimiento, seguridad, recuperación, esfuerzo, despliegue, etc.
\end{itemize}


\section{Administración de la Configuración}\label{sec:administracion-de-la-configuracion}
\begin{definicion}
    Conjunto de actividades para gestionar los cambios durante el ciclo de vida del software, garantizando:
    \begin{itemize}
        \item Trazabilidad
        \item Control de versiones
        \item Información actualizada
    \end{itemize}

    Se diferencia del mantenimiento en que este aplica cambios, y la administración de configuración controla y registra dichos cambios.
\end{definicion}

\subsection*{Conceptos clave}
\begin{itemize}
    \item \textbf{Elemento de configuración (EC / ICS):} Unidad identificable que debe ser controlada (código, documentación, etc.).
    \item \textbf{Base de datos de configuración (BCD):} Lugar donde se almacenan los EC y sus versiones.
    \item \textbf{Línea base:} Versión estable y acordada de un conjunto de EC\@.
    \item \textbf{Cambio:} Modificación propuesta a un EC\@.
    \item \textbf{Control de cambios:} Proceso de evaluar y aprobar cambios.
\end{itemize}

\subsection{Ítems de Configuración (IC)}\label{subsec:items-de-configuracion-(ic)}
\begin{definicion}
    Cada uno de los elementos que comprenden toda la información
    producida como parte del proceso de software.
\end{definicion}
\begin{itemize}
    \item Programas de cómputo
    \item Productos de trabajo
    \item Contenido
\end{itemize}

\subsection{Sistema de Administración}\label{subsec:sistema-de-administracion}
\begin{itemize}
    \item \textbf{Elementos componentes:} Herramientas que permiten el acceso y gestión de cada ítem de configuración del software.
    \item \textbf{Elementos de proceso:} Acciones y tareas necesarias para realizar una gestión efectiva del cambio.
    \item \textbf{Elementos de construcción:} Herramientas que automatizan la compilación, empaquetado y generación de versiones correctas del software.
    \item \textbf{Elementos humanos:} Procesos y herramientas que utiliza el equipo para implementar de manera efectiva la administración de configuración.
\end{itemize}

\subsection{Repositorio ACS}\label{subsec:repositorio-acs}
\begin{definicion}
    Almacén centralizado o distribuido donde se gestionan los ítems de configuración.
\end{definicion}

\paragraph{Características clave:}
\begin{itemize}
    \item Control de versiones.
    \item Rastreo de dependencias entre componentes.
    \item Trazabilidad de requisitos y cambios.
    \item Apoyo a auditorías e inspecciones de calidad.
\end{itemize}

\subsection{Proceso de Administración de la Configuración}\label{subsec:proceso-de-administracion-de-la-configuracion}
\begin{enumerate}
    \item \textbf{Identificación:} Asignar nombres unívocos a los ítems de configuración.
    \item \textbf{Control de cambios:} Registrar, aprobar e implementar modificaciones.
    \item \textbf{Control de versiones:} Gestionar diferentes versiones de todos los objetos.
    \item \textbf{Auditoría:} Verificar que los cambios cumplen con los estándares establecidos.
    \item \textbf{Reporte de estado:} Informar del estado actual y del histórico de cambios.
\end{enumerate}

\subsection{Administración del contenido}\label{subsec:administracion-del-contenido}
\begin{itemize}
    \item \textbf{Subsistema de recopilación:} Permite almacenar y organizar el contenido asociado a cada Ítem de Configuración del Software (ICS).
    \item \textbf{Subsistema de administración:} Gestiona los cambios realizados sobre el contenido de cada ICS, manteniendo un histórico completo y trazable.
    \item \textbf{Subsistema de publicación:} Hace accesible el contenido relevante a los distintos interesados del proyecto, permitiendo su consulta o reutilización.
\end{itemize}

\subsection{Conflictos dominantes}\label{subsec:conflictos-dominantes}
\begin{itemize}
    \item \textbf{Contenido:} No está claro qué información debe formar parte del sistema de administración de configuración.
    \item \textbf{Personas:} El equipo desconoce las herramientas o procesos de ACS, o no sabe cuándo aplicarlos.
    \item \textbf{Escalabilidad:} A medida que crece el proyecto, se complica la gestión del gran volumen de cambios.
    \item \textbf{Políticas:} Faltan reglas claras sobre quién es responsable de llevar a cabo la administración de configuración y cómo aplicarla de manera coherente.
\end{itemize}





    \chapter{Mantenimiento}\label{ch:mantenimiento}
    \localtableofcontents

\section{Métricas de producto}\label{sec:tema-6.1---metricas-de-producto}
    \begin{definicion}
        \textbf{Métrica:} Medida cuantitativa del grado en que un sistema, componente o proceso posee un atributo determinado (IEEE Standard).
    \end{definicion}

    \subsection{Conceptos clave}\label{subsec:conceptos-clave}
    \begin{itemize}
        \item \textbf{Medida:} Un único dato.
        \item \textbf{Medición:} Recolección de varias medidas.
        \item \textbf{Indicador:} Combinación de métricas que proporciona comprensión sobre procesos, proyectos o productos.
    \end{itemize}

    \subsection{Principios de medición}\label{subsec:principios-de-medicion}
    \begin{enumerate}
        \item \textbf{Formulación:} Derivación de medidas apropiadas
        \item \textbf{Recolección:} Mecanismo para acumular datos
        \item \textbf{Análisis:} Cálculo de métricas con herramientas matemáticas
        \item \textbf{Interpretación:} Evaluación de resultados
        \item \textbf{Retroalimentación:} Recomendaciones al equipo
    \end{enumerate}

    \subsection{Atributos de métricas}\label{subsec:atributos-de-metricas}
    \begin{itemize}
        \item Medible
        \item Intuitiva
        \item Objetiva
        \item Coherente \item
        Tecnológicamente agnóstica
        \item Accionable
    \end{itemize}

    \subsection{Métricas específicas}\label{subsec:metricas-especificas}

    \begin{definicion}

        \textit{Los puntos de definición son una forma de medir el tamaño, la complejidad y la calidad del software}.

    \end{definicion}
    \begin{itemize}
        \item \textbf{Modelo de requisitos:}
        \[PF = conteo \cdot (0.65 + (0.01 \cdot \text{FAV}))\]
        \begin{itemize}
            \item Componentes:
            \begin{description}
                \item[Entradas externas (EE)]: Información que se origina de un usuario o se transmite desde otra aplicación.
                Se usan para actualizar archivos lógicos internos (ALI). Las entradas deben distinguirse de las consultas.
                \item[Salidas externas (SE)]: Datos derivados dentro de la aplicación que ofrecen información al usuario.
                \item [Consultas externas (CE)]: Entrada que da como resultado la generación de alguna respuesta (con frecuencia recuperada de un ALI).
                \item[Número de archivos lógicos internos (ALI)]:Agrupamiento lógico de datos que reside dentro de la frontera de la aplicación y se mantiene mediante entradas externas
                \item[Número de archivos de interfaz externos (AIE)]:
                Agrupamiento lógico de datos que reside fuera de la aplicación, pero que proporciona información que puede usar la aplicación.
                \item [Factor de ajuste de valor (FAV)]: Indica la complejidad del sistema en su conjunto en base a unas preguntas a las que se asigna un valor entre 0 (irrelevante) y 5 (esencial).
            \end{description}
        \end{itemize}

        \item \textbf{Diseño arquitectónico:}
        \begin{itemize}
            \item Complejidad de módulo: $S(i) = f^2_{out}(i)$, $D(i) = \frac{v(i)}{f_{out}(i) + 1}$ (Estructural (S) y Datos (D))
            \item Complejidad de sistema: $C(i) = S(i) + D(i)$
        \end{itemize}
        Donde:
        \item $f_{out}(i)$ es el número de módulos que dependen del módulo $i$.
        \item $v(i)$ es el número de módulos de los que depende el módulo $i$.
        \item $\text{Tamaño} = n + a$; $n$ es el número de nodos y $a$ es el número de arcos.
        \item Profundidad: trayectoria más larga desde el nodo raíz hasta un nodo hoja
        \item Ancho: número máximo de nodos en cualquier nivel de la arquitectura

        \item \textbf{Orientadas a clase (CK):}
        \begin{table}[!ht]
            \centering
            \caption{Métricas de la Suite CK}
            \label{tab:ck_metrics}
            \begin{tabularx}{\linewidth}{lX}
                \toprule
                \textbf{Acrónimo} & \textbf{Descripción}                                                                                                                                                          \\
                \midrule
                MPC               & Métodos Ponderados por Clase: Suma de las complejidades nominales de todos los métodos de una clase ($MPC = \sum c_i$), donde $c_i$ es la complejidad nominal de cada método. \\
                \addlinespace[0.3cm]
                PAH               & Profundidad del Árbol de Herencia: Longitud máxima desde el nodo de la clase actual hasta la raíz del árbol de herencia.                                                      \\
                \addlinespace[0.3cm]
                NDH               & Número de Hijos Directos: Cantidad de subclases que heredan directamente de la clase actual (subclases inmediatas).                                                           \\
                \addlinespace[0.3cm]
                FCOM              & Falta de Cohesión en Métodos: Número de pares de métodos que acceden a uno o más atributos de clase en común.                                                                 \\
                \bottomrule
            \end{tabularx}
        \end{table}

        \item \textbf{Código fuente, métricas de Halstead:}
        \begin{itemize}
            \begin{table}[!ht]
                \centering
                \caption{Métricas de Halstead}
                \label{tab:halstead_metrics}
                \begin{tabular}{>{\bfseries}l l l}
                    \toprule
                    \multicolumn{1}{c}{\textbf{Símbolo}} &
                    \multicolumn{1}{c}{\textbf{Descripción}} &
                    \multicolumn{1}{c}{\textbf{Fórmula}} \\
                    \midrule
                    n1 & Número de operadores únicos    & Cantidad de tipos diferentes de operadores        \\
                    & en un programa                 &                                                   \\
                    \addlinespace

                    n2 & Número de operandos únicos     & Cantidad de tipos diferentes de operandos         \\
                    & en un programa                 &                                                   \\
                    \addlinespace

                    N1 & Frecuencia total de operadores & $N1 = \sum (\text{ocurrencias de cada operador})$ \\
                    &                                &                                                   \\
                    \addlinespace

                    N2 & Frecuencia total de operandos  & $N2 = \sum (\text{ocurrencias de cada operando})$ \\
                    &                                &                                                   \\
                    \bottomrule
                \end{tabular}
            \end{table}

            \item Longitud: $N = n_1 \log_2 n_1 + n_2 \log_2 n_2$
            \begin{nota}
                \textit{(Coa estimación de $N$ anterior. Se se usa $N = N_1 + N_2$, o volume sería real.)}
            \end{nota}
            \item Volumen: $V = N \log_2 (n_1 + n_2)$
        \end{itemize}

        \item \textbf{Pruebas:}
        \begin{itemize}
            \item Nivel de programa: $PL = 1 / ((n_1/2) \cdot (N_2/n_2))$
            \item Esfuerzo de prueba: $e = V / PL$
        \end{itemize}

        \item \textbf{Mantenimiento:}
        \begin{itemize}
            \item Índice de madurez: $IMS = (M_1 - (F_a + F_c + F_d)) / M_1$
        \end{itemize}

        \item \textbf{SLA/SLO/SLI:}
        \begin{itemize}
            \item SLA: Acuerdo con cliente (Ejemplo: disponibilidad >99.9\%)
            \item SLO: Objetivo específico dentro de SLA
            \item SLI: Medida real de cumplimiento
        \end{itemize}
    \end{itemize}

% todo: completar


\section{Tipos de mantenimiento}\label{sec:tipos-de-mantenimiento}
\subsection{Leyes de Lehman}\label{subsec:leyes-de-lehman}
\begin{itemize}
    \item \textbf{Ley del cambio continuo.} En un entorno real, un sistema debe necesariamente cambiar para mantener su utilidad.
    \item \textbf{Ley de complejidad creciente.} Cuando el sistema evoluciona se hace más complejo.
    Hay que tomar medidas para evitarlo.
    \item \textbf{Ley de evolución. }La evolución es un proceso autorregulado.
    El tamaño, tiempo entre versiones, errores detectados, etc., se mantienen en el tiempo.
    \item \textbf{Ley de estabilidad organizacional.} Durante el tiempo de vida del sistema su velocidad de desarrollo es constante e independiente de los recursos dedicados su desarrollo.
    \item \textbf{Ley de conservación de la familiaridad.} A medida que un sistema evoluciona todo lo que está asociado con ello debe mantener un conocimiento total de su contenido y su comportamiento.
    \item \textbf{Ley de crecimiento continuado. }La funcionalidad ofrecida por los sistemas tiene que crecer continuamente para
    mantener la satisfacción de los usuarios.
    \item \textbf{Ley de decremento de la calidad.} La calidad de los sistemas software comenzará a disminuir a menos que dichos
    sistemas se adapten a los cambios de su entorno de funcionamiento.
    \item \textbf{Ley de retroalimentación.} Los procesos de evolución incorporan sistemas de retroalimentación.
\end{itemize}

\subsection{Proceso de mantenimiento}\label{subsec:proceso-de-mantenimiento}
\begin{itemize}
    \item \textbf{Causas de modificación:}
    \begin{itemize}
        \item Nuevos requisitos/cambios solicitados
        \item Corrección de errores
    \end{itemize}

    \item \textbf{Incorporación al desarrollo:}
    \begin{itemize}
        \item Implementación formal u hotfixes
    \end{itemize}

    \item \textbf{Factores de esfuerzo:}
    \begin{itemize}
        \item Diseño del sistema
        \item  Mecanismos de prueba
        \item Documentación
        \item  Estabilidad del personal
    \end{itemize}
\end{itemize}

\subsection{Clasificación de mantenimiento}\label{subsec:clasificacion-de-mantenimiento}
\begin{tabular}{|l|l|l|p{6cm}|c|}
    \hline
    \textbf{Categoría} & \textbf{Tipo} & \textbf{Descripción}          & \textbf{Esfuerzo} \\
    \hline
    Evolutivo          & Perfectivo    & Añadir nuevas funcionalidades & 50\%              \\
    \hline
    & Adaptativo    & Adaptar a nuevos entornos     & 25\%              \\
    \hline
    & Preventivo    & Mejorar mantenibilidad futura & 5\%               \\
    \hline
    Tradicional        & Correctivo    & Corregir errores              & 20\%              \\
    \hline
\end{tabular}

\subsection{Release Notes}\label{subsec:release-notes}
\begin{itemize}
    \item \textbf{Contenido esencial:}
    \begin{itemize}
        \item Novedades
        \item Mejoras
        \item Correcciones de errores
    \end{itemize}
\end{itemize}

\subsection{Sistemas heredados}\label{subsec:sistemas-heredados}
\begin{itemize}
    \item \textbf{Problemas comunes:}
    \begin{itemize}
        \item Código espagueti
        \item  Falta de documentación
        \item Estructura deficiente
        \item Especificaciones ausentes
    \end{itemize}

    \item \textbf{Solución:} Ingeniería inversa
\end{itemize}

\subsection{Reingeniería de sistemas}\label{subsec:reingenieria-de-sistemas}
\begin{definicion}
    Reestructuración, reescritura o re-documentación sin cambiar funcionalidad.
\end{definicion}

\begin{itemize}
    \item \textbf{Ventajas:} Más económico que desarrollo nuevo • Reemplazo gradual
    \item \textbf{Factores coste:} Personal experto, herramientas disponibles
    \item \textbf{¿Por qué?:} Más barato que el desarrollo.
\end{itemize}

\subsection{Tipos de reingeniería}\label{subsec:tipos-de-reingenieria}
\begin{itemize}
    \item Traducción de código
    \item Ingeniería inversa
    \item Reestructuración
    \item Ingeniería hacia adelante
    \item Migración de datos
\end{itemize}

\subsection{Flujo de reingeniería}\label{subsec:flujo-de-reingenieria}
\begin{enumerate}
    \item Código fuente sucio $\rightarrow$ Reestructuración $\rightarrow$ Código limpio
    \item Extracción de abstracciones $\rightarrow$ Especificación inicial
    \item Refinamiento $\rightarrow$ Especificación final
\end{enumerate}



    \chapter*{Anexos}
    \addcontentsline{toc}{chapter}{Anexos}
    \phantomsection

    \clearpage


    \section*{Anexo 1: Explicación de la Administración de Configuración de Software (ACS)}
    \addcontentsline{toc}{section}{Anexo 1: Explicación de la Administración de Configuración de Software (ACS)}
    \phantomsection

    \begin{multicols}{2}

        \begin{cajaazul}[¿Qué es ACS?]
            La Administración de Configuración de Software (ACS) es un conjunto de \textbf{procesos y herramientas} para gestionar los \textbf{cambios} en productos software durante todo su ciclo de vida.

            Va más allá de controlar solo versiones, integrando \textbf{documentación, dependencias, auditorías y trazabilidad}.
        \end{cajaazul}

        \vspace{0.5em}

        \begin{cajaverde}[Componentes principales]
            \begin{itemize}[leftmargin=*]
                \item \textbf{Control de versiones}: Histórico y gestión de versiones.
                \item \textbf{Gestión de dependencias}: Control de relaciones entre componentes.
                \item \textbf{Gestión de cambios}: Registro, planificación y aprobación.
                \item \textbf{Auditoría y trazabilidad}: Seguimiento detallado de cambios.
            \end{itemize}
        \end{cajaverde}

        \vspace{0.5em}
        \deactivatequoting
        \begin{cajanaranja}[Diferencias con Git]
            \begin{itemize}[leftmargin=*]
                \item \textbf{Alcance mayor}: Incluye código, documentación, requisitos, configuraciones\ldots
                \item \textbf{Procesos integrados}: Control de flujos y aprobaciones.
                \item \textbf{Trazabilidad}: Desde requisitos hasta entregables.
            \end{itemize}
        \end{cajanaranja}
        \activatequoting
        \vspace{0.5em}

        \begin{cajarosa}[Beneficios]
            \begin{itemize}[leftmargin=*]
                \item Mejora la calidad y estabilidad.
                \item Facilita el trabajo en equipos grandes.
                \item Reduce errores por cambios no controlados.
                \item Permite auditorías y cumplimiento.
            \end{itemize}
        \end{cajarosa}

        \vspace{0.5em}

        \begin{cajaroja}[¿Por dónde empezar?]
            \begin{itemize}[leftmargin=*]
                \item Usa un sistema de control de versiones (ej.\ Git).
                \item Documenta los elementos de configuración.
                \item Define políticas y procesos claros.
                \item Usa herramientas de seguimiento (p.ej.\ Jira).
                \item Forma al equipo en buenas prácticas.
            \end{itemize}
        \end{cajaroja}

    \end{multicols}

    \vspace{1em}
    \begin{center}
        \small\textit{Este anexo es una introducción visual y resumida para entender la Administración de Configuración de Software.}
    \end{center}


\end{document}
