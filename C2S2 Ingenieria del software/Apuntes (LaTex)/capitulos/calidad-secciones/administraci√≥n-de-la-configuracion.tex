\begin{definicion}
    Conjunto de actividades para gestionar los cambios durante el ciclo de vida del software, garantizando:
    \begin{itemize}
        \item Trazabilidad
        \item Control de versiones
        \item Información actualizada
    \end{itemize}

    Se diferencia del mantenimiento en que este aplica cambios, y la administración de configuración controla y registra dichos cambios.
\end{definicion}

\subsection*{Conceptos clave}
\begin{itemize}
    \item \textbf{Elemento de configuración (EC / ICS):} Unidad identificable que debe ser controlada (código, documentación, etc.).
    \item \textbf{Base de datos de configuración (BCD):} Lugar donde se almacenan los EC y sus versiones.
    \item \textbf{Línea base:} Versión estable y acordada de un conjunto de EC\@.
    \item \textbf{Cambio:} Modificación propuesta a un EC\@.
    \item \textbf{Control de cambios:} Proceso de evaluar y aprobar cambios.
\end{itemize}

\subsection{Ítems de Configuración (IC)}\label{subsec:items-de-configuracion-(ic)}
\begin{definicion}
    Cada uno de los elementos que comprenden toda la información
    producida como parte del proceso de software.
\end{definicion}
\begin{itemize}
    \item Programas de cómputo
    \item Productos de trabajo
    \item Contenido
\end{itemize}

\subsection{Sistema de Administración}\label{subsec:sistema-de-administracion}
\begin{itemize}
    \item \textbf{Elementos componentes:} Herramientas que permiten el acceso y gestión de cada ítem de configuración del software.
    \item \textbf{Elementos de proceso:} Acciones y tareas necesarias para realizar una gestión efectiva del cambio.
    \item \textbf{Elementos de construcción:} Herramientas que automatizan la compilación, empaquetado y generación de versiones correctas del software.
    \item \textbf{Elementos humanos:} Procesos y herramientas que utiliza el equipo para implementar de manera efectiva la administración de configuración.
\end{itemize}

\subsection{Repositorio ACS}\label{subsec:repositorio-acs}
\begin{definicion}
    Almacén centralizado o distribuido donde se gestionan los ítems de configuración.
\end{definicion}

\paragraph{Características clave:}
\begin{itemize}
    \item Control de versiones.
    \item Rastreo de dependencias entre componentes.
    \item Trazabilidad de requisitos y cambios.
    \item Apoyo a auditorías e inspecciones de calidad.
\end{itemize}

\subsection{Proceso de Administración de la Configuración}\label{subsec:proceso-de-administracion-de-la-configuracion}
\begin{enumerate}
    \item \textbf{Identificación:} Asignar nombres unívocos a los ítems de configuración.
    \item \textbf{Control de cambios:} Registrar, aprobar e implementar modificaciones.
    \item \textbf{Control de versiones:} Gestionar diferentes versiones de todos los objetos.
    \item \textbf{Auditoría:} Verificar que los cambios cumplen con los estándares establecidos.
    \item \textbf{Reporte de estado:} Informar del estado actual y del histórico de cambios.
\end{enumerate}

\subsection{Administración del contenido}\label{subsec:administracion-del-contenido}
\begin{itemize}
    \item \textbf{Subsistema de recopilación:} Permite almacenar y organizar el contenido asociado a cada Ítem de Configuración del Software (ICS).
    \item \textbf{Subsistema de administración:} Gestiona los cambios realizados sobre el contenido de cada ICS, manteniendo un histórico completo y trazable.
    \item \textbf{Subsistema de publicación:} Hace accesible el contenido relevante a los distintos interesados del proyecto, permitiendo su consulta o reutilización.
\end{itemize}

\subsection{Conflictos dominantes}\label{subsec:conflictos-dominantes}
\begin{itemize}
    \item \textbf{Contenido:} No está claro qué información debe formar parte del sistema de administración de configuración.
    \item \textbf{Personas:} El equipo desconoce las herramientas o procesos de ACS, o no sabe cuándo aplicarlos.
    \item \textbf{Escalabilidad:} A medida que crece el proyecto, se complica la gestión del gran volumen de cambios.
    \item \textbf{Políticas:} Faltan reglas claras sobre quién es responsable de llevar a cabo la administración de configuración y cómo aplicarla de manera coherente.
\end{itemize}
