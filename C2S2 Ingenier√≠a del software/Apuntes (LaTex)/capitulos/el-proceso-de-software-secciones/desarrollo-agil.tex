    \subsection{Definiciones}\label{subsec:definiciones-agiles}
    \begin{definicion}
        \textbf{Agilidad:} Capacidad de \textbf{adaptación al cambio} (requisitos, equipo, tecnologías,…)
    \end{definicion}

    \begin{definicion}
        \textbf{Agilismo:} Métodos para alcanzar agilidad.
    \end{definicion}

    \subsection{El coste del cambio}\label{subsec:el-coste-del-cambio}
    Los métodos ágiles buscan reducir el coste del cambio a lo largo del ciclo de vida del proyecto, evitando que aumente exponencialmente en las fases tardías.

    \subsection{Características de los métodos ágiles}\label{subsec:caracteristicas-de-los-metodos-agiles}

    \begin{itemize}
        \item \textbf{Compatibles con estándares}: CMMI, ISO, etc.
        \item \textbf{Diferenciación}: Los estándares indican el QUÉ hacer, el agilismo indica el CÓMO hacerlo
        \item \textbf{Modelos iterativos y adaptativos}: pueden ser incrementales o evolutivos
        \item \textbf{Equipos multidisciplinares}: autónomos y autoorganizados.
        \item \textbf{Guiados por el Manifiesto Ágil} (2001)
        \item \textbf{Múltiples metodologías}: adopción flexible según necesidades
    \end{itemize}

    \subsection{Manifiesto ÁGIL}\label{subsec:manifiesto-agil}

    Cuatro valores fundamentales (priorizando los de la izquierda sobre los de la derecha):

    \begin{enumerate}
        \item \textbf{Individuos e interacciones} sobre procesos y herramientas
        \item \textbf{Software funcionando} sobre documentación extensiva
        \item \textbf{Colaboración con el cliente} sobre negociación contractual
        \item \textbf{Respuesta ante el cambio} sobre seguir un plan
    \end{enumerate}

    \subsection{Principios del Manifiesto Ágil}\label{subsec:principios-del-manifiesto-agil}

    \begin{enumerate}
        \item \textbf{Prioridad}: Satisfacer al cliente mediante entrega temprana y continua de software con valor.
        \item \textbf{Cambios}: Aceptar que los requisitos cambien, incluso en etapas tardías.
        \item \textbf{Entregas frecuentes}: Software funcional cada 2 semanas a 2 meses (preferiblemente más corto).
        \item \textbf{Colaboración diaria}: Responsables de negocio y desarrolladores trabajan juntos.
        \item \textbf{Individuos motivados}: Dar entorno y apoyo, confiar en la ejecución.
        \item \textbf{Comunicación cara a cara}: Método más eficiente y efectivo.
        \item \textbf{Software funcionando}: Principal medida de progreso.
        \item \textbf{Desarrollo sostenible}: Mantener ritmo constante indefinidamente.
        \item \textbf{Excelencia técnica}: Atención continua al buen diseño mejora la agilidad.
        \item \textbf{Simplicidad}: Arte de maximizar la cantidad de trabajo no realizado.
        \item \textbf{Equipos auto-organizados}: Las mejores arquitecturas, requisitos y diseños emergen de ellos.
        \item \textbf{Reflexión regular}: El equipo reflexiona para ser más efectivo y ajustar comportamiento.
    \end{enumerate}


    \subsection{Roles en el desarrollo ágil}\label{subsec:roles-en-el-desarrollo-agil}

    \begin{itemize}
        \item \textbf{Agile Coach}: Experto en agilismo que ayuda a los empleados a adoptar metodologías ágiles
        \item \textbf{Product Owner}: Gestor de la pila de trabajo y su prioridad para maximizar valor entregado
        \item \textbf{Scrum Master}: Facilitador de los equipos que siguen metodología Scrum
        \item \textbf{Equipo de desarrollo}: Conjunto de miembros que desarrollan y entregan software en incrementos de valor
    \end{itemize}


    \subsection{Programación extrema (XP)}\label{subsec:programacion-extrema-(xp)}

    La Programación Extrema fue creada por Kent Beck, quien también fue contribuidor al manifiesto ágil.
    Se caracteriza por llevar las buenas prácticas de programación a sus límites extremos.

    Busca \textbf{retroalimentación continua} del usuario mediante entregas cortas y frecuentes, lo que permite \textbf{detectar y corregir problemas rápidamente}.
    La \textbf{documentación es simple} y se basa en tres principios: mantener \textbf{código simple y mantenible}, priorizar \textbf{código autodocumentado} sobre comentarios extensos, y usar\textbf{ tests unitarios }como mecanismo de diseño y documentación.

    La \textbf{programación por parejas} implica que dos desarrolladores trabajen juntos en el mismo código, lo que mejora la calidad y facilita la transferencia de conocimiento.
    El \textbf{énfasis en pruebas} se materializa a través del Test Driven Development (TDD), donde las pruebas se escriben antes que el código, y se eliminan defectos antes de añadir nueva funcionalidad.

    El principio \textbf{YAGNI} ("You Aren't Gonna Need It") promueve programar solo para las prioridades inmediatas, evitando la sobreingeniería y el desarrollo de funcionalidades que podrían no ser necesarias.


    \subsection{SCRUM}\label{subsec:scrum}

    Scrum fue concebido a principios de la década de 1990 por Jeff Sutherland, otro contribuidor al manifiesto ágil.
    Se estructura en \textbf{iteraciones de 2 a 4 semanas} de duración, donde cada iteración debe terminar con una entrega de valor tangible al cliente.

    Una característica fundamental es que \textbf{el alcance no puede modificarse} durante el desarrollo de la iteración, lo que proporciona estabilidad al equipo.
    Incorpora un \textbf{proceso de mejora continua} para incrementar la eficiencia del equipo mediante retrospectivas regulares.


    \subsubsection{Ceremonias SCRUM}

    El \textbf{Sprint Planning} es la sesión de planificación donde el equipo decide qué elementos del product backlog se desarrollarán en la siguiente iteración, basándose en la priorización establecida por el Product Owner.

    El \textbf{Daily Scrum} es una reunión diaria de máximo 15 minutos donde cada miembro del equipo comparte tres elementos: el \textbf{progreso realizado desde la anterior reunión}, el \textbf{progreso esperado hasta la siguiente reunión}, y \textbf{cualquier bloqueo o impedimento que esté enfrentando.}

    El \textbf{Sprint Review} es la sesión donde se revisa la iteración completada mediante una demostración del software funcional desarrollado.

    El \textbf{Refinamiento} son sesiones dedicadas a revisar y clarificar requisitos para alcanzar un entendimiento común entre todos los miembros del equipo.

    La \textbf{Retrospectiva} es una sesión de revisión del proceso utilizado durante la iteración, donde el equipo identifica qué funcionó bien, qué puede mejorarse, y define acciones concretas de mejora.



    \subsubsection{Artefactos SCRUM}

    El \textbf{Product Backlog} es la pila de trabajo que contiene todos los requisitos y funcionalidades a cumplir, ordenados por prioridad y valor de negocio.

    El \textbf{Sprint Backlog} contiene específicamente los requisitos que se van a desarrollar en la iteración próxima, con el nivel de detalle necesario para su implementación.

    El \textbf{Scrum Board} es un tablero visual que muestra el estado actual de todas las tareas, típicamente organizado en columnas como \textquote{Por hacer}, \textquote{En progreso} y \textquote{Terminado}.

    El \textbf{Burndown Chart} es un gráfico que compara el progreso ideal de una iteración con el progreso real, permitiendo identificar desviaciones y tomar medidas correctivas.

    La \textbf{Definition of Ready (DoR)} establece los criterios que debe cumplir una tarea para estar lista para ser iniciada por el equipo de desarrollo.

    La \textbf{Definition of Done (DoD)} define los criterios que debe cumplir una tarea para poder ser marcada como completada y entregada.



    \subsection{KANBAN}\label{subsec:kanban}

    Kanban fue definido por primera vez en 2007 y está basado en el proceso de gestión visual desarrollado por Toyota para la manufactura.
    Sigue el proceso \textbf{Kaizen} de mejora continua y se caracteriza por \textbf{no tener iteraciones} \textbf{ni ceremonias preestablecidas}, siendo \textbf{más flexible en su estructura}.

    \subsubsection{Principios KANBAN:}

    Los \textbf{principios de gestión del cambio} establecen que se debe comenzar con lo que se hace actualmente (sin cambios disruptivos), aceptar el cambio incremental y evolutivo (evitando transformaciones radicales), y fomentar actos de liderazgo a todos los niveles de la organización:
    \begin{itemize}
        \item Comienza con lo que haces ahora
        \item Aceptar el cambio incremental y evolutivo
        \item Fomentar los actos de liderazgo a todos los niveles
    \end{itemize}

    Los \textbf{principios de prestación de servicios} se centran en las necesidades y expectativas del cliente como foco principal, gestionar el trabajo y los procesos en lugar de microgestionar a los trabajadores, y revisar periódicamente toda la red de servicios para optimizar el flujo de valor:
    \begin{itemize}
        \item Centrarte en las necesidades y expectativas del cliente
        \item  Gestionar el trabajo, no los trabajadores
        \item Revisar periódicamente la red de servicios
    \end{itemize}

    \subsubsection{Prácticas KANBAN:}

    \textbf{Visualizar el flujo de trabajo} mediante tableros que muestren claramente el estado de todas las tareas y su progreso a través del proceso.

    \textbf{Limitar el trabajo en curso} (WIP - Work In Progress) para evitar la sobrecarga del sistema y mejorar el flujo.

    \textbf{Gestionar el flujo} monitorizando y optimizando el movimiento del trabajo a través del sistema.

    \textbf{Explicitar las políticas de procesos} para que todos entiendan claramente cómo funciona el sistema.

    \textbf{Aplicar bucles de retroalimentación} para obtener información sobre el rendimiento del sistema y áreas de mejora.

    \textbf{Mejorar en colaboración} mediante el trabajo conjunto de todo el equipo para optimizar el sistema.


    \subsection{Otras metodologías ágiles}\label{subsec:otras-metodologias-agiles}

    \textbf{Lean Software Development (LSD)} está basado en los principios del Lean Startup y se enfoca en eliminar desperdicios, amplificar el aprendizaje y entregar valor rápidamente.

    \textbf{Desarrollo Adaptativo de Software (DAS)} es un enfoque que asume que los proyectos de software son inherentemente impredecibles y se adapta continuamente a los cambios.

    \textbf{Agile Unified Process} es una versión simplificada y ágil del Proceso Unificado tradicional, manteniendo sus fortalezas pero eliminando su rigidez.

    \textbf{Crystal Clear} es una metodología ligera diseñada específicamente para equipos pequeños, enfocándose en la comunicación y la simplicidad.

    \textbf{PMI Agile} es el enfoque ágil desarrollado por el Project Management Institute, integrando prácticas ágiles con la gestión de proyectos tradicional.





% ****************** CAPITULO -- FIN -- CAPITULO -- FIN **************