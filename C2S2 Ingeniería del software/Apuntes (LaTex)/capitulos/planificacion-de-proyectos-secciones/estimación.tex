\subsection{Objetivo y proceso de la planificación}\label{subsec:objetivo-y-proceso-de-la-planificacion}

La planificación busca determinar los recursos necesarios para completar el proyecto en plazo y con calidad aceptable.
Los elementos principales son:

\begin{itemize}
    \item Estimación del esfuerzo y tiempo
    \item Asignación de tareas
    \item Identificar las dependencias
    \item Gestión de riesgos
\end{itemize}

Sus fases son:

\begin{enumerate}
    \item \textbf{Establecer ámbito:} ¿Qué se va a hacer?
    (casos de uso + requisitos no funcionales)
    \item \textbf{Determinar viabilidad:} ¿Es posible?
    (tecnología, finanzas, tiempo)
    \item \textbf{Analizar riesgos:} ¿Qué puede salir mal?
    \item \textbf{Definir recursos:} Personal, hardware, herramientas y componentes reutilizables
    \item \textbf{Estimar coste/esfuerzo:} Técnicas como COCOMO o puntos de historia
    \item \textbf{Desarrollar calendario:} Con tareas, hitos y dependencias
\end{enumerate}

\subsection{Ámbito vs Factibilidad}\label{subsec:ambito-vs-factibilidad}

El ámbito puede ser tanto una descripción escrita como un conjunto de casos de uso.
Se deben tener en cuenta los detalles de los casos de uso y considerar las restricciones.

Por otro lado, la factibilidad es hasta qué punto es realizable el proyecto.
Esta puede ser:

\begin{itemize}
    \item Tecnológica
    \item Financiera
    \item Temporal
    \item Material (recursos)
\end{itemize}

Los recursos se pueden dividir de la siguiente forma dentro del proyecto:

\begin{itemize}
    \item \textbf{Personal:}
    \begin{itemize}
        \item Número
        \item Habilidades
        \item Ubicación
    \end{itemize}
    \item \textbf{Entorno:}
    \begin{itemize}
        \item Herramientas software
        \item Hardware
        \item Recursos de red
    \end{itemize}
    \item \textbf{Software reutilizable:}
    \begin{itemize}
        \item \textbf{Componentes nuevos}
        \item \textbf{Componentes de experiencia parcial:} Las especificaciones, los diseños, código o pruebas existentes de proyectos anteriores podrán ser usados para el proyecto actual, pero requerirán una modificación sustancial y los miembros del equipo han limitado su experiencia sólo al área de aplicación representada por los componentes.
        Por eso, las modificaciones tendrán un mayor riesgo
        \item \textbf{Componentes con experiencia completa:} Ya existentes y desarrollados para proyectos anteriores similares al software que se va a construir para el proyecto actual.
        Los miembros del equipo ya tienen experiencia en el área.
        Bajo riesgo
        \item \textbf{Componentes COTS:} Diseñados para un uso inmediato, no requiere modificaciones.
        Sin riesgo
    \end{itemize}
\end{itemize}

\subsection{Técnicas de Estimación}\label{subsec:tecnicas-de-estimacion}

\subsubsection{Ley de Parkinson}

El mismo trabajo requiere más tiempo cuando hay plazos más largos.

\subsubsection{Precio Oportunista}

\begin{itemize}
    \item Basado en lo que el cliente está dispuesto a pagar
    \item Común en concursos públicos y licitaciones
    \item Riesgo: Puede no reflejar el esfuerzo real requerido
\end{itemize}

\subsection{Técnicas de Descomposición}\label{subsec:tecnicas-de-descomposicion}

\subsubsection{Tipos de Descomposición}

\begin{itemize}
    \item \textbf{Lógica difusa:}
    \begin{itemize}
        \item Identificar tipo de aplicación
        \item Establecer magnitud en escala cualitativa
        \item Refinar dentro del rango original
    \end{itemize}

    \item \textbf{Puntos de Función (PF):}
    \begin{itemize}
        \item Medir características del dominio de información
        \item Componentes: Entradas, Salidas, Consultas, Archivos, Interfaces
    \end{itemize}

    \item \textbf{Componente Estándar:}
    \begin{itemize}
        \item Contar ocurrencias de componentes estándar
        \item Usar datos históricos para estimar tamaño por componente
    \end{itemize}

    \item \textbf{Dimensionamiento del Cambio:}
    \begin{itemize}
        \item Para proyectos que modifican software existente
        \item Estimar número y tipo de modificaciones requeridas
    \end{itemize}
\end{itemize}

\subsubsection{Estimación Basada en Problema}

\begin{itemize}
    \item \textbf{Métricas clave:}
    \begin{itemize}
        \item Líneas de Código (LOC)
        \item Puntos de Función (PF)
    \end{itemize}

    \item \textbf{Fórmula de estimación ponderada:}
    \[
        S = \frac{S_{\text{opt}} + 4S_m + S_{\text{pes}}}{6}
    \]
    donde:
    \begin{itemize}
        \item $S_{\text{opt}}$ = Estimación optimista
        \item $S_m$ = Estimación más probable
        \item $S_{\text{pes}}$ = Estimación pesimista
    \end{itemize}

    \item \textbf{Ejemplo práctico:}
    \begin{center}
        \begin{tabular}{lcccc}
            \toprule
            Componente     & Optimista & Probable & Pesimista & Estimación         \\
            \midrule
            GUI            & 4600      & 6900     & 8600      & 6800               \\
            Servicio 1     & 2200      & 2750     & 3300      & 2750               \\
            Servicio 2     & 2500      & 3300     & 4400      & 3350               \\
            Servicio 3     & 1800      & 2250     & 2700      & 2250               \\
            \midrule
            \textbf{Total} &           &          &           & \textbf{15150 LOC} \\
            \bottomrule
        \end{tabular}
    \end{center}

    \item \textbf{Cálculo de coste:}
    \[
        \text{Esfuerzo} = \frac{\text{Total LOC}}{\text{Productividad}} = \frac{15150}{750} = 20.2\ \text{persona-meses}
    \]
\end{itemize}

\subsubsection{Estimación Basada en Proceso}

\begin{itemize}
    \item Descomposición en actividades del proceso de software
    \item Asignación de esfuerzo a cada actividad
\end{itemize}

\subsubsection{Modelos Empíricos}
\label{subsubsec:empiricos}

\begin{itemize}
    \item \textbf{Fórmula general:}
    \[
        E = A + B \cdot (e_v)^C
    \]
    \begin{itemize}
        \item $E$: Esfuerzo en personas-mes.
        \item $e_v$: Variable de estimación (LOC o PF).
        \item $A, B, C$: Constantes derivadas de datos históricos.
    \end{itemize}
    \item \textbf{Ventaja:} Predictibilidad mediante regresión sobre proyectos pasados.
\end{itemize}

\subsection{COCOMO II}
\label{subsec:cocomo}

\subsubsection{Conceptos Básicos}
\label{subsubsec:cocomo-basico}

\begin{itemize}
    \item \textbf{Objetivo:} Estimar esfuerzo ($E$) y tiempo ($D$) en función del tamaño (KLOC) para ($N$) personas
    \item \textbf{Ecuaciones:}
    \begin{align*}
        E &= a \cdot (\text{KLOC})^b \\
        D &= c \cdot (E)^d
    \end{align*}
\end{itemize}

\subsubsection{Tipos de Proyectos}
\label{subsubsec:tipos-proyectos}

\begin{center}
    \begin{tabular}{lcccc}
        \toprule
        \textbf{Tipo de Proyecto}                         & \textbf{a} & \textbf{b} & \textbf{c} & \textbf{d} \\
        \midrule
        \textbf{Orgánico} (pequeño, requisitos flexibles) & 2,4        & 1,05       & 2,5        & 0,38       \\
        \textbf{Semi-acoplado} (complejidad media)         & 3,0        & 1,12       & 2.5        & 0,35       \\
        \textbf{Empotrado} (requisitos rígidos)           & 3,6        & 1,20       & 2,5        & 0,32       \\
        \bottomrule
    \end{tabular}
\end{center}

\subsubsection{Ejemplo Práctico}
\label{subsubsec:ejemplo-cocomo}

\begin{itemize}
    \item \textbf{Tamaño total:} 15.150 LOC = 15,15 KLOC
    \item \textbf{Proyecto orgánico:}
    \begin{align*}
        E &= 2.4 \cdot (15.15)^{1.05} = 42\ \text{personas-mes} \\
        D &= 2.5 \cdot (42)^{0.38} = 11\ \text{meses} \\
        N &= \frac{E}{D} = \frac{42}{11} = 4\ \text{personas}
    \end{align*}
\end{itemize}

\subsubsection{Limitaciones}
\label{subsubsec:limitaciones-cocomo}

\begin{itemize}
    \item No considera reusabilidad en programación orientada a objetos
    \item Basado en muestras limitadas (no aplicable a todos los entornos)
    \item Ignora paralelización de tareas y factores de productividad
\end{itemize}

\subsection{Estimación Ágil}
\label{subsec:agil}

\subsubsection{Poker Planning}
\label{subsubsec:poker}

\begin{description}
    \item[\textbf{Paso 1:}] Seleccionar una historia de usuario
    \item[\textbf{Paso 2:}] Discusión breve del equipo
    \item[\textbf{Paso 3:}] Estimación individual con tarjetas (ejemplo: Fibonacci: 1, 2, 3, 5, 8)
    \item[\textbf{Paso 4:}] Revelar estimaciones simultáneamente
    \item[\textbf{Paso 5:}] Si hay discrepancia, debatir y repetir
\end{description}

\paragraph{Puntos vs. Horas}
\label{par:puntos-vs-horas}

\begin{center}
    \begin{tabularx}{\textwidth}{lXlX}
        \toprule
        & \textbf{Puntos} & & \textbf{Horas} \\
        \midrule
        \textbf{Ventajas} &
        \begin{itemize}
            \item Capturan complejidad, riesgo y esfuerzo
            \item Enfocados en valor (no tiempo)
        \end{itemize} &
        \textbf{Ventajas} &
        \begin{itemize}[leftmargin=*]
            \item Fácil medición del trabajo
            \item Cálculo directo de productividad
        \end{itemize} \\
        \midrule
        \textbf{Desventajas} &
        \begin{itemize}[leftmargin=*]
            \item Abstractos (requieren equipo consolidado)
        \end{itemize} &
        \textbf{Desventajas} &
        \begin{itemize}[leftmargin=*]
            \item Ignoran trabajo no relacionado (reuniones, etc.)
        \end{itemize} \\
        \bottomrule
    \end{tabularx}
\end{center}
