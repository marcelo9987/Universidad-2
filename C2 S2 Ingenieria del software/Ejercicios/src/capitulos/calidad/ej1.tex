% !TeX root = ../calidad.tex

\begin{enunciado}
    Calcula el número de errores que verán los usuarios finales para un proyecto en el que sucede lo siguiente:
    \begin{itemize}
        \item En el modelo de requerimientos se han cometido 10 errores y cada uno se amplifica en un factor de 2:1 en el diseño.
        \item En el diseño se cometen otros 20 errores adicionales, que luego se amplifican en un factor de 1.5:1 en el código.
        \item En el código se cometen otros 30 errores adicionales.
        \item Las pruebas unitarias encuentran un 20\% de todos los errores.
        \item Las pruebas de integración descubren el 50\% de los errores restantes.
        \item Las pruebas de validación/aceptación hallan el 50\% de los que queden.
        \item Tras las pruebas se pone el nuevo software en marcha.
    \end{itemize}
\end{enunciado}

\subsection{Datos}\label{subsec:datos}
\begin{itemize}
    \item Errores en requisitos: $10$ errores.

    Amplificación en diseño: $10 \cdot 2 = 20$ errores.


    \item Errores de diseño adicionales: $20$ errores.


    Total de errores diseño: $20 + 20 = 40$ errores.

    Amplificación en código: $40 \cdot 1.5 = 60$ errores.


    \item Errores adicionales en código: $30$ errores.

    Total de errores en código: $60 + 30 = 90$ errores.


\end{itemize}

\hrulefill

\subsection{Cálculo de errores tras las pruebas}\label{subsec:calculo-de-errores-tras-las-pruebas}
\begin{itemize}
    \item Pruebas unitarias detectan el 20\%: $90 \cdot 0.20 = 18$ errores detectados.
    Errores restantes: $90 - 18 = 72$.
    \item Pruebas de integración detectan el 50\% de los restantes: $72 \cdot 0.50 = 36$ errores detectados.
    Errores restantes: $72 - 36 = 36$.
    \item Pruebas de validación detectan el 50\% de los restantes: $36 \cdot 0.50 = 18$ errores detectados.
    Errores restantes: $36 - 18 = 18$.
\end{itemize}

\begin{solucion}
    El total de errores que verán los usuarios finales es de \boxed{\textbf{18 errores}}.
\end{solucion}
