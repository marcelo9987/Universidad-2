\addcontentsline{toc}{chapter}{Anexos}
\phantomsection

\clearpage

\section*{Anexo 1: Clasificación de procesos según ISO/IEC 15504}
\addcontentsline{toc}{section}{Anexo 1: Clasificación de procesos según ISO/IEC 15504}
\phantomsection
\begin{table}[H]
    \centering
    \begin{tabular}{|c|p{12cm}|}
        \hline
        \textbf{Nivel} & \textbf{Criterios (checklist)} \\ \hline
        \textbf{0 – Incompleto} &
        \begin{itemize}
%                \item[$\square$] Non se segue un proceso claro ou estruturado.
            \item[$\square$] No se sigue un proceso claro o estructurado.
            \item[$\square$] No se puede saber si se alcanzan los objetivos del proceso.
        \end{itemize} \\ \hline

        \textbf{1 – Realizado} &
        \begin{itemize}
            \item [$\square$] El proceso se realiza de manera consistente.
            \item [$\square$] Los resultados son medibles.
        \end{itemize} \\ \hline

        \textbf{2 – Gestionado} &
        \begin{itemize}
            \item[$\square$] Las actividades están planificadas y controladas.
            \item[$\square$] Los productos de trabajo (artefactos) están controlados.
        \end{itemize} \\ \hline

        \textbf{3 – Establecido} &
        \begin{itemize}
            \item[$\square$] El proceso está documentado y definido.
            \item[$\square$] Está basado en estándares o modelos.
        \end{itemize} \\ \hline

        \textbf{4 – Predecible} &
        \begin{itemize}
            \item[$\square$] Se recogen datos cuantitativos (tempos, esfuerzo, etcétera\ldots).
            \item[$\square$] Se establecen objetivos medibles en base a esos datos.
        \end{itemize} \\ \hline

        \textbf{5 – Optimizado} &
        \begin{itemize}
            \item[$\square$] Hay mejora continua del proceso.
            \item[$\square$] Se identifican puntos débiles y buenas prácticas.
            \item[$\square$] Se implementan cambios activamente para mejorar el proceso.
        \end{itemize} \\ \hline

    \end{tabular}
    \caption{Checklist para clasificación de procesos segundo ISO/IEC 15504}\label{tab:IEC15504_checklist}
\end{table}

\clearpage

\section*{Anexo 2: Explicación de la Administración de Configuración de Software (ACS)}
\addcontentsline{toc}{section}{Anexo 2: Explicación de la Administración de Configuración de Software (ACS)}
\phantomsection

\begin{multicols}{2}

    \begin{cajaazul}[¿Qué es ACS?]
        La Administración de Configuración de Software (ACS) es un conjunto de \textbf{procesos y herramientas} para gestionar los \textbf{cambios} en productos software durante todo su ciclo de vida.

        Va más allá de controlar solo versiones, integrando \textbf{documentación, dependencias, auditorías y trazabilidad}.
    \end{cajaazul}

    \vspace{0.5em}

    \begin{cajaverde}[Componentes principales]
        \begin{itemize}[leftmargin=*]
            \item \textbf{Control de versiones}: Histórico y gestión de versiones.
            \item \textbf{Gestión de dependencias}: Control de relaciones entre componentes.
            \item \textbf{Gestión de cambios}: Registro, planificación y aprobación.
            \item \textbf{Auditoría y trazabilidad}: Seguimiento detallado de cambios.
        \end{itemize}
    \end{cajaverde}

    \vspace{0.5em}
    \deactivatequoting
    \begin{cajanaranja}[Diferencias con Git]
        \begin{itemize}[leftmargin=*]
            \item \textbf{Alcance mayor}: Incluye código, documentación, requisitos, configuraciones\ldots
            \item \textbf{Procesos integrados}: Control de flujos y aprobaciones.
            \item \textbf{Trazabilidad}: Desde requisitos hasta entregables.
        \end{itemize}
    \end{cajanaranja}
    \activatequoting
    \vspace{0.5em}

    \begin{cajarosa}[Beneficios]
        \begin{itemize}[leftmargin=*]
            \item Mejora la calidad y estabilidad.
            \item Facilita el trabajo en equipos grandes.
            \item Reduce errores por cambios no controlados.
            \item Permite auditorías y cumplimiento.
        \end{itemize}
    \end{cajarosa}

    \vspace{0.5em}

    \begin{cajaroja}[¿Por dónde empezar?]
        \begin{itemize}[leftmargin=*]
            \item Usa un sistema de control de versiones (ej.\ Git).
            \item Documenta los elementos de configuración.
            \item Define políticas y procesos claros.
            \item Usa herramientas de seguimiento (p.ej.\ Jira).
            \item Forma al equipo en buenas prácticas.
        \end{itemize}
    \end{cajaroja}

\end{multicols}

\vspace{1em}
\begin{center}
    \small\textit{Este anexo es una introducción visual y resumida para entender la Administración de Configuración de Software.}
\end{center}