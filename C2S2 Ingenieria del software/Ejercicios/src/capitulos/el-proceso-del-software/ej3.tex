% !TeX root = ../el-proceso-de-software.tex


\begin{enunciado}
    Indica qué flujo/s de fases siguen los siguientes modelos del proceso.
    Justifica la respuesta.
    \begin{enumerate}
        \item Modelo de la cascada - Versión V
        \item Modelo de proceso incremental
        \item Modelo en espiral
        \item Modelo de ciclo de producto
    \end{enumerate}
\end{enunciado}

\begin{solucion}
    \begin{enumerate}
        \item \textbf{Modelo de la cascada - Versión V:} Sigue un flujo lineal, ya que cada fase espera a que la anterior sea completada y no se itera \textquote{hacia atrás}.
        \item \textbf{Modelo de proceso incremental:} Este modelo sigue un flujo evolutivo, ya que permite desarrollar el software en incrementos o versiones, donde cada incremento añade funcionalidad al producto existente.
        \item \textbf{Modelo en espiral:} Evolutivo, ya que tiene un enfoque iterativo y cíclico, donde cada ciclo incluye fases de planificación, análisis de riesgos, desarrollo y evaluación, permitiendo adaptarse a cambios en los requisitos.
        \item \textbf{Modelo de ciclo de producto:} Lineal, ya que sigue un flujo secuencial desde la concepción del producto hasta su entrega final, sin iteraciones ni retrocesos significativos.
    \end{enumerate}
\end{solucion}
