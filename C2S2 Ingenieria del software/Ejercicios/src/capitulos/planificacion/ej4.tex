% !TeX root = ../planificacion.tex

\begin{enunciado}
    Clasifica los siguientes riesgos según su tipo:
    \begin{enumerate}
        \item Problemas de disponibilidad de personal capacitado.
        \item Tiempos de respuesta lentos.
        \item Requisitos imprecisos.
        \item Cambios en las condiciones del mercado.
        \item Problemas de comunicación entre el equipo del proyecto y los interesados.
        \item Falta de liderazgo en el proyecto.
        \item Problemas de integración con otras aplicaciones o sistemas.
        \item Que la aplicación sea difícil de usar o frustrante para el usuario final.
        \item Fallos que pueden afectar negativamente a la percepción del usuario.
        \item Vulnerabilidades en la aplicación que pueden permitir el acceso a los datos del usuario.
        \item Falta de financiación para el desarrollo y mantenimiento.
        \item Problemas de escalabilidad que pueden surgir durante la implementación del software.
    \end{enumerate}
\end{enunciado}

\begin{solucion}

    \begin{itemize}
        \item \textbf{Personal:}
        \begin{itemize}
            \item (1) Problemas de disponibilidad de personal capacitado.
        \end{itemize}

        \item \textbf{Técnico:}
        \begin{itemize}
            \item (2) Tiempos de respuesta lentos.
            \item (7) Problemas de integración con otras aplicaciones o sistemas.
            \item (10) Vulnerabilidades en la aplicación que pueden permitir el acceso a los datos del usuario.
            \item (12) Problemas de escalabilidad que pueden surgir durante la implementación del software.
        \end{itemize}


        \item \textbf{Requisitos:}
        \begin{itemize}
            \item (3) Requisitos imprecisos.
        \end{itemize}
    \end{itemize}
    Sigue debajo
\end{solucion}
\begin{solucion}[title=Solucion (2)]
    \begin{itemize}
        \item \textbf{Negocio:}
        \begin{itemize}
            \item (4) Cambios en las condiciones del mercado.
        \end{itemize}

        \item \textbf{Organizativos:}
        \begin{itemize}
            \item (5) Problemas de comunicación entre el equipo del proyecto y los interesados.
            \item (6) Falta de liderazgo en el proyecto.
            \item (11) Falta de financiación para el desarrollo y mantenimiento.
        \end{itemize}

        \item \textbf{Producto:}
        \begin{itemize}
            \item (8) Que la aplicación sea difícil de usar o frustrante para el usuario final.
            \item (9) Fallos que pueden afectar negativamente a la percepción del usuario.
        \end{itemize}

    \end{itemize}

\end{solucion}