% !TeX root = ../calidad.tex


\begin{enunciado}
    Clasifica los siguientes tipos de deuda técnica:
    \begin{enumerate}
        \item Comenzar a realizar el código de la aplicación en base a los requisitos sin tener un diagrama de arquitectura.
        \item Decidir no ejecutar todas las pruebas de validación definidas para poder entregar un cambio regulatorio a tiempo.
        \item Identificar que un componente debería haberse dividido en dos para poder escalar independientemente cada uno de ellos.
        \item Hacer un mal uso de una tecnología de la que no se tiene suficiente conocimiento.
        \item No disponer de un listado actualizado de componentes del sistema.
        \item Des-priorizar la solución de un problema de rendimiento para enfocarse en completar toda la funcionalidad.
    \end{enumerate}
\end{enunciado}
\begin{solucion}
    \begin{description}
        \item[1] \textbf{Temeraria inadvertida.}
        Se comienza el desarrollo sin una arquitectura definida, lo cual genera riesgos importantes no previstos en la integración, escalabilidad y mantenimiento.

        \item[2] \textbf{Temeraria deliberada.}
        Se omiten pruebas de validación conscientemente para cumplir un plazo, aceptando riesgos que comprometen la calidad del producto en producción.

        \item[3] \textbf{Inadvertida (con posible gestión prudente).}
        La necesidad de dividir el componente no se identificó inicialmente.
        Al detectarse, se puede gestionar de forma prudente mediante refactorización.

        \item[4] \textbf{Temeraria inadvertida.}
        Se utiliza incorrectamente una tecnología por falta de conocimiento, lo cual introduce riesgos ocultos que pueden emerger en fases avanzadas.

        \item[5] \textbf{Temeraria (inadvertida o deliberada).}
        La ausencia de un listado actualizado de componentes compromete la trazabilidad y gestión del sistema.
        Puede deberse tanto a descuido como a una decisión consciente de priorizar otras tareas.

        \item[6] \textbf{Prudente deliberada.}
        Se pospone la mejora del rendimiento para priorizar la entrega funcional.
        La deuda está identificada y se asume con la intención de abordarla en el futuro.
    \end{description}
\end{solucion}