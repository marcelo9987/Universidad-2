% !TeX root = ../planificacion.tex

\begin{enunciado}
    Desarrolla el diagrama de Gantt del siguiente conjunto de tareas:

    \begin{itemize}[leftmargin=*]
        \item \textbf{A.} 2 días (No depende de ninguna tarea)
        \item \textbf{B.} 5 días (Depende de la tarea A)
        \item \textbf{C.} 7 días (Depende de la tarea A)
        \item \textbf{D.} 3 días (Depende de la tarea B)
        \item \textbf{E.} 3 días (Depende de la tarea C)
        \item \textbf{F.} 2 días (Depende de la tarea E)
        \item \textbf{G.} 10 días (Depende de la tarea D y F)
        \item \textbf{H.} 5 días (Depende de la tarea G)
        \item \textbf{I.} 3 días (Depende de la tarea H)
        \item \textbf{J.} 1 día (Depende de la tarea I)
    \end{itemize}
\end{enunciado}

\subsection{Análisis de dependencias}\label{subsec:analisis-de-dependencias}

\begin{table}[hbtp]
    \centering
    \caption{Tabla de tareas, duración y dependencias}
    \label{tab:dependencias}
    \begin{tabular}{@{}ccc@{}}
        \toprule
        \textbf{Tarea} & \textbf{Duración (días)} & \textbf{Dependencias} \\
        \midrule
        A              & 2                        & $\emptyset$           \\
        B              & 5                        & A                     \\
        C              & 7                        & A                     \\
        D              & 3                        & B                     \\
        E              & 3                        & C                     \\
        F              & 2                        & E                     \\
        G              & 10                       & D y F                 \\
        H              & 5                        & G                     \\
        I              & 3                        & H                     \\
        J              & 1                        & I                     \\
        \bottomrule
    \end{tabular}


\end{table}

\deactivatequoting

\subsection{Diagrama de Gantt}\label{subsec:diagrama-de-gantt}

\begin{ganttchart}[
    canvas/.append style={fill=none, draw=black!10, line width=.75pt},
    hgrid style/.style={draw=black!10, line width=.75pt},
    vgrid={*1{draw=black!10, line width=.75pt}},
    title/.style={draw=none, fill=none},
    title label font=\bfseries\footnotesize,
    title label anchor/.style={below=7pt},
    include title in canvas=false,
    bar label font=\mdseries\small\color{darkgray},
    bar label anchor/.style={left=2cm},
    bar/.append style={draw=none, fill=primaryblue!70},
    bar height=0.6,
    group height=.5,
    inline
]{1}{33}

% Título principal
    \gantttitle{\textbf{\textcolor{primaryblue}{Diagrama de Gantt del Proyecto}}}{33} \\

% Subtítulos por semanas
    \gantttitle{\textcolor{darkgray}{Semana 1}}{7}
    \gantttitle{\textcolor{darkgray}{Semana 2}}{7}
    \gantttitle{\textcolor{darkgray}{Semana 3}}{7}
    \gantttitle{\textcolor{darkgray}{Semana 4}}{7}
    \gantttitle{\textcolor{darkgray}{Semana 5}}{5} \\

% Días
    \gantttitlelist{1,...,33}{1} \\

% Tareas con colores diferenciados
    \ganttbar[name=tareaA, bar/.append style={fill=red!70}]{A}{1}{2} \\
    \ganttbar[name=tareaB, bar/.append style={fill=blue!70}]{B}{3}{7} \\
    \ganttbar[name=tareaC, bar/.append style={fill=green!70}]{C}{3}{9} \\
    \ganttbar[name=tareaD, bar/.append style={fill=orange!70}]{D}{8}{10} \\
    \ganttbar[name=tareaE, bar/.append style={fill=purple!70}]{E}{10}{12} \\
    \ganttbar[name=tareaF, bar/.append style={fill=pink!70}]{F}{13}{14} \\
    \ganttbar[name=tareaG, bar/.append style={fill=cyan!70}]{G}{15}{24} \\
    \ganttbar[name=tareaH, bar/.append style={fill=yellow!70}]{H}{25}{29} \\
    \ganttbar[name=tareaI, bar/.append style={fill=teal!70}]{I}{30}{32} \\
    \ganttbar[name=tareaJ, bar/.append style={fill=gray!70}]{J}{33}{33} \\

% Enlaces de dependencias
    \ganttlink[link/.style={->, thick, primaryblue!80}]{tareaA}{tareaB}
    \ganttlink[link/.style={->, thick, primaryblue!80}, link mid=0.7]{tareaA}{tareaC}
    \ganttlink[link/.style={->, thick, primaryblue!80}]{tareaB}{tareaD}
    \ganttlink[link/.style={->, thick, primaryblue!80}]{tareaC}{tareaE}
    \ganttlink[link/.style={->, thick, primaryblue!80}]{tareaE}{tareaF}
    \ganttlink[link/.style={->, thick, primaryblue!80}]{tareaD}{tareaG}
    \ganttlink[link/.style={->, thick, primaryblue!80}, link mid=0.3]{tareaF}{tareaG}
    \ganttlink[link/.style={->, thick, primaryblue!80}]{tareaG}{tareaH}
    \ganttlink[link/.style={->, thick, primaryblue!80}]{tareaH}{tareaI}
    \ganttlink[link/.style={->, thick, primaryblue!80}]{tareaI}{tareaJ}

\end{ganttchart}

% \begin{importante}
% La duración total del proyecto es de \textbf{33 días}, determinada por el camino crítico: A → C → E → F → G → H → I → J.
% \end{importante}

\activatequoting