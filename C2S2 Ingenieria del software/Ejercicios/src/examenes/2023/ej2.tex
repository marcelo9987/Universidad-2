% !TeX root = ../examen-parcial-2023.tex

\begin{itemize}
    \item \textbf{Puntos:} 3
\end{itemize}

\begin{enunciado}
    Se han identificado los siguientes requisitos como parte de la mejora de la gestión de la
    producción:
    \begin{enumerate}
        \item Los gestores deben poder acceder al sistema a través de una interfaz web mientras que
        los agricultores deben poder hacerlo mediante una aplicación móvil disponible para iOS\@.
        \item Los gestores de producción deben poder añadir y eliminar campos de cultivo al sistema.
        \item Los agricultores deben poder registrar las labores realizadas en los campos de cultivo
        (arado, siembra, riego, abonado, recolección, etc.) mediante geolocalización.
        \item Los agricultores deben poder notificar incidencias que afecten a la producción (plagas,
        eventos climatológicos, etc.).
        \item El equipo de desarrollo debe poder saber si el sistema está funcionando correctamente.
        \item Los gestores deben poder anotar la producción recolectada en cada campo.
        \item La aplicación debe tener un porcentaje de disponibilidad anual del 99.99\%.
        \item Los gestores deben poder marcar el estado de un campo (barbecho, activo, etc.).
    \end{enumerate}
    Lee detenidamente los requisitos y:
    \begin{enumerate}
        \item Clasifica los requisitos en funcionales, no funcionales u otros.
        \item $0.2$ puntos por cada respuesta correcta.
        \item Desarrolla la especificación del caso de uso de uno de los requisitos que hayas
        clasificado como funcional.
        \item $0.2$ puntos por cada campo simple; $0.3$ puntos por cada campo
    \end{enumerate}
\end{enunciado}
\begin{solucion}
    \begin{enumerate}
        \item Clasificación de los requisitos:
        \begin{itemize}
            \item Requisito 1: No funcional.
            \item Requisito 2: Funcional.
            \item Requisito 3: Funcional.
            \item Requisito 4: Funcional.
            \item Requisito 5: Otros.
            \item Requisito 6: Funcional.
            \item Requisito 7: No funcional.
            \item Requisito 8: Funcional.
        \end{itemize}

        \item Especificación del caso de uso (ejemplo para el requisito 2):
        \begin{itemize}
            \item Nombre: Añadir campo de cultivo.
            \item Actor: Gestor de producción.
            \item Descripción: Los gestores de producción deben poder añadir y eliminar campos de cultivo al sistema.
            \item Precondiciones: El gestor de producción debe haber iniciado sesión en el sistema.
            \item Dependencias: No especificado.
            \item Escenario:
            \begin{enumerate}
                \item El gestor de campo comienza el proceso de añadir un campo.
                \item El gestor de campo rellena los detalles del campo (nombre, localización,\ldots ).
                \item El gestor graba el campo en el sistema.
            \end{enumerate}
            \item Excepciones:
            \begin{enumerate}
                \item El gestor de campo comienza el proceso de añadir un campo.
                \item El gestor de campo no rellena todos los detalles del campo.
                \item El gestor de campo intenta grabar el campo en el sistema.
                \item El sistema indica que faltan detalles del campo.
            \end{enumerate}
        \end{itemize}
    \end{enumerate}
    \begin{itemize}
        \item Prioridad: No especificado.
    \end{itemize}
\end{solucion}