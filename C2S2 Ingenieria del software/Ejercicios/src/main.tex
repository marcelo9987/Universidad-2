%!root = ./main.tex
\documentclass[a4paper,11pt]{article}

% Codificación e idioma
%\usepackage[T1]{fontenc}
\usepackage[spanish]{babel}

% Estilo y herramientas
\usepackage{geometry}
\geometry{margin=2.5cm}
\usepackage{amsmath, amssymb}
\usepackage{enumitem}
\usepackage{tcolorbox}
\usepackage{fancyhdr}
\usepackage{tikz}
\usepackage{hyperref}
\usepackage{graphicx}
\usepackage{booktabs}
\usepackage{csquotes}
\usepackage{xcolor}
\usepackage{pgf-umlsd}

\usetikzlibrary{graphs,shapes,arrows,positioning,shadows,calc}

% Definir estilos
\tikzset{
    service/.style={
        rectangle,
        draw=black,
        fill=white,
        minimum width=2.5cm,
        minimum height=1cm,
        align=center,
        font=\small,
        drop shadow
    },
    database/.style={
        cylinder,
        shape border rotate=90,
        aspect=0.25,
        draw=black,
        fill=gray!20,
        minimum width=0.8cm,
        minimum height=1cm
    },
    interface/.style={
        rectangle,
        draw=black,
        fill=blue!20,
        minimum width=4cm,
        minimum height=0.8cm,
        font=\small\bfseries
    },
    arrow/.style={
        ->,
        >=stealth,
        thick
    },
    bidirectional/.style={
        <->,
        >=stealth,
        thick
    }
}

% Configuración de colores personalizados
\definecolor{primaryblue}{RGB}{25, 118, 210}
\definecolor{lightblue}{RGB}{227, 242, 253}
\definecolor{darkgray}{RGB}{66, 66, 66}
\definecolor{lightgray}{RGB}{245, 245, 245}
\definecolor{successgreen}{RGB}{76, 175, 80}
\definecolor{warningorange}{RGB}{255, 152, 0}

% Configuración mejorada de tcolorbox
\tcbuselibrary{skins,breakable}

% Estilos personalizados para cajas
\newtcolorbox{enunciado}{
    enhanced,
    breakable,
    colback=lightblue,
    colframe=primaryblue,
    arc=4mm,
    boxrule=1.5pt,
    title={\textbf{Enunciado}},
    fonttitle=\Large\bfseries,
    attach boxed title to top left={yshift=-3mm, xshift=4mm},
    boxed title style={
        enhanced,
        colback=primaryblue,
        colframe=primaryblue,
        arc=2mm
    },
    drop shadow,
    left=8pt,
    right=8pt,
    top=12pt,
    bottom=8pt
}

\newtcolorbox{solucion}[1][]{
    enhanced,
    breakable,
    colback=white,
    colframe=successgreen,
    arc=3mm,
    boxrule=1pt,
    title={\textbf{Solución}},
    fonttitle=\large\bfseries,
    attach boxed title to top left={yshift=-2mm, xshift=4mm},
    boxed title style={
        enhanced,
        colback=successgreen,
        colframe=successgreen,
        arc=2mm
    },
    left=6pt,
    right=6pt,
    top=8pt,
    bottom=6pt,
    #1
}

\newtcolorbox{importante}{
    enhanced,
    breakable,
    colback=orange!10,
    colframe=warningorange,
    arc=3mm,
    boxrule=1pt,
    title={\textbf{¡Importante!}},
    fonttitle=\large\bfseries,
    attach boxed title to top left={yshift=-2mm, xshift=4mm},
    boxed title style={
        enhanced,
        colback=warningorange,
        colframe=warningorange,
        arc=2mm
    },
    left=6pt,
    right=6pt,
    top=8pt,
    bottom=6pt
}

%-- --

% Configuración de siunitx
\usepackage[per-mode=symbol, group-minimum-digits=4]{siunitx}
\sisetup{group-separator = {\:\!}}
\DeclareSIUnit\puntosFuncion{PF}
\DeclareSIUnit\porMes{pm}
\DeclareSIUnit\linesofcode{LOC}
\DeclareSIUnit\personameses{persona\text{\textbf{-}}meses}
\DeclareSIUnit\meses{meses}

% Configuración de pgfgantt
\usepackage{pgfgantt}

% Encabezado y pie de página mejorado
\pagestyle{fancy}
\fancyhf{}
\fancyhead[L]{\textcolor{primaryblue}{\textbf{Ingeniería del Software}}}
\fancyhead[R]{\textcolor{darkgray}{Ejercicios para el examen}}
\fancyfoot[C]{\textcolor{darkgray}{\thepage}}
\renewcommand{\headrulewidth}{0.8pt}
\renewcommand{\footrulewidth}{0.4pt}
\renewcommand{\headrule}{\hbox to\headwidth{\color{primaryblue}\leaders\hrule height \headrulewidth\hfill}}

% Configuración de secciones
\setcounter{secnumdepth}{0}

%configuracion de indice
\setcounter{tocdepth}{1}

% para que se calle el pesado de fancyhdr
\setlength{\headheight}{24pt}

% Configuración de hyperref
\hypersetup{
    colorlinks=true,
    linkcolor=primaryblue,
    filecolor=primaryblue,
    urlcolor=primaryblue,
    citecolor=primaryblue,
    pdfpagemode=FullScreen,
}

% Metadatos
\title{\Huge\textbf{Ejercicios} \\ \Large\textcolor{primaryblue}{Ingeniería del Software}}
\author{\textbf{Marcelo Fort Muñoz}}
\date{\today}

\begin{document}

% Página de título personalizada
    \begin{titlepage}
        \centering
        \vspace*{2cm}

        {\Huge\textbf{Ejercicios}}\\[0.5cm]
        {\Large\textcolor{primaryblue}{\textbf{Ingeniería del Software}}}\\[2cm]

        \begin{tikzpicture}
            \draw[primaryblue, line width=2pt] (0,0) -- (8,0);
        \end{tikzpicture}\\[2cm]

        {\Large\textbf{Marcelo Fort Muñoz}}\\[1cm]
        {\large\textcolor{darkgray}{\today}}

        \vfill

        \begin{tikzpicture}
            \draw[lightgray, line width=1pt] (0,0) -- (8,0);
        \end{tikzpicture}
    \end{titlepage}

    \tableofcontents
    \newpage


    \part{\textcolor{primaryblue}{Introducción}}\label{part:introduccion}
    %\setcounter{minitocdepth}{3} % 1 = seccións, 2 = subseccións, 3 = subsubseccións
\localtableofcontents

\section{Conceptos básicos}\label{sec:tema-1.1-conceptos-basicos}

\subsection{Historia: la crisis del software}\label{subsec:historia:-la-crisis-del-software}
\begin{itemize}
    \item Término acuñado en 1968 en una conferencia de la OTAN\@.
    \item Refleja la dificultad para desarrollar software útil y eficiente en tiempo y forma.
    \item Ejemplos de grandes fracasos:
    \begin{itemize}
        \item \textbf{Mariner 1 (1962):} pérdida de $18\text{.}500\text{.}000\$$.
        \item \textbf{Therac-25 (1982):} 3 fallecidos y 3 con secuelas por errores de software.
        \item \textbf{Caída AT\&T (1990):} 75 millones de llamadas afectadas.
        \item \textbf{Knight Capital (2012):} pérdida de $500\text{.}000\text{.}000\$$.
    \end{itemize}
\end{itemize}

\subsection{Definiciones}\label{subsec:definiciones}


\begin{definicion}
    \textbf{Software:}
    \begin{itemize}
        \item Instrucciones que al ejecutarse proporcionan funciones, características y rendimiento.
        \item Estructuras de datos que permiten manipular información adecuadamente.
    \end{itemize}
\end{definicion}

\begin{definicion}
    \textbf{Ingeniería del Software (IEEE, 1993):} Aplicación de un enfoque sistemático, disciplinado y cuantificable al desarrollo, operación y mantenimiento del software.
\end{definicion}

\begin{definicion}
    \textbf{Ingeniería del Software (Fritz Bauer, 1969):} Uso de principios fundamentales de la ingeniería para desarrollar software fiable y eficiente de forma económica.
\end{definicion}

\subsection{Conceptos clave}\label{subsec:conceptos-clave-conceptos-basicos}

\begin{itemize}
    \item \textbf{Personas:} Creadores del producto.
    \item \textbf{Producto:} Resultado final del desarrollo.
    \item \textbf{Usuarios:} Receptores del producto.
    \item \textbf{Proyecto:} Conjunto de hitos hacia un resultado.
    \item \textbf{Proceso:} Fases del desarrollo del software.
    \item \textbf{Ciclo de vida:} Evoluciones del producto en el tiempo.
\end{itemize}

\subsection{Roles en el proceso}\label{subsec:roles-en-el-proceso}
\begin{itemize}
    \item \textbf{Gestor de producto:} Define funcionalidades y prioridades.
    \item \textbf{Gestor de proyecto:} Asegura recursos para cumplir el plan.
    \item \textbf{Ingeniero de software:} Diseña e implementa el software.
    \item \textbf{Ingeniero de calidad:} Verifica el correcto funcionamiento del producto.
\end{itemize}


\section{Estándares y organizaciones}\label{sec:estandares-y-organizaciones}

\subsection{Necesidad de estandarizar}\label{subsec:necesidad-de-estandarizar}
\begin{itemize}
    \item \textbf{Proceso tipo:} Marco para valorar e identificar mejoras.
    \item \textbf{Buenas prácticas:} Lo que funciona se institucionaliza.
    \item \textbf{Lenguaje común:} Mejora la comunicación entre roles.
\end{itemize}

\subsection{Usos de los estándares}\label{subsec:usos-de-los-estandares}
\begin{itemize}
    \item \textbf{Certificación:} Requisito para:
    \begin{itemize}
        \item Acceder a proyectos regulados.
        \item Comercializar productos y servicios.
    \end{itemize}
\end{itemize}

\subsection{Organizaciones relevantes}\label{subsec:organizaciones-relevantes}
\begin{itemize}
    \item \textbf{IEEE:} Institute of Electrical and Electronics Engineers (global).
    \item \textbf{SEI:} Software Engineering Institute (EUA).
    \item \textbf{ISO:} International Organization for Standardization (global).
    \item \textbf{IEC:} International Electrotechnical Commission (global).
    \item \textbf{EIA:} Electronic Industries Alliance (EUA).
\end{itemize}

\subsection{Principales estándares ISO}\label{subsec:principales-estandares-iso}

\begin{itemize}
    \item \textbf{ISO 9126:} Evalúa la calidad del software en:
    \begin{itemize}
        \item Funcionalidad, fiabilidad, usabilidad, eficiencia, mantenibilidad, portabilidad y satisfacción.
    \end{itemize}
    \item \textbf{Familia ISO 9000:} Gestión de calidad
    \begin{itemize}
        \item \textbf{ISO 9000:} Fundamentos y vocabulario.
        \item \textbf{ISO 9001:} Requisitos de calidad.
        \item \textbf{ISO 9004:} Mejora continua.
    \end{itemize}

    \item \textbf{ISO/IEC 12207:} Procesos del ciclo de vida del software: procesos principales, de apoyo y organizativos.

    \item \textbf{ISO/IEC 15504 (SPICE):} Evalúa y mejora procesos de ingeniería del software.
\end{itemize}

\subsection{SPICE – Niveles de madurez}\label{subsec:spice--niveles-de-madurez}

\begin{description}
    \item[Nivel 0 - Incompleto:] Sin implementación efectiva.
    \item[Nivel 1 - Realizado:] Procesos implementados y objetivos alcanzados.
    \item[Nivel 2 - Gestionado:] Procesos y productos controlados.
    \item[Nivel 3 - Establecido:] Procesos basados en estándares.
    \item[Nivel 4 - Predecible:] Gestión cuantitativa con objetivos.
    \item[Nivel 5 - Optimizado:] Mejora continua y búsqueda de buenas prácticas.
\end{description}

\subsection{CMMI (Capability Maturity Model Integration)}\label{subsec:cmmi-(capability-maturity-model-integration)}

\begin{itemize}
    \item Establecido por el SEI\@.
    \item Actualmente gestionado por el CMMI Institute.
    \item Basado en SPICE, muy popular en EUA\@.
\end{itemize}


    \clearpage


    \part{\textcolor{primaryblue}{El proceso del software}}\label{part:el-proceso-del-software}
    %!root =  ../main.tex
\minitoc
    \section{Fases del desarrollo}\label{sec:fases-del-desarrollo}
    \subsection{Metodología y procesos}\label{subsec:metodologia-y-procesos}
Para el desarrollo exitoso de un software se requiere de ciertos \textquote{componentes} que permiten que la \textquote{máquina} funciones sin \textquote{atascarse}.
Estos \textquote{engranajes} son:
\begin{enumerate}
    \item Herramientas: Tanto control de versiones (git, subversion,\dots), como IDEs y otros \textbf{programas o dispositivos} que ayudan al desarrollo.
    \item Métodos: \textbf{Técnicas} usadas para facilitar el desarrollo.
    \item Proceso: \textbf{Actividades, acciones y tareas} realizadas con el fin de crear productos.
\end{enumerate}

\subsection{Proceso esencial}\label{subsec:proceso-esencial}
En un desarrollo habitual, el proceso suele ser el siguiente:

\begin{enumerate}
    \item \textbf{Entender el problema:} Comunicación y análisis (participantes, incógnitas).
    \item \textbf{Planear la solución:} Modelado y diseño (descomposición del problema).
    \item \textbf{Ejecutar el plan:} Creación de código (seguimiento y revisión).
    \item \textbf{Examinar el resultado:} Pruebas y aseguramiento de la calidad.
\end{enumerate}

\subsection{Definiciones}\label{subsec:definiciones}
\begin{definicion}
    Un \textbf{proceso} es aquel \textbf{conjunto de actividades, acciones y tareas} que se ejecutan cuando va a crearse algún producto del trabajo.
\end{definicion}
\begin{definicion}
    Una \textbf{actividad} busca lograr un \textbf{objetivo amplio} sin importar el dominio de la aplicación, tamaño del proyecto, complejidad del esfuerzo.
\end{definicion}

\begin{definicion}
    Una \textbf{acción} es un \textbf{conjunto de tareas} que producen un producto importante del trabajo.
\end{definicion}
\begin{definicion}
    Una \textbf{tarea} se centra en un objetivo \textbf{pequeño pero bien definido} que produce un \textbf{resultado tangible}.
\end{definicion}
\begin{definicion}
    Un \textbf{proceso} no es una prescripción rígida.
    Es un enfoque\textbf{adaptable} que permite entregar el software de \textbf{forma oportuna} y con \textbf{calidad suficiente}.
\end{definicion}

\subsection{Fases del proceso}\label{subsec:fases-del-proceso}
El proceso de ingeniería del software se divide en 5 sencillas fases: \textbf{Comunicación}, \textbf{planificación}, \textbf{diseño}, \textbf{desarrollo} y, \textbf{despliegue}.
Se puede observar que es y que objetivo tiene cada fase en la \autoref{tab:fases-desarrollo}

\begin{table}[hbtp]
    \centering

    \begin{tabularx}{\textwidth}{|l l X|}\toprule
    Fase & Objetivo& Detalles \\\midrule

    \textbf{Comunicación} & Entender requisitos.& Colaboración con participantes, definición de características. \\
    \textbf{Planificación} & Definir el traballo& Tareas técnicas, riesgos, recursos, cronograma.\\
    \textbf{Diseño}& Plantear solución & Diagramas de alto nivel, refinamiento iterativo.\\
    \textbf{Desarrollo}& Construír y verificar& Porgramación y pruebas.\\
    \textbf{Despliegue}& Entregar produto & Evaluación y \textit{feedback}.\\ \bottomrule

    \end{tabularx}
    \caption{Comparación de las distintas fases del proceso de desarrollo de software}
    \label{tab:fases-desarrollo}

\end{table}


\subsection{Tipos de flujo de fases}\label{subsec:tipos-de-flujo-de-fases}
Existen varios tipos de flujo de fases.
A continuación, varios diagramas con los distintos flujos:

\subsubsection{Lineal}

De una en una, sencillo.

\deactivatequoting
\tikz
{
    \node [rectangle, draw] (A) {Comunicación};
    \node [rectangle, draw] (B) [right= of A] {Planeación};
    \node [rectangle, draw] (C) [right=of B] {Modelado};
    \node [rectangle, draw] (D) [right=of C] {Construcción};
    \node [rectangle, draw] (E) [right=of D] {Despliegue};

    \draw[
        -{Latex}
    ,draw=black
    , thick
    ]
    % Básicos
    (A) edge (B)
    (B) edge (C)
    (C) edge (D)
    (D) edge (E)
}
\activatequoting

\subsubsection{Iterativo}

\deactivatequoting
\tikz
{
    \node [rectangle, draw] (A) {Comunicación};
    \node [rectangle, draw] (B) [right= of A] {Planeación};
    \node [rectangle, draw] (C) [right=of B] {Modelado};
    \node [rectangle, draw] (D) [right=of C] {Construcción};
    \node [rectangle, draw] (E) [right=of D] {Despliegue};

    \draw[
        -{Latex}
    ,draw=black
    , thick
    ]
    % Básicos
    (A) edge (B)
    (B) edge (C)
    (C) edge (D)
    (D) edge (E)

%Iterativo
    (B) edge[bend left = 45] (A)
    (C) edge[in=-170, out=-10,looseness=5] (C)
    (D) edge[bend left = 45] (A)
}
\activatequoting

\subsubsection{Evolutivo}
Permite un prototipado rápido.
Resiliente y con una planificación adaptativa.

\deactivatequoting
\tikz
{
    \node [rectangle, draw] (A) {Comunicación};
    \node [rectangle, draw] (B) [above right    = of A] {Planeación};
    \node [rectangle, draw] (C) [below right    = of B] {Modelado};
    \node [rectangle, draw] (D) [below          = of C] {Construcción};
    \node [rectangle, draw] (E) [below left     = of D] {Despliegue};
    \node [rectangle]       (F) [left           = of E] {Incremento obtenido};

    \draw[
        -{Latex}
    ,draw=black
    , thick
    ]
    % Básicos
    (A) edge (B)
    (B) edge (C)
    (C) edge (D)
    (D) edge (E)
    (E) edge (A)
    (E) edge (F)

}
\activatequoting

\subsubsection{Paralelo}
Fases en paralelo para optimizar la eficiencia

\deactivatequoting
\tikz
{
    \node [rectangle, draw] (A) {Comunicación};
    \node [rectangle, draw] (B) [right          = of A] {Planeación};
    \node [rectangle, draw] (C) [below right    = of A] {Modelado};
    \node [rectangle, draw] (D) [below right    = of C] {Construcción};
    \node [rectangle, draw] (E) [right          = of D] {Despliegue};
    \node [rectangle]       (F) [right          = of C] {Tiempo};

    \draw[
        -{Latex}
    ,draw=black
    , thick
    ]
    % Básicos
    (A) edge (B)
    (A) edge (C)
    (B) edge (C)
    (C) edge (D)
    (D) edge (E)

}
\activatequoting

\subsection{Actividades transversales}\label{subsec:actividades-transversales}
Las siguientes tareas se llevan a cabo de forma mas o menos contínua durante la duración total del proyecto:
\begin{itemize}
    \item \textbf{Seguimiento y control:} Evaluación del estado del proyecto con respecto al plan original.

    \item \textbf{Administración de riesgos:} Identificación y mitigación.

    \item \textbf{Seguimiento de calidad del software:} Revisiones y estándares que \textbf{garantizan} que el software tiene la calidad prometida.


    \item \textbf{Medición:} Métricas de proceso, proyecto y produto.

    \item \textbf{Gestión de la configuración:} Control de cambios.

    \item \textbf{Reutilización:} Creación e uso de componentes reutilizables.
    \item \textbf{Preparación y producción del producto del trabajo:} Actividades  para crear productos del trabajo, tales como modelos y documentación.

\end{itemize}

    \section{Modelos del proceso}\label{sec:modelos-del-proceso}
        No todos los proyectos son iguales.
    En este tema veremos los modelos de procesos mas usados.

    \clearpage
    \subsection{Modelo cascada: Versión V}\label{subsec:modelo-cascada:-version-v}

    Tiene fases secuenciales y lineales.
    Las pruebas y el desarrollo funcionan en paralelo, de ahí la forma de V\@.

    \begin{tikzpicture}[
        node distance=1.5cm,
        box/.style={rectangle, draw, thick, minimum width=3cm, minimum height=1.2cm, align=center, fill=white},
        arrow/.style={->, thick, >=Stealth}
    ]

% Lado izquierdo (desarrollo)
        \node[box] (req) {Modelado de los\\requerimientos};
        \node[box, below=of req] (arch) {Diseño de la\\arquitectura};
        \node[box, below=of arch] (comp) {Diseño de los\\componentes};
        \node[box, below=of comp] (code) {Generación\\de código};

% Lado derecho (pruebas)
        \node[box, right=6cm of req] (accept) {Pruebas de\\aceptación};
        \node[box, below=of accept] (system) {Pruebas\\del sistema};
        \node[box, below=of system] (integration) {Pruebas de\\integración};
        \node[box, below=of integration] (unit) {Pruebas\\unitarias};

% Nodo final
        \node[below=1.5cm of code, xshift=3cm] (software) {Software ejecutable};

% Flechas verticales lado izquierdo
        \draw[arrow] (req) -- (arch);
        \draw[arrow] (arch) -- (comp);
        \draw[arrow] (comp) -- (code);

% Flechas verticales lado derecho
        \draw[arrow] (unit) -- (integration);
        \draw[arrow] (integration) -- (system);
        \draw[arrow] (system) -- (accept);

% Flechas horizontales (correspondencias)
        \draw[arrow]  (accept)      -- (req);
        \draw[arrow]  (system)      -- (arch);
        \draw[arrow]  (integration) -- (comp);
        \draw[arrow] (unit)         -- (code);

% Flechas hacia el software ejecutable
        \draw[arrow] (code) -- (software);
        \draw[arrow]  (software) -- (unit);

% Flechas diagonales de entrada y salida
        \draw[arrow] (-1.5, 2) -- (req);
        \draw[arrow] (accept) -- (9.5, 2);

    \end{tikzpicture}

    \clearpage
    \subsection{Modelo de proceso incremental}\label{subsec:modelo-de-proceso-incremental}
    El proceso se divide en incrementos que incluyen todas las fases.
    Esto implica que todas las entregas intermedias (incremento x) aseguran un cierto nivel de calidad.

    \begin{tikzpicture}[
        node distance=0.3cm,
        phase/.style={rectangle, draw, thick, minimum width=1.2cm, minimum height=0.8cm, align=center},
        arrow/.style={->, thick, >=Stealth},
        legend/.style={rectangle, draw, minimum width=0.4cm, minimum height=0.4cm}
    ]

% Ejes
        \draw[thick, ->] (0,0) -- (13,0) node[midway,right,anchor=north] {Calendario del proyecto};
        \draw[thick, ->] (0,0) -- (0,14) node[midway,above, rotate=90, anchor=south] {Funcionalidad y características del software};

% Leyenda
        \node[legend, fill=white] at (1,10) {};
        \node[right=0.1cm] at (1.2,10) {Comunicación};

        \node[legend, fill=gray!20] at (1,9.5) {};
        \node[right=0.1cm] at (1.2,9.5) {Planeación};

        \node[legend, fill=gray!40] at (1,9) {};
        \node[right=0.1cm] at (1.2,9) {Modelado (análisis, diseño)};

        \node[legend, fill=gray!60] at (1,8.5) {};
        \node[right=0.1cm] at (1.2,8.5) {Construcción (código, prueba)};

        \node[legend, fill=gray!80] at (1,8) {};
        \node[right=0.1cm] at (1.2,8) {Despliegue (entrega, retroalimentación)};

        \newcommand{\desplaltura}{1.00cm}

% Incremento #1
        \node[phase, fill=white] (c1) at (2,2) {};
        \node[phase, fill=gray!20, below right = 0.5cm of c1,yshift=\desplaltura] (p1) {};
        \node[phase, fill=gray!40, below right = 0.5cm of p1,yshift=\desplaltura] (m1) {};
        \node[phase, fill=gray!60, below right = 0.5cm of m1,yshift=\desplaltura] (co1) {};
        \node[phase, fill=gray!80, below right = 0.5cm of  co1,yshift=\desplaltura] (d1) {};

        \draw[arrow] (c1) -- (p1);
        \draw[arrow] (p1) -- (m1);
        \draw[arrow] (m1) -- (co1);
        \draw[arrow] (co1) -- (d1);

        \node[right=2.0cm of co1] (e1) {entrega del primer};
        \node[right=2.0cm of co1, yshift=-0.4cm] {incremento};

        \node[above = 0.25cm of c1] {incremento \# 1};

% Incremento #2
        \node[phase, fill=white] (c2) at (3.5,4.00) {};
        \node[phase, fill=gray!20, below right = 0.5cm of c2,yshift=\desplaltura] (p2) {};
        \node[phase, fill=gray!40, below right = 0.5cm of p2,yshift=\desplaltura] (m2) {};
        \node[phase, fill=gray!60, below right = 0.5cm of m2,yshift=\desplaltura] (co2) {};
        \node[phase, fill=gray!80, below right = 0.5cm of co2,yshift=\desplaltura] (d2) {};

        \draw[arrow] (c2) -- (p2);
        \draw[arrow] (p2) -- (m2);
        \draw[arrow] (m2) -- (co2);
        \draw[arrow] (co2) -- (d2);

        \node[above = 0.2cm of c2] {incremento \# 2};
        \node[right = 0.5cm of d2] {entrega del segundo};
        \node[right = 0.5cm of d2,yshift=-0.4cm] {incremento};

% Puntos suspensivos
        \fill (8,4.2) circle (0.08);
        \fill (8.3,4.6) circle (0.08);
        \fill (8.6,5) circle (0.08);

% Incremento #n
        \node[phase, fill=white] (cn) at (7,6) {};
        \node[phase, fill=gray!20, right=of cn] (pn) {};
        \node[phase, fill=gray!40, right=of pn] (mn) {};
        \node[phase, fill=gray!60, right=of mn] (con) {};
        \node[phase, fill=gray!80, right=of con] (dn) {};

        \draw[arrow] (cn) -- (pn);
        \draw[arrow] (pn) -- (mn);
        \draw[arrow] (mn) -- (con);
        \draw[arrow] (con) -- (dn);

        \node[above left=0.2cm of cn] {incremento \# n};
        \node[right = 0.1cm of dn] {entrega del n-ésimo};
        \node[right = 0.1cm of dn,yshift=-0.4cm] {incremento};

    \end{tikzpicture}
    \clearpage

    \clearpage
    \subsection{Proceso evolutivo}\label{subsec:proceso-evolutivo}
    Con forma de \textquote{espiral}, permite adaptar los requisitos y solucionar los problemas de iteraciones antiguas.

    \begin{tikzpicture}[
        node distance=2cm,
        box/.style={rectangle, draw, thick, minimum width=2.5cm, minimum height=1.5cm, align=center, fill=white},
        arrow/.style={->, very thick, >=Stealth, gray!70, fill=gray!50},
        cycle arrow/.style={very thick, gray!70, fill=gray!50}
    ]

% Nodos principales
        \node[box] (comunicacion) at (0,3) {Comunicación};
        \node[box] (plan) at (4,5) {Plan rápido};
        \node[box] (modelado) at (7,2) {Modelado\\Diseño rápido};
        \node[box] (construccion) at (4,-1) {Construcción\\del\\prototipo};
        \node[box] (despliegue) at (0,0) {Despliegue\\Entrega y\\Retroalimentación};

% Flecha curva grande de comunicación a plan
        \draw[cycle arrow] (comunicacion.north east)
        to[out=45, in=135]
        node[single arrow, draw, thick, fill=gray!50, minimum height=1.5cm, single arrow head extend=0.3cm, rotate=45] {}
        (plan.north west);

% Flecha de plan a modelado
        \draw[cycle arrow] (plan.south east)
        to[out=-45, in=45]
        node[single arrow, draw, thick, fill=gray!50, minimum height=1.2cm, single arrow head extend=0.3cm, rotate=-45] {}
        (modelado.north east);

% Flecha de modelado a construcción
        \draw[cycle arrow] (modelado.south)
        to[out=-90, in=45]
        node[single arrow, draw, thick, fill=gray!50, minimum height=1.5cm, single arrow head extend=0.3cm, rotate=-90] {}
        (construccion.east);

% Flecha de construcción a despliegue
        \draw[cycle arrow] (construccion.west)
        to[out=180, in=-45]
        node[single arrow, draw, thick, fill=gray!50, minimum height=1.2cm, single arrow head extend=0.3cm, rotate=135] {}
        (despliegue.south east);

% Flecha de despliegue a comunicación (completando el ciclo)
        \draw[cycle arrow] (despliegue.north)
        to[out=135, in=225]
        node[single arrow, draw, thick, fill=gray!50, minimum height=1.2cm, single arrow head extend=0.3cm, rotate=90] {}
        (comunicacion.south west);

    \end{tikzpicture}


    \clearpage
    \subsection{Modelo Concurrente}\label{subsec:modelo-concurrente}
    Funciona con el diseño y el desarrollo en paralelo.
    Muy versátil.


    \begin{tikzpicture}[
        node distance=2.5cm,
        state/.style={rectangle, draw, very thick, rounded corners=0.5cm, minimum width=2.5cm, minimum height=1cm, align=center, fill=white},
        arrow/.style={->, thick, >=Stealth},
        container/.style={rectangle, draw, very thick, rounded corners=1cm, fill=gray!20, minimum width=12cm, minimum height=10cm}
    ]

% Contenedor principal
        \node[container] (container) at (4,0) {};


% Estado inicial (fuera del contenedor)
        \node[state] (inactivo) at (4,6.5) {Inactivo};

% Estados dentro del contenedor
        \node[state] (desarrollo) at (4,3) {En\\desarrollo};
        \node[state] (espera) at (0.5,0.5) {Cambios\\en espera};
        \node[state] (evaluacion) at (0.5,-2) {En\\evaluación};
        \node[state] (revision) at (7.5,0.5) {En revisión};
        \node[state] (alcance) at (7.5,-2) {Alcance mínimo};
        \node[state] (terminado) at (4,-3.5) {Terminado};

% Flecha de entrada
        \draw[arrow] (inactivo) -- (desarrollo);

% Flechas internas del ciclo
        \draw[arrow] (desarrollo) -- (espera);
        \draw[arrow] (espera) -- (evaluacion);
        \draw[arrow] (evaluacion) -- (terminado);
        \draw[arrow] (desarrollo) -- (revision);
        \draw[arrow] (revision) -- (alcance);
        \draw[arrow] (alcance) -- (terminado);
        \draw[arrow] (revision) -- (evaluacion);

% Flecha de retroalimentación
        \draw[arrow] (evaluacion) to[out=120, in=240] (espera);

% Etiqueta explicativa
        \node[right=0.3cm of desarrollo, text width=3.5cm, font=\small] {Representa el estado\\de una actividad o\\tarea de la ingeniería\\de software};

% Línea de conexión de la etiqueta
        \draw[-] (desarrollo.east) -- ++(0.3,0);

    \end{tikzpicture}

    \clearpage
    \subsection{Proceso unificado}\label{subsec:proceso-unificado}

    Compuesta por fases múltiples con ciclos cortos.

    \begin{tikzpicture}[
        node distance=2cm,
        phase/.style={rectangle, draw, very thick, minimum width=2cm, minimum height=0.8cm, align=center, fill=gray!30, drop shadow={shadow xshift=0.2cm, shadow yshift=-0.2cm, fill=black}},
        arrow/.style={->, very thick, >=Stealth},
        spiral/.style={very thick, black}
    ]

% Fases en espiral (empezando desde comunicación)
        \node[phase] (comunicacion) at (-3,-1) {comunicación};
        \node[phase] (planeacion) at (-1,2) {planeación};
        \node[phase] (modelado) at (3,2.5) {modelado};
        \node[phase] (construccion) at (4,-1) {construcción};
        \node[phase] (despliegue) at (1,-3) {despliegue};

% Incremento del software (centro-abajo)
        \node[phase, fill=white] (incremento) at (0,-5) {incremento del software};

% Etiquetas de las grandes fases
        \node[left=0.3cm of comunicacion,yshift=2cm, font=\large\bfseries\itshape] (concepcion) {Concepción};
        \node[above=1.0cm of planeacion,xshift=1.2cm, font=\large\bfseries\itshape](elaboracion) {Elaboración};
        \node[above=0.3cm of construccion,xshift=3.5cm, font=\large\bfseries\itshape] (construccion_) {Construcción};
        \node[below=0.3cm of incremento, font=\large\bfseries\itshape] (produccion) {Producción};

        \draw (concepcion) -- (comunicacion);
        \draw (concepcion) -- (planeacion);

        \draw (elaboracion) -- (planeacion);
        \draw (elaboracion) -- (modelado);

        \draw (construccion_) -- (construccion);




        \node[left=0.5cm of incremento, font=\large\bfseries] {Lanzamiento};
        \node[right=1cm of construccion, font=\large\bfseries\itshape](transicion) {Transición};

        \draw (transicion) -- (construccion);
        \draw (transicion) -- (despliegue);

        \draw (produccion) -- (incremento);

% Espiral principal
        \draw[spiral, ->] (comunicacion.center)
        to[out=60, in=180] (planeacion.center)
        to[out=0, in=120] (modelado.center)
        to[out=-60, in=60] (construccion.center)
        to[out=-120, in=30] (despliegue.center)
        to[out=210, in=90] (incremento.center);

% Flecha curva de retroalimentación
        \draw[spiral, ->] (despliegue.west)
        to[out=180, in=-60, looseness=1.5] (comunicacion.south);

    \end{tikzpicture}

    \subsection{Otros modelos}\label{subsec:otros-modelos}

    \subsubsection{Modelos especializados}

    \begin{itemize}
        \item \textbf{Desarrollo basado en componentes}: utiliza un enfoque evolutivo y iterativo
        \item \textbf{Modelo de métodos formales}: basado en modelos matemáticos para especificación rigurosa
        \item \textbf{Desarrollo orientado a aspectos}: combina enfoques evolutivos y concurrentes
    \end{itemize}

    \subsubsection{Modelos personales y de equipo}

    \begin{itemize}
        \item \textbf{Proceso Personal del Software (PPS)}: metodología individual que incluye:
        \begin{itemize}
            \item Planificación
            \item Diseño
            \item Revisión
            \item Desarrollo
            \item Post mórtem
        \end{itemize}
        \item \textbf{Proceso del Equipo de Software (PES)}: equipos auto-dirigidos que se gestionan autónomamente.
    \end{itemize}


    \subsection{Ciclo de vida del producto}\label{subsec:ciclo-de-vida-del-producto}

    \textbf{Etapas}: Introdución → Crecemento → Madurez → Declive.

    \section{Desarrollo ágil}\label{sec:desarrollo-agil}
        \subsection{Definiciones}\label{subsec:definiciones-agiles}
    \begin{definicion}
        \textbf{Agilidad:} Capacidad de \textbf{adaptación al cambio} (requisitos, equipo, tecnologías,…)
    \end{definicion}

    \begin{definicion}
        \textbf{Agilismo:} Métodos para alcanzar agilidad.
    \end{definicion}

    \subsection{El coste del cambio}\label{subsec:el-coste-del-cambio}
    Los métodos ágiles buscan reducir el coste del cambio a lo largo del ciclo de vida del proyecto, evitando que aumente exponencialmente en las fases tardías.

    \subsection{Características de los métodos ágiles}\label{subsec:caracteristicas-de-los-metodos-agiles}

    \begin{itemize}
        \item \textbf{Compatibles con estándares}: CMMI, ISO, etc.
        \item \textbf{Diferenciación}: Los estándares indican el QUÉ hacer, el agilismo indica el CÓMO hacerlo
        \item \textbf{Modelos iterativos y adaptativos}: pueden ser incrementales o evolutivos
        \item \textbf{Equipos multidisciplinares}: autónomos y autoorganizados.
        \item \textbf{Guiados por el Manifiesto Ágil} (2001)
        \item \textbf{Múltiples metodologías}: adopción flexible según necesidades
    \end{itemize}

    \subsection{Manifiesto ÁGIL}\label{subsec:manifiesto-agil}

    Cuatro valores fundamentales (priorizando los de la izquierda sobre los de la derecha):

    \begin{enumerate}
        \item \textbf{Individuos e interacciones} sobre procesos y herramientas
        \item \textbf{Software funcionando} sobre documentación extensiva
        \item \textbf{Colaboración con el cliente} sobre negociación contractual
        \item \textbf{Respuesta ante el cambio} sobre seguir un plan
    \end{enumerate}

    \subsection{Principios del Manifiesto Ágil}\label{subsec:principios-del-manifiesto-agil}

    \begin{enumerate}
        \item \textbf{Prioridad}: Satisfacer al cliente mediante entrega temprana y continua de software con valor.
        \item \textbf{Cambios}: Aceptar que los requisitos cambien, incluso en etapas tardías.
        \item \textbf{Entregas frecuentes}: Software funcional cada 2 semanas a 2 meses (preferiblemente más corto).
        \item \textbf{Colaboración diaria}: Responsables de negocio y desarrolladores trabajan juntos.
        \item \textbf{Individuos motivados}: Dar entorno y apoyo, confiar en la ejecución.
        \item \textbf{Comunicación cara a cara}: Método más eficiente y efectivo.
        \item \textbf{Software funcionando}: Principal medida de progreso.
        \item \textbf{Desarrollo sostenible}: Mantener ritmo constante indefinidamente.
        \item \textbf{Excelencia técnica}: Atención continua al buen diseño mejora la agilidad.
        \item \textbf{Simplicidad}: Arte de maximizar la cantidad de trabajo no realizado.
        \item \textbf{Equipos auto-organizados}: Las mejores arquitecturas, requisitos y diseños emergen de ellos.
        \item \textbf{Reflexión regular}: El equipo reflexiona para ser más efectivo y ajustar comportamiento.
    \end{enumerate}


    \subsection{Roles en el desarrollo ágil}\label{subsec:roles-en-el-desarrollo-agil}

    \begin{itemize}
        \item \textbf{Agile Coach}: Experto en agilismo que ayuda a los empleados a adoptar metodologías ágiles
        \item \textbf{Product Owner}: Gestor de la pila de trabajo y su prioridad para maximizar valor entregado
        \item \textbf{Scrum Master}: Facilitador de los equipos que siguen metodología Scrum
        \item \textbf{Equipo de desarrollo}: Conjunto de miembros que desarrollan y entregan software en incrementos de valor
    \end{itemize}


    \subsection{Programación extrema (XP)}\label{subsec:programacion-extrema-(xp)}

    La Programación Extrema fue creada por Kent Beck, quien también fue contribuidor al manifiesto ágil.
    Se caracteriza por llevar las buenas prácticas de programación a sus límites extremos.

    Busca \textbf{retroalimentación continua} del usuario mediante entregas cortas y frecuentes, lo que permite \textbf{detectar y corregir problemas rápidamente}.
    La \textbf{documentación es simple} y se basa en tres principios: mantener \textbf{código simple y mantenible}, priorizar \textbf{código autodocumentado} sobre comentarios extensos, y usar\textbf{ tests unitarios }como mecanismo de diseño y documentación.

    La \textbf{programación por parejas} implica que dos desarrolladores trabajen juntos en el mismo código, lo que mejora la calidad y facilita la transferencia de conocimiento.
    El \textbf{énfasis en pruebas} se materializa a través del Test Driven Development (TDD), donde las pruebas se escriben antes que el código, y se eliminan defectos antes de añadir nueva funcionalidad.

    El principio \textbf{YAGNI} ("You Aren't Gonna Need It") promueve programar solo para las prioridades inmediatas, evitando la sobreingeniería y el desarrollo de funcionalidades que podrían no ser necesarias.


    \subsection{SCRUM}\label{subsec:scrum}

    Scrum fue concebido a principios de la década de 1990 por Jeff Sutherland, otro contribuidor al manifiesto ágil.
    Se estructura en \textbf{iteraciones de 2 a 4 semanas} de duración, donde cada iteración debe terminar con una entrega de valor tangible al cliente.

    Una característica fundamental es que \textbf{el alcance no puede modificarse} durante el desarrollo de la iteración, lo que proporciona estabilidad al equipo.
    Incorpora un \textbf{proceso de mejora continua} para incrementar la eficiencia del equipo mediante retrospectivas regulares.


    \subsubsection{Ceremonias SCRUM}

    El \textbf{Sprint Planning} es la sesión de planificación donde el equipo decide qué elementos del product backlog se desarrollarán en la siguiente iteración, basándose en la priorización establecida por el Product Owner.

    El \textbf{Daily Scrum} es una reunión diaria de máximo 15 minutos donde cada miembro del equipo comparte tres elementos: el \textbf{progreso realizado desde la anterior reunión}, el \textbf{progreso esperado hasta la siguiente reunión}, y \textbf{cualquier bloqueo o impedimento que esté enfrentando.}

    El \textbf{Sprint Review} es la sesión donde se revisa la iteración completada mediante una demostración del software funcional desarrollado.

    El \textbf{Refinamiento} son sesiones dedicadas a revisar y clarificar requisitos para alcanzar un entendimiento común entre todos los miembros del equipo.

    La \textbf{Retrospectiva} es una sesión de revisión del proceso utilizado durante la iteración, donde el equipo identifica qué funcionó bien, qué puede mejorarse, y define acciones concretas de mejora.



    \subsubsection{Artefactos SCRUM}

    El \textbf{Product Backlog} es la pila de trabajo que contiene todos los requisitos y funcionalidades a cumplir, ordenados por prioridad y valor de negocio.

    El \textbf{Sprint Backlog} contiene específicamente los requisitos que se van a desarrollar en la iteración próxima, con el nivel de detalle necesario para su implementación.

    El \textbf{Scrum Board} es un tablero visual que muestra el estado actual de todas las tareas, típicamente organizado en columnas como \textquote{Por hacer}, \textquote{En progreso} y \textquote{Terminado}.

    El \textbf{Burndown Chart} es un gráfico que compara el progreso ideal de una iteración con el progreso real, permitiendo identificar desviaciones y tomar medidas correctivas.

    La \textbf{Definition of Ready (DoR)} establece los criterios que debe cumplir una tarea para estar lista para ser iniciada por el equipo de desarrollo.

    La \textbf{Definition of Done (DoD)} define los criterios que debe cumplir una tarea para poder ser marcada como completada y entregada.



    \subsection{KANBAN}\label{subsec:kanban}

    Kanban fue definido por primera vez en 2007 y está basado en el proceso de gestión visual desarrollado por Toyota para la manufactura.
    Sigue el proceso \textbf{Kaizen} de mejora continua y se caracteriza por \textbf{no tener iteraciones} \textbf{ni ceremonias preestablecidas}, siendo \textbf{más flexible en su estructura}.

    \subsubsection{Principios KANBAN:}

    Los \textbf{principios de gestión del cambio} establecen que se debe comenzar con lo que se hace actualmente (sin cambios disruptivos), aceptar el cambio incremental y evolutivo (evitando transformaciones radicales), y fomentar actos de liderazgo a todos los niveles de la organización:
    \begin{itemize}
        \item Comienza con lo que haces ahora
        \item Aceptar el cambio incremental y evolutivo
        \item Fomentar los actos de liderazgo a todos los niveles
    \end{itemize}

    Los \textbf{principios de prestación de servicios} se centran en las necesidades y expectativas del cliente como foco principal, gestionar el trabajo y los procesos en lugar de microgestionar a los trabajadores, y revisar periódicamente toda la red de servicios para optimizar el flujo de valor:
    \begin{itemize}
        \item Centrarte en las necesidades y expectativas del cliente
        \item  Gestionar el trabajo, no los trabajadores
        \item Revisar periódicamente la red de servicios
    \end{itemize}

    \subsubsection{Prácticas KANBAN:}

    \textbf{Visualizar el flujo de trabajo} mediante tableros que muestren claramente el estado de todas las tareas y su progreso a través del proceso.

    \textbf{Limitar el trabajo en curso} (WIP - Work In Progress) para evitar la sobrecarga del sistema y mejorar el flujo.

    \textbf{Gestionar el flujo} monitorizando y optimizando el movimiento del trabajo a través del sistema.

    \textbf{Explicitar las políticas de procesos} para que todos entiendan claramente cómo funciona el sistema.

    \textbf{Aplicar bucles de retroalimentación} para obtener información sobre el rendimiento del sistema y áreas de mejora.

    \textbf{Mejorar en colaboración} mediante el trabajo conjunto de todo el equipo para optimizar el sistema.


    \subsection{Otras metodologías ágiles}\label{subsec:otras-metodologias-agiles}

    \textbf{Lean Software Development (LSD)} está basado en los principios del Lean Startup y se enfoca en eliminar desperdicios, amplificar el aprendizaje y entregar valor rápidamente.

    \textbf{Desarrollo Adaptativo de Software (DAS)} es un enfoque que asume que los proyectos de software son inherentemente impredecibles y se adapta continuamente a los cambios.

    \textbf{Agile Unified Process} es una versión simplificada y ágil del Proceso Unificado tradicional, manteniendo sus fortalezas pero eliminando su rigidez.

    \textbf{Crystal Clear} es una metodología ligera diseñada específicamente para equipos pequeños, enfocándose en la comunicación y la simplicidad.

    \textbf{PMI Agile} es el enfoque ágil desarrollado por el Project Management Institute, integrando prácticas ágiles con la gestión de proyectos tradicional.





% ****************** CAPITULO -- FIN -- CAPITULO -- FIN **************




    \clearpage


    \part{\textcolor{primaryblue}{Modelado}}\label{part:modelado}
    % !TeX root = ../main.tex


\section{Ejercicio 1: Requisitos}\label{sec:ejercicio-1:-requisitos}
% !TeX root = ../calidad.tex

\begin{enunciado}
    Calcula el número de errores que verán los usuarios finales para un proyecto en el que sucede lo siguiente:
    \begin{itemize}
        \item En el modelo de requerimientos se han cometido 10 errores y cada uno se amplifica en un factor de 2:1 en el diseño.
        \item En el diseño se cometen otros 20 errores adicionales, que luego se amplifican en un factor de 1.5:1 en el código.
        \item En el código se cometen otros 30 errores adicionales.
        \item Las pruebas unitarias encuentran un 20\% de todos los errores.
        \item Las pruebas de integración descubren el 50\% de los errores restantes.
        \item Las pruebas de validación/aceptación hallan el 50\% de los que queden.
        \item Tras las pruebas se pone el nuevo software en marcha.
    \end{itemize}
\end{enunciado}

\subsection{Datos}\label{subsec:datos}
\begin{itemize}
    \item Errores en requisitos: $10$ errores.

    Amplificación en diseño: $10 \cdot 2 = 20$ errores.


    \item Errores de diseño adicionales: $20$ errores.


    Total de errores diseño: $20 + 20 = 40$ errores.

    Amplificación en código: $40 \cdot 1.5 = 60$ errores.


    \item Errores adicionales en código: $30$ errores.

    Total de errores en código: $60 + 30 = 90$ errores.


\end{itemize}

\hrulefill

\subsection{Cálculo de errores tras las pruebas}\label{subsec:calculo-de-errores-tras-las-pruebas}
\begin{itemize}
    \item Pruebas unitarias detectan el 20\%: $90 \cdot 0.20 = 18$ errores detectados.
    Errores restantes: $90 - 18 = 72$.
    \item Pruebas de integración detectan el 50\% de los restantes: $72 \cdot 0.50 = 36$ errores detectados.
    Errores restantes: $72 - 36 = 36$.
    \item Pruebas de validación detectan el 50\% de los restantes: $36 \cdot 0.50 = 18$ errores detectados.
    Errores restantes: $36 - 18 = 18$.
\end{itemize}

\begin{solucion}
    El total de errores que verán los usuarios finales es de \boxed{\textbf{18 errores}}.
\end{solucion}



\section{Ejercicio 2: Casos de uso e historias de usuario}\label{sec:ejercicio-2:-casos-de-uso-e-historias-de-usuario}
% !TeX root = ../examen-parcial-2023.tex

\begin{itemize}
    \item \textbf{Puntos:} 3
\end{itemize}

\begin{enunciado}
    Se han identificado los siguientes requisitos como parte de la mejora de la gestión de la
    producción:
    \begin{enumerate}
        \item Los gestores deben poder acceder al sistema a través de una interfaz web mientras que
        los agricultores deben poder hacerlo mediante una aplicación móvil disponible para iOS\@.
        \item Los gestores de producción deben poder añadir y eliminar campos de cultivo al sistema.
        \item Los agricultores deben poder registrar las labores realizadas en los campos de cultivo
        (arado, siembra, riego, abonado, recolección, etc.) mediante geolocalización.
        \item Los agricultores deben poder notificar incidencias que afecten a la producción (plagas,
        eventos climatológicos, etc.).
        \item El equipo de desarrollo debe poder saber si el sistema está funcionando correctamente.
        \item Los gestores deben poder anotar la producción recolectada en cada campo.
        \item La aplicación debe tener un porcentaje de disponibilidad anual del 99.99\%.
        \item Los gestores deben poder marcar el estado de un campo (barbecho, activo, etc.).
    \end{enumerate}
    Lee detenidamente los requisitos y:
    \begin{enumerate}
        \item Clasifica los requisitos en funcionales, no funcionales u otros.
        \item $0.2$ puntos por cada respuesta correcta.
        \item Desarrolla la especificación del caso de uso de uno de los requisitos que hayas
        clasificado como funcional.
        \item $0.2$ puntos por cada campo simple; $0.3$ puntos por cada campo
    \end{enumerate}
\end{enunciado}
\begin{solucion}
    \begin{enumerate}
        \item Clasificación de los requisitos:
        \begin{itemize}
            \item Requisito 1: No funcional.
            \item Requisito 2: Funcional.
            \item Requisito 3: Funcional.
            \item Requisito 4: Funcional.
            \item Requisito 5: Otros.
            \item Requisito 6: Funcional.
            \item Requisito 7: No funcional.
            \item Requisito 8: Funcional.
        \end{itemize}

        \item Especificación del caso de uso (ejemplo para el requisito 2):
        \begin{itemize}
            \item Nombre: Añadir campo de cultivo.
            \item Actor: Gestor de producción.
            \item Descripción: Los gestores de producción deben poder añadir y eliminar campos de cultivo al sistema.
            \item Precondiciones: El gestor de producción debe haber iniciado sesión en el sistema.
            \item Dependencias: No especificado.
            \item Escenario:
            \begin{enumerate}
                \item El gestor de campo comienza el proceso de añadir un campo.
                \item El gestor de campo rellena los detalles del campo (nombre, localización,\ldots ).
                \item El gestor graba el campo en el sistema.
            \end{enumerate}
            \item Excepciones:
            \begin{enumerate}
                \item El gestor de campo comienza el proceso de añadir un campo.
                \item El gestor de campo no rellena todos los detalles del campo.
                \item El gestor de campo intenta grabar el campo en el sistema.
                \item El sistema indica que faltan detalles del campo.
            \end{enumerate}
        \end{itemize}
    \end{enumerate}
    \begin{itemize}
        \item Prioridad: No especificado.
    \end{itemize}
\end{solucion}


\section{Ejercicio 3: Diagrama de secuencia}\label{sec:ejercicio-3:-diagrama-de-secuencia}
% !TeX root = ../mantenimiento.tex
\begin{enunciado}
    Calcula si los siguientes conjuntos de datos cumplen con el SLO definido:
    \begin{itemize}
        \item Relación de éxito de peticiones del 95\%
        \begin{itemize}
            \item Peticiones exitosas: 37
            \item Peticiones fallidas: 5
        \end{itemize}
        \item Tiempo de respuesta P90 inferior a 2s:
        \begin{itemize}
            \item Tiempos de respuesta de cada petición realizada: 1.3s; 1.2s; 2.3s; 1.9s; 1.7s; 1.8s; 0.9s; 2.1s; 1.1s; 1.4s.
        \end{itemize}
        \item Número de compras medio cada 10 minutos: 50
        \begin{itemize}
            \item Compras por minuto: 1, 7, 3, 5, 8, 9, 0, 10, 11, 7
        \end{itemize}
    \end{itemize}
\end{enunciado}

\subsection{Métrica 1: Relación de éxito}\label{subsec:metrica-1:-relacion-de-exito}
Calculamos la tasa de éxito:
\[
    T_{\text{éxito}} = \frac{37}{37 + 5} = \frac{37}{42} \approx 0.881 \Rightarrow 88.1\%
\]

\begin{solucion}[parte 1]
    Como $88.1\% < 95\%$, esta métrica \textbf{no} cumple el SLO\@.
\end{solucion}

\subsection{Métrica 2: Tiempo de respuesta P90}\label{subsec:metrica-2:-tiempo-de-respuesta-p90}
Tiempos recogidos:
\[
    T = (1.3,\ 1.2,\ 2.3,\ 1.9,\ 1.7,\ 1.8,\ 0.9,\ 2.1,\ 1.1,\ 1.4)
\]
Ordenamos:
\[
    t = (0.9,\ 1.1,\ 1.2,\ 1.3,\ 1.4,\ 1.7,\ 1.8,\ 1.9,\ 2.1,\ 2.3)
\]
Para calcular el percentil $P90$ con $n = 10$, usamos la posición:
\[
    i = \lceil 0.9 \cdot n \rceil = \lceil 9 \rceil = 9
\]
Entonces:
\[
    P90 = t_9 = 2.1\,\text{s}
\]
\begin{solucion}[parte 2]

    Como $2.1\,\text{s} > 2\,\text{s}$, esta métrica \textbf{no} cumple el SLO\@.
\end{solucion}

\subsection{Métrica 3: Compras medias}\label{subsec:metrica-3:-compras-medias}
Suma total de compras:
\[
    \sum c = 1 + 7 + 3 + 5 + 8 + 9 + 0 + 10 + 11 + 7 = 61
\]
Promedio por minuto:
\[
    \mu = \frac{61}{10} = 6.1\quad\Rightarrow\quad 10 \cdot \mu = 61
\]
\begin{solucion}[parte 3]

    Como $61 \geq 50$, esta métrica \textbf{sí} cumple el SLO\@.
\end{solucion}
\begin{solucion}
    \subsection*{Conclusión}
    Aunque se cumple la tercera métrica, las dos primeras no lo hacen, por lo tanto:
    \[
        \textbf{El SLO no se cumple.}
    \]
\end{solucion}


\section{Ejercicio 5: Tiempo medio entre fallos y tiempo de recuperación}\label{sec:ejercicio-5:-tiempo-medio-entre-fallos-y-tiempo-de-recuperacion}
\begin{solucion}
    \textbf{Versión 1.1 — 12 de junio de 2025}

    \textbf{Novedades:}
    \begin{itemize}
        \item Se han añadido pruebas unitarias para el servicio de búsqueda.
        \item Se ha integrado Apple Pay como nuevo método de pago.
        \item Se ha integrado un nuevo proveedor de anuncios en la aplicación.
    \end{itemize}

    \textbf{Mejoras:}
    \begin{itemize}
        \item Se ha mejorado la fluidez de la animación al pasar a la siguiente carátula.
        \item Se ha mejorado el registro de logs del servicio de login.
        \item Se ha simplificado la función de ordenación de películas por evaluación.
        \item Se ha optimizado el renderizado de carátulas.
    \end{itemize}

    \textbf{Cambios:}
    \begin{itemize}
        \item Se ha actualizado el precio de la suscripción.
    \end{itemize}

    \textbf{Errores corregidos:}
    \begin{itemize}
        \item Se ha corregido la visualización errónea de caracteres ISO-8859-1.
    \end{itemize}
\end{solucion}


\clearpage


\section{Ejercicio 6: Requisitos}\label{sec:ejercicio-6-:-requisitos}
% !TeX root = ../modelado.tex


\begin{enunciado}
    Clasifica los siguientes requisitos no funcionales en su correspondiente subcategoría:
    \begin{enumerate}
        \item La aplicación debe funcionar en Windows, Unix, MacOs, Android e iOS
        \item La aplicación debe usar lenguaje inclusivo
        \item La aplicación debe estar disponible de lunes a viernes de 8h a 18h
        \item La operación de registro debe realizarse en menos de 1 segundo
        \item La versión móvil de la aplicación debe ocupar menos de 100MB
        \item La aplicación debe ser compatible con el sistema de videoconferencia Zoom
        \item Se deben entregar tanto los ejecutables de las diferentes versiones como el código fuente
        \item La aplicación debe ser accesible para personas con discapacidades visuales o motoras.
    \end{enumerate}
\end{enunciado}

\begin{solucion}
    \begin{enumerate}
        \item De proceso, implementación, usabilidad
        \item Externos, legislativo, seguridad
        \item Producto, usabilidad
        \item Producto, eficiencia, espacio
        \item Producto, usabilidad, implementación
        \item Proceso, delivery
        \item Ética?
    \end{enumerate}
\end{solucion}


\clearpage


\section{Ejercicio 7: Diagrama de arquitectura}\label{sec:ejercicio-7:-diagrama-de-arquitectura}
% !TeX root = ../modelado.tex

\begin{enunciado}
    Completa el diagrama de arquitectura para añadir la siguiente funcionalidad:

    Aplicación de ofertas en función de la cantidad de productos comprados y el país desde el
    que se realiza la compra.
\end{enunciado}


\begin{solucion}

    \begin{tikzpicture}[node distance=1.25cm]
% Interfaz Gráfica
        \hspace{1em}
        \node[interface] (gui) {Interfaz Gráfica};

% Primera fila de servicios
        \node[service, below left=2cm and 1.5cm of gui] (catalog) {CatalogService};
        \node[service, below=2cm of gui] (cart) {CartService};
        \node[service, below right=2cm and 1.5cm of gui] (order) {OrderService};

% Segunda fila de servicios
        \node[service, below =2.0cm of catalog] (pricing) {PricingService};
        \node[service, below =2cm of order] (payment) {PaymentService};

% Nuevo servicio de ofertas (añadido)
        \node[service, below=2.0cm of cart, fill=green!20] (offers) {OffersService};

% Servicio de geolocalización (añadido)
        \node[service, below left=1.5cm and -0.5cm of offers, fill=orange!20] (geo) {GeoService};

% Bases de datos
        \node[database, left=0.5cm of catalog] (db1) {};
        \node[database, above= 0.25cm of cart,xshift=0.5cm] (db2) {};
        \node[database, right=0.5cm of order] (db3) {};
        \node[database, left=0.5cm of pricing] (db4) {};
        \node[database, right=0.5cm of offers] (db6) {};
        \node[database, left=0.5cm of geo] (db7) {};

% Conexiones desde la interfaz
        \draw[arrow] (gui) -- (catalog);
        \draw[arrow] (gui) -- (cart);
        \draw[arrow] (gui) -- (order);

% Conexiones entre servicios
        \draw[bidirectional] (catalog) -- (cart);
        \draw[bidirectional] (cart) -- (order);
        \draw[bidirectional] (catalog) -- (pricing);
        \draw[bidirectional] (order) -- (payment);

% Nuevas conexiones para ofertas
        \draw[bidirectional] (cart) -- (offers);
        \draw[bidirectional] (order) -- (offers);
        \draw[bidirectional] (offers) -- (geo);
        \draw[bidirectional] (offers) -- (pricing);

% Conexiones a bases de datos
        \draw[arrow] (catalog) -- (db1);
        \draw[arrow] (cart) -- (db2);
        \draw[arrow] (order) -- (db3);
        \draw[arrow] (pricing) -- (db4);
        \draw[arrow] (offers) -- (db6);
        \draw[arrow] (geo) -- (db7);

% Etiquetas para las nuevas funcionalidades
        \node[above=0.2cm of offers, font=\tiny, text=green!60!black] {Gestión de ofertas};
        \node[above=0.2cm of geo, font=\tiny, text=orange!60!black] {Geolocalización};

    \end{tikzpicture}

\end{solucion}


\section{Ejercicio 8: Interfaz textual}\label{sec:ejercicio-8:-interfaz-textual}
% !TeX root = ../modelado.tex


\begin{enunciado}
    Diseña una interfaz textual para las siguientes funcionalidades:

    \begin{itemize}
        \item Registrar películas vistas con su título y fecha de visualización
        \item Listar las películas vistas permitiendo ordenarlas por título o por fecha de visualización
    \end{itemize}
\end{enunciado}

\begin{solucion}
    Componente visual, imagen de la portada de la película


    Componente caja de texto para introducir el texto


    Componente campo de texto para introducir la fecha


    Búsqueda de datos para almacenar películas


    Componente visual para elegir si mostrar por fecha o por titulo
\end{solucion}


    \clearpage


    \part{\textcolor{primaryblue}{Planificación}}\label{part:planificacion}
    % !TeX root = ../main.tex


\section{Ejercicio 1: Estimación basada en problema}\label{sec:ejercicio-1:-estimacion-basada-en-problema}
% !TeX root = ../calidad.tex

\begin{enunciado}
    Calcula el número de errores que verán los usuarios finales para un proyecto en el que sucede lo siguiente:
    \begin{itemize}
        \item En el modelo de requerimientos se han cometido 10 errores y cada uno se amplifica en un factor de 2:1 en el diseño.
        \item En el diseño se cometen otros 20 errores adicionales, que luego se amplifican en un factor de 1.5:1 en el código.
        \item En el código se cometen otros 30 errores adicionales.
        \item Las pruebas unitarias encuentran un 20\% de todos los errores.
        \item Las pruebas de integración descubren el 50\% de los errores restantes.
        \item Las pruebas de validación/aceptación hallan el 50\% de los que queden.
        \item Tras las pruebas se pone el nuevo software en marcha.
    \end{itemize}
\end{enunciado}

\subsection{Datos}\label{subsec:datos}
\begin{itemize}
    \item Errores en requisitos: $10$ errores.

    Amplificación en diseño: $10 \cdot 2 = 20$ errores.


    \item Errores de diseño adicionales: $20$ errores.


    Total de errores diseño: $20 + 20 = 40$ errores.

    Amplificación en código: $40 \cdot 1.5 = 60$ errores.


    \item Errores adicionales en código: $30$ errores.

    Total de errores en código: $60 + 30 = 90$ errores.


\end{itemize}

\hrulefill

\subsection{Cálculo de errores tras las pruebas}\label{subsec:calculo-de-errores-tras-las-pruebas}
\begin{itemize}
    \item Pruebas unitarias detectan el 20\%: $90 \cdot 0.20 = 18$ errores detectados.
    Errores restantes: $90 - 18 = 72$.
    \item Pruebas de integración detectan el 50\% de los restantes: $72 \cdot 0.50 = 36$ errores detectados.
    Errores restantes: $72 - 36 = 36$.
    \item Pruebas de validación detectan el 50\% de los restantes: $36 \cdot 0.50 = 18$ errores detectados.
    Errores restantes: $36 - 18 = 18$.
\end{itemize}

\begin{solucion}
    El total de errores que verán los usuarios finales es de \boxed{\textbf{18 errores}}.
\end{solucion}



\section{Ejercicio 2: Tiempo de desarrollo usando COCOMO II}\label{sec:ejercicio-2:-tiempo-de-desarrollo-usando-cocomo-ii}
% !TeX root = ../examen-parcial-2023.tex

\begin{itemize}
    \item \textbf{Puntos:} 3
\end{itemize}

\begin{enunciado}
    Se han identificado los siguientes requisitos como parte de la mejora de la gestión de la
    producción:
    \begin{enumerate}
        \item Los gestores deben poder acceder al sistema a través de una interfaz web mientras que
        los agricultores deben poder hacerlo mediante una aplicación móvil disponible para iOS\@.
        \item Los gestores de producción deben poder añadir y eliminar campos de cultivo al sistema.
        \item Los agricultores deben poder registrar las labores realizadas en los campos de cultivo
        (arado, siembra, riego, abonado, recolección, etc.) mediante geolocalización.
        \item Los agricultores deben poder notificar incidencias que afecten a la producción (plagas,
        eventos climatológicos, etc.).
        \item El equipo de desarrollo debe poder saber si el sistema está funcionando correctamente.
        \item Los gestores deben poder anotar la producción recolectada en cada campo.
        \item La aplicación debe tener un porcentaje de disponibilidad anual del 99.99\%.
        \item Los gestores deben poder marcar el estado de un campo (barbecho, activo, etc.).
    \end{enumerate}
    Lee detenidamente los requisitos y:
    \begin{enumerate}
        \item Clasifica los requisitos en funcionales, no funcionales u otros.
        \item $0.2$ puntos por cada respuesta correcta.
        \item Desarrolla la especificación del caso de uso de uno de los requisitos que hayas
        clasificado como funcional.
        \item $0.2$ puntos por cada campo simple; $0.3$ puntos por cada campo
    \end{enumerate}
\end{enunciado}
\begin{solucion}
    \begin{enumerate}
        \item Clasificación de los requisitos:
        \begin{itemize}
            \item Requisito 1: No funcional.
            \item Requisito 2: Funcional.
            \item Requisito 3: Funcional.
            \item Requisito 4: Funcional.
            \item Requisito 5: Otros.
            \item Requisito 6: Funcional.
            \item Requisito 7: No funcional.
            \item Requisito 8: Funcional.
        \end{itemize}

        \item Especificación del caso de uso (ejemplo para el requisito 2):
        \begin{itemize}
            \item Nombre: Añadir campo de cultivo.
            \item Actor: Gestor de producción.
            \item Descripción: Los gestores de producción deben poder añadir y eliminar campos de cultivo al sistema.
            \item Precondiciones: El gestor de producción debe haber iniciado sesión en el sistema.
            \item Dependencias: No especificado.
            \item Escenario:
            \begin{enumerate}
                \item El gestor de campo comienza el proceso de añadir un campo.
                \item El gestor de campo rellena los detalles del campo (nombre, localización,\ldots ).
                \item El gestor graba el campo en el sistema.
            \end{enumerate}
            \item Excepciones:
            \begin{enumerate}
                \item El gestor de campo comienza el proceso de añadir un campo.
                \item El gestor de campo no rellena todos los detalles del campo.
                \item El gestor de campo intenta grabar el campo en el sistema.
                \item El sistema indica que faltan detalles del campo.
            \end{enumerate}
        \end{itemize}
    \end{enumerate}
    \begin{itemize}
        \item Prioridad: No especificado.
    \end{itemize}
\end{solucion}


\section{Ejercicio 3: Diagrama de Gantt}\label{sec:ejercicio-3:-diagrama-de-gantt}
% !TeX root = ../mantenimiento.tex
\begin{enunciado}
    Calcula si los siguientes conjuntos de datos cumplen con el SLO definido:
    \begin{itemize}
        \item Relación de éxito de peticiones del 95\%
        \begin{itemize}
            \item Peticiones exitosas: 37
            \item Peticiones fallidas: 5
        \end{itemize}
        \item Tiempo de respuesta P90 inferior a 2s:
        \begin{itemize}
            \item Tiempos de respuesta de cada petición realizada: 1.3s; 1.2s; 2.3s; 1.9s; 1.7s; 1.8s; 0.9s; 2.1s; 1.1s; 1.4s.
        \end{itemize}
        \item Número de compras medio cada 10 minutos: 50
        \begin{itemize}
            \item Compras por minuto: 1, 7, 3, 5, 8, 9, 0, 10, 11, 7
        \end{itemize}
    \end{itemize}
\end{enunciado}

\subsection{Métrica 1: Relación de éxito}\label{subsec:metrica-1:-relacion-de-exito}
Calculamos la tasa de éxito:
\[
    T_{\text{éxito}} = \frac{37}{37 + 5} = \frac{37}{42} \approx 0.881 \Rightarrow 88.1\%
\]

\begin{solucion}[parte 1]
    Como $88.1\% < 95\%$, esta métrica \textbf{no} cumple el SLO\@.
\end{solucion}

\subsection{Métrica 2: Tiempo de respuesta P90}\label{subsec:metrica-2:-tiempo-de-respuesta-p90}
Tiempos recogidos:
\[
    T = (1.3,\ 1.2,\ 2.3,\ 1.9,\ 1.7,\ 1.8,\ 0.9,\ 2.1,\ 1.1,\ 1.4)
\]
Ordenamos:
\[
    t = (0.9,\ 1.1,\ 1.2,\ 1.3,\ 1.4,\ 1.7,\ 1.8,\ 1.9,\ 2.1,\ 2.3)
\]
Para calcular el percentil $P90$ con $n = 10$, usamos la posición:
\[
    i = \lceil 0.9 \cdot n \rceil = \lceil 9 \rceil = 9
\]
Entonces:
\[
    P90 = t_9 = 2.1\,\text{s}
\]
\begin{solucion}[parte 2]

    Como $2.1\,\text{s} > 2\,\text{s}$, esta métrica \textbf{no} cumple el SLO\@.
\end{solucion}

\subsection{Métrica 3: Compras medias}\label{subsec:metrica-3:-compras-medias}
Suma total de compras:
\[
    \sum c = 1 + 7 + 3 + 5 + 8 + 9 + 0 + 10 + 11 + 7 = 61
\]
Promedio por minuto:
\[
    \mu = \frac{61}{10} = 6.1\quad\Rightarrow\quad 10 \cdot \mu = 61
\]
\begin{solucion}[parte 3]

    Como $61 \geq 50$, esta métrica \textbf{sí} cumple el SLO\@.
\end{solucion}
\begin{solucion}
    \subsection*{Conclusión}
    Aunque se cumple la tercera métrica, las dos primeras no lo hacen, por lo tanto:
    \[
        \textbf{El SLO no se cumple.}
    \]
\end{solucion}


\section{Ejercicio 4: Clasificación de riesgos}\label{sec:ejercicio-4:-clasificacion-de-riesgos}
%Ejercicio 4 (Puntos: 2)
%El director de producción quiere poder consultar, a través de un bot de Telegram, la última labor
%realizada en un campo según la localización en que se encuentre en ese momento.
%Desarrolla la interfaz textual del bot teniendo en cuenta:
%•
%Los requisitos funcionales del Ejercicio 2.
%•
%Las condiciones de error que pueden darse.
%•
%La información mostrada debe incluir: nombre del campo, labor realizada, fecha en que
%se realizó y nombre del agricultor que hizo la labor.
%Bot (B): Bienvenido al Labrija. Selección opción.
%Usuario (U): Ver la última labor del campo en el que estoy.
%B: Esta operación requiere acceso a la localización del dispositivo. ¿Permitir?
%U: Sí.
%B: Se ha detectado que el campo en el que se encuentra es Campo 1. ¿Es correcto?
%U: Sí.
%B: Aquí está la última labor del campo:
%-
%Campo: Campo 1
%-
%Labor: Siembra
%-
%Fecha: 2 de febrero de 2023
%-
%Agricultor: Felipe Gómez Pérez

\begin{itemize}
    \item \textbf{Puntos:} 2
\end{itemize}

\begin{enunciado}
    El director de producción quiere poder consultar, a través de un bot de Telegram, la última labor
    realizada en un campo según la localización en que se encuentre en ese momento.
    Desarrolla la interfaz textual del bot teniendo en cuenta:
    \begin{itemize}
        \item Los requisitos funcionales del Ejercicio 2.
        \item Las condiciones de error que pueden darse.
        \item La información mostrada debe incluir: nombre del campo, labor realizada, fecha en que
        se realizó y nombre del agricultor que hizo la labor.
    \end{itemize}
\end{enunciado}

\begin{solucion}
    \begin{itemize}
        \item \textbf{Bot (B):} Bienvenido al Labrija.
        Selección opción.
        \item \textbf{Usuario (U):} Ver la última labor del campo en el que estoy.
        \item \textbf{B:} Esta operación requiere acceso a la localización del dispositivo.
        ¿Permitir?
        \item \textbf{U:} Sí.
        \item \textbf{B:} Se ha detectado que el campo en el que se encuentra es Campo 1.
        ¿Es correcto?
        \item \textbf{U:} Sí.
        \item \textbf{B:} Aquí está la última labor del campo:
        \begin{itemize}
            \item Campo: Campo 1
            \item Labor: Siembra
            \item Fecha: 2 de febrero de 2023
            \item Agricultor: Felipe Gómez Pérez
        \end{itemize}
    \end{itemize}
\end{solucion}



\section{Ejercicio 5: Camino crítico PERT}\label{sec:ejercicio-5:-camino-critico-pert}
\begin{solucion}
    \textbf{Versión 1.1 — 12 de junio de 2025}

    \textbf{Novedades:}
    \begin{itemize}
        \item Se han añadido pruebas unitarias para el servicio de búsqueda.
        \item Se ha integrado Apple Pay como nuevo método de pago.
        \item Se ha integrado un nuevo proveedor de anuncios en la aplicación.
    \end{itemize}

    \textbf{Mejoras:}
    \begin{itemize}
        \item Se ha mejorado la fluidez de la animación al pasar a la siguiente carátula.
        \item Se ha mejorado el registro de logs del servicio de login.
        \item Se ha simplificado la función de ordenación de películas por evaluación.
        \item Se ha optimizado el renderizado de carátulas.
    \end{itemize}

    \textbf{Cambios:}
    \begin{itemize}
        \item Se ha actualizado el precio de la suscripción.
    \end{itemize}

    \textbf{Errores corregidos:}
    \begin{itemize}
        \item Se ha corregido la visualización errónea de caracteres ISO-8859-1.
    \end{itemize}
\end{solucion}



    \clearpage


    \part{\textcolor{primaryblue}{Calidad}}\label{part:calidad}
    \minitoc
\begin{definicion}
    La calidad en software se define como un proceso eficaz que, al ser bien aplicado, crea un producto útil con valor mesurable tanto para los productores como para los usuarios.
\end{definicion}

La ingeniería de software tiene como objetivo garantizar esa calidad a lo largo de todo el ciclo de vida:

\begin{itemize}
    \item Planificación
    \item Diseño
    \item Desarrollo
    \item Pruebas
    \item Despliegue
    \item Mantenimiento
\end{itemize}


\section{Aseguramiento de la Calidad}\label{sec:aseguramiento-de-la-calidad}
\begin{definicion}
    Según la norma ISO/IEC 9126, la calidad de un producto software se puede evaluar mediante seis atributos principales:
\end{definicion}

\begin{enumerate}
    \item \textbf{Funcionalidad:} Grado en que el software cumple los requisitos funcionales esperados.
    \item \textbf{Confiabilidad:} Estabilidad del software ante fallos, por ejemplo, el tiempo medio de funcionamiento antes de un fallo.
    \item \textbf{Usabilidad:} Facilidad de uso para los usuarios.
    \item \textbf{Eficiencia:} Uso óptimo de los recursos disponibles (CPU, memoria, etc.).
    \item \textbf{Mantenibilidad:} Facilidad para modificar, corregir o mejorar el software.
    \item \textbf{Portabilidad:} Facilidad para trasladar el software entre diferentes entornos.
\end{enumerate}

\begin{nota}
    Existe un dilema clásico entre producir software rápido y barato o producir software de alta calidad, ya que la calidad requiere tiempo y recursos.
\end{nota}

\begin{nota}
    Los errores pequeños no detectados a tiempo pueden amplificarse y causar problemas mayores en fases posteriores, por lo que la detección temprana es fundamental.
\end{nota}

\subsection{Principios rectores de la calidad}\label{subsec:principios-rectores-de-la-calidad}
\begin{itemize}
    \item \textbf{Formulación.} La derivación de medidas y métricas de software apropiadas
    \item para la representación del software que se está construyendo.
    \item \textbf{Recolección. }Mecanismo que se usa para acumular datos requeridos para derivar las métricas formuladas.
    \item \textbf{Análisis.} El cálculo de métricas y la aplicación de herramientas
    \item matemáticas.
    \item \textbf{Interpretación.} Evaluación de las métricas resultantes para comprender la calidad de la representación.
    \item \textbf{Retroalimentación.} Recomendaciones derivadas de la interpretación de las métricas del producto, transmitidas al equipo de software.
\end{itemize}

\subsection{Atributos de las métricas}\label{subsec:atributos-de-las-metricas}
\begin{enumerate}
    \item \textbf{Medible. }Debe ser simple poder recolectar los datos que componen la métrica y realizar su cálculo.
    \item \textbf{Intuitiva.} Los usuarios de la métrica deben poder identificar su significado y su valor.
    \item \textbf{Objetiva.} Siempre debe producir resultados que no tengan ambigüedades.
    \item \textbf{Coherente.} El cálculo matemático de la métrica debe usar medidas que no conduzcan a combinaciones extrañas de unidades.
    \item \textbf{Tecnológicamente agnóstica.} Debe basarse en el modelo de requerimientos, el modelo de diseño o la estructura del programa en sí.
    \item \textbf{Accionable.} Debe proporcionar información que pueda conducir a un producto final de mayor calidad.
\end{enumerate}

\subsection{Control vs. Aseguramiento de la Calidad}\label{subsec:control-vs.-aseguramiento-de-la-calidad}

\begin{center}
    \begin{tabularx}{\textwidth}{|X|X|}
        \hline
        \textbf{Control de Calidad (QC)}                         & \textbf{Aseguramiento de Calidad (QA)}               \\
        \hline
        Reactivo                                                 & Proactivo                                            \\
        Detección y corrección de errores después de que ocurren & Prevención de errores mediante estándares y procesos \\
        Inspección de productos                                  & Mejora continua de procesos                          \\
        \hline
    \end{tabularx}
\end{center}

\subsection{Revisiones técnicas}\label{subsec:revisiones-tecnicas}

\begin{itemize}
    \item \textbf{Informales:} Conversaciones espontáneas, revisiones en escritorio; baja eficacia que mejora con listas de verificación.
    \item \textbf{Formales:} Reuniones estructuradas y preparadas; alta eficacia, más tiene un alto coste en tiempo y esfuerzo; se utiliza normalmente una muestra representativa.
\end{itemize}

\subsection{Revisiones durante el desarrollo}\label{subsec:revisiones-durante-el-desarrollo}

\begin{itemize}
    \item \textbf{Revisión de código:} Entre compañeros (pair review) para detectar errores y mejorar el aprendizaje del equipo.
    \item \textbf{Análisis estático de código:} Herramientas automáticas que detectan errores potenciales, complejidad y duplicaciones.
\end{itemize}

\subsection{Métricas de calidad (DORA Metrics)}\label{subsec:metricas-de-calidad-(dora-metrics)}

\begin{center}
    \begin{tabular}{|l|l|}
        \hline
        \textbf{Métrica}                  & \textbf{Significado}                                 \\
        \hline
        MTTR (Mean Time to Recover)       & Tiempo medio para recuperar un sistema tras un fallo \\
        MTBF (Mean Time Between Failures) & Tiempo medio entre fallos                            \\
        Disponibilidad                    & Proporción de tiempo que el sistema está disponible  \\
        \hline
    \end{tabular}
\end{center}

\begin{definicion}
    La disponibilidad se calcula con la fórmula:
    \[
        \text{Disponibilidad} = \frac{\text{MTBF}}{\text{MTBF} + \text{MTTR}} \cdot 100\%
    \]
\end{definicion}

\subsection{Buenas prácticas de desarrollo}\label{subsec:buenas-practicas-de-desarrollo}

\begin{itemize}
    \item \textbf{Clean Code} (Robert C. Martin - Uncle Bob):
    \begin{itemize}
        \item KISS: “Keep It Simple, Stupid”
        \item DRY: “Don’t Repeat Yourself”
        \item YAGNI: “You Aren’t Gonna Need It”
        \item SoC: “Separation of Concerns”
    \end{itemize}
    \item \textbf{Documentación:} Comentarios explicativos (no descriptivos), explicaciones en los commits y ejemplos de uso en tests.
    \item \textbf{Control de versiones:} Uso de herramientas como Git para seguir cambios y facilitar la colaboración.
\end{itemize}

\subsection{Métricas de desarrollo}\label{subsec:metricas-de-desarrollo}

\begin{center}
    \begin{tabular}{|l|l|}
        \hline
        \textbf{Métrica}                             & \textbf{Descripción}                       \\
        \hline
        Densidad de comentarios                      & \% de comentarios respecto al código total \\
        Duplicidad de código                         & Código repetido                            \\
        Cobertura de pruebas                         & \% de código ejecutado durante pruebas     \\
        Complejidad ciclomática                      & Mide rutas lógicas (condiciones y bucles)  \\
        IMS (Índice de Madurez del código(Software)) & Evalúa la estabilidad de una release       \\
        \hline
    \end{tabular}
\end{center}

\begin{definicion}
    La fórmula para calcular el IMS es:
    \[
        IMS = \frac{M_T - (F_a + F_c + F_d)}{M_T}
    \]
    Donde:
    \begin{itemize}
        \item $M_T$: Número total de pruebas planificadas.
        \item $F_a$: Fallos críticos encontrados.
        \item $F_c$: Fallos menores encontrados.
        \item $F_d$: Fallos detectados en desarrollo.
    \end{itemize}
\end{definicion}

\subsection{Deuda técnica}\label{subsec:deuda-tecnica}

\begin{definicion}
    Según Martin Fowler, la deuda técnica es una metáfora financiera: tomar atajos en el diseño genera un \textquote{interés} que se paga con mayor esfuerzo futuro.
    Se puede:
    \begin{itemize}
        \item Seguir pagando intereses (mantener mal diseño).
        \item Pagar el principal (refactorizar y mejorar).
    \end{itemize}
\end{definicion}

Se clasifica según dos ejes:

\begin{center}
    \begin{tabular}{|c|c|c|}
        \hline
        & \textbf{Temeraria}                              & \textbf{Prudente}                                    \\
        \hline
        \textbf{Deliberada}  & \textquote{No tenemos tiempo, entregamos ahora} & \textquote{Lo haremos rápido y mejoraremos después}  \\
        \hline
        \textbf{Inadvertida} & \textquote{¿Qué componentes tiene esto?}        & \textquote{Ahora sabemos cómo debería haberse hecho} \\
        \hline
    \end{tabular}
\end{center}


\section{Estrategias de Prueba}\label{sec:estrategias-de-prueba}

\subsection{Verificación vs. Validación}\label{subsec:verificacion-vs.-validacion}

\begin{center}
    %! suppress = LineBreak
    \begin{tabular}{|l|l|}
        \hline
        \textbf{Verificación}                     & \textbf{Validación}                         \\
        \hline
        ¿Construimos \textbf{bien} el producto?   & ¿Construimos el \textbf{producto correcto}? \\
        Garantía de implementación correcta       & Cumplimiento de requisitos del cliente      \\
        Incluye pruebas, revisiones, simulaciones & Incluye pruebas de aceptación y prototipos  \\
        \hline
    \end{tabular}
\end{center}

\subsection{Malas prácticas}\label{subsec:malas-practicas}

\begin{itemize}
    \item Suponer que hay partes que no se pueden probar.
    \item Que el desarrollador no haga pruebas.
    \item Aislar al equipo de pruebas.
    \item Involucrar a los testers solo al final.
\end{itemize}

\subsection{Estrategias de prueba (pirámide de pruebas)}\label{subsec:estrategias-de-prueba-(piramide-de-pruebas)}

Martin Fowler propone priorizar así:

\begin{enumerate}
    \item Pruebas unitarias (muchas, automáticas).
    \item Pruebas de integración.
    \item Pruebas de interfaz/aceptación (pocas, más caras de automatizar).
\end{enumerate}

\subsection{Dimensiones de la prueba}\label{subsec:dimensiones-de-la-prueba}

\begin{center}
    \begin{tabularx}{\textwidth}{|l|l|X|}
        \hline
        \textbf{Dimensión} & \textbf{Subdimensión}                   & \textbf{Descripción}                                    \\
        \hline
        Tipo               & Funcional / No funcional                & Comprobación de funciones / parámetros como rendimiento \\
        Granularidad       & Unitaria / Integración / Validación     & Nivel del sistema probado                               \\
        Alcance            & Progresión / Regresión / Smoke / Sanity & Cobertura funcional                                     \\
        Ejecución          & Manual / Asistida / Automática          & Nivel de automatización                                 \\
        Metodología        & Guiada / Exploratoria                   & Nivel de formalización                                  \\
        \hline
    \end{tabularx}
\end{center}

\subsection{Pruebas específicas}\label{subsec:pruebas-especificas}

\begin{itemize}
    \item \textbf{Unitarias:} \textbf{Caja blanca}.
    Son las pruebas más simples y frecuentes (sobre código individual).
    Uso de \emph{stubs} para módulos dependientes.
    Ejemplos: métodos, condiciones de frontera, errores.
    \item \textbf{Integración:} \textbf{Caja gris}.
    Pruebas de cómo interactúan los componentes.
    Importante la integración incremental (mayor control de errores).
    Puede ser ascendente o descendente.
    \item \textbf{Validación (aceptación):} \textbf{Caja negra}.
    Realistas y orientadas al usuario final.
    Difíciles de automatizar.
    Usadas en fases alfa y beta.
    \item \textbf{No funcionales:} Rendimiento, seguridad, recuperación, esfuerzo, despliegue, etc.
\end{itemize}


\section{Administración de la Configuración}\label{sec:administracion-de-la-configuracion}
\begin{definicion}
    Conjunto de actividades para gestionar los cambios durante el ciclo de vida del software, garantizando:
    \begin{itemize}
        \item Trazabilidad
        \item Control de versiones
        \item Información actualizada
    \end{itemize}

    Se diferencia del mantenimiento en que este aplica cambios, y la administración de configuración controla y registra dichos cambios.
\end{definicion}

\subsection*{Conceptos clave}
\begin{itemize}
    \item \textbf{Elemento de configuración (EC / ICS):} Unidad identificable que debe ser controlada (código, documentación, etc.).
    \item \textbf{Base de datos de configuración (BCD):} Lugar donde se almacenan los EC y sus versiones.
    \item \textbf{Línea base:} Versión estable y acordada de un conjunto de EC\@.
    \item \textbf{Cambio:} Modificación propuesta a un EC\@.
    \item \textbf{Control de cambios:} Proceso de evaluar y aprobar cambios.
\end{itemize}

\subsection{Ítems de Configuración (IC)}\label{subsec:items-de-configuracion-(ic)}
\begin{definicion}
    Cada uno de los elementos que comprenden toda la información
    producida como parte del proceso de software.
\end{definicion}
\begin{itemize}
    \item Programas de cómputo
    \item Productos de trabajo
    \item Contenido
\end{itemize}

\subsection{Sistema de Administración}\label{subsec:sistema-de-administracion}
\begin{itemize}
    \item \textbf{Elementos componentes:} Herramientas que permiten el acceso y gestión de cada ítem de configuración del software.
    \item \textbf{Elementos de proceso:} Acciones y tareas necesarias para realizar una gestión efectiva del cambio.
    \item \textbf{Elementos de construcción:} Herramientas que automatizan la compilación, empaquetado y generación de versiones correctas del software.
    \item \textbf{Elementos humanos:} Procesos y herramientas que utiliza el equipo para implementar de manera efectiva la administración de configuración.
\end{itemize}

\subsection{Repositorio ACS}\label{subsec:repositorio-acs}
\begin{definicion}
    Almacén centralizado o distribuido donde se gestionan los ítems de configuración.
\end{definicion}

\paragraph{Características clave:}
\begin{itemize}
    \item Control de versiones.
    \item Rastreo de dependencias entre componentes.
    \item Trazabilidad de requisitos y cambios.
    \item Apoyo a auditorías e inspecciones de calidad.
\end{itemize}

\subsection{Proceso de Administración de la Configuración}\label{subsec:proceso-de-administracion-de-la-configuracion}
\begin{enumerate}
    \item \textbf{Identificación:} Asignar nombres unívocos a los ítems de configuración.
    \item \textbf{Control de cambios:} Registrar, aprobar e implementar modificaciones.
    \item \textbf{Control de versiones:} Gestionar diferentes versiones de todos los objetos.
    \item \textbf{Auditoría:} Verificar que los cambios cumplen con los estándares establecidos.
    \item \textbf{Reporte de estado:} Informar del estado actual y del histórico de cambios.
\end{enumerate}

\subsection{Administración del contenido}\label{subsec:administracion-del-contenido}
\begin{itemize}
    \item \textbf{Subsistema de recopilación:} Permite almacenar y organizar el contenido asociado a cada Ítem de Configuración del Software (ICS).
    \item \textbf{Subsistema de administración:} Gestiona los cambios realizados sobre el contenido de cada ICS, manteniendo un histórico completo y trazable.
    \item \textbf{Subsistema de publicación:} Hace accesible el contenido relevante a los distintos interesados del proyecto, permitiendo su consulta o reutilización.
\end{itemize}

\subsection{Conflictos dominantes}\label{subsec:conflictos-dominantes}
\begin{itemize}
    \item \textbf{Contenido:} No está claro qué información debe formar parte del sistema de administración de configuración.
    \item \textbf{Personas:} El equipo desconoce las herramientas o procesos de ACS, o no sabe cuándo aplicarlos.
    \item \textbf{Escalabilidad:} A medida que crece el proyecto, se complica la gestión del gran volumen de cambios.
    \item \textbf{Políticas:} Faltan reglas claras sobre quién es responsable de llevar a cabo la administración de configuración y cómo aplicarla de manera coherente.
\end{itemize}





    \clearpage


    \part{\textcolor{primaryblue}{Mantenimiento}}\label{part:mantenimiento}
    \minitoc

    \section{Métricas de producto}\label{sec:tema-6.1---metricas-de-producto}
        \begin{definicion}
        \textbf{Métrica:} Medida cuantitativa del grado en que un sistema, componente o proceso posee un atributo determinado (IEEE Standard).
    \end{definicion}

    \subsection{Conceptos clave}\label{subsec:conceptos-clave}
    \begin{itemize}
        \item \textbf{Medida:} Un único dato.
        \item \textbf{Medición:} Recolección de varias medidas.
        \item \textbf{Indicador:} Combinación de métricas que proporciona comprensión sobre procesos, proyectos o productos.
    \end{itemize}

    \subsection{Principios de medición}\label{subsec:principios-de-medicion}
    \begin{enumerate}
        \item \textbf{Formulación:} Derivación de medidas apropiadas
        \item \textbf{Recolección:} Mecanismo para acumular datos
        \item \textbf{Análisis:} Cálculo de métricas con herramientas matemáticas
        \item \textbf{Interpretación:} Evaluación de resultados
        \item \textbf{Retroalimentación:} Recomendaciones al equipo
    \end{enumerate}

    \subsection{Atributos de métricas}\label{subsec:atributos-de-metricas}
    \begin{itemize}
        \item Medible
        \item Intuitiva
        \item Objetiva
        \item Coherente \item
        Tecnológicamente agnóstica
        \item Accionable
    \end{itemize}

    \subsection{Métricas específicas}\label{subsec:metricas-especificas}

    \begin{itemize}
        \item \textbf{Modelo de requisitos:}
        \[PF = conteo \cdot (0.65 + (0.01 \cdot \text{FAV}))\]
        \begin{itemize}
            \item Componentes:
            \begin{description}
                \item[Entradas externas (EE)]: Información que se origina de un usuario o se transmite desde otra aplicación.
                Se usan para actualizar archivos lógicos internos (ALI). Las entradas deben distinguirse de las consultas.
                \item[Salidas externas (SE)]: Datos derivados dentro de la aplicación que ofrecen información al usuario.
                \item [Consultas externas (CE)]: Entrada que da como resultado la generación de alguna respuesta (con frecuencia recuperada de un ALI).
                \item[Número de archivos lógicos internos (ALI)]:Agrupamiento lógico de datos que reside dentro de la frontera de la aplicación y se mantiene mediante entradas externas
                \item[Número de archivos de interfaz externos (AIE)]:
                Agrupamiento lógico de datos que reside fuera de la aplicación, pero que proporciona información que puede usar la aplicación.
                \item [Factor de ajuste de valor (FAV)]: Indica la complejidad del sistema en su conjunto en base a unas preguntas a las que se asigna un valor entre 0 (irrelevante) y 5 (esencial).
            \end{description}
        \end{itemize}

        \item \textbf{Diseño arquitectónico:}
        \begin{itemize}
            \item Complejidad de módulo: $S(i) = f^2_{out}(i)$, $D(i) = v(i) / (f_{out}(i) + 1)$
            \item Complejidad de sistema: $C(i) = S(i) + D(i)$
        \end{itemize}

        \item \textbf{Orientadas a clase (CK):}
        \begin{table}[!ht]
            \centering
            \caption{Métricas de la Suite CK}
            \label{tab:ck_metrics}
            \begin{tabularx}{\linewidth}{lX}
                \toprule
                \textbf{Acrónimo} & \textbf{Descripción}                                                                                                                                                          \\
                \midrule
                MPC               & Métodos Ponderados por Clase: Suma de las complejidades nominales de todos los métodos de una clase ($MPC = \sum c_i$), donde $c_i$ es la complejidad nominal de cada método. \\
                \addlinespace[0.3cm]
                PAH               & Profundidad del Árbol de Herencia: Longitud máxima desde el nodo de la clase actual hasta la raíz del árbol de herencia.                                                      \\
                \addlinespace[0.3cm]
                NDH               & Número de Hijos Directos: Cantidad de subclases que heredan directamente de la clase actual (subclases inmediatas).                                                           \\
                \addlinespace[0.3cm]
                FCOM              & Falta de Cohesión en Métodos: Número de pares de métodos que acceden a uno o más atributos de clase en común.                                                                 \\
                \bottomrule
            \end{tabularx}
        \end{table}

        \item \textbf{Código fuente:}
        \begin{itemize}
            \begin{table}[!ht]
                \centering
                \caption{Métricas de Halstead}
                \label{tab:halstead_metrics}
                \begin{tabular}{>{\bfseries}l l l}
                    \toprule
                    \multicolumn{1}{c}{\textbf{Símbolo}} &
                    \multicolumn{1}{c}{\textbf{Descripción}} &
                    \multicolumn{1}{c}{\textbf{Fórmula}} \\
                    \midrule
                    n1 & Número de operadores únicos    & Cantidad de tipos diferentes de operadores        \\
                    & en un programa                 &                                                   \\
                    \addlinespace

                    n2 & Número de operandos únicos     & Cantidad de tipos diferentes de operandos         \\
                    & en un programa                 &                                                   \\
                    \addlinespace

                    N1 & Frecuencia total de operadores & $N1 = \sum (\text{ocurrencias de cada operador})$ \\
                    &                                &                                                   \\
                    \addlinespace

                    N2 & Frecuencia total de operandos  & $N2 = \sum (\text{ocurrencias de cada operando})$ \\
                    &                                &                                                   \\
                    \bottomrule
                \end{tabular}
            \end{table}

            \item Longitud: $N = n_1 \log_2 n_1 + n_2 \log_2 n_2$
            \item Volumen: $V = N \log_2 (n_1 + n_2)$
        \end{itemize}

        \item \textbf{Pruebas:}
        \begin{itemize}
            \item Nivel de programa: $PL = 1 / ((n_1/2) \cdot (N_2/n_2))$
            \item Esfuerzo de prueba: $e = V / PL$
        \end{itemize}

        \item \textbf{Mantenimiento:}
        \begin{itemize}
            \item Índice de madurez: $IMS = (M_1 - (F_a + F_c + F_d)) / M_1$
        \end{itemize}

        \item \textbf{SLA/SLO/SLI:}
        \begin{itemize}
            \item SLA: Acuerdo con cliente (Ejemplo: disponibilidad >99.9\%)
            \item SLO: Objetivo específico dentro de SLA
            \item SLI: Medida real de cumplimiento
        \end{itemize}
    \end{itemize}

% todo: completar

    \section{Tipos de mantenimiento}\label{sec:tipos-de-mantenimiento}
    \subsection{Leyes de Lehman}\label{subsec:leyes-de-lehman}
\begin{itemize}
    \item \textbf{Ley del cambio continuo.} En un entorno real, un sistema debe necesariamente cambiar para mantener su utilidad.
    \item \textbf{Ley de complejidad creciente.} Cuando el sistema evoluciona se hace más complejo.
    Hay que tomar medidas para evitarlo.
    \item \textbf{Ley de evolución. }La evolución es un proceso autorregulado.
    El tamaño, tiempo entre versiones, errores detectados, etc., se mantienen en el tiempo.
    \item \textbf{Ley de estabilidad organizacional.} Durante el tiempo de vida del sistema su velocidad de desarrollo es constante e independiente de los recursos dedicados su desarrollo.
    \item \textbf{Ley de conservación de la familiaridad.} A medida que un sistema evoluciona todo lo que está asociado con ello debe mantener un conocimiento total de su contenido y su comportamiento.
    \item \textbf{Ley de crecimiento continuado. }La funcionalidad ofrecida por los sistemas tiene que crecer continuamente para
    mantener la satisfacción de los usuarios.
    \item \textbf{Ley de decremento de la calidad.} La calidad de los sistemas software comenzará a disminuir a menos que dichos
    sistemas se adapten a los cambios de su entorno de funcionamiento.
    \item \textbf{Ley de retroalimentación.} Los procesos de evolución incorporan sistemas de retroalimentación.
\end{itemize}

\subsection{Proceso de mantenimiento}\label{subsec:proceso-de-mantenimiento}
\begin{itemize}
    \item \textbf{Causas de modificación:}
    \begin{itemize}
        \item Nuevos requisitos/cambios solicitados
        \item Corrección de errores
    \end{itemize}

    \item \textbf{Incorporación al desarrollo:}
    \begin{itemize}
        \item Implementación formal u hotfixes
    \end{itemize}

    \item \textbf{Factores de esfuerzo:}
    \begin{itemize}
        \item Diseño del sistema
        \item  Mecanismos de prueba
        \item Documentación
        \item  Estabilidad del personal
    \end{itemize}
\end{itemize}

\subsection{Clasificación de mantenimiento}\label{subsec:clasificacion-de-mantenimiento}
\begin{tabular}{|l|l|l|p{6cm}|c|}
    \hline
    \textbf{Categoría} & \textbf{Tipo} & \textbf{Descripción}          & \textbf{Esfuerzo} \\
    \hline
    Evolutivo          & Perfectivo    & Añadir nuevas funcionalidades & 50\%              \\
    \hline
    & Adaptativo    & Adaptar a nuevos entornos     & 25\%              \\
    \hline
    & Preventivo    & Mejorar mantenibilidad futura & 5\%               \\
    \hline
    Tradicional        & Correctivo    & Corregir errores              & 20\%              \\
    \hline
\end{tabular}

\subsection{Release Notes}\label{subsec:release-notes}
\begin{itemize}
    \item \textbf{Contenido esencial:}
    \begin{itemize}
        \item Novedades
        \item Mejoras
        \item Correcciones de errores
    \end{itemize}
\end{itemize}

\subsection{Sistemas heredados}\label{subsec:sistemas-heredados}
\begin{itemize}
    \item \textbf{Problemas comunes:}
    \begin{itemize}
        \item Código espagueti
        \item  Falta de documentación
        \item Estructura deficiente
        \item Especificaciones ausentes
    \end{itemize}

    \item \textbf{Solución:} Ingeniería inversa
\end{itemize}

\subsection{Reingeniería de sistemas}\label{subsec:reingenieria-de-sistemas}
\begin{definicion}
    Reestructuración, reescritura o re-documentación sin cambiar funcionalidad.
\end{definicion}

\begin{itemize}
    \item \textbf{Ventajas:} Más económico que desarrollo nuevo • Reemplazo gradual
    \item \textbf{Factores coste:} Personal experto, herramientas disponibles
    \item \textbf{¿Por qué?:} Más barato que el desarrollo.
\end{itemize}

\subsection{Tipos de reingeniería}\label{subsec:tipos-de-reingenieria}
\begin{itemize}
    \item Traducción de código
    \item Ingeniería inversa
    \item Reestructuración
    \item Ingeniería hacia adelante
    \item Migración de datos
\end{itemize}

\subsection{Flujo de reingeniería}\label{subsec:flujo-de-reingenieria}
\begin{enumerate}
    \item Código fuente sucio $\rightarrow$ Reestructuración $\rightarrow$ Código limpio
    \item Extracción de abstracciones $\rightarrow$ Especificación inicial
    \item Refinamiento $\rightarrow$ Especificación final
\end{enumerate}



\end{document}
