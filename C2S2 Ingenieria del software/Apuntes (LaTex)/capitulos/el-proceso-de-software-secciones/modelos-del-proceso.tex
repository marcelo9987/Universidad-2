    No todos los proyectos son iguales.
    En este tema veremos los modelos de procesos mas usados.

    \clearpage
    \subsection{Modelo cascada: Versión V}\label{subsec:modelo-cascada:-version-v}

    Tiene fases secuenciales y lineales.
    Las pruebas y el desarrollo funcionan en paralelo, de ahí la forma de V\@.

    \begin{tikzpicture}[
        node distance=1.5cm,
        box/.style={rectangle, draw, thick, minimum width=3cm, minimum height=1.2cm, align=center, fill=white},
        arrow/.style={->, thick, >=Stealth}
    ]

% Lado izquierdo (desarrollo)
        \node[box] (req) {Modelado de los\\requerimientos};
        \node[box, below=of req] (arch) {Diseño de la\\arquitectura};
        \node[box, below=of arch] (comp) {Diseño de los\\componentes};
        \node[box, below=of comp] (code) {Generación\\de código};

% Lado derecho (pruebas)
        \node[box, right=6cm of req] (accept) {Pruebas de\\aceptación};
        \node[box, below=of accept] (system) {Pruebas\\del sistema};
        \node[box, below=of system] (integration) {Pruebas de\\integración};
        \node[box, below=of integration] (unit) {Pruebas\\unitarias};

% Nodo final
        \node[below=1.5cm of code, xshift=3cm] (software) {Software ejecutable};

% Flechas verticales lado izquierdo
        \draw[arrow] (req) -- (arch);
        \draw[arrow] (arch) -- (comp);
        \draw[arrow] (comp) -- (code);

% Flechas verticales lado derecho
        \draw[arrow] (unit) -- (integration);
        \draw[arrow] (integration) -- (system);
        \draw[arrow] (system) -- (accept);

% Flechas horizontales (correspondencias)
        \draw[arrow]  (accept)      -- (req);
        \draw[arrow]  (system)      -- (arch);
        \draw[arrow]  (integration) -- (comp);
        \draw[arrow] (unit)         -- (code);

% Flechas hacia el software ejecutable
        \draw[arrow] (code) -- (software);
        \draw[arrow]  (software) -- (unit);

% Flechas diagonales de entrada y salida
        \draw[arrow] (-1.5, 2) -- (req);
        \draw[arrow] (accept) -- (9.5, 2);

    \end{tikzpicture}

    \clearpage
    \subsection{Modelo de proceso incremental}\label{subsec:modelo-de-proceso-incremental}
    El proceso se divide en incrementos que incluyen todas las fases.
    Esto implica que todas las entregas intermedias (incremento x) aseguran un cierto nivel de calidad.

    \begin{tikzpicture}[
        node distance=0.3cm,
        phase/.style={rectangle, draw, thick, minimum width=1.2cm, minimum height=0.8cm, align=center},
        arrow/.style={->, thick, >=Stealth},
        legend/.style={rectangle, draw, minimum width=0.4cm, minimum height=0.4cm}
    ]

% Ejes
        \draw[thick, ->] (0,0) -- (13,0) node[midway,right,anchor=north] {Calendario del proyecto};
        \draw[thick, ->] (0,0) -- (0,14) node[midway,above, rotate=90, anchor=south] {Funcionalidad y características del software};

% Leyenda
        \node[legend, fill=white] at (1,10) {};
        \node[right=0.1cm] at (1.2,10) {Comunicación};

        \node[legend, fill=gray!20] at (1,9.5) {};
        \node[right=0.1cm] at (1.2,9.5) {Planeación};

        \node[legend, fill=gray!40] at (1,9) {};
        \node[right=0.1cm] at (1.2,9) {Modelado (análisis, diseño)};

        \node[legend, fill=gray!60] at (1,8.5) {};
        \node[right=0.1cm] at (1.2,8.5) {Construcción (código, prueba)};

        \node[legend, fill=gray!80] at (1,8) {};
        \node[right=0.1cm] at (1.2,8) {Despliegue (entrega, retroalimentación)};

        \newcommand{\desplaltura}{1.00cm}

% Incremento #1
        \node[phase, fill=white] (c1) at (2,2) {};
        \node[phase, fill=gray!20, below right = 0.5cm of c1,yshift=\desplaltura] (p1) {};
        \node[phase, fill=gray!40, below right = 0.5cm of p1,yshift=\desplaltura] (m1) {};
        \node[phase, fill=gray!60, below right = 0.5cm of m1,yshift=\desplaltura] (co1) {};
        \node[phase, fill=gray!80, below right = 0.5cm of  co1,yshift=\desplaltura] (d1) {};

        \draw[arrow] (c1) -- (p1);
        \draw[arrow] (p1) -- (m1);
        \draw[arrow] (m1) -- (co1);
        \draw[arrow] (co1) -- (d1);

        \node[right=2.0cm of co1] (e1) {entrega del primer};
        \node[right=2.0cm of co1, yshift=-0.4cm] {incremento};

        \node[above = 0.25cm of c1] {incremento \# 1};

% Incremento #2
        \node[phase, fill=white] (c2) at (3.5,4.00) {};
        \node[phase, fill=gray!20, below right = 0.5cm of c2,yshift=\desplaltura] (p2) {};
        \node[phase, fill=gray!40, below right = 0.5cm of p2,yshift=\desplaltura] (m2) {};
        \node[phase, fill=gray!60, below right = 0.5cm of m2,yshift=\desplaltura] (co2) {};
        \node[phase, fill=gray!80, below right = 0.5cm of co2,yshift=\desplaltura] (d2) {};

        \draw[arrow] (c2) -- (p2);
        \draw[arrow] (p2) -- (m2);
        \draw[arrow] (m2) -- (co2);
        \draw[arrow] (co2) -- (d2);

        \node[above = 0.2cm of c2] {incremento \# 2};
        \node[right = 0.5cm of d2] {entrega del segundo};
        \node[right = 0.5cm of d2,yshift=-0.4cm] {incremento};

% Puntos suspensivos
        \fill (8,4.2) circle (0.08);
        \fill (8.3,4.6) circle (0.08);
        \fill (8.6,5) circle (0.08);

% Incremento #n
        \node[phase, fill=white] (cn) at (7,6) {};
        \node[phase, fill=gray!20, right=of cn] (pn) {};
        \node[phase, fill=gray!40, right=of pn] (mn) {};
        \node[phase, fill=gray!60, right=of mn] (con) {};
        \node[phase, fill=gray!80, right=of con] (dn) {};

        \draw[arrow] (cn) -- (pn);
        \draw[arrow] (pn) -- (mn);
        \draw[arrow] (mn) -- (con);
        \draw[arrow] (con) -- (dn);

        \node[above left=0.2cm of cn] {incremento \# n};
        \node[right = 0.1cm of dn] {entrega del n-ésimo};
        \node[right = 0.1cm of dn,yshift=-0.4cm] {incremento};

    \end{tikzpicture}
    \clearpage

    \clearpage
    \subsection{Proceso evolutivo}\label{subsec:proceso-evolutivo}
    Con forma de \textquote{espiral}, permite adaptar los requisitos y solucionar los problemas de iteraciones antiguas.

    \begin{tikzpicture}[
        node distance=2cm,
        box/.style={rectangle, draw, thick, minimum width=2.5cm, minimum height=1.5cm, align=center, fill=white},
        arrow/.style={->, very thick, >=Stealth, gray!70, fill=gray!50},
        cycle arrow/.style={very thick, gray!70, fill=gray!50}
    ]

% Nodos principales
        \node[box] (comunicacion) at (0,3) {Comunicación};
        \node[box] (plan) at (4,5) {Plan rápido};
        \node[box] (modelado) at (7,2) {Modelado\\Diseño rápido};
        \node[box] (construccion) at (4,-1) {Construcción\\del\\prototipo};
        \node[box] (despliegue) at (0,0) {Despliegue\\Entrega y\\Retroalimentación};

% Flecha curva grande de comunicación a plan
        \draw[cycle arrow] (comunicacion.north east)
        to[out=45, in=135]
        node[single arrow, draw, thick, fill=gray!50, minimum height=1.5cm, single arrow head extend=0.3cm, rotate=45] {}
        (plan.north west);

% Flecha de plan a modelado
        \draw[cycle arrow] (plan.south east)
        to[out=-45, in=45]
        node[single arrow, draw, thick, fill=gray!50, minimum height=1.2cm, single arrow head extend=0.3cm, rotate=-45] {}
        (modelado.north east);

% Flecha de modelado a construcción
        \draw[cycle arrow] (modelado.south)
        to[out=-90, in=45]
        node[single arrow, draw, thick, fill=gray!50, minimum height=1.5cm, single arrow head extend=0.3cm, rotate=-90] {}
        (construccion.east);

% Flecha de construcción a despliegue
        \draw[cycle arrow] (construccion.west)
        to[out=180, in=-45]
        node[single arrow, draw, thick, fill=gray!50, minimum height=1.2cm, single arrow head extend=0.3cm, rotate=135] {}
        (despliegue.south east);

% Flecha de despliegue a comunicación (completando el ciclo)
        \draw[cycle arrow] (despliegue.north)
        to[out=135, in=225]
        node[single arrow, draw, thick, fill=gray!50, minimum height=1.2cm, single arrow head extend=0.3cm, rotate=90] {}
        (comunicacion.south west);

    \end{tikzpicture}


    \clearpage
    \subsection{Modelo Concurrente}\label{subsec:modelo-concurrente}
    Funciona con el diseño y el desarrollo en paralelo.
    Muy versátil.


    \begin{tikzpicture}[
        node distance=2.5cm,
        state/.style={rectangle, draw, very thick, rounded corners=0.5cm, minimum width=2.5cm, minimum height=1cm, align=center, fill=white},
        arrow/.style={->, thick, >=Stealth},
        container/.style={rectangle, draw, very thick, rounded corners=1cm, fill=gray!20, minimum width=12cm, minimum height=10cm}
    ]

% Contenedor principal
        \node[container] (container) at (4,0) {};


% Estado inicial (fuera del contenedor)
        \node[state] (inactivo) at (4,6.5) {Inactivo};

% Estados dentro del contenedor
        \node[state] (desarrollo) at (4,3) {En\\desarrollo};
        \node[state] (espera) at (0.5,0.5) {Cambios\\en espera};
        \node[state] (evaluacion) at (0.5,-2) {En\\evaluación};
        \node[state] (revision) at (7.5,0.5) {En revisión};
        \node[state] (alcance) at (7.5,-2) {Alcance mínimo};
        \node[state] (terminado) at (4,-3.5) {Terminado};

% Flecha de entrada
        \draw[arrow] (inactivo) -- (desarrollo);

% Flechas internas del ciclo
        \draw[arrow] (desarrollo) -- (espera);
        \draw[arrow] (espera) -- (evaluacion);
        \draw[arrow] (evaluacion) -- (terminado);
        \draw[arrow] (desarrollo) -- (revision);
        \draw[arrow] (revision) -- (alcance);
        \draw[arrow] (alcance) -- (terminado);
        \draw[arrow] (revision) -- (evaluacion);

% Flecha de retroalimentación
        \draw[arrow] (evaluacion) to[out=120, in=240] (espera);

% Etiqueta explicativa
        \node[right=0.3cm of desarrollo, text width=3.5cm, font=\small] {Representa el estado\\de una actividad o\\tarea de la ingeniería\\de software};

% Línea de conexión de la etiqueta
        \draw[-] (desarrollo.east) -- ++(0.3,0);

    \end{tikzpicture}

    \clearpage
    \subsection{Proceso unificado}\label{subsec:proceso-unificado}

    Compuesta por fases múltiples con ciclos cortos.

    \begin{tikzpicture}[
        node distance=2cm,
        phase/.style={rectangle, draw, very thick, minimum width=2cm, minimum height=0.8cm, align=center, fill=gray!30, drop shadow={shadow xshift=0.2cm, shadow yshift=-0.2cm, fill=black}},
        arrow/.style={->, very thick, >=Stealth},
        spiral/.style={very thick, black}
    ]

% Fases en espiral (empezando desde comunicación)
        \node[phase] (comunicacion) at (-3,-1) {comunicación};
        \node[phase] (planeacion) at (-1,2) {planeación};
        \node[phase] (modelado) at (3,2.5) {modelado};
        \node[phase] (construccion) at (4,-1) {construcción};
        \node[phase] (despliegue) at (1,-3) {despliegue};

% Incremento del software (centro-abajo)
        \node[phase, fill=white] (incremento) at (0,-5) {incremento del software};

% Etiquetas de las grandes fases
        \node[left=0.3cm of comunicacion,yshift=2cm, font=\large\bfseries\itshape] (concepcion) {Concepción};
        \node[above=1.0cm of planeacion,xshift=1.2cm, font=\large\bfseries\itshape](elaboracion) {Elaboración};
        \node[above=0.3cm of construccion,xshift=3.5cm, font=\large\bfseries\itshape] (construccion_) {Construcción};
        \node[below=0.3cm of incremento, font=\large\bfseries\itshape] (produccion) {Producción};

        \draw (concepcion) -- (comunicacion);
        \draw (concepcion) -- (planeacion);

        \draw (elaboracion) -- (planeacion);
        \draw (elaboracion) -- (modelado);

        \draw (construccion_) -- (construccion);




        \node[left=0.5cm of incremento, font=\large\bfseries] {Lanzamiento};
        \node[right=1cm of construccion, font=\large\bfseries\itshape](transicion) {Transición};

        \draw (transicion) -- (construccion);
        \draw (transicion) -- (despliegue);

        \draw (produccion) -- (incremento);

% Espiral principal
        \draw[spiral, ->] (comunicacion.center)
        to[out=60, in=180] (planeacion.center)
        to[out=0, in=120] (modelado.center)
        to[out=-60, in=60] (construccion.center)
        to[out=-120, in=30] (despliegue.center)
        to[out=210, in=90] (incremento.center);

% Flecha curva de retroalimentación
        \draw[spiral, ->] (despliegue.west)
        to[out=180, in=-60, looseness=1.5] (comunicacion.south);

    \end{tikzpicture}

    \subsection{Otros modelos}\label{subsec:otros-modelos}

    \subsubsection{Modelos especializados}

    \begin{itemize}
        \item \textbf{Desarrollo basado en componentes}: utiliza un enfoque evolutivo y iterativo
        \item \textbf{Modelo de métodos formales}: basado en modelos matemáticos para especificación rigurosa
        \item \textbf{Desarrollo orientado a aspectos}: combina enfoques evolutivos y concurrentes
    \end{itemize}

    \subsubsection{Modelos personales y de equipo}

    \begin{itemize}
        \item \textbf{Proceso Personal del Software (PPS)}: metodología individual que incluye:
        \begin{itemize}
            \item Planificación
            \item Diseño
            \item Revisión
            \item Desarrollo
            \item Post mórtem
        \end{itemize}
        \item \textbf{Proceso del Equipo de Software (PES)}: equipos auto-dirigidos que se gestionan autónomamente.
    \end{itemize}


    \subsection{Ciclo de vida del producto}\label{subsec:ciclo-de-vida-del-producto}

    \textbf{Etapas}: Introdución → Crecemento → Madurez → Declive.