\begin{itemize}
    \item \textbf{RAE:} \textquote{Contingencia o proximidad de un daño}
    \item \textbf{Ingeniería del software:}
    \begin{itemize}
        \item Posibilidad de eventos imprevistos con impacto negativo en el proyecto
        \item Implica previsión del futuro, cambio y toma de decisiones
    \end{itemize}

    \item \textbf{Elementos clave:}
    \begin{itemize}
        \item \textbf{Fuente:} Origen del riesgo (ejemplo: nueva tecnología)
        \item \textbf{Evento:} Qué podría ocurrir (ejemplo: incompatibilidad)
        \item \textbf{Impacto:} Consecuencias (ejemplo: retraso 2 semanas)
    \end{itemize}
\end{itemize}

\subsection{Clasificación de Riesgos}
\label{subsec:clasificacion}

\begin{center}
    \begin{tabular}{p{3.5cm}p{9cm}}
        \toprule
        \textbf{Tipo de Riesgo} & \textbf{Descripción y Ejemplos} \\
        \midrule
        \textbf{Proyecto} &
        \begin{itemize}[leftmargin=*]
            \item Calendario o recursos insuficientes
            \item Ejemplo: Desarrollo por detrás de las metas acordadas con el cliente
        \end{itemize} \\
        \midrule
        \textbf{Técnico} &
        \begin{itemize}[leftmargin=*]
            \item Amenaza calidad o planificación
            \item ejemplo: Rendimiento inadecuado bajo carga máxima
        \end{itemize} \\
        \midrule
        \textbf{Producto} &
        \begin{itemize}[leftmargin=*]
            \item Incumplimiento expectativas usuario
            \item ejemplo: Interfaz poco intuitiva para usuarios finales
        \end{itemize} \\
        \midrule
        \textbf{Negocio} &
        \begin{itemize}[leftmargin=*]
            \item Problemas organizativos o de mercado
            \item ejemplo: Cambio en estrategia corporativa que reduce la prioridad del proyecto
        \end{itemize} \\
        \bottomrule
    \end{tabular}
\end{center}

\begin{center}
    \begin{tabular}{p{3.5cm}p{9cm}}
        \toprule
        \textbf{Tipo de Riesgo} & \textbf{Descripción y Ejemplos} \\
        \midrule
        \textbf{Tecnologías} &
        \begin{itemize}[leftmargin=*]
            \item Librerías obsoletas sin soporte
            \item Dependencias con licencias restrictivas
        \end{itemize} \\
        \midrule
        \textbf{Personal} &
        \begin{itemize}[leftmargin=*]
            \item Baja capacitación en tecnología clave
            \item Conflictos interpersonales en equipo
        \end{itemize} \\
        \midrule
        \textbf{Organizativos} &
        \begin{itemize}[leftmargin=*]
            \item Cambios frecuentes en liderazgo
            \item Presupuesto inestable
        \end{itemize} \\
        \midrule
        \textbf{Herramientas} &
        \begin{itemize}[leftmargin=*]
            \item Incompatibilidad entre entornos
            \item Bugs en software de testing
        \end{itemize} \\
        \midrule
        \textbf{Requisitos} &
        \begin{itemize}[leftmargin=*]
            \item Alcance no definido (scope creep)
            \item Requerimientos contradictorios
        \end{itemize} \\
        \midrule
        \textbf{Estimación} &
        \begin{itemize}[leftmargin=*]
            \item Subestimación de complejidad
            \item Falta de datos históricos
        \end{itemize} \\
        \bottomrule
    \end{tabular}
\end{center}

\subsection{Descripción de Riesgos}\label{subsec:descripcion-de-riesgos}

Para cada riesgo identificado, se recomienda recopilar la siguiente información:

\begin{itemize}
    \item \textbf{Descripción:} Cuál es el riesgo y cuál es su origen.

    \item \textbf{Prioridad (impacto):} \textit{alta} | \textit{media} | \textit{baja}

    \item \textbf{Probabilidad:} \textit{alta} | \textit{media} | \textit{baja}

    \item \textbf{Plan de acción:}
    \begin{itemize}
        \item Tipo: \textit{contingencia}, \textit{minimización} o \textit{prevención}
        \item Estrategia: \textit{soportar}, \textit{reducir el impacto}, \textit{evitar}
        \item Acciones planificadas: descripción específica de los pasos a seguir.
    \end{itemize}
    \item \textbf{Responsable:} Persona a cargo de la implementación del plan
    \item \textbf{Estado:} \textit{abierto} | \textit{cerrado}
\end{itemize}

\begin{exemplo}
    \textbf{Descripción:} Posible ausencia prolongada de una persona clave del equipo.

    \textbf{Impacto:} Alto

    \textbf{Probabilidad:} Promedio

    \textbf{Plan de contingencia:}
    \begin{itemize}
        \item Tipo: Contingencia
        \item Estrategia: Reducir el impacto
        \item Acciones: Documentar el trabajo clave y la capacitación cruzada dentro del equipo
    \end{itemize}

    \textbf{Responsable:} Coordinador del proyecto

    \textbf{Estado:} Abierto
\end{exemplo}

\subsection{Estrategias de Gestión de Riesgos}\label{subsec:estrategias-de-gestion-de-riesgos}

La gestión de riesgos puede adoptar diferentes enfoques según el momento oportuno para actuar sobre el riesgo:

\begin{itemize}
    \item \textbf{Estrategias reactivas:} Se actúa tras la materialización del riesgo, evaluando sus consecuencias y tomando medidas correctivas

    \item \textbf{Estrategias proactivas:} El riesgo se considera un posible evento futuro.
    Se establece un plan de contingencia para evitarlo o, en caso de ocurrir, minimizar su impacto
\end{itemize}

Un análisis temprano, sistemático y profundo de los riesgos favorece una estrategia proactiva global y reduce la ocurrencia de problemas imprevistos graves.

\medskip

\textbf{La gestión de riesgos} es el proceso continuo que permite:

\begin{itemize}
    \item Identificar riesgos potenciales
    \item Analizar su naturaleza e impacto
    \item Proponer soluciones antes de que se conviertan en problemas reales
\end{itemize}

Este proceso debe aplicarse continuamente durante todo el ciclo de vida del proyecto.

\begin{nota}
    La integración temprana de la gestión de riesgos en la planificación permite anticipar fallos, tomar decisiones más informadas y reducir costos inesperados.
\end{nota}

\subsection{Proceso de Gestión de Riesgos}
\label{subsec:proceso}
\deactivatequoting
\begin{center}
    \begin{tikzpicture}[node distance=1.5cm, auto]
        \tikzstyle{block} = [rectangle, draw, text width=6em, text centered, rounded corners, minimum height=4em]
        \tikzstyle{line} = [draw, -latex']

        \node [block] (identificar) {Identificación};
        \node [block, right of=identificar, node distance=4cm] (analizar) {Análisis};
        \node [block, right of=analizar, node distance=4cm] (planificar) {Planificación};
        \node [block, below of=analizar, node distance=3cm] (monitorear) {Seguimiento};

        \path [line] (identificar) -- (analizar);
        \path [line] (analizar) -- (planificar);
        \path [line] (planificar) |- (monitorear);
        \path [line] (monitorear) -| (identificar);
        \path [line] (monitorear) -- (analizar);
    \end{tikzpicture}
\end{center}
\activatequoting

\subsection{Actividades por fase}\label{subsec:actividades-por-fase}
\begin{itemize}
    \item \textbf{Identificación}: Listar riesgos potenciales
    \item \textbf{Análisis}: Calcular probabilidad/impacto, priorizar
    \item \textbf{Planificación}: Definir estrategias y planes de acción
    \item \textbf{Seguimiento}:
    \begin{itemize}
        \item Revisión mensual de riesgos activos
        \item Actualizar matriz de riesgos
        \item Activar planes de contingencia si es necesario
    \end{itemize}
\end{itemize}

\clearpage
