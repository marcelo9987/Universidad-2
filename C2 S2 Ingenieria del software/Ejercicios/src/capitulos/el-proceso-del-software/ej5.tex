% !TeX root = ../el-proceso-de-software.tex


\begin{enunciado}
    Indica las similitudes y diferencias entre Scrum y Kanban.
\end{enunciado}


Ambas metodologías ágiles comparten varios principios fundamentales.
Son enfoques iterativos e incrementales que priorizan la entrega continua de valor al cliente y fomentan la mejora continua del proceso.
Tanto Scrum como Kanban enfatizan la transparencia en el trabajo, la colaboración del equipo y la adaptabilidad ante los cambios.
Ambos utilizan tableros visuales para gestionar el flujo de trabajo y permiten que los equipos se autoorganicen en cierta medida.

\subsection{Diferencias Principales}\label{subsec:diferencias-principales}

\subsubsection{Estructura Temporal}
Scrum opera con \textbf{sprints de duración fija} (típicamente 2-4 semanas), mientras que Kanban es un \textbf{flujo continuo} sin iteraciones predefinidas.
En Scrum, el trabajo se planifica y ejecuta en bloques temporales específicos, mientras que en Kanban las tareas fluyen constantemente a través del sistema.

\subsubsection{Roles Definidos}
Scrum establece \textbf{roles específicos} como Product Owner, Scrum Master y Development Team.
Kanban no prescribe roles particulares y se adapta a la estructura organizacional existente.

\subsubsection{Ceremonias}
Scrum incluye \textbf{eventos formales} como Sprint Planning, Daily Standups, Sprint Review y Sprint Retrospective.
Kanban no tiene ceremonias obligatorias, aunque puede incorporar reuniones según las necesidades del equipo.

\subsubsection{Planificación}
En Scrum, la planificación ocurre al \textbf{inicio de cada sprint} con compromisos específicos.
Kanban permite una \textbf{planificación más flexible y continua}, priorizando según la capacidad disponible.

\subsubsection{Limitación del Trabajo}
Kanban utiliza \textbf{límites WIP} (Work in Progress) para controlar la cantidad de trabajo simultáneo.
Scrum limita el trabajo a través de la \textbf{capacidad del sprint}.

\subsubsection{Métricas}
Scrum se enfoca en la \textbf{velocidad del equipo} y burndown charts.
Kanban prioriza el \textbf{tiempo de ciclo}, lead time y el flujo de trabajo.

\subsubsection{Cambios}
Scrum \textbf{protege el sprint} de cambios externos una vez iniciado.
Kanban permite \textbf{cambios continuos} en las prioridades según surgen nuevas necesidades.

\subsection{Conclusión}
\label{subsec:conclusion-scrum-kanban}
\begin{solucion}
    Ambas metodologías son efectivas, pero se adaptan mejor a diferentes contextos organizacionales y tipos de trabajo.
\end{solucion}

\begin{table}[h]
    \centering
    \begin{tabular}{@{}p{4cm}p{5cm}p{5cm}@{}}
        \toprule
        \textbf{Aspecto}    & \textbf{Scrum}            & \textbf{Kanban}         \\
        \midrule
        Estructura temporal & Sprints fijos             & Flujo continuo          \\
        Roles               & Definidos específicamente & Flexibles               \\
        Ceremonias          & Obligatorias              & Opcionales              \\
        Planificación       & Por sprint                & Continua                \\
        Limitación trabajo  & Capacidad sprint          & Límites WIP             \\
        Métricas            & Velocidad, burndown       & Tiempo ciclo, lead time \\
        Cambios             & Protegidos durante sprint & Continuos               \\
        \bottomrule
    \end{tabular}
    \caption{Comparación resumida entre Scrum y Kanban}\label{tab:comparacion-scrum-kanban}
\end{table}
