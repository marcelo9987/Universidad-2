% !TeX root = ../modelado.tex


\begin{enunciado}
    Desarrolla el siguiente requisito como 1) caso de uso y como 2) historias de usuario.

    \textquote{Búsqueda por nombre de un aula para programar una clase en un campus dado}.
\end{enunciado}

\subsection{1. Caso de uso}\label{subsec:1.-caso-de-uso}
\begin{solucion}[Caso de uso]
    \textbf{Nombre}: Búsqueda de aula por nombre para programar una clase

    \textbf{Actor principal}: Personal docente o administrativo autorizado

    \textbf{Interesados y necesidades}:

    \begin{itemize}
        \item \textbf{Personal docente}: Necesita encontrar rápidamente un aula concreta por su nombre para poder planificar una clase.
        \item \textbf{Sistema de programación de aulas}: Debe permitir búsquedas eficientes y devolver resultados precisos dentro del campus seleccionado.
    \end{itemize}
    \textbf{Precondiciones}:

    \begin{itemize}
        \item El usuario ha iniciado sesión correctamente.
        \item El sistema contiene información actualizada de aulas y su disponibilidad.
        \item Se ha seleccionado un campus.
    \end{itemize}
    \textbf{Postcondiciones}:

    \begin{itemize}
        \item Se muestran los datos del aula correspondiente si existe y pertenece al campus indicado.
        \item Si no existe, se notifica al usuario.
    \end{itemize}
    \textbf{Escenario principal (flujo básico)}:

    \begin{enumerate}
        \item El usuario accede a la funcionalidad de programación de clases.
        \item El sistema solicita el campus y el nombre del aula.
        \item El usuario introduce el nombre del aula y selecciona el campus.
        \item El sistema busca el aula correspondiente dentro del campus.
        \item El sistema muestra la información del aula si existe.
        \item El usuario puede continuar con la programación de la clase.
    \end{enumerate}
    \textbf{Flujos alternativos}:

    4a) Si el nombre del aula no existe en ese campus, el sistema notifica que no se ha encontrado ninguna coincidencia.
\end{solucion}

\subsection{2. Historias de usuario}\label{subsec:2.-historias-de-usuario}
\begin{solucion}[Historias de usuario]
    \textbf{H1}

    \textbf{Como} miembro del personal docente

    \textbf{quiero} buscar un aula por su nombre dentro de un campus

    \textbf{para} poder programar una clase de forma rápida sin revisar toda la lista.

    \textbf{Criterios de aceptación}:
    \begin{itemize}
        \item Puedo introducir el nombre del aula y seleccionar un campus.
        \item El sistema me muestra la información del aula si existe.
        \item Si el aula no existe, me informa adecuadamente.
    \end{itemize}

\end{solucion}