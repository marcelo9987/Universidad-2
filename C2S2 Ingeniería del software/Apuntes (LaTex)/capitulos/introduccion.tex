\setcounter{minitocdepth}{3} % 1 = seccións, 2 = subseccións, 3 = subsubseccións
\minitoc

\section{Conceptos básicos}\label{sec:tema-1.1---conceptos-basicos}

\subsection{Historia: la crisis del software}
\begin{itemize}
    \item Término acuñado en 1968 en una conferencia de la OTAN\@.
    \item Refleja la dificultad para desarrollar software útil y eficiente en tiempo y forma.
    \item Ejemplos de grandes fracasos:
    \begin{itemize}
        \item \textbf{Mariner 1 (1962):} pérdida de $18\text{.}500\text{.}000\$$.
        \item \textbf{Therac-25 (1982):} 3 fallecidos y 3 con secuelas por errores de software.
        \item \textbf{Caída AT\&T (1990):} 75 millones de llamadas afectadas.
        \item \textbf{Knight Capital (2012):} pérdida de $500\text{.}000\text{.}000\$$.
    \end{itemize}
\end{itemize}

\subsection{Definiciones}


\begin{definicion}
    \textbf{Software:}
    \begin{itemize}
        \item Instrucciones que al ejecutarse proporcionan funciones, características y rendimiento.
        \item Estructuras de datos que permiten manipular información adecuadamente.
    \end{itemize}
\end{definicion}

\begin{definicion}
    \textbf{Ingeniería del Software (IEEE, 1993):} Aplicación de un enfoque sistemático, disciplinado y cuantificable al desarrollo, operación y mantenimiento del software.
\end{definicion}

\begin{definicion}
    \textbf{Ingeniería del Software (Fritz Bauer, 1969):} Uso de principios fundamentales de la ingeniería para desarrollar software fiable y eficiente de forma económica.
\end{definicion}

\subsection{Conceptos clave}

\begin{itemize}
    \item \textbf{Personas:} Creadores del producto.
    \item \textbf{Producto:} Resultado final del desarrollo.
    \item \textbf{Usuarios:} Receptores del producto.
    \item \textbf{Proyecto:} Conjunto de hitos hacia un resultado.
    \item \textbf{Proceso:} Fases del desarrollo del software.
    \item \textbf{Ciclo de vida:} Evoluciones del producto en el tiempo.
\end{itemize}

\subsection{Roles en el proceso}
\begin{itemize}
    \item \textbf{Gestor de producto:} Define funcionalidades y prioridades.
    \item \textbf{Gestor de proyecto:} Asegura recursos para cumplir el plan.
    \item \textbf{Ingeniero de software:} Diseña e implementa el software.
    \item \textbf{Ingeniero de calidad:} Verifica el correcto funcionamiento del producto.
\end{itemize}


\section{Estándares y organizaciones}\label{sec:estandares-y-organizaciones}

\subsection{Necesidad de estandarizar}
\begin{itemize}
    \item \textbf{Proceso tipo:} Marco para valorar e identificar mejoras.
    \item \textbf{Buenas prácticas:} Lo que funciona se institucionaliza.
    \item \textbf{Lenguaje común:} Mejora la comunicación entre roles.
\end{itemize}

\subsection{Usos de los estándares}
\begin{itemize}
    \item \textbf{Certificación:} Requisito para:
    \begin{itemize}
        \item Acceder a proyectos regulados.
        \item Comercializar productos y servicios.
    \end{itemize}
\end{itemize}

\subsection{Organizaciones relevantes}
\begin{itemize}
    \item \textbf{IEEE:} Institute of Electrical and Electronics Engineers (global).
    \item \textbf{SEI:} Software Engineering Institute (EUA).
    \item \textbf{ISO:} International Organization for Standardization (global).
    \item \textbf{IEC:} International Electrotechnical Commission (global).
    \item \textbf{EIA:} Electronic Industries Alliance (EUA).
\end{itemize}

\subsection{Principales estándares ISO}

\begin{itemize}
    \item \textbf{ISO 9126:} Evalúa la calidad del software en:
    \begin{itemize}
        \item Funcionalidad, fiabilidad, usabilidad, eficiencia, mantenibilidad, portabilidad y satisfacción.
    \end{itemize}
    \item \textbf{Familia ISO 9000:} Gestión de calidad
    \begin{itemize}
        \item \textbf{ISO 9000:} Fundamentos y vocabulario.
        \item \textbf{ISO 9001:} Requisitos de calidad.
        \item \textbf{ISO 9004:} Mejora continua.
    \end{itemize}

    \item \textbf{ISO/IEC 12207:} Procesos del ciclo de vida del software: procesos principales, de apoyo y organizativos.

    \item \textbf{ISO/IEC 15504 (SPICE):} Evalúa y mejora procesos de ingeniería del software.
\end{itemize}

\subsection{SPICE – Niveles de madurez}

\begin{description}
    \item[Nivel 0 - Incompleto:] Sin implementación efectiva.
    \item[Nivel 1 - Realizado:] Procesos implementados y objetivos alcanzados.
    \item[Nivel 2 - Gestionado:] Procesos y productos controlados.
    \item[Nivel 3 - Establecido:] Procesos basados en estándares.
    \item[Nivel 4 - Predecible:] Gestión cuantitativa con objetivos.
    \item[Nivel 5 - Optimizado:] Mejora continua y búsqueda de buenas prácticas.
\end{description}

\subsection{CMMI (Capability Maturity Model Integration)}

\begin{itemize}
    \item Establecido por el SEI\@.
    \item Actualmente gestionado por el CMMI Institute.
    \item Basado en SPICE, muy popular en EUA\@.
\end{itemize}