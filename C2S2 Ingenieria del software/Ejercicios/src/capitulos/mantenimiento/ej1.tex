% !TeX root = ../mantenimiento.tex

\begin{enunciado}
    Calcula la complejidad y tamaño del siguiente sistema:
    \begin{enumerate}
        \item Componente A
        \begin{enumerate}
            \item Variables de entrada: 3
            \item Variables de salida: 2
            \item Depende de: 1 servicio
        \end{enumerate}

        \item Componente B
        \begin{enumerate}
            \item Variables de entrada: 2
            \item Variables de salida: 3
            \item Depende de: 2 servicios
        \end{enumerate}
        \item Componente C
        \begin{enumerate}
            \item Variables de entrada: 1
            \item Variables de salida: 1
            \item Depende de: 0 servicios
        \end{enumerate}
    \end{enumerate}
\end{enunciado}

\subsection{Análisis usando Métricas de Diseño Arquitectónico}\label{subsec:analisis-usando-metricas-de-diseno-arquitectonico}
Para aplicar las fórmulas de complejidad de módulo, interpretamos:
\begin{itemize}
    \item Variables de salida $\rightarrow$ $f_{out}(i)$ (módulos que dependen del módulo $i$)
    \item Servicios $\rightarrow$ $v(i)$ (número de dependencias del módulo)
\end{itemize}

\subsection{Cálculos de Complejidad por Componente}\label{subsec:calculos-de-complejidad-por-componente}

\subsubsection{Componente A}
\begin{align}
    f_{out}(A) &= 2 \text{ (variables de salida)} \\
    v(A) &= 1 \text{ (servicios de los que depende)}\\
    S(A) &= f_{out}^2(A) = 2^2 = 4 \\
    D(A) &= \frac{v(A)}{f_{out}(A) + 1} = \frac{1}{2 + 1} = \frac{1}{3} = 0.33 \\
    C(A) &= S(A) + D(A) = 4 + 0.33 = 4.33
\end{align}

\subsubsection{Componente B}
\begin{align}
    f_{out}(B) &= 3 \text{ (variables de salida)} \\
    v(B) &= 2 \text{ (servicios de los que depende)} \\
    S(B) &= f_{out}^2(B) = 3^2 = 9 \\
    D(B) &= \frac{v(B)}{f_{out}(B) + 1} = \frac{2}{3 + 1} = \frac{2}{4} = 0.5 \\
    C(B) &= S(B) + D(B) = 9 + 0.5 = 9.5
\end{align}

\subsubsection{Componente C}
\begin{align}
    f_{out}(C) &= 1 \text{ (variables de salida)} \\
    v(C) &= 0 \text{ (servicios de los que depende)} \\
    S(C) &= f_{out}^2(C) = 1^2 = 1 \\
    D(C) &= \frac{v(C)}{f_{out}(C) + 1} = \frac{0}{1 + 1} = 0 \\
    C(C) &= S(C) + D(C) = 1 + 0 = 1
\end{align}

\subsection{Complejidad Total del Sistema}\label{subsec:complejidad-total-del-sistema}
\[C_{total} = C(A) + C(B) + C(C) = 4.33 + 9.5 + 1 = 14.83\]

\subsection{Cálculo del Tamaño del Sistema}\label{subsec:calculo-del-tamano-del-sistema}

\subsubsection{Número de nodos (n)}
\begin{itemize}
    \item Componentes: A, B, C = 3 nodos
    \item Servicios externos: 1 (para A) + 2 (para B) + 0 (para C) = 3 servicios
    \item \textbf{Total de nodos}: $n = 3 + 3 = 6$
\end{itemize}

\subsubsection{Número de arcos (a)}
Los arcos representan las dependencias:
\begin{itemize}
    \item A depende de 1 servicio: 1 arco
    \item B depende de 2 servicios: 2 arcos
    \item C no depende de servicios: 0 arcos
    \item \textbf{Total de arcos}: $a = 1 + 2 + 0 = 3$
\end{itemize}

\subsubsection{Tamaño total}
\[\text{Tamaño} = n + a = 6 + 3 = \boxed{9}\]



\begin{solucion}
    \subsection{Resumen}\label{subsec:resumen}
    \begin{gather*}
        C_{total} = C(A) + C(B) + C(C) = 4.33 + 9.5 + 1 = 14.83\\
        \text{Tamaño} = n + a = 6 + 3 = \boxed{9}\\
    \end{gather*}
\end{solucion}

\subsection{Resumen de Resultados}\label{subsec:resumen-de-resultados}
\begin{table}[h!]
    \center
    \begin{tabular}{|l|l|}
        \hline
        \textbf{Métrica}                       & \textbf{Valor} \\
        \hline
        Complejidad del Componente A           & 4.33           \\
        Complejidad del Componente B           & 9.5            \\
        Complejidad del Componente C           & 1.0            \\
        \hline
        \textbf{Complejidad Total del Sistema} & \textbf{14.83} \\
        \textbf{Tamaño del Sistema}            & \textbf{9}     \\
        \hline
    \end{tabular}\label{tab:table}
\end{table}